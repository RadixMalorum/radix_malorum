%
\chapter{Introduzione}
\section{Prefazione}
%%
\subsection{Cos'\`e un Gioco di Ruolo}
Quello che avete fra le mani e che vi apprestate a leggere \`e il
manuale di un Gioco di Ruolo, che gli addetti ai lavori chiamano {\bf
GdR}. A beneficio di chi non conosce il significato del termine,
forniamo una breve spiegazione.

Il modo pi\`u semplice per descrivere il contenuto e lo scopo del
GdR consiste nell'analizzare il termine stesso ``Gioco di Ruolo''. In
quanto {\bf gioco} in esso \`e predominante l'aspetto ludico e ricreativo
con lo scopo di divertire i partecipanti; il fine principale di ciascuno \`e
quello di interpretare una parte, un {\bf ruolo}, che gli permetta
di calarsi, con l'aiuto della propria fantasia, nei panni di un
personaggio, e di interpretarlo, muovendolo con l'ausilio di regole
all'interno della trama intessuta da un altro giocatore regista.
\iffullversion

Questo mette in evidenza una prima distinzione fra i giocatori:
coloro che si calano nei panni dei cosiddetti Personaggi Giocanti o
{\bf PG}, e colui che crea la storia, la trama all'interno del quale i PG si
muoveranno, come il regista di un'opera. Il ``regista'' nella
maggior parte dei GdR prende il nome di {\bf Master} (o Arbitro, o qualcosa
del genere).

Il GdR esalta la creativit\`a delle persone portandole a
sperimentare i modi di interagire pi\`u adatti alle situazioni in
cui il Master decide di far imbattere i PG. L'interpretazione del
ruolo \`e cardine fondamentale del GdR e non viene sminuita nel
ruolo del Master, il quale non \`e il creatore di una trama sterile
e prestabilita come potrebbe essere quella di un film, ma rappresenta
tutto il mondo dei PG, comprese tutte le persone che i questi potranno
incontrare durante lo svilupparsi del gioco. 

In altre parole, il
Master stende un canovaccio, una trama di cui decide i punti salienti
e i personaggi ``complementari'' che egli dovr\`a gestire (detti
Personaggi Non Giocanti o {\bf PNG}), mentre i giocatori hanno il compito di
sviluppare le linee guida della storia, tramite le proprie azioni, la
reciproca interazione e l'interazione con il Master; il tutto mediante
l'applicazione delle regole che permettono di far vivere e agire i PG,
che restano comunque personaggi di fantasia.
\fi

\subsection{Il GdR Fantasy}

Il gioco di ruolo ha conosciuto nella sua storia adattamenti pi\`u o
meno riusciti a molteplici ambientazioni, reali o parto della fantasia
di autori pi\`u o meno noti, quali Tolkien, Gibson, Rice, solo per
citarne alcuni.

Sul mercato si trovano GdR per tutti i palati: Horror, Gotico,
Fantascientifico, Cyberpunk, Demenziale, Storico ecc. Tuttavia ci
sentiamo di affermare che il genere pi\`u diffuso \`e il Fantasy,
dove storia e fantasia si mischiano in una alchimia elettrizzante,
dove la spada si incrocia con la magia (il classico \textit{Sword \&
  Sorcery}) e in cui abbondano creature fantastiche e mitologiche.

Il progenitore del GdR Fantasy \`e senza dubbio Dungeons \& Dragons,
che ha lanciato una moda e segnato un'epoca.

Il gioco che state per conoscere, \textbf{Radix Malorum}, appartiene a
tale categoria. Esso rispecchia, con le dovute differenze, i capisaldi
del genere fantasy. Radix Malorum \`e, ciononostante, frutto della
fantasia dei suoi autori e la sua ambientazione \`e totalmente
originale e scaturita insieme alle altre regole da un duro lavoro di
sperimentazione durato quasi nove anni, senza negare il contributo
datoci dagli autori che hanno fatto nascere in noi la passione per
questo genere.

{\sloppy\raggedright \subsection{Le caratteristiche di Radix Malorum}}

Fra i punti di forza di Radix Malorum possiamo ricordare:

\begin{itemize}

\item la dettagliata ambientazione \textbf{originale}

\item il sistema di \textbf{miglioramento} delle
abilit\`a, proporzionato alla complessit\`a delle abilit\`a
stesse;

\item il \textbf{sistema di combattimento}, studiato per garantire
un buon equilibrio tra realismo e spettacolarit\`a, in cui anche un
solo colpo di pugnale pu\`o uccidere, in cui \`e prevista la
possibilit\`a per le armature di danneggiarsi e logorarsi, e per le
armi di rompersi; 

\item infine, un sistema di \textbf{creazione immediata
della magia} che permette di attivare incantesimi personalizzati da
adattare ad ogni situazione.

\end{itemize}

Vi sveliamo a questo punto un ``segreto'' che potrebbe soddisfare le
curiosit\`a che sicuramente vi sorgeranno nella lettura: gli autori di
Radix Malorum sono sardi e molto legati alla loro terra. In pi\`u
punti del manuale troverete riferimenti a miti, luoghi e tradizioni
della Sardegna, riferimenti che sono stati ritenuti dagli autori un
doveroso omaggio alla loro isola.

Per scoprire un nuovo meraviglioso mondo, non dovrete far altro che
addentrarvi nella lettura, senza spaventarvi alle prime difficolt\`a
di applicazione delle regole. E vi garantiamo che ne sar\`a valsa la
pena.

%%
{\sloppypar\raggedright \section{Il Mondo di Quadrantal}}
%%
Il mondo in cui vi apprestate ad entrare, per scoprirne misteri,
meraviglie, passioni e paure, \`e stato chiamato, dai suoi abitanti
originali, Quadrantal.  Il nome \textbf{Quadrantal} significa nell'antica
lingua ``dado'', infatti gli indigeni pensavano che esso presentasse
tale forma.

\iffullversion
Attorno a Quadrantal orbitano due satelliti naturali, due lune
chiamate \textbf{Mohwr} e \textbf{Alphan}. La prima appare di colore
rosso, mentre la seconda di colore bianco.

L'anno solare dura 300 giorni. 
\fi

\subsection{L'Arcipelago}

Il cosiddetto ``mondo conosciuto'', chiamato \textbf{Arcipelago},
\`e un continente, situato interamente nell'emisfero settentrionale
del pianeta, costituito da 7 grandi isole disposte in cerchio e
delimitanti un grande specchio d'acqua chiamato \textbf{Mare Interno}.
Al centro del Mare Interno sorge una piccola isola detta Isola del
\textbf{Gran Consiglio}.

Gli abitanti dell'Arcipelago ignorano quanto si trovi nel resto del
pianeta, oltre il cosiddetto \textbf{Mare Esterno}. Questo, infatti,
non \`e da essi navigabile se non nelle immediate vicinanze delle
isole. 

Ogni nazione nell'Arcipelago ha un proprio calendario, ma durante la
\textbf{Dominazione Levante}, per facilitare i fiorenti scambi
commerciali, \`e stato adottato un calendario comune, detto Calendario
Commerciale, utilizzato in tutto il continente, in cui l'anno solare
\`e diviso in 10 mesi di 30 giorni ciascuno.

\iffullversion
I mesi sono: 

\begin{enumerate}\itemsep -6pt
\item Titheea
\item Tebidow
\item Kalhentee
\item Bahska
\item Bowdhyow
\item Proydee
\item Skow-Thowladah
\item Fryow-Sow
\item Nyee
\item Jelhow 
\end{enumerate}

Il passaggio tra le fasce climatiche non \`e netto, ma graduale. \`E
possibile comunque individuare quattro fasce principali.

Nella \textbf{zona artica} la temperatura \`e sempre molto bassa e
non piove mai, perch\'e nevica.

Nella \textbf{zona fredda} i mesi 1,2,9,10 sono invernali, 3,4
primaverili, 5 estivo e 6,7,8,9 autunnali. Le precipitazioni sono
distribuite pi\`u o meno uniformemente su tutto l'arco dell'anno.

Nella \textbf{zona temperata} le stagioni sono ben definite, i mesi
1,9,10 sono invernali, 2 e 3 primaverili, 4,5,6 estivi e 7,8
autunnali. La stagione estiva \`e secca.  l'alternarsi delle
stagioni \`e graduale.

Nella \textbf{zona tropicale}, in cui la temperatura \`e sempre
alta, esistono soltanto due stagioni, la stagione secca, nei mesi
9,10,1,2,3 e la stagione delle piogge nei mesi restanti. Durante la
stagione delle piogge piove 5 giorni su 6. Nei deserti tropicali,
anche durante la cosiddetta stagione delle piogge, non piove
praticamente mai.

Troverete la Mappa dell'Arcipelago, in cui sono indicate anche le zone
climatiche, allegata al manuale.

\subsection{La Fauna}
Nella tabella a pagina \pageref{tabanimali} sono indicate le razze
animali pi\`u comuni nell'Arcipelago. Una descrizione dettagliata
\`e fornita soltanto per le specie caratteristiche. Per le altre,
basta consultare un buon testo di zoologia.

Nella tabella troverete indicati:
\begin{description}
\item {la \textbf{Specie}} o gruppo di specie a cui appartengono;
\item {\textbf{CAT}}, i valori delle caratteristiche INT,
    CoNC, COS, AGI, FOR, OSS;
  
  \item {la \textbf{AMM}} Difficolt\`a del tiro Ammaestrare;
  \item{la \textbf{CON}} Difficolt\`a del tiro Conoscere Animali,
    che pu\`o aumentare per casi particolari a discrezione del
    Master, ad es. per sottospecie particolarmente rare;
  \item{\textbf{Danni}} che gli esemplari della specie possono
    infliggere con gli attacchi naturali (ad essi andr\`a aggiunto
    il bonus forza).
  \item{\textbf{Esempi}} di animali di quella specie.
  \item{\textbf{Note}} e Particolarit\`a della specie.

\end{description}

Il TOT di ogni abilit\`a di combattimento (Artigliata, Morso,
Schivata ecc.) se non indicato diversamente nella descrizione, \`e
uguale all'AGI. 

\nb{Non lasciatevi spaventare da queste ``strane sigle''! TOT e AGI
  indicano rispettivamente TOTale di un'abilit\`a e la caratteristica
  AGIlit\`a. Per il momento sono nomi oscuri, ma la mente vi si
  aprir\`a dopo aver letto il capitolo '' Personaggio'' a pagina
  \pageref{personaggio}}

\subsubsection{Fortral}
\pinupp{fortral.eps}{Fortral}{b}

Sono le cavalcature bipedi pi\`u diffuse. Sono dei massicci uccelli
rapaci corridori, incapaci di volare, dotati di un grosso becco adunco
usato per attaccare.

I Fortral vivono in climi temperati, sono ricoperti di penne su tutto
il corpo tranne che nella parte inferiore delle zampe, terminanti con
robusti artigli che migliorano la presa sui terreni fortemente
scoscesi. Il piumaggio \`e grigio-bruno, pi\`u scuro sul dorso.

\subsubsection{Nortral}

I Nortral sono una variet\`a di Fortral adattatasi ai climi rigidi
delle terre del nord.  Differiscono dai loro cugini del Sud sia per il
colore delle penne che si presenta grigio-bianco che per le loro ali
che sono leggermente pi\`u sviluppate, permettendo loro di spingersi
in alto nell'eventualit\`a che restino bloccati nella neve troppo
soffice e profonda.

Le loro zampe sono interamente coperte da piume e terminano con degli
artigli palmati che garantiscono una distribuzione del peso tale da
facilitare l'andatura su terreni innevati.

\subsubsection{Reptis} 

Sono le cavalcature volanti pi\`u diffuse. Sono simili a piccoli
pterodattili con ali membranose ed un collo esile e allungato che li
rende inadatti alle battaglie.

Non essendo dotati di coda, il loro timone \`e costituito da
un'escrescenza cartilaginea della testa, il loro becco \`e lungo e
robusto e dotato di piccoli denti.

\subsubsection{Krenay} 

Sono grosse lucertole umanoidi che possono facilmente raggiungere i
due metri di altezza.  Hanno un collo tozzo e robusto; i loro arti
superiori, allungati, terminano con dei potenti artigli.

\pinup{krenay.eps}{Krenay}

Sono organizzati e cacciano in branco, e pu\`o capitare che a volte
brandiscano dei bastoni o delle ossa come armi da botta. 

La loro pelle verde \`e squamosa ed \`e un'ottima protezione
naturale.

\`E facile incontrarli sul ``suolo'' di Umbrosa; tuttavia sono stati
avvistati anche nelle foreste di Terranova e Norda. Inoltre si \`e
trovata una specie a loro simile nelle isole a Sud di Loydi-Genya.

Gli Umbra li hanno soprannominati ``mangiatori di cuori'' data la
predilezione che nutrono per il cuore delle loro vittime.

I krenay temono il fuoco, ma non fuggono davanti ad esso. Attaccano
chiunque gli si trovi davanti, soprattutto se sono affamati.

\subsubsection{Tartarughe Tsyu-betch} 

Sono tartarughe marine lunghe fino a quattro metri che vivono nel Mare
Esterno e si nutrono di pesci e calamari. Ogni 5 anni si avvicinano
alla terraferma per depositare le uova.

Sono rare e piuttosto ricercate visto che il loro guscio viene
utilizzato per costruire oggetti magici. I pi\`u capaci cacciatori
di Tsyu-betch sono i Chire.

\subsubsection{Ragni giganti}

Nell'Arcipelago vivono due principali variet\`a di ragni giganti.

I primi sono chiamati Marhya-Far-Hank. Sono diffusi nelle terre
Reuben. Non sono velenosi, ma il loro morso \`e dolorosissimo e
pu\`o trasmettere la malattia detta ``Mows Sihadura''.

Il guscio dell'addome viene utilizzato dai Reuben per costruire
corazze.  La durezza del guscio aumenta di pari passo con l'et\`a
del ragno. Questi ragni hanno zampe molto lunghe e affusolate, non
sono pelosi e possono raggiungere due metri di lunghezza.

I secondi, che sono chiamati Tray-Toree dagli elfi, vivono ad Umbrosa
e a Bahuney, nelle caverne o fra le fronde basse degli alberi di
Segram. Hanno grossi addomi e zampe tozze e pelose e possono
raggiungere i due metri di lunghezza.

I Tray-Toree sono di colore verde scuro o marrone e si mimetizzano
facilmente fra le fronde, il loro morso \`e molto doloroso e inietta
un veleno chiamato Kow-how, che porta alla paralisi. Questi ragni
conservano le vittime all'interno di bozzoli.

\subsubsection{Insetti giganti} 

Gli insetti giganti abitano solitamente le foreste di segram di
Umbrosa e Bahuney, ma si possono incontrare anche nelle altre terre
conosciute. Ve ne sono di ogni tipo e possono raggiungere i due metri
di lunghezza.

\subsubsection{Troll} 

I troll sono grossi umanoidi dalla pelle color ocra, bitorzoluti,
butterati e con il naso schiacciato che abitano la giungla
sull'altopiano di Loydi-Genya. Sono molto stupidi ma in compenso
possono raggiungere i 3 metri e mezzo di altezza.  Temono il fuoco e
solitamente fuggono, se non sono troppo affamati, alla sua vista.  A
volte brandiscono grosse clave ricavate da ossi o alberi.  I troll si
nutrono di tutto quello che capita loro a tiro.

I troll sono nemici naturali dei giganti anche se hanno poche
occasioni di incontrarli; quando accade gli scontri sono furiosi e
all'ultimo sangue.

\subsubsection{Stelle Marine Kalareyna}

Sono stelle marine di circa 20 cm di diametro, che si trovano sul
fondo del Mare Esterno, 

Vengono pescate perch\'e, essiccate, sono adatte alla costruzione di
oggetti magici. Le kalareyna essiccate pesano da 10 a 20 grammi.

{

\setlength{\tabcolsep}{0.25em}
\onecolumn\sloppy
\centering
{\Large\sc Animali}
\label{tabanimali}
\newcommand\tabcell{\raggedleft}
\footnotesize

%% \begin{longtable}[c]{|p{1.7cm}*{8}{|p{0.75cm}}*{3}{|p{1.8cm}}|}
\begin{longtable}[c]{|p{1.7cm}*{8}{|p{0.75cm}}*{3}{|p{2cm}}|}
  \par
  \hline
  Specie&INT&CNC&COS&AGI&FOR&OSS&AMM&CON&Danni&Esempi&Note \\ \hline\hline
  \endfirsthead
  \hline
  Specie&INT&CNC&COS&AGI&FOR&OSS&AMM&CON&Danni&Esempi&Note \\ \hline\hline
  \endhead
  \raggedright Equini  Piccoli & \raggedright 1d4 & \raggedright 4+ 2d8 & \raggedright 10+ 1d10 & \raggedright 10+ 1d10 & \raggedright 10+ 1d10 & \raggedright 10+ 1d10 & \raggedright 20 & \raggedright 5 & \raggedright 1d4 (Calcio) & \raggedright Pony, Asini & \ \tabularnewline \hline
  \raggedright Equini\linebreak Medi & \raggedright 1d4 & \raggedright 4+ 2d8 & \raggedright 14+ 1d8 & \raggedright 10+ 1d10 & \raggedright 14+ 1d8 & \raggedright 10+ 1d10 & \raggedright 20 & \raggedright 5 & \raggedright 1d6 (Calcio) & \raggedright Muli, Cavalli, Zebre & \ \tabularnewline \hline
  \raggedright Equini\linebreak Grandi & \raggedright 1d4 & \raggedright 4+ 2d8 & \raggedright 17+ 1d8 & \raggedright 10+ 1d10 & \raggedright 17+ 1d8 & \raggedright 10+ 1d10 & \raggedright 20 & \raggedright 5 & \raggedright 2d6 (Calcio) & \raggedright Cavallo da guerra, cavallo da tiro & \tabularnewline \hline
  \raggedright Cammello & \raggedright 1d4 & \raggedright 4+ 2d8 & \raggedright 17+ 1d8 & \raggedright 10+ 1d10 & \raggedright 17+ 1d8 & \raggedright 10+ 1d10 & \raggedright 20 & \raggedright 10 & \raggedright 2d6 (Calcio) & \raggedright Cammello, dromedario & \tabularnewline \hline
  \raggedright Fortral\linebreak e Nortral & \raggedright 1d4 & \raggedright 4+ 2d8 & \raggedright 12+ 1d8 & \raggedright 15+ 1d10 & \raggedright 14+ 1d8 & \raggedright 10+ 1d10 & \raggedright 20 & \raggedright 5 & \raggedright 1d8 (Becco) & \raggedright V. Descrizione & \tabularnewline \hline
  \raggedright Reptis & \raggedright 1d4 & \raggedright 4+ 2d8 & \raggedright 10+ 1d10 & \raggedright 15+ 1d10 & \raggedright 16+ 1d6 & \raggedright 15+ 1d10 & \raggedright 23 & \raggedright 15 & \raggedright 1d6 (Becco o Artiglio) & \raggedright V. Descrizione & \tabularnewline \hline
  \raggedright Felini\linebreak piccoli & \raggedright 1d6 & \raggedright 4+ 2d8 & \raggedright 2+ 1d8 & \raggedright 13+ 2d8 & \raggedright 2+ 1d8 & \raggedright 15+ 1d10 & \raggedright 30 & \raggedright 5-20 & \raggedright 1d4 (Morso o Artiglio) & \raggedright Gatti, linci & \tabularnewline \hline
  \raggedright Felini\linebreak grandi & \raggedright 1d6 & \raggedright 4+ 2d8 & \raggedright 11+ 2d8 & \raggedright 13+ 2d6 & \raggedright 11+ 2d8 & \raggedright 15+ 1d10 & \raggedright 30 & \raggedright 10-25 & \raggedright 2d6 (Morso o Artiglio) & \raggedright Leoni, Tigri, Pantere, Leopardi ecc. & \tabularnewline \hline
  \raggedright Canidi\linebreak piccoli & \raggedright 1d6 & \raggedright 4+ 2d8 & \raggedright 3+ 1d8 & \raggedright 10+ 1d10 & \raggedright 3+ 1d8 & \raggedright 10+ 1d10 & \raggedright 10 & \raggedright 5-20 & \raggedright 1d6+ 1 (Morso) & \raggedright Piccoli Cani, Dingo, Coyote, Volpe & \tabularnewline \hline
  \raggedright Canidi\linebreak grandi & \raggedright 1d6 & \raggedright 4+ 2d8 & \raggedright 11+ 1d10 & \raggedright 10+ 1d10 & \raggedright 11+ 1d10 & \raggedright 10+ 1d10 & \raggedright 10 & \raggedright 10-20 & \raggedright 1d8+ 1 (Morso) & \raggedright Grossi cani, lupi & \tabularnewline \hline
  \raggedright Elefante & \raggedright 1d4 & \raggedright 15+ 1d10 & \raggedright 40+ 2d10 & \raggedright 6+ 1d6 & \raggedright 30+ 2d10 & \raggedright 6+ 1d10 & \raggedright 22 & \raggedright 15 & \raggedright 1d10 (Testata, in Carica), 2d8 (Calcio) & \raggedright Elefante Africano, Indiano. & \raggedright Attacco in carica: somma la FOR al Danno; pelle 3 PP \tabularnewline \hline
  \raggedright Rinoceronte & \raggedright 1d4 & \raggedright 4+ 2d8 & \raggedright 30+ 1d10 & \raggedright 9+ 1d6 & \raggedright 25+ 1d10 & \raggedright 2+ 1d10 & \raggedright 40 & \raggedright 20 & \raggedright 2d8 (Corno, in Carica) & \raggedright Rinoceronte Nero, indiano ecc. & \raggedright Attacco in carica: somma la FOR al Danno; pelle 5 PP \tabularnewline \hline
  \raggedright Krenay & \raggedright 4+ 1d8 & \raggedright 4+ 2d8 & \raggedright 10+ 2d6 & \raggedright 10+ 2d6 & \raggedright 10+ 2d6 & \raggedright 10+ 2d6 & \raggedright 40 & \raggedright 25 & \raggedright 2d6 (Artiglio), 1d6 (Morso) & \raggedright V. Descrizione & \raggedright Pelle 3 PP\tabularnewline \hline
  \raggedright Orsi & \raggedright 1d4 & \raggedright 4+ 2d8 & \raggedright 5+ 2d10 & \raggedright 8+ 1d6 & \raggedright 5+ 2d10 & \raggedright 10+ 1d10 & \raggedright 25 & \raggedright 20 & \raggedright 2d6 (Artiglio), 1d8 (Morso) & \raggedright Grizzly, orso polare, panda & \tabularnewline \hline
  \raggedright Bovini & \raggedright 1d4 & \raggedright 4+ 2d8 & \raggedright 15+ 2d6 & \raggedright 1d10 & \raggedright 10+ 2d6 & \raggedright 10+ 1d10 & \raggedright 5 & \raggedright 10 & \raggedright 1d6 (Cornata, in Carica) & \raggedright Buoi, Tori, Yak, Bisonti & \raggedright Attacco in carica: somma la FOR al Danno \tabularnewline \hline
  \raggedright Suini & \raggedright 1d4 & \raggedright 4+ 2d8 & \raggedright 8+ 3d6 & \raggedright 2+ 1d10 & \raggedright 8+ 2d6 & \raggedright 10+ 1d10 & \raggedright 15 & \raggedright 10 & \raggedright 1d6 (Testate, in Carica), 1d6 (Morso) & \raggedright Maiali, Cinghiali & \raggedright Attacco in carica: somma la FOR al Danno \tabularnewline \hline
  \raggedright Ragni\linebreak piccoli & \raggedright - & \raggedright - & \raggedright - & \raggedright 1d10 & \raggedright - & \raggedright 1d6 & \raggedright - & \raggedright 15-30 & \raggedright - & \raggedright Vedove nere & \raggedright Iniettano veleno se il TPC va a segno \tabularnewline \hline
  \raggedright Ragni\linebreak grandi & \raggedright - & \raggedright 1 & \raggedright 1 & \raggedright 1d10 & \raggedright 1 & \raggedright 1d6 & \raggedright - & \raggedright 15-30 & \raggedright 1(Morso) & \raggedright Migali & \raggedright Iniettano veleno se il TPC va a segno \tabularnewline \hline
  \raggedright Ragni\linebreak giganti & \raggedright 1 & \raggedright 1d6 & \raggedright 8+ 2d6 & \raggedright 2d10 & \raggedright 8+ 2d6 & \raggedright 4+ 1d6 & \raggedright 40 & \raggedright 20-35 & \raggedright 1d8+ 2 (Morso) & \raggedright V. Descrizione & \raggedright Iniettano veleno se il TPC va a segno; protezione 3-10 PP \tabularnewline \hline
  \raggedright Scimmie\linebreak piccole & \raggedright 1d6 & \raggedright 4+ 2d8 & \raggedright 1d6 & \raggedright 17+ 1d10 & \raggedright 1d6 & \raggedright 10+ 1d10 & \raggedright 10 & \raggedright 15-25 & \raggedright 1d4 (Morso) & \raggedright Bertucce, lemuri & \tabularnewline \hline
  \raggedright Scimmie\linebreak medie & \raggedright 1d10 & \raggedright 10+ 1d10 & \raggedright 2+ 3d6 & \raggedright 15+ 1d10 & \raggedright 2+ 3d6 & \raggedright 10+ 1d10 & \raggedright 10 & \raggedright 15-25 & \raggedright 1d4 (Pugno), 1d6 (Morso) & \raggedright Scimpanz\`e, Macachi, Babbuini & \tabularnewline \hline
  \raggedright Scimmie\linebreak grandi & \raggedright 1d10 & \raggedright 10+ 1d10 & \raggedright 15+ 1d10 & \raggedright 13+ 1d10 & \raggedright 15+ 1d10 & \raggedright 10+ 1d10 & \raggedright 25 & \raggedright 15-30 & \raggedright 1d4 (Pugno o Presa) 1d8 (Morso) & \raggedright Oranghi, mandrilli, gorilla & \tabularnewline \hline
  \raggedright Rettili\linebreak piccoli & \raggedright 1 & \raggedright 1d6 & \raggedright 1 & \raggedright 2d10 & \raggedright 1 & \raggedright 5+ 1d10 & \raggedright - & \raggedright 10-30 & \raggedright 1 (Morso) & \raggedright Lucertole, Bisce, Gechi ecc. & \raggedright Iniettano veleno se il TPC va a segno \tabularnewline \hline
  \raggedright Rettili\linebreak medi & \raggedright 1 & \raggedright 1d6 & \raggedright 1d6 & \raggedright 2d10 & \raggedright 1d6 & \raggedright 5+ 1d10 & \raggedright - & \raggedright 10-30 & \raggedright 1d4 (Morso) & \raggedright Vipere, Cobra, Iguane & \raggedright Iniettano veleno se il TPC va a segno \tabularnewline \hline
  \raggedright Rettili\linebreak grandi & \raggedright 1 & \raggedright 2d8 & \raggedright 4+ 3d6 & \raggedright 2d10 & \raggedright 4+ 3d6 & \raggedright 5+ 1d10 & \raggedright - & \raggedright 10-30 & \raggedright 1d8 (Presa), 2d8 (Morso) & \raggedright Varani, Coccodrilli, Anaconda, Boa, Tartarughe marine & \tabularnewline \hline
  \raggedright Uccelli\linebreak piccoli & \raggedright 1d4 & \raggedright 4+ 2d8 & \raggedright 1 & \raggedright 20+ 1d6 & \raggedright 1 & \raggedright 15+ 1d10 & \raggedright 20 & \raggedright 5-40 & \raggedright - & \raggedright Colibr\`\i\, passeri, canarini ecc. & \tabularnewline \hline
  \raggedright Uccelli\linebreak medi & \raggedright 1d4 & \raggedright 4+ 2d8 & \raggedright 1d8 & \raggedright 20+ 1d6 & \raggedright 1d8 & \raggedright 15+ 1d10 & \raggedright 15 & \raggedright 5-40 & \raggedright 1d4 (Becco), 1d6 (Artigli) & \raggedright Corvo, Falco, Civetta & \tabularnewline \hline
  \raggedright Uccelli\linebreak grandi & \raggedright 1d4 & \raggedright 4+ 2d8 & \raggedright 3+ 2d6 & \raggedright 14+ 2d6 & \raggedright 3+ 2d6 & \raggedright 15+ 1d10 & \raggedright 20 & \raggedright 5-40 & \raggedright 1d8 (Becco), 1d8+ 1 (Artigli) & \raggedright Aquila, Condor, Gufo, Avvoltoio & \tabularnewline \hline
  \raggedright Pesci\linebreak piccoli & \raggedright 1 & \raggedright 1d6 & \raggedright - & \raggedright 20+ 1d6 & \raggedright - & \raggedright 1d20 & \raggedright - & \raggedright 5-40 & \raggedright - & \raggedright Ghiozzi, Triglie, Orate & \tabularnewline \hline
  \raggedright Pesci\linebreak medi & \raggedright 1 & \raggedright 1d6 & \raggedright 1d4 & \raggedright 20+ 1d6 & \raggedright 1d4 & \raggedright 1d20 & \raggedright - & \raggedright 5-40 & \raggedright - & \raggedright Dentici, Cernie & \tabularnewline \hline
  \raggedright Pesci\linebreak grandi & \raggedright 1 & \raggedright 1d6 & \raggedright 12+ 2d10 & \raggedright 14+ 2d6 & \raggedright 6+ 3d10 & \raggedright 1d10 & \raggedright - & \raggedright 5-40 & \raggedright 2d8+ d (Morso) & \raggedright Grandi squali, Mante & \tabularnewline \hline
  \raggedright Balene & \raggedright 2+ 1d6 & \raggedright 4+ 2d8 & \raggedright 60+ 4d10 & \raggedright 2+ 1d8 & \raggedright 40+ 4d10 & \raggedright 10+ 1d10 & \raggedright - & \raggedright 20 & \raggedright 1d10 (Testata, in Carica) & \raggedright Balenottere, Capodogli & \raggedright Attacco in carica: somma la FOR al Danno\tabularnewline \hline
  \raggedright Orca & \raggedright 4+ 1d10 & \raggedright 4+ 2d8 & \raggedright 40+ 3d10 & \raggedright 10+ 1d8 & \raggedright 30+ 3d10 & \raggedright 10+ 1d10 & \raggedright 30 & \raggedright 25 & \raggedright 3d10 (Morso) & \raggedright Orcinus Orca & \raggedright Attacco in carica: somma la FOR al Danno\tabularnewline \hline
  \raggedright Delfino & \raggedright 10+ 1d10 & \raggedright 4+ 2d8 & \raggedright 20+ 1d10 & \raggedright 20+ 1d6 & \raggedright 15+ 1d10 & \raggedright 15+ 1d10 & \raggedright 25 & \raggedright 20 & \raggedright 1d10 (Testata, in Carica) & \raggedright Delfini, Narvali,  & \raggedright Attacco in carica: somma la FOR al Danno\tabularnewline \hline
  \raggedright Insetti\linebreak piccoli & \raggedright - & \raggedright - & \raggedright - & \raggedright 2d20 & \raggedright - & \raggedright 10+ 1d10 & \raggedright - & \raggedright 5-40 & \raggedright - & \raggedright Mosche, zanzare, formiche e molte altre & \raggedright Iniettano veleno se il TPC va a segno\tabularnewline \hline
  \raggedright Insetti\linebreak giganti & \raggedright 1 & \raggedright 1 & \raggedright 8+ 2d6 & \raggedright 1d20 & \raggedright 8+ 2d6 & \raggedright 10+ 1d10 & \raggedright - & \raggedright 25-40 & \raggedright 1d10 (Attacco generico) & \raggedright V. Descrizione & \tabularnewline \hline
  \raggedright Troll & \raggedright 1d4 & \raggedright 4+ 2d8 & \raggedright 9+ 2d8 & \raggedright 6+ 1d10 & \raggedright 9+ 2d8 & \raggedright 10+ 1d10 & \raggedright 30 & \raggedright 15 & \raggedright 2d6 (Morso), possono avere CAC o ABL & \raggedright V. Descrizione & \tabularnewline \hline
  \raggedright Roditori\linebreak piccoli & \raggedright 1d4 & \raggedright 1d6 & \raggedright 1 & \raggedright 10+ 1d10 & \raggedright 1 & \raggedright 10+ 1d10 & \raggedright 15 & \raggedright 10-30 & \raggedright 1d6 (Morso) & \raggedright Topi, criceti, lemmings & \tabularnewline \hline
  \raggedright Roditori\linebreak grandi & \raggedright 1d4 & \raggedright 1d6 & \raggedright 1d6 & \raggedright 6+ 1d10 & \raggedright 1d6 & \raggedright 10+ 1d10 & \raggedright 25 & \raggedright 20-30 & \raggedright 1d6 (Morso) & \raggedright Conigli, lepri, ratti & \tabularnewline \hline
  \raggedright Ovini\linebreak piccoli & \raggedright 1d4 & \raggedright 1d6 & \raggedright 2+ 1d6 & \raggedright 10+ 1d10 & \raggedright 2+ 1d6 & \raggedright 10+ 1d10 & \raggedright 20 & \raggedright 5-20 & \raggedright 1d4 (Testata, in carica) & \raggedright Pecore, Capre & \raggedright Attacco in carica: somma la FOR al Danno\tabularnewline \hline
  \raggedright Ovini\linebreak grandi & \raggedright 1d4 & \raggedright 1d8 & \raggedright 8+ 2d6 & \raggedright 10+ 1d10 & \raggedright 8+ 2d6 & \raggedright 10+ 1d10 & \raggedright 20 & \raggedright 5-20 & \raggedright 1d6 (Testata o cornata, in carica) & \raggedright M\`uvara, Stambecchi, Lama & \raggedright Attacco in carica: somma la FOR al Danno\tabularnewline \hline
  \caption[Tabella Animali]{Tabella Animali}\\
\end{longtable}

\twocolumn


%%% Local Variables: 
%%% mode: plain-tex
%%% TeX-master: "manual"
%%% TeX-master: "manual"
%%% End: 

}
\fi
%%% Local Variables: 
%%% mode: latex
%%% TeX-master: "manual"
%%% End: 
