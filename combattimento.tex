\iffullversion \onecolumn\fi
\chapter{Il Combattimento}
\label{combattimento}
\newcommand{\persA} {\textbf{A}rt}
\newcommand{\persB} {\textbf{B}art}
\newcommand{\persC} {\textbf{C}art}
\newcommand{\persD} {\textbf{D}art}

\iffullversion
\vfill
\begin{racconto}
  ``--Per tutti i miei antenati...--
  
  THUMP!!
  
  Takashi-san non se l'aspettava proprio di venire sbattuto a terra da
  un cavallo... 
  
  --Se trovo quel bastardo che lascia in giro delle
  bestie del genere dovranno raccogliere i suoi pezzi sparsi in giro
  per due mesi!-- disse l'uomo rialzandosi, mentre una voce ruvida
  raggiungeva le sue orecchie:
  
  --...credo che tu stia cercando me, amico!-- Un uomo a cavallo
  proiett\`o la sua ombra sull'orientale, che non sembr\`o
  soddisfatto della risposta del suo involontario interlocutore: --Io
  non sono l'amico di nessuno, e bada a come parli o ti strappo la
  lingua.-- disse, quasi senza accorgersene...
  
  --PRRRRRRR!-- fu la risposta dell'uomo a cavallo.
  
  Nessun'altra parola sarebbe potuta essere pi\`u insolente e
  presuntuosa, si disse l'orientale guardando l'uomo a cavallo con
  aria di sfida.
  
  --Prima di ucciderti voglio sapere almeno come ti chiami, uomo-- fu la
  replica di Takashi.
  
  --Sono William Wicker...-- Disse l'uomo scendendo a terra --Ma non
  ho sentito bene il tuo...--
  
  --Bene io sono Takashi Horuru, discendo da una famiglia di signori
  delle terre orientali, e questo \`e il momento di divertirmi...--
  
  L'orientale sfoder\`o il katana in un istante, lasciando il tempo
  al suo avversario di estrarre con pi\`u lentezza la sua spada
  bastarda. Wicker cal\`o con potenza la lama della sua spada in
  direzione del Signore Orientale che non ebbe alcuna difficolt\`a
  nello schivare il colpo e proiettare a terra il malcapitato
  avversario.
  
  Gli ci volle qualche istante per capire ci\`o che gli era
  effettivamente accaduto, ma durante quel lasso di tempo la gelida
  lama del katana del suo avversario gli aveva gi\`a provocato un
  brivido alla base del collo. --Sei un avversario troppo scadente,
  uccidendoti ti farei solo un favore...-- Disse con aria cupa
  l'orientale.
  
  --Sei una delle poche persone che potr\`a dire di aver osato
  sfidarmi, e questo non potr\`a che servire da esempio ad altri che
  vorranno imitarti... Sto aspettando le tue scuse, e fino a quando
  non arriveranno tu non andrai via di qui... vivo--
  
  --OK, amico ho capito, ti chiedo scusa... va bene cos\`{\i}?--
  
  --Certo, ora puoi andare... Wicker--''
\end{racconto}

\pinupbig{orco_nani.eps}{}{p}

\twocolumn 

Questo pu\`o essere un tipico esempio di combattimento
tra due personaggi durante una sessione. Vi capiter\`a sicuramente
di scontrarvi con esseri pi\`u o meno umani e pi\`u o meno
antipatici che tenteranno in tutti i modi di causarvi come minimo un
po' di dolore. Oppure potreste essere voi a voler causare danni
pi\`u o meno mortali ad altri poveri malcapitati... In tali
occasioni sarebbe opportuno sapere come fare per combattere. Potreste
avere bisogno di armi, armature e altri simpatici artefatti, ma,
soprattutto, dovete saperli usare.

Il capitolo che segue vi spiega nei particolari il funzionamento del
sistema di combattimento in Radix Malorum.
\fi

\section{Abilit\`a di combattimento (AdC)}

 Sulla scheda \`e riportato un elenco di
abilit\`a specifiche per il combattimento

\abi{Armi da taglio lunghe (ATL)}{1}{AGI} Specifica il grado di
abilit\`a di un personaggio nel maneggiare spade, spade bastarde,
spadoni a due mani, asce e, in genere, armi lunghe dotate di lama.

\abi{Armi da taglio corte (ATC)}{1}{AGI} Indica quanto un personaggio
sia bravo nell'utilizzare pugnali, spadini, daghe, stiletti, falcetti,
o qualsiasi arma corta dotata di lama.

\iffullversion

\abi{Armi da botta lunghe (ABL)}{1}{AGI} 
Esprime la capacit\`a nell'usare armi come bastoni,
pertiche, martelli da guerra, e qualsivoglia arma lunga che infligga
danni da botta.

\abi{Armi da botta corte (ABC)}{1}{AGI} Specifica il
grado di abilit\`a nell'utilizzare mazze, manganelli, mazzafrusti,
ton-fa, nunchaku, e simili.
\fi

\abi{Armi da lancio (ADL)}{1}{AGI}
Indica la capacit\`a nel lanciare armi come shuriken, pugnali da
lancio, giavellotti, martelli da lancio, bolas, reti e armi abilitate
al lancio.

\abi{Armi da tiro (ADT)}{1}{OSS} Esprime la capacit\`a
di usare archi, balestre, fionde, fromboli, cerbottane e armi che
scagliano proiettili.

\iffullversion
\abi{Armi in asta (AIA)}{1}{AGI} Specifica la
bravura nell'utilizzare armi come lance, alabarde, picche, naginata e
in genere tutte le armi con manico lungo e con una o pi\`u
punte/lame alle estremit\`a.

\abi{Armi da fuoco (ADF)}{1}{OSS}
Indica quanto un personaggio sia bravo nell'utilizzare armi come
archibugi e pistole ad avancarica.

\abi{Artiglieria (ART)}{1}{OSS}
Esprime la capacit\`a di usare armi di grosso calibro come catapulte
e cannoni.

\abi{Armi occasionali (AO)}{1}{AGI} 

Specifica quanto un personaggio sia bravo nell'utilizzare, come armi,
oggetti che non rientrano nelle precedenti categorie come comodini,
tavoli, armadi, animali domestici e non, ecc.
\fi

\abi{Scudo (SCD)}{1}{AGI} Esprime la
capacit\`a di usare scudi piccoli, medi o grandi. Lo scudo pu\`o
essere usato dai destri con la mano sinistra e dai mancini con la
destra senza subire malus.

\abi{Corpo a Corpo (CAC)}{1}{AGI} 
Indica l'abilit\`a nel combattimento a mani nude 
\iffullversion
(i colpi
permessi sono descritti nel capitolo seguente).

\section{Le Armi}
Nella tabella Armi che troverete a pagina \pageref{tabarmi} sono
indicate 13 caselle per ciascun'arma, ognuna delle quali mostra un
particolare dell'arma.

% \pinup{picca.eps}{Picca}

\pinup{picca1.eps}{Picca}

\begin{description}

\item{\bf Arma} Indica di che arma si tratta.    
  
\item{\bf TIP} (Tipologia): Indica a che Abilit\`a di Combattimento si
  deve fare riferimento per il suo utilizzo.

\item{\bf TCI} (Tipo di Colpo Inferto): Tipo di Colpo che
  pu\`o infliggere un arma: Punta, Taglio, Botta. Ad ogni TCI \`e
  riferita una Tipologia di Danno (TDA) indicata nel Paragrafo
  ``Tipologia del Danno'' a pagina \pageref{tipodanno}
  
\item{\bf Danno}: Indica il numero ed il tipo di dadi da lanciare per
  stabilire il danno inferto.  

\item{\bf PSA} Punti Struttura Arma: indica la
  resistenza dell'arma agli urti durante il combattimento.
  
\item{\bf BP} (Bonus Parata): Indica il Bonus Parata conferito all'azione
  Parare con l'arma.
  
\item{\bf PA} (Penalit\`a Arma): Indica la penalit\`a data dall'arma
  all'iniziativa.

\item{\bf REQ} (Requisiti): Indica i punteggi minimi che \`e necessario
  avere in talune caratteristiche, per maneggiare l'arma senza
  ulteriori penalit\`a.  Per ogni punto in meno del PG in una delle
  caratteristiche rispetto a quelli minimi si avr\`a un malus di -1
  a tutti i tiri nelle azioni in combattimento con quell'arma.
  
\item{\bf PES} (Peso): Peso in Kg dell'arma; fa parte del Peso
  Trasportabile.

\item{\bf LG} (Lunghezza): Lunghezza totale dell'arma.  
  
\item{\bf REP\%} (Reperibilit\`a): Probabilit\`a di trovare
  l'arma in vendita in un posto qualsiasi. Per sapere se l'arma \`e
  disponibile il Master deve tirare 1d100: un risultato minore o
  uguale alla REP indica che l'arma \`e in commercio.
  
\item{\bf Costo}: Prezzo di mercato dell'arma di media fattura.  I
  costi potranno essere maggiori se l'arma \`e di fattura migliore,
  ha un certo valore artistico o \`e costituita da materiali
  preziosi, oppure inferiori se l'arma \`e in cattive condizioni,
  difettosa o scadente. 
  
\item{\bf GI} (Gittata): Distanza a cui pu\`o essere lanciata
  un'arma o scagliato un proiettile, \textbf{senza malus al TPC}.
  Ogni 3 metri oltre la GI il TPC \`e penalizzato di -1. La gittata
  massima a cui un proiettile pu\`o arrivare \`e il doppio della
  GI.  L'arma non pu\`o essere utilizzata se il bersaglio si trova
  ad una distanza inferiore ad 1/10 (un decimo) della GI.

\item{\bf Note}:
  In questa casella sono indicate le descrizioni delle armi pi\`u
  particolari ed eventuali variazioni alle regole generali.
\end{description}



\subsection{Armi per i giganti} 

Normalmente i giganti sono soliti utilizzare armi artigianali ricavate
da tronchi o pietre, come spiegato a pagina \pageref{armigiganti}.
Tuttavia \`e possibile che nel corso della sua vita un gigante
riesca a farsi costruire da un vero armaiolo un'arma adatta alle sue
dimensioni. 

Un'arma di questo tipo infligge, \textbf{oltre al danno indicato nella
  tabella delle armi, un altro dado dello stesso tipo}. 

\goodbreak
\es{Una comune alabarda infligge un danno pari a 3d6+2, mentre una
alabarda costruita per un gigante infligge 4d6+2. Analogamente una
spada infligge 2d6+1 se di dimensioni normali, 3d6+1 se costruita per
un gigante}

\clearpage\onecolumn{\setlength{\tabcolsep}{0.2em}
\centering

{\Large\sc Armi}
\label{tabarmi}

\footnotesize
\begin{longtable}{|p{1.5cm}|p{0.7cm}|p{0.9cm}|p{0.9cm}|l|l|l|p{0.9cm}|p{0.8cm}|p{1.0cm}|l|l|p{0.75cm}|p{3.2cm}|}
  \par
  \hline
  Arma& \raggedright TIP& \raggedright TCI& \raggedright Danno& \raggedright PSA& \raggedright BP& \raggedright PA& \raggedright REQ& \raggedright PES& \raggedright LG& \raggedright Rep\%& \raggedright Costo& \raggedright GI& \raggedright Note\tabularnewline \hline\hline
  \endfirsthead
  \hline
  Arma& \raggedright TIP& \raggedright TCI& \raggedright Danno& \raggedright PSA& \raggedright BP& \raggedright PA& \raggedright REQ& \raggedright PES& \raggedright LG& \raggedright Rep\%& \raggedright Costo& \raggedright GI& \raggedright Note\tabularnewline \hline\hline
  \endhead
  
  \hline\multicolumn{14}{|c|}{\normalsize\sc Armi da Taglio Corte (ATC)}\tabularnewline \hline\hline
  \raggedright Stiletto& \raggedright ATC& \raggedright Punta& \raggedright 1d4& \raggedright 9& \raggedright -& \raggedright -& \raggedright -& \raggedright 100 g& \raggedright 15-25 cm& \raggedright 90& \raggedright 3S& \raggedright -& \raggedright \tabularnewline \hline
  \raggedright Pugnale& \raggedright ATC ADL& \raggedright Punta, Taglio& \raggedright 1d6& \raggedright 11& \raggedright 1& \raggedright -& \raggedright -& \raggedright 200 g& \raggedright 15-40 cm& \raggedright 100& \raggedright 5S& \raggedright -/ 15m& \raggedright Pu\`o essere lanciata come una normale ADL\tabularnewline \hline
  \raggedright Daga& \raggedright ATC& \raggedright Punta, Taglio& \raggedright 1d6+1& \raggedright 12& \raggedright 2& \raggedright -& \raggedright FOR 5& \raggedright 400 g& \raggedright 40-50 cm& \raggedright 80& \raggedright 8S& \raggedright -& \raggedright \tabularnewline \hline
  \raggedright Spada Corta& \raggedright ATC& \raggedright Punta, Taglio& \raggedright 1d8+1& \raggedright 14& \raggedright 3& \raggedright -1& \raggedright FOR 6& \raggedright 700 g& \raggedright 40-50 cm& \raggedright 100& \raggedright 8S& \raggedright -& \raggedright \tabularnewline \hline
  \raggedright Scure& \raggedright ATC& \raggedright Taglio& \raggedright 1d6+1& \raggedright 12& \raggedright 2& \raggedright -& \raggedright -& \raggedright 400 g& \raggedright 40-50 cm& \raggedright 100& \raggedright 2S& \raggedright -& \raggedright \tabularnewline \hline
  \raggedright Ascia& \raggedright ATC ADL& \raggedright Taglio& \raggedright 1d8+1& \raggedright 14& \raggedright 3& \raggedright -1& \raggedright FOR 5& \raggedright 600 g& \raggedright 40-50 cm& \raggedright 70& \raggedright 8S& \raggedright -/ 10m& \raggedright Ha due lame; pu\`o essere lanciata come una normale ADL\tabularnewline \hline
  \raggedright Artiglio& \raggedright ATC& \raggedright Punta, Taglio& \raggedright 1d6+1& \raggedright 8& \raggedright 1& \raggedright -& \raggedright -& \raggedright 200 g& \raggedright 20-30 cm& \raggedright 20& \raggedright 10S& \raggedright -& \raggedright Si indossa sul dorso della mano. Non impedisce di impugnare altre armi\tabularnewline \hline
  \raggedright Falcetto& \raggedright ATC& \raggedright Taglio& \raggedright 1d6+1& \raggedright 12& \raggedright 1& \raggedright -& \raggedright -& \raggedright 300 g& \raggedright 40-50 cm& \raggedright 100& \raggedright 3S& \raggedright -& \raggedright \tabularnewline \hline
  \raggedright Sai& \raggedright ATC& \raggedright Punta& \raggedright 1d6& \raggedright 14& \raggedright 3& \raggedright -& \raggedright -& \raggedright 300 g& \raggedright 30-50 cm& \raggedright 40& \raggedright 7S& \raggedright -& \raggedright Piccolo tridente con il dente centrale allungato e acuminato. La reperibilit\`a \`e 70\% nelle terre orientali.\tabularnewline \hline
  \raggedright Shi Tai& \raggedright ATC& \raggedright Punta& \raggedright 1d6& \raggedright 11& \raggedright 1& \raggedright -& \raggedright -& \raggedright 100 g& \raggedright 15-40 cm& \raggedright 10& \raggedright 6S& \raggedright -& \raggedright Pugnale che si indossa all'estremit\`a della coda. Ha reperibilit\`a del 100\% nel territorio Reuben\tabularnewline \hline
  \hline \multicolumn{14}{|c|}{\normalsize\sc Armi da Taglio Lunghe (ATL)}\tabularnewline \hline\hline
  \raggedright Spada/ Sciabola& \raggedright ATL& \raggedright Punta, Taglio& \raggedright 2d6+1& \raggedright 18& \raggedright 4& \raggedright -2& \raggedright FOR 8& \raggedright 800 g& \raggedright 50-100 cm& \raggedright 90& \raggedright 12S& \raggedright -& \raggedright \tabularnewline \hline
  \raggedright Spada bastarda& \raggedright ATL& \raggedright Punta, Taglio, Botta& \raggedright 2d8/ 2d8+2& \raggedright 23& \raggedright 6& \raggedright -5& \raggedright FOR 13& \raggedright 1.5 kg& \raggedright 70-120 cm& \raggedright 60& \raggedright 15S& \raggedright -& \raggedright Se usata a 1 mano infligge 2d8, se usata a 2 mani infligge 2d8+2\tabularnewline \hline
  \raggedright Katana& \raggedright ATL& \raggedright Punta, Taglio& \raggedright 2d8/ 2d8+2& \raggedright 18& \raggedright 6& \raggedright -4& \raggedright FOR 10& \raggedright 1.1 kg& \raggedright 80-120 cm& \raggedright 20& \raggedright 20S& \raggedright -& \raggedright Se usata a 1 mano infligge 2d8, se usata a 2 mani infligge 2d8+2. La reperibilit\`a \`e 60\% nelle terre orientali.\tabularnewline \hline
  \raggedright Ascia bastarda& \raggedright ATL& \raggedright Taglio& \raggedright 2d8+1/ 2d8+3& \raggedright 23& \raggedright 5& \raggedright -6& \raggedright FOR 13& \raggedright 1.7 kg& \raggedright 60-100 cm& \raggedright 20& \raggedright 20S& \raggedright -& \raggedright Se usata a 1 mano infligge 2d8+1, se usata a 2 mani infligge 2d8+3\tabularnewline \hline
  \raggedright Ascia bipenne& \raggedright ATL& \raggedright Taglio& \raggedright 3d6+4& \raggedright 27& \raggedright 6& \raggedright -6& \raggedright FOR 18& \raggedright 4 kg& \raggedright 60-100 cm& \raggedright 10& \raggedright 30S& \raggedright -& \raggedright Deve essere usata a 2 mani\tabularnewline \hline
  \raggedright Spadone& \raggedright ATL& \raggedright Punta, Taglio, Botta& \raggedright 3d6+1& \raggedright 24& \raggedright 7& \raggedright -6& \raggedright FOR 16& \raggedright 4 kg& \raggedright 90-150 cm& \raggedright 30& \raggedright 30S& \raggedright -& \raggedright Deve essere usata a 2 mani\tabularnewline \hline
  \raggedright Falcione& \raggedright ATL& \raggedright Taglio& \raggedright 2d6+2& \raggedright 19& \raggedright 4& \raggedright -4& \raggedright FOR 13& \raggedright 3 kg& \raggedright 70-80 cm& \raggedright 100& \raggedright 5S& \raggedright -& \raggedright Deve essere usata a 2 mani\tabularnewline \hline
  \raggedright Shi Rik& \raggedright ATL& \raggedright Taglio, Punta& \raggedright 2d8& \raggedright 21& \raggedright 6& \raggedright -5& \raggedright Tiro in Con\-tor\-sio\-nismo a diff. 15; FOR 10& \raggedright 1.2 kg& \raggedright 60-70 cm& \raggedright 10& \raggedright 30S& \raggedright -& \raggedright \`E una specie di forbice che si indossa sull'avambraccio. Il suo meccanismo richiede una innaturale flessione del polso, quindi per usarla \`e richiesto un tiro a difficolt\`a pari a 15 in contorsionismo. La reperibilit\`a \`e 90\% nel territorio Reuben.\tabularnewline \hline
  \pagebreak\hline\multicolumn{14}{|c|}{\normalsize\sc Armi da Botta Corte (ABC)}\tabularnewline \hline\hline
  \raggedright Randello& \raggedright ABC& \raggedright Botta& \raggedright 1d6& \raggedright 11& \raggedright 1& \raggedright -& \raggedright -& \raggedright 200 g& \raggedright 30-40 cm& \raggedright 100& \raggedright 1S& \raggedright -& \raggedright \tabularnewline \hline
  \raggedright Mazzetta& \raggedright ABC& \raggedright Botta& \raggedright 1d8& \raggedright 13& \raggedright 1& \raggedright -& \raggedright -& \raggedright 400 g& \raggedright 30-40 cm& \raggedright 60& \raggedright 2S& \raggedright -& \raggedright \tabularnewline \hline
  \raggedright Maglio& \raggedright ABC& \raggedright Botta& \raggedright 1d6+2& \raggedright 13& \raggedright 2& \raggedright -1& \raggedright FOR 6& \raggedright 500 g& \raggedright 30-40 cm& \raggedright 100& \raggedright 1S& \raggedright -& \raggedright \tabularnewline \hline
  \raggedright Martello& \raggedright ABC& \raggedright Botta& \raggedright 1d6+1& \raggedright 12& \raggedright 1& \raggedright -& \raggedright -& \raggedright 400 g& \raggedright 30-50 cm& \raggedright 100& \raggedright 1S& \raggedright -& \raggedright \tabularnewline \hline
  \raggedright Mazza\-frusto& \raggedright ABC& \raggedright Botta& \raggedright 1d8+1& \raggedright 14& \raggedright -& \raggedright -1& \raggedright FOR 6& \raggedright 500 g& \raggedright 30-50 cm& \raggedright 70& \raggedright 3S& \raggedright -& \raggedright \tabularnewline \hline
  \raggedright Ton-Fa& \raggedright ABC& \raggedright Botta& \raggedright 1d6& \raggedright 14& \raggedright 5& \raggedright -& \raggedright -& \raggedright 200 g& \raggedright 30-40 cm& \raggedright 30& \raggedright 3S& \raggedright -& \raggedright Randello formato da due bastoni incrociati a T. La reperibilit\`a \`e 80\% nelle terre orientali\tabularnewline \hline
  \raggedright Nunchaku& \raggedright ABC& \raggedright Botta& \raggedright 1d8+1& \raggedright 14& \raggedright 1& \raggedright -1& \raggedright -& \raggedright 400 g& \raggedright 50-70 cm& \raggedright 50& \raggedright 6S& \raggedright -& \raggedright Due randelli uniti da una catena. La reperibilit\`a \`e 80\% nelle terre orientali\tabularnewline \hline
  \raggedright Tirapugni& \raggedright ABC CAC& \raggedright Botta, Punta& \raggedright +2& \raggedright 5& \raggedright -& \raggedright -& \raggedright -& \raggedright 50 g& \raggedright -& \raggedright 90& \raggedright 1S& \raggedright -& \raggedright Conferisce un bonus di +2 al danno di PUGno\tabularnewline \hline
  \raggedright Clavetta& \raggedright ABC ADL& \raggedright Botta& \raggedright 1d4+1& \raggedright 10& \raggedright 1& \raggedright -& \raggedright -& \raggedright 100 g& \raggedright 20-30 cm& \raggedright 60& \raggedright 1S& \raggedright -/ 15m& \raggedright Pu\`o essere lanciata come una normale ADL\tabularnewline \hline
  \hline\multicolumn{14}{|c|}{\normalsize\sc Armi da Botta Lunghe (ABL)}\tabularnewline \hline\hline
  \raggedright Bastone& \raggedright ABL& \raggedright Botta& \raggedright 1d8+1& \raggedright 14& \raggedright 6& \raggedright -2& \raggedright -& \raggedright 300 g& \raggedright 150-300 cm& \raggedright 100& \raggedright 1S& \raggedright -& \raggedright \tabularnewline \hline
  \raggedright Mazza& \raggedright ABL& \raggedright Botta& \raggedright 2d8+1& \raggedright 22& \raggedright 5& \raggedright -4& \raggedright FOR 13& \raggedright 2 kg& \raggedright 100-150 cm& \raggedright 90& \raggedright 2S& \raggedright -& \raggedright \tabularnewline \hline
  \raggedright Stella del mattino& \raggedright ABL& \raggedright Botta& \raggedright 3d6+3& \raggedright 26& \raggedright 3& \raggedright -6& \raggedright FOR 18& \raggedright 5 kg& \raggedright 80-120 cm& \raggedright 30& \raggedright 20S& \raggedright -& \raggedright Deve essere usata a 2 mani\tabularnewline \hline
  \raggedright Martello da guerra& \raggedright ABL& \raggedright Botta& \raggedright 3d6+1& \raggedright 24& \raggedright 5& \raggedright -5& \raggedright FOR 16& \raggedright 4 kg& \raggedright 80-120 cm& \raggedright 50& \raggedright 20S& \raggedright -& \raggedright Deve essere usata a 2 mani\tabularnewline \hline
  \raggedright Maglio da guerra& \raggedright ABL& \raggedright Botta& \raggedright 2d8& \raggedright 21& \raggedright 5& \raggedright -4& \raggedright FOR 13& \raggedright 2.5 kg& \raggedright 80-130 cm& \raggedright 50& \raggedright 20S& \raggedright -& \raggedright \tabularnewline \hline
  \raggedright Clava& \raggedright ABL& \raggedright Botta& \raggedright 2d6+1& \raggedright 18& \raggedright 2& \raggedright -6& \raggedright FOR 10& \raggedright 3 kg& \raggedright 70-200 cm& \raggedright 100& \raggedright -& \raggedright -& \raggedright Si pu\`o ricavare da un pezzo di legno o simili\tabularnewline \hline
  \hline\multicolumn{14}{|c|}{\normalsize\sc Armi da Lancio (ADL)}\tabularnewline \hline\hline
  \raggedright Trottola& \raggedright ADL& \raggedright Botta, Punta& \raggedright 1d4& \raggedright 18& \raggedright -& \raggedright -& \raggedright -& \raggedright 50 g& \raggedright 10 cm& \raggedright 20& \raggedright 1S& \raggedright 5 m& \raggedright Richiede 2 round per essere "caricata", o un tiro Incoccare Veloce a diff. 25.\tabularnewline \hline
  \raggedright Giavellotto& \raggedright ADL& \raggedright Punta& \raggedright 1d8& \raggedright 10& \raggedright 3& \raggedright -1& \raggedright FOR 6& \raggedright 500 g& \raggedright 180-300 cm& \raggedright 40& \raggedright 5S& \raggedright 30 m& \raggedright \tabularnewline \hline
  \raggedright Bolas& \raggedright ADL& \raggedright Botta& \raggedright 1d4& \raggedright 4& \raggedright -& \raggedright -1& \raggedright FOR 8& \raggedright 600 g& \raggedright 60-120 cm& \raggedright 70& \raggedright 3S& \raggedright 20 m& \raggedright Se usate contro parti del corpo scoperte infliggono 1d4 di danno da presa a round, fino alla riuscita del tiro Divincolarsi a difficolt\`a pari al TPC.\tabularnewline \hline
  \raggedright Shuriken& \raggedright ADL& \raggedright Punta& \raggedright 1d6& \raggedright 24& \raggedright -& \raggedright -& \raggedright -& \raggedright 60 g& \raggedright 10 cm& \raggedright 20& \raggedright 2S& \raggedright 20 m& \raggedright Stelle da lancio\tabularnewline \hline
  \raggedright Yo Yo& \raggedright ADL& \raggedright Botta, Taglio& \raggedright 1d4& \raggedright 4& \raggedright -& \raggedright -& \raggedright -& \raggedright 60 g& \raggedright 10 cm& \raggedright 5& \raggedright 2S& \raggedright 3 m& \raggedright Torna indietro dopo ogni colpo, pu\`o essere dotato di bordi taglienti\tabularnewline \hline
  \raggedright Arpione& \raggedright ADL AIA& \raggedright Punta& \raggedright 1d8+1& \raggedright 14& \raggedright -/ 6& \raggedright -1/ -3& \raggedright FOR '9/ 7& \raggedright 400 g& \raggedright 120-200 cm& \raggedright 80& \raggedright 2S& \raggedright 15 m& \raggedright Pu\`o essere usato sia come ADL che come AIA\tabularnewline \hline
  \pagebreak\hline\multicolumn{14}{|c|}{\normalsize\sc Armi in Asta (AIA)}\tabularnewline \hline\hline
  \raggedright Falce& \raggedright AIA& \raggedright Taglio, Punta& \raggedright 3d6+1& \raggedright 24& \raggedright 5& \raggedright -6& \raggedright FOR 7& \raggedright 2 kg& \raggedright 120-180 cm& \raggedright 100& \raggedright 3S& \raggedright -& \raggedright \tabularnewline \hline
  \raggedright Picca& \raggedright AIA& \raggedright Punta, Botta& \raggedright 2d8+2& \raggedright 23& \raggedright 3& \raggedright -5& \raggedright FOR 10& \raggedright 2 kg& \raggedright 150-250 cm& \raggedright 60& \raggedright 15S& \raggedright -& \raggedright \tabularnewline \hline
  \raggedright Alabarda (Naginata)& \raggedright AIA& \raggedright Taglio, Punta& \raggedright 3d6+2& \raggedright 25& \raggedright 5& \raggedright -6& \raggedright FOR 13& \raggedright 3 kg& \raggedright 150-250 cm& \raggedright 60& \raggedright 30S& \raggedright -& \raggedright \tabularnewline \hline
  \raggedright Lancia& \raggedright AIA& \raggedright Punta& \raggedright 2d8+1/ 2d8+2& \raggedright 22& \raggedright 3& \raggedright -4& \raggedright FOR 8& \raggedright 1.8 kg& \raggedright 150-300 cm& \raggedright 70& \raggedright 10S& \raggedright -& \raggedright \tabularnewline \hline
  \raggedright Forcone& \raggedright AIA& \raggedright Punta& \raggedright 2d8+1& \raggedright 22& \raggedright 4& \raggedright -5& \raggedright FOR 8& \raggedright 1.8 kg& \raggedright 120-200 cm& \raggedright 100& \raggedright 2S& \raggedright -& \raggedright \tabularnewline \hline
  \hline\multicolumn{14}{|c|}{\normalsize\sc Armi da Tiro (ADT)}\tabularnewline \hline\hline
  \raggedright Frombolo& \raggedright ADT& \raggedright Botta& \raggedright 1d6& \raggedright 2& \raggedright -& \raggedright -1& \raggedright -& \raggedright 50 g& \raggedright 150 cm& \raggedright 100& \raggedright 1S& \raggedright 30 m& \raggedright \tabularnewline \hline
  \raggedright Arco corto& \raggedright ADT& \raggedright Punta& \raggedright 1d8& \raggedright 10& \raggedright 2& \raggedright -1& \raggedright FOR 5& \raggedright 300 g& \raggedright 80-120 cm& \raggedright 70& \raggedright 12S& \raggedright 25 m& \raggedright \tabularnewline \hline
  \raggedright Arco lungo& \raggedright ADT& \raggedright Punta& \raggedright 2d6+1& \raggedright 13& \raggedright 6& \raggedright -1& \raggedright FOR 10& \raggedright 400 g& \raggedright 100-130 cm& \raggedright 70& \raggedright 15S& \raggedright 30 m& \raggedright \tabularnewline \hline
  \raggedright Arco elfico& \raggedright ADT& \raggedright Punta& \raggedright 3d6+2& \raggedright 12& \raggedright 4& \raggedright -2& \raggedright FOR 13& \raggedright 400 g& \raggedright 120-170 cm& \raggedright 20& \raggedright 40S& \raggedright 40 m& \raggedright A Bahuney e Malagana la reperibilit\`a \`e 70\%\tabularnewline \hline
  \raggedright Balestrina& \raggedright ADT& \raggedright Punta& \raggedright 1d8& \raggedright 6& \raggedright -& \raggedright -1& \raggedright -& \raggedright 200 g& \raggedright 30-40 cm& \raggedright 20& \raggedright 15S& \raggedright 20 m& \raggedright \tabularnewline \hline
  \raggedright Balestra& \raggedright ADT& \raggedright Punta& \raggedright 2d6+2& \raggedright 15& \raggedright 4& \raggedright -2& \raggedright FOR 7& \raggedright 1.2 kg& \raggedright 80-110 cm& \raggedright 40& \raggedright 20S& \raggedright 35 m& \raggedright Necessita di 2 round per essere caricata o un tiro Incoccare Veloce a diff. 25\tabularnewline \hline
  \raggedright Balestra pesante& \raggedright ADT& \raggedright Punta& \raggedright 4d6+3& \raggedright -& \raggedright -& \raggedright -3& \raggedright -& \raggedright 6 kg& \raggedright 120-170 cm& \raggedright 20& \raggedright 80S& \raggedright 45 m& \raggedright Postazione fissa. Pu\`o essere usata in movimento con un minimo in FOR di 23. Neccesita di 3 round per essere caricata\tabularnewline \hline
  \hline\multicolumn{14}{|c|}{\normalsize\sc Armi da Fuoco (ADF)}\tabularnewline \hline\hline
  \raggedright Archibugio& \raggedright ADF& \raggedright Punta& \raggedright 4d6+3& \raggedright 14& \raggedright 4& \raggedright -& \raggedright -& \raggedright 3 kg& \raggedright 130-180 cm& \raggedright 1& \raggedright 3000S& \raggedright 15 m& \raggedright Necessita di 2 round per il caricamento. La reperibilit\`a \`e del 2\% a Loydi-Genya. Il TPC ha un malus di -2. Il Fallimento Catastrofico del TPC comporta la distruzione dell'arma e 4d6 PV/ PSC di danno all'utilizzatore\tabularnewline \hline
  \raggedright Pistola ad avancarica& \raggedright ADF& \raggedright Punta& \raggedright 3d6+2& \raggedright 6& \raggedright -& \raggedright -& \raggedright -& \raggedright 1.3 kg& \raggedright 25-40 cm& \raggedright 1& \raggedright 1200S& \raggedright 5 m& \raggedright Necessita di 2 round per il caricamento. La reperibilit\`a \`e del 5\% a Loydi-Genya. Il TPC ha un malus di -5. Il Fallimento Catastrofico del TPC comporta la distruzione dell'arma e 3d6 PV/ PSC di danno all'utilizzatore\tabularnewline \hline
  \pagebreak\hline\multicolumn{14}{|c|}{\normalsize\sc Artiglieria (ART)}\tabularnewline \hline\hline
  \raggedright Catapulta& \raggedright ART& \raggedright Botta& \raggedright 3d20& \raggedright -& \raggedright -& \raggedright -& \raggedright -& \raggedright 2000 kg& \raggedright 3-8 m& \raggedright 30& \raggedright 4000S& \raggedright 70 m& \raggedright Infligge il danno a tutti coloro che si trovano a 3 metri o meno dal punto di impatto. Il Danno va tirato per ognuna delle vittime. Richiede 4 persone per 10 round per il caricamento. \tabularnewline \hline
  \raggedright Ballista& \raggedright ART& \raggedright Punta, Botta& \raggedright 8d6& \raggedright -& \raggedright -& \raggedright -& \raggedright -& \raggedright 600 kg& \raggedright 4-7 m& \raggedright 30& \raggedright 4000S& \raggedright 50 m& \raggedright Balestra gigantesca. Va indirizzata su un solo bersaglio. Il caricamento richiede 8 round a 2 persone.\tabularnewline \hline
  \raggedright Bombarda& \raggedright ART& \raggedright Botta& \raggedright 10d6& \raggedright -& \raggedright -& \raggedright -& \raggedright -& \raggedright 2000 kg& \raggedright 2-3 m& \raggedright 1& \raggedright 8000S& \raggedright 100m& \raggedright Infligge il danno a tutti coloro che si trovano a 3 metri o meno dal punto di impatto. Richiede 2 persone per 6 round per il caricamento. Ha un malus di -5 al TPC. Il Fallimento catastrofico fa esplodere l'arma infliggendo 10d6 PV/ PSC in 3 metri di raggio\tabularnewline \hline
  \hline\multicolumn{14}{|c|}{\normalsize\sc Scudi (SCD)}\tabularnewline \hline\hline
  \raggedright Scudo piccolo& \raggedright SCD& \raggedright Botta& \raggedright 1d4& \raggedright 20& \raggedright 6& \raggedright -1& \raggedright -& \raggedright 500 g& \raggedright 20-30 cm& \raggedright 90& \raggedright 3S& \raggedright -& \raggedright Di forma circolare, si porta sull'avambraccio. Pu\`o essere portato al braccio che regge l'arma.\tabularnewline \hline
  \raggedright Scudo medio& \raggedright SCD& \raggedright Botta& \raggedright 1d6& \raggedright 30& \raggedright 9& \raggedright -5& \raggedright FOR 9& \raggedright 1.2 kg& \raggedright 50-60 cm& \raggedright 90& \raggedright 12S& \raggedright -& \raggedright Scudo da cavaliere. Si usa con un braccio, tiene occupata una mano.\tabularnewline \hline
  \raggedright Scudo grande& \raggedright SCD& \raggedright Botta& \raggedright 1d8& \raggedright 40& \raggedright 12& \raggedright -9& \raggedright FOR 13& \raggedright 5 kg& \raggedright 100-150 cm& \raggedright 90& \raggedright 20S& \raggedright -& \raggedright Si usa con una o due mani.\tabularnewline \hline
  \caption{Armi}{Tabella Armi}\tabularnewline
\end{longtable}
}
  
%%% Local Variables: 
%%% mode: latex
%%% TeX-master: "manual"
%%% End: 
\twocolumn
\fi

\section{Combattere}

Ogni round di combattimento si suddivide in due parti: iniziativa e
azione.

\subsection{L'Iniziativa}

L'iniziativa \textbf{stabilisce l'ordine con il quale i personaggi
  agiscono}.  Ogni round di combattimento, tutti i personaggi, compresi i
PNG, tirano 1d20 (tiro aperto) e sommano \textbf{il punteggio} ottenuto
all'\textbf{AGIlit\`a}, sottraendo i malus all'iniziativa
\textbf{(PA) dell'arma utilizzata}, quello dovuto al \textbf{Peso
  dell'armatura}, quello dovuto al \textbf{Peso Trasportabile}, e gli
ulteriori malus o bonus decisi dal Master o dal regolamento.

Ovviamente chi non utilizza armi non subir\`a il malus per l'arma.  Le
azioni procedono in ordine \textbf{decrescente} dei valori ottenuti.
In caso di punteggio uguale fra due o pi\`u personaggi l'azione
avverr\`a simultaneamente.

Una volta stabilito l'ordine, i giocatori dichiareranno la loro azione
al Master (in caso di azione simultanea \`e preferibile che i
giocatori coinvolti facciano la dichiarazione in segreto).

  
Queste regole per l'iniziativa sono valide anche in caso di lancio
degli incantesimi (pag. \pageref{lanciareincantesimo}).
Tuttavia chi lancia un incantesimo non subisce malus per il peso
dell'armatura in quanto la sua \`e un'azione statica.  Ovviamente non
ha malus per l'utilizzo delle armi, mentre subisce gli eventuali malus
aggiuntivi previsti a seconda delle situazioni (perdita punti vita,
perdita punti fisico, ecc.).

In sostanza quindi il \textbf{Tiro Iniziativa} \`e composto da:

\begin{itemize}
  \itemsep -6pt
\item 1d20
\item AGIlit\`a
\item Malus dell'armatura (da non considerare per il lancio degli
  incantesimi)
\item Malus dell'arma: PA (da non considerare per il lancio degli
  incantesimi e per il combattimento a mani nude)
\item Malus o bonus dovuti a situazioni particolari del PG o decisi
  dal Master (Perdita punti vita, punti fisico, ecc.)
\end{itemize}

L'iniziativa \`e importante perch\'e \`e qui che dovete comunicare
se intendete utilizzare l'arma o meno. Se infatti tirate l'iniziativa
\textbf{senza} calcolare il malus dell'arma, essa non pu\`o essere utilizzata
in quel round. Al contrario per\`o potete tirare l'iniziativa
calcolando il malus dell'arma e decidere ugualmente di attaccare a
mani nude.

Se state usando lo scudo per difendervi dovete tirare l'iniziativa
calcolando il malus dell'arma con cui attaccate e poi difendervi
normalmente con lo scudo (utilizzando l'abilit\`a Scudo).

Se usate lo scudo per attaccare (eh, s\`i! \`E possibile anche
questo!), esso viene considerato come una qualunque altra arma e
l'iniziativa verr\`a calcolata conteggiando la PA dello scudo.

Un Fallimento Catastrofico all'iniziativa comporta la perdita della
possibilit\`a di compiere azioni in quel round.

Tutte le armi richiedono inoltre un tempo tecnico per essere estratte,
imbracciate o caricate. Tutte le ATL, ATC, ABL, ABC, ADL, AIA, SCD, AO
necessitano, qualora non siano gi\`a in mano, di 1 round per essere
sfoderate o un tiro realizzato nell'abilit\`a \textbf{Prontezza}.

Un discorso a parte meritano le ADT e le ADF, queste oltre a
richiedere 1 round per essere imbracciate, necessitano, salvo
diversamente indicato nelle note, di un ulteriore round per essere
caricate o un tiro realizzato nell'abilit\`a \textbf{Incoccare Veloce}. La PA
per queste armi si usa come malus iniziativa solo in seguito alla
riuscita del tiro Incoccare Veloce.

Se l'arma \`e stata caricata in un round precedente la PA \`e
zero.

\nb{Le ADT e le ADF non possono essere imbracciate e caricate nel
  medesimo round}

\subsection{L'Azione} 

Per ora ci limitiamo a descrivere il combattimento a mani nude e con
l'utilizzo delle armi: in questi casi l'azione pu\`o essere di
\textbf{attacco} o di \textbf{difesa}. Per il lancio degli incantesimi
vi rimandiamo al paragrafo ``Lanciare un incantesimo'' a pagina
\pageref{lanciareincantesimo}.

In linea di massima per il Combattimento a mani nude o con le armi si
effettua un confronto fra il Tiro Per Colpire (TPC) dell'attaccante ed
il Tiro Per Difendersi (TPD) del difensore. Se il TPC \`e maggiore
del TPD l'attacco \`e andato a segno.  In caso di parit\`a vince
la difesa.
  
\nb{Il TPC deve essere superiore a 20, in caso contrario esso \`e
  considerato inefficace (nessun danno inflitto) o fuori bersaglio, e
  non sar\`a necessario realizzare un TPD}

\subsection{Attacco} 

Chi \textbf{vince} l'iniziativa rispetto all'avversario o agli
avversari che ha di fronte, pu\`o scegliere:

\begin{enumerate}
  \itemsep 6pt
\item di attaccare
\item di attendere che sia l'altro a
  compiere la prima mossa. 
\end{enumerate}

\subsubsection{Attaccare}
Chi decide di attaccare dichiarer\`a al Master di voler effettuare
l'attacco e lancer\`a 1d20.  Al risultato ottenuto sommer\`a:

\begin{itemize}
\item Il TOT dell'Abilit\`a di Combattimento (AdC) relativa all'arma
  che ha intenzione di utilizzare (ATL, ADT, ABL, ecc.) o, nel caso
  non intenda utilizzarne alcuna, l'abilit\`a di CAC.

\item Malus dato dall'armatura indossata (secondo la tabella ``Malus per il
peso dell'armatura'' nel paragrafo ``Le Armature'').

\item Malus dovuti a stanchezza, danni subiti o carico eccessivo.

\item Eventuali altri malus e bonus decisi dal Master a seconda della
situazione o di ci\`o che il PG effettuer\`a (due azioni, Tiro
Mirato, ecc.).
\end{itemize}

La somma di tutti questi valori (1d20+ Abilit\`a + Malus e Bonus)
\`e il Valore del Tiro Per Colpire (TPC) realizzato.

\es{Lariath ha un TOT di 15 in ATL ed indossa
  un'armatura di piastre completa (malus di -5).  Vuole attaccare
  l'avversario che ha davanti. Lancia il dado ed ottiene 15,
  totalizzando: 15 (ATL) - 5 (Armatura) + 15 (1d20) = 25 Il suo TPC
  \`e pari a 25.}

\pinup{spada_2_mani.eps}{Spadone}

\subsubsection{Attendere}

Chi decide di attendere l'attacco avversario
beneficier\`a di un bonus di +5 ad un eventuale TPD.

\subsection{Colpi
  permessi nel Combattimento a mani nude} 

Se si decide di combattere a
mani nude e quindi di utilizzare l'abilit\`a CAC, i colpi permessi
sono:

\iffullversion
\subsubsection{Testata} Comprende tutti i colpi inferti con il capo.
Una testata infligge 1d4 + BON FOR PV/PSC di danno nel round in cui va
a segno.
\fi

\subsubsection{Pugno}
Comprende tutti i colpi inferti con l'arto superiore (colpo a mano
aperta, pugno, gomitata, spallata).  Un pugno infligge 1d4 + BON FOR
PV/PSC di danno nel round in cui va a segno.  l'attacco di Pugno
beneficia di un bonus di +1 al Tiro Per Colpire.

\es{Marius ha 20 in CAC e vuole sferrare un pugno
  all'avversario che ha di fronte. Marius non indossa nessuna armatura.
  Lancia 1d20 e ottiene 15. Il suo TPC \`e perci\`o pari a: 16
  (1d20+1) + 20 (Abilit\`a di CAC) = 36}

\subsubsection{Calcio} Comprende
  tutti i colpi inferti con l'arto inferiore (calcio, ginocchiata,
  colpi di stinco). Un calcio infligge 1d4+1 + BON FOR PV/PSC di danno
  nel round in cui va a segno.

\iffullversion
\subsubsection{Presa} Comprende tutte le mosse
che consentono di immobilizzare una o pi\`u parti del corpo
dell'avversario e successivamente di infliggere danni. Comprende
atterramenti, leve, morse.

Questo tipo di mossa si svolge su pi\`u round: durante il primo il
difensore pu\`o Schivare o Parare la Presa. Se non ci riesce,
durante il primo round \textbf{non} subisce alcun danno, ma la parte del corpo
sulla quale viene effettuata la presa risulta immobilizzata. 

Per ogni round successivo, fino alla fine della presa, chi la subisce
pu\`o effettuare un TPD Divincolarsi a difficolt\`a pari al TPC
della presa iniziale. Se il tiro di difesa fallisce o il difensore
decide di compiere azioni differenti, subisce \textbf{all'inizio del round} un
danno pari a 1d6 + BON FOR PV/PSC.

\es{Lariath vuole immobilizzare l'avversario che
  ha davanti mediante una presa. Lariath ha 20 in CAC e indossa
  un'armatura di piastre che gli da un malus di -5. Lancia il dado
  e ottiene 12. Il suo TPC \`e perci\`o pari a 27. Il suo
  avversario prova a schivare, ma il suo TPD \`e pari a 25, che non
  gli consente di evitare l'attacco di Lariath.  Per questo round non
  subisce danni, ma si ritrova una parte del corpo immobilizzata (per
  esempio il braccio). 
  
  Il round successivo prova a divincolarsi, ma di nuovo il suo TPD
  \`e soltanto di 23. In questo caso subisce 1d6 + BON FOR PV/PSC di
  danno. Al Round successivo dovr\`a tentare di nuovo di
  divincolarsi se vorr\`a evitare i danni.}

L'attaccante pu\`o decidere all'inizio del round di effettuare un
nuovo TPC per ``aggiustare'' la presa o ``tenere'' la presa precedente
col vecchio TPC.

\es{Continuando l'esempio precedente: Lariath non \`e contento del
  suo TPC, nonostante abbia sortito degli effetti, cos\`{\i} prova ad
  aggiustare la sua presa realizzando un altro TPC.  Tira il dado ed
  ottiene 15, totalizzando quindi 30. Ora il suo avversario dovr\`a
  realizzare un TPD a difficolt\`a 30 e non pi\`u 27}

Il difensore pu\`o \textbf{decidere di subire il Danno} da presa senza
tentare di Divincolarsi e colpire l'attaccante con una parte del corpo
non immobilizzata, se resiste al Danno inferto dalla presa

\es{l'avversario di Lariath si accorge che la presa \`e troppo
  stretta perch\'e lui riesca a divincolarsi cosicch\'e subisce il
  danno inflitto da Lariath senza provare a divincolarsi e decide di
  colpirlo alla testa. Lariath non ha possibilit\`a di difendersi
  perch\'e nel suo round ha continuato la presa.  L'attacco del suo
  avversario va a segno e Lariath subisce il danno.}

La Presa ha termine anche quando l'attaccante non \`e pi\`u in
grado di effettuarla per svenimento, rottura degli arti interessati
per effetto di attacchi di terzi, ecc.

\subsubsection{Proiezione}
Comprende tutte le tecniche che possono infliggere dei danni
all'avversario scagliandolo contro oggetti, pareti, pavimenti,
soffitti (!) ecc.

Sono proiezioni gli sgambetti, i lanci, le catapulte ecc. \`E possibile
evitare la proiezione con una Schivata (vedi ``Difesa'') o, se questa
non riesce, evitare almeno parte dei danni, se la proiezione riesce, con
un tiro in Caduta nello stesso round (vedi ''Limitare i danni in
seguito ad una proiezione'' a pagina \pageref{limitaredanni}).  

La proiezione infligge 1d4+1 + BONus FORza PV/\-PSC di Danno nella
parte del corpo interessata alla caduta. 

In seguito a una proiezione il difensore si pu\`o trovare in
svantaggio perch\'e ``a terra'' nei round successivi.
Ci\`o accade quando la Caduta non riesce. 
\fi

Tutti i colpi (d'attacco) permessi con CAC producono danni da
\textbf{botta} (vedi paragrafo Tipologia del Danno a pagina \pageref{tipodanno}).

\subsection{Difesa}

Chi perde l'iniziativa e decide di difendersi o vince
l'iniziativa e decide di avvalersi del bonus di attesa pu\`o:

\begin{enumerate}
  \itemsep -6pt
\item Parare
\item Schivare
\item Divincolarsi
\item Subire automaticamente il danno avversario e successivamente
  attaccare (se rimane vivo!)
\end{enumerate}

La somma dei valori della difesa (1d20 + Abilit\`a di CAC o
dell'arma utilizzata + Bonus e Malus) costituisce il valore del
\textbf{Tiro Per Difendersi (TPD)}.

\subsubsection{Parata}
La parata consente di evitare di essere colpiti o afferrati. Pu\`o
essere effettuata con l'arma, con lo scudo o senza arma. Se viene
effettuata \textbf{con l'arma} (o con lo scudo) bisogner\`a
\textbf{tirare 1d20} e sommare al punteggio ottenuto:

\begin{itemize}
\itemsep -6pt
\item Abilit\`a di combattimento correlata all'arma in uso (Es.
  ATL, ADT, ABL, SCD, ecc.)
\item Bonus parata dell'arma o dello scudo
\item Malus dell'armatura
\item Eventuali altri malus o bonus (per esempio il bonus di attesa).
\end{itemize}

La parata pu\`o provocare danni alle armi, come vedrete nel paragrafo
``Produrre dei danni alle armi'' a pagina \pageref{danniarmi}.

\es{Joseph ha un TOT di 14 in ATL ed indossa un'armatura di cuoio
  (malus -2). Vuole provare a parare con la spada bastarda (Bonus
  Parata +6) l'attacco del suo avversario. \par Lancia il dado ed ottiene
  16. Il suo TPD \`e pari a: 14 (ATL) + 6 (Bonus Parata) - 2
  (armatura) + 16 (1d20) = 34}

Se viene effettuata senz'arma bisogner\`a sempre tirare un 1d20 ma si
dovr\`a sommare al punteggio: 

\begin{itemize}
\itemsep -6pt
\item Abilit\`a di CAC
\item Malus dell'armatura
\item Eventuali altri malus o bonus.  
\end{itemize}

Ricordiamo che una parata ha successo se \textbf{il risultato del TPD \`e
maggiore o uguale al TPC} dell'attaccante.

Parare frecce, proiettili e simili \`e sempre
un'azione a difficolt\`a 20 + TPC dell'avversario.  Questo
significa che se il vostro avversario vi lancia un pugnale o vi
scaglia contro una freccia ed il suo TPC \`e 23, il vostro TPD
(parare o schivare) dovr\`a essere almeno pari a: 20 + 23 = 43.

\subsubsection{Schivata} \`E possibile tentare di schivare qualunque attacco.
Nel caso in cui si voglia effettuare una schivata bisogner\`a
sommare al tiro del d20: 

\begin{itemize}
\itemsep -6pt
\item Abilit\`a di CAC
\item Bonus dato dall'Abilit\`a Standard \textbf{Acrobazia}
\item Malus dell'armatura
\item Eventuali altri malus o bonus.
\end{itemize}

Ricordiamo che una schivata ha successo se il risultato del tiro
\textbf{Schivare} (TPD) \`e \textbf{maggiore o uguale al TPC}
dell'attaccante.  Schivare frecce, proiettili e simili \`e sempre
un'azione a difficolt\`a 20 + TPC dell'avversario.

\iffullversion
\subsubsection{Divincolarsi} 

Serve per sfuggire all'azione di una Presa. Al TPC della Presa va
confrontato il TPD risultante dalla somma di: 

\begin{itemize}
\itemsep -6pt
\item 1d20 
\item Abilit\`a di CAC
\item Bonus dato dall'abilit\`a Contorsionismo
\item Malus dell'armatura
\item Eventuali altri malus o bonus.
\end{itemize}

Se il \textbf{risultato del TPD Divincolarsi \`e maggiore o uguale al
  TPC di Presa} ci si \`e liberati dalla Presa stessa e non si
subiscono ulteriori danni.  \fi

{\raggedright \subsubsection{Subire l'attacco avversario e successivamente attaccare}}

Quando il vostro avversario vi attacca voi potete decidere di
attaccarlo a vostra volta, senza sprecare l'azione per difendervi
(\`e la tattica di solito utilizzata dai giganti).

In questo caso subite l'attacco (se \`e andato a segno) e
contrattaccate l'avversario, considerando per\`o il fatto che nel
frattempo potreste aver perso dei Punti Vita o dei Punti Struttura
Corpo e che quindi il vostro attacco potrebbe non essere possibile,
per esempio se il braccio con cui tenete l'arma diventa
inutilizzabile, oppure siete svenuti o morti.

Qualora possiate effettuare l'attacco di ritorno, anche il vostro
avversario, ovviamente, non potr\`a difendersi perch\'e ha gi\`a
sprecato l'azione del round per attaccarvi
\iffullversion
 (a meno che non abbia
deciso di Compiere 2 o pi\`u azioni: vedi paragrafo ``Compiere
pi\`u di un'azione in un round'' a pagina \pageref{azionemultipla}).


\subsubsection{Limitare i danni in seguito ad una proiezione}
\label{limitaredanni}
Se il tiro Schivare in seguito ad una Proiezione viene fallito, la
vittima subisce dei danni. Essa pu\`o per\`o limitare tali danni
effettuando con successo una Caduta, come specificato qui di seguito.

\subsubsection{Caduta} Se si \`e stati Proiettati si pu\`o evitare di
farsi male confrontando col TPC della Proiezione il risultato della
somma di 1d20 con 

\begin{itemize}
\itemsep -6pt
\item Abilit\`a di CAC
\item Malus dell'armatura 
\item Eventuali altri malus o bonus
\end{itemize}

 Se il risultato del tiro Caduta \`e maggiore
del risultato del TPC Proiezione si ``risparmia'' un numero di PSC/PV
di Danno pari alla differenza tra Caduta (TPD) e Proiezione (TPC), in
caso contrario si subisce interamente il danno.

\nb{La Caduta \`e a tutti gli effetti un'azione di combattimento,
 quindi non potr\`a essere tentata se non si hanno azioni a
 disposizione, ad esempio se non \`e stata dichiarata un'Azione
 Multipla ed \`e gi\`a stato tentato un TPD nello stesso round.}

\es{Tarkos ha subito una proiezione che gli avrebbe inflitto 5 PSC/PV
  al torace e vuole limitare i danni. Il TPC Proiezione del suo
  avversario era pari a 28.
  
  Tarkos ha 15 in CAC e indossa un'armatura di cuoio (malus -2).
  Lancia 1d20 e ottiene 17.  Il suo tiro Caduta \`e pari a: 17
  (1d20) + 15 (CAC) - 2 (malus dell'armatura) = 30. Tarkos
  ``risparmia'' 30 - 28 = 2 PSC/PV di danno e perci\`o ne subir\`a
  soltanto 3}
\fi

{\sloppy\raggedright \section{Esempio di combattimento}}
\es{
Questo esempio \`e abbastanza complesso. Vi suggeriamo di ``simularlo''
usando PNG creati alla bisogna, mentre lo leggete.

In un dato scontro si hanno 2 PG: \persA, \persB\ e 2 PNG (avversari):
\persC\ e \persD. \persA\ e \persB\ indossano un'armatura di cuoio (malus
-2), mentre \persC\ e \persD\ un'armatura di piastre non completa
(malus -4). Secondo il procedimento sopra descritto si deve stabilire
l'ordine con il quale si svolgono le azioni. \persA\ e \persB\ hanno un
punteggio di AGIlit\`a pari a 18, mentre \persC\ e \persD\ hanno 20.

Tutti e quattro i personaggi utilizzano una spada bastarda (malus
-3/-5 e bonus parata +6), mentre soltanto \persA\ e \persB\ dispongono di
uno scudo.  Ogni personaggio ha 18 in ATL. Nessuno ha bonus di
specializzazione nell'arma.

Procediamo con l'iniziativa: \persA\ totalizza 23 (12 col dado + 18 AGI -
5 spada bastarda - 2 armatura); \persB\ totalizza 30 (19 col dado + 18
AGI - 2 armatura - 5 spada bastarda); \persC\ totalizza 25 (14 col dado + 20 AGI
- 4 armatura - 5 spada); \persD\ totalizza 27 (16 col dado + 20 AGI - 4
armatura - 5 spada).

L'ordine di azione \`e
quindi \persB\ , \persD, \persC, \persA. \persB\ decide di attaccare \persC.
\persC\ decide di difendersi. \persD\ vuole invece attendere le mosse di
\persA\ che sceglie di attaccare \persD. A questo punto tutti i
personaggi coinvolti hanno comunicato le loro intenzioni per questo
round e si pu\`o procedere con le azioni vere e proprie. \persB\ 
attacca \persC\ e totalizza 30 (14 col dado + 18 ATL - 2 armatura). 

A questo punto l'azione spetterebbe a \persD\ ma poich\'e egli ha deciso di
attendere le mosse di \persA\ agir\`a dopo di lui e quindi per ultimo.
\`E quindi \persC\ a potersi muovere provando a difendersi dall'attacco
di \persB. Lancia il dado e ottiene 9 a questo valore aggiunge 18 ATL +
6 bonus parata - 4 armatura e totalizza 29.  La sua difesa non ha
effetto, il suo 29 \`e inferiore al 30 realizzato da \persB\ per cui
quest'ultimo gli infligger\`a dei danni.  l'azione successiva
spetta, come detto, ad \persA\ che attaccando totalizza 28. 

L'ultimo a muoversi pertanto \`e \persD\ che si difende.

Lancia il dado ed ottiene 5, a questo valore sommer\`a, oltre al 18
in ATL al + 6 bonus parata e al - 4 per l'armatura, il +5 che egli
ha conseguito per il fatto che ha atteso l'azione dell'avversario.

In totale quindi ottiene un 30 che supera il 28 di \persA\ e gli
permette di non subire danni. Si proceder\`a in modo analogo anche
per i combattimenti Corpo A Corpo. 

\`E importantissimo notare che se l'azione che si intendeva compiere
non \`e pi\`u possibile per qualsiasi motivo, si ha diritto a
``cambiare idea''. Se avevate per esempio deciso di difendervi da un
attacco, ma questo \`e andato fuori bersaglio (perch\'e il TPC non
ha raggiunto il 20), potete tranquillamente cambiare la vostra
decisione e compiere un attacco o un qualsiasi altro tipo di azione.

Naturalmente se l'iniziativa \`e stata tirata senza il malus dovuto
all'utilizzo dell'arma, quest'ultima non potr\`a essere utilizzata
per quel round}

\iffullversion
{\sloppy\raggedright \section{Compiere pi\`u di un'azione in un round}}
\label{azionemultipla}

Quando abbiamo parlato dei malus che si devono conteggiare in caso di
attacco e di difesa, abbiamo accennato al fatto che si possono
compiere 2 azioni (o pi\`u) in un determinato round.

Pu\`o capitare per esempio che un solo personaggio venga attaccato
da due avversari e che quindi debba fronteggiare due attacchi in uno
stesso round. Se il PG compisse un'azione potrebbe difendersi
soltanto da uno di essi, mentre l'altro avversario andrebbe
tranquillamente a segno senza particolari complicazioni.

\begin{table}[b]
\begin{center}

  {\Large\sc Malus alle \\Azioni Multiple}\medskip
  
  \begin{tabular}{|c|c|}
    \hline
    Numero Azioni& Malus per Azione \\ \hline \hline
    2& -5 \\ \hline
    3& -7 \\ \hline
    4& -9 \\ \hline
    5& -11 \\ \hline
    6& -13 \\ \hline
  \end{tabular}
  \caption{Malus alle Azioni Multiple}
  \label{tabazionimultiple}
\end{center}
\end{table}

Per evitare questo spiacevole inconveniente dovete dichiarare al
Master, prima di effettuare il vostro attacco o la vostra difesa, che
intendete avvalervi della possibilit\`a di effettuare due (o pi\`u)
azioni.

Queste azioni multiple naturalmente saranno pi\`u difficili da
compiere e pertanto saranno assoggettate ad un malus,
tanto pi\`u elevato quanto maggiore sar\`a il numero di
azioni che intendete compiere, come riportato in tabella
``Malus alle Azioni Multiple''.

Compiendo due azioni, per esempio, il malus per ciascuna di esse
sar\`a -5; compiendone 3 il malus sar\`a pari a -7; compiendone 4
sar\`a pari a -9, e cos\`{\i} via

Ricordatevi che potrete eseguire la vostra seconda azione solo per
effettuare un TPD, oppure dopo che tutti hanno completato la loro
prima azione; la terza solo dopo che tutti hanno compiuto la seconda,
ecc.

\nb{Non \`e possibile compiere pi\`u di un'azione a round quando
  una di esse \`e Lanciare un Incantesimo}


\es{\persA\ \`e un PG e deve vedersela contro due PNG: \persB\ e \persC.
  L'ordine di iniziativa \`e \persB, \persA, \persC. \persA\ decide di
  effettuare due azioni. \persB\ lo attacca e ottiene 23, \persA\ come
  prima cosa decide di difendersi e tirando il dado ottiene 10.
  
  Egli indossa un'armatura di cuoio ed ha 15 in ATL.  Quindi la sua
  difesa ha come risultato: 10 (1d20) + 15 (ATL) - 2 (armatura) -5 (2
  azioni) = 18; non abbastanza per fronteggiare l'attacco avversario
  che va a segno. \persA\ pu\`o comunque effettuare la sua seconda
  azione contro \persC\ e decide di attaccarlo (avrebbe anche potuto
  aspettare l'attacco di \persC\ e agire di conseguenza).
  
  Lancia il dado e ottiene 15 totalizzando 23 (15 col dado + 15 ATL -
  2 armatura - 5 per 2 azioni). \persC\ tenta di difendersi ottenendo un
  misero 21 e subendo il danno.}

\subsection{Schivata multipla} 

Un caso particolare di azione multipla \`e rappresentato dalla
Schivata multipla. Essa consente di schivare pi\`u attacchi
contemporanei.  Il TPD schivare sar\`a penalizzato di un malus (come
da Tabella Malus alle azioni multiple), dipendente dal numero di
attacchi da schivare. La schivata multipla usufruisce per\`o di un
bonus di +2.

Ci\`o significa che se un PG volesse schivare 3 attacchi
contemporanei il suo TPD schivare sarebbe penalizzato di -5, se
volesse schivarne 4 avrebbe -7, ecc.

Nel round in cui si compie una schivata multipla non si possono
compiere altre azioni.

\nb{Si effettua un solo TPD, che si confronta con tutti i TPC degli
  attacchi.  Verranno schivati gli attacchi con TPC inferiore o uguale
  al TPD Schivare}

\es{\persA\ vuole schivare gli attacchi contemporanei di \persB, \persC\ e
\persD. \persA\ ha un TOT di 15 in CAC, un TOT di 15 in Acrobazia ed indossa
un'armatura di cuoio (malus -2). Il TPC di \persB\ \`e pari a 28, quello
di \persC\ \`e pari a 35 e quello di \persD\ \`e pari a 22. \persA\ lancia
1d20 e ottiene 19 totalizzando 29 al TPD (15 CAC + 4 Bonus Acrobazia - 2 Armatura
- 7 per schivare 3 attacchi). \persA\ riesce cos\`{\i} a schivare gli attacchi
di \persB\ e \persD\ ma non quello di \persC, che andr\`a a segno.}
\fi

\section{Il Fantoccio} 
\label{fantoccio}

Ogni attacco \`e rivolto ad una particolare parte del corpo, che deve
essere dichiarata prima di effettuare il TPC. Se non eseguite un
attacco mirato, il Master stabilir\`a quale parte del corpo \`e stata
colpita lanciando 1d20 e facendo riferimento alla tabella ``Il
Fantoccio'' a pagina \pageref{tabfantoccio}

In tale tabella sono indicate le varie parti del corpo che \`e
possibile colpire ed il valore del dado ad esse associato. Tirate ``a
pari o dispari'' con un dado qualunque per sapere se la parte colpita
\`e destra (pari) o sinistra (dispari).

\pinupp{spadone_senza_sfondo.eps}{Spada a una mano}{bth}

\nb{Alcune parti del corpo del bersaglio potrebbero non essere
  raggiungibili, se stabilito dal Master. Un attacco mirato a tali
  parti non sar\`a possibile.}

\es{L'attacco di Tarkus \`e andato a segno ma Tarkus non aveva
  mirato l'attacco in nessuna zona particolare del corpo. Il Master
  perci\`o tira 1d20 ed ottiene 1. Il valore 1 corrisponde alla
  Testa. l'attacco di Tarkus ha colpito la testa del suo avversario.
  Il Master potr\`a tirare un successivo d20 per stabilire quale
  parte della testa. Tarkus avrebbe potuto indirizzare l'attacco anche
  ad un punto specifico del capo.}


\section{Tiro Mirato} Al
momento dell'attacco, quindi, potete decidere di colpire il vostro
avversario in una determinata parte del corpo. Per fare ci\`o dovete
comunicare al Master la vostra intenzione (prima di lanciare i dadi
per l'attacco), e considerare il fatto che a seconda della zona che
volete colpire avete un malus al TPC. 

Nelle tabelle sono indicate le
parti del corpo che \`e possibile colpire, ed i relativi malus.
Naturalmente \`e possibile compiere 2 (o pi\`u) azioni mirate
sommando i relativi malus. Il Master pu\`o apportare modifiche al
Fantoccio e al Malus al TPC in relazione alla diversa conformazione
fisica di certe creature (es. Animali, Creature Magiche). 

\iffullversion
\section{Ulteriori Malus e Bonus}

Altre variazioni sono relative alle condizioni di combattimento. A seconda
delle situazioni in cui vi trovate a combattere, \`e possibile che
usufruire di bonus o essere penalizzati da malus.

La tabella sottostante indica queste situazioni e quantifica i bonus e
i malus. I bonus ed i malus previsti possono essere cumulativi.

\begin{radtable}{Tabella Modifiche al Combattimento}{|p{3.5cm}|p{2.5cm}|}
Attacco& Variazione \tabularnewline \hline\hline
\raggedright Attacco di sorpresa&\raggedright  da +5 a +20 all'iniziativa \tabularnewline \hline
\raggedright Subire Attacco di sorpresa&\raggedright  da -5 a -20 al TPD\tabularnewline \hline
\raggedright Subire Attacco alle spalle&\raggedright  -5 al TPD \tabularnewline \hline
\raggedright Mirare &\raggedright  +1 per Round al TPC \tabularnewline \hline
\raggedright ADT e ADL in corsa&\raggedright  -5 al TPC \tabularnewline \hline
\raggedright Azione se si \`e a terra&\raggedright -5 all'iniziativa, -5 TPD, -5 TPC.\tabularnewline \hline
\raggedright Attacco/Difesa al buio&\raggedright  -20 al TPC/TPD \tabularnewline \hline
\raggedright Bersagli molto piccoli rispetto all'attaccante&\raggedright  da -1 a -10 al TPC\tabularnewline \hline 
\raggedright Bersagli molto grandi rispetto all'attaccante& \raggedright da +1 a +10 al TPC \tabularnewline \hline
\raggedright Bersaglio in zona sopraelevata&\raggedright  da -1 a -5 al TPC \tabularnewline \hline
\raggedright Bersaglio pi\`u in basso&\raggedright  da +1 a +5 al TPC \tabularnewline \hline
\raggedright Bersaglio in corsa&\raggedright  da -1 a -3 al TPC\tabularnewline \hline
\end{radtable}
\label{tabmodifiche}
\fi



\section{Danni}

\subsection{Ferire}

Qualora l'attacco sia andato a segno in qualche particolare parte del
corpo, sar\`a necessario calcolare i Danni Inflitti. Ricordiamo che
qualora \textbf{non} si tratti di armi da tiro (ADT), da fuoco (ADF),
da lancio (ADL) e artiglieria (ART) verr\`a sommato al valore ottenuto
dal lancio dei dadi (il cui numero e tipo \`e indicato nella Tabella
Appendice 8 nella casella Danno) il \textbf{BON FOR}.

\begin{table}
\small
\begin{radtable}{Tabella I Fantoccio}{|l|l|r|}
1d20&  Parte del corpo & Malus al TPC \\ \hline\hline
1 & Testa& -5 \\ \hline
2 &Collo& -6 \\ \hline
3,4,5,6 &Torace& -2 \\ \hline
7,8& Addome& -3 \\ \hline
9,10& Braccio& -3 \\ \hline
11,12& Avambraccio& -3 \\ \hline
13& Mano& -6 \\ \hline
14,15,16& Coscia& -2 \\ \hline
17,18& Gamba& -3 \\ \hline
19& Piede& -6 \\ \hline
20& Coda& -5\\ \hline
\end{radtable}
\bigskip
\begin{radtable}{Testa}{|l|l|r|}
1d20& Parte della Testa& Malus al TPC \\ \hline\hline
1-10& Calotta cranica& -6 \\ \hline
11,12& Occhi& -8 (-15 di spalle) \\ \hline
13& Naso& -7 (-14 di spalle) \\ \hline
14& Bocca& -7 (-14 di spalle) \\ \hline
15,16& Orecchie& -7\\ \hline
17& Nuca& -10 \\ \hline
18& Tempia& -8 \\ \hline
19,20& Mento e guancia& -7 \\ \hline
\end{radtable}
\caption{Il Fantoccio}
\label{tabfantoccio}
\normalsize
\end{table}

Il danno viene inflitto alla parte del corpo coinvolta nell'attacco,
determinata dal Tiro Mirato o dal Master. Come detto precedentemente,
ogni arma (ed ogni colpo di Corpo a Corpo) infligge un certo danno per
ogni attacco che viene messo a segno. Tale danno potr\`a essere subito
totalmente dalla vittima, oppure potr\`a essere assorbito in tutto o
in parte dell'armatura che chi subisce l'attacco indossa.

Pi\`u schematicamente possiamo dire che i danni
inflitti ad una persona possono calcolarsi in questo modo: 

\begin{itemize}
  \itemsep -6pt
\item \textbf{Danno base} dell'arma (o del colpo di CAC)
\item \textbf{+\ BON FOR} dell'attaccante (quando previsto)
\item \textbf{-\ PP}, Valore di protezione dell'armatura del difensore (PP)
\item \textbf{-\ Eventuali altre protezioni} (per es. Pelle d'acciaio, 
  vedi ``Tecniche Speciali'' e ``La Magia'').
\end{itemize}

Se il danno inflitto supera il valore di protezione dell'armatura (PP)
e le eventuali altre protezioni, la vittima subir\`a un danno netto e
perder\`a una quantit\`a di Punti Struttura Corpo (PSC) alla parte del
corpo interessata e, contestualmente, una pari quantit\`a di Punti
Vita (PV), pari al danno inflitto meno il valore di protezione
dell'armatura (PP), meno le Eventuali altre protezioni.

Inoltre il danno netto pu\`o essere di entit\`a tale da causare
Danni Addizionali come indicato nel Paragrafo ``Danni Addizionali'' a
pagina \pageref{danniaddizionali}.

\es{Continuando l'esempio precedente: \persB\ causa dei danni con una
  spada bastarda (che infligge 2d8 in quanto \`e stata usata ad una
  mano) a \persC\ che \`e protetto da un'armatura di piastre.
  
  Per semplicit\`a consideriamo i due contendenti con un valore
  rispettivamente in FORza e COStituzione di 14.
  
  Poich\'e l'attaccante \textbf{non} effettua un \textbf{tiro mirato},
  il Master decide che \persB\ colpisce all'addome del suo avversario.
  La tabella ``PUNTI STRUTTURA DEL CORPO'' ci dice che un personaggio
  con un valore in COStituzione di 14 ha, all'addome 8 punti
  struttura, mentre dal paragrafo ``Descrizione delle Armature''
  ricaviamo che un'armatura di piastre ha come valore di protezione
  10 PP.
  
  \persB\ tira il danno e ottiene 13 (12 con 2d8 + 1 di bonus forza).
  \persC\ perder\`a 3 PSC all'addome e 3 punti vita (PV), in quanto
  la sua armatura gli ha ``assorbito'' 10 punti.
  
  Va detto comunque che ogni armatura non magica non \`e eterna, ma
  subisce anch'essa dei danni quando viene colpita. A questo proposito
  vi rimandiamo al paragrafo ``Punti Struttura delle Armature''}

\nb{Chi porta l'attacco pu\`o decidere di \textbf{ridurre} il
  proprio Danno ottenuto dal lancio dei dadi, realizzando un TCONC a
  difficolt\`a 20.}

\es{Browslee ha 18 in FOR, un TOT di 20 in CAC. Sta insegnando una
  tecnica al suo allievo, e vuole dimostrargli la sua abilit\`a.
  Durante il combattimento simulato, Browslee riesce a mettere a segno
  un pugno sul viso del suo allievo. Tira per il danno ottenendo 3
  (1d4) +3 (BON FOR) = 6. Con 6 PV/PSC potrebbe ucciderlo, per cui
  decide di ridurre il danno. Browslee realizza il TCONC. Decide di
  ridurlo di 5 PV/PSC, infliggendo 1 solo PV/PSC. l'allievo viene
  colpito, ma non messo KO}

\iffullversion
{\raggedright \subsection{Produrre dei danni agli oggetti}}

Tutti gli oggetti possiedono un certo numero di Punti Struttura (PS).
Se si escludono Armi e Armature (per i quali esiste una regola
particolare, come indicato in seguito), quando essi vengono colpiti
perdono un numero di PS pari al danno inflitto. Portare a 0 i PS
comporta la distruzione dell'oggetto.

I Punti Struttura degli oggetti devono essere stabiliti
dal Master in relazione alla robustezza dell'oggetto. I linea di massima possiamo
dire che: una serratura ha 5 PS, una porta da 20 a 50, un muro pu\`o avere
a seconda dello spessore da 30 a 10000 PS, un tavolo da 15 a 50, un bicchiere
di vetro 3 PS.

\subsection{Produrre dei danni alle armi}
\label{danniarmi}

Se la parata con l'arma ha successo, l'attaccante colpisce l'arma del
difensore infliggendole i danni che sarebbero stati inflitti al
difensore qualora l'attacco fosse andato a segno.

Se i danni subiti dall'arma superano il suo valore di Punti Struttura Arma (PSA)
l'arma del difensore viene distrutta, altrimenti non succede nulla. In genere
ogni arma ha, in punti struttura, il valore del danno massimo che \`e in grado
di infliggere $+5$. 

In ogni caso i PSA sono indicati nella Tabella Armi a pagina
\pageref{tabarmi}. Possono essere comunque trovate, se il Master lo
ritiene opportuno, armi pi\`u robuste nel corso del gioco.

\pinup{spada_nera.eps}{Spada}

\es{Spada Bastarda contro Spada Bastarda \persA\ para un attacco di
  \persB. \persB\ ha 16 in FORza ed infligge come danno 13 punti (11 con
  2d8+2, + 2 di Bonus Forza), l'arma di \persA\ ha 23 Punti Struttura
  (massimo di (2d8+2) + 5). l'arma non si spezza. Es. Ascia Bipenne
  contro Spada Bastarda \persA\ para con la spada un attacco di \persB
  con l'ascia Bipenne. \persB\ ha 18 in FORza ed infligge come danno 24
  (21 con 3d6+4 + 3 Bonus Forza). l'arma di \persA\ ha 23 Punti
  Struttura e pertanto viene distrutta divenendo inutilizzabile}

Se la parata viene effettuata con lo scudo il discorso non cambia, ed
i suoi punti struttura sono indicati nella Tabella Armi

\es{Spadone a 2 mani contro Scudo. \persA\ para con uno Scudo Medio un
  attacco di \persB\ con uno Spadone a 2 mani. Lo Scudo Medio ha 30
  Punti Struttura. \persB\ ha 19 in forza ed infligge come danno 19 (15
  con 3d6+1 + 4 bonus FORza). Lo Scudo resta intatto.}

Se la parata \`e a mani nude, non si usufruisce del bonus parata
dello scudo o dell'arma ma non si subisce nessun tipo di danno.  La
parata a mani nude presuppone una deviazione del colpo volta ad
evitare il danno.

Se l'attacco \`e a mani nude e la difesa avviene con un'arma sar\`a
l'attaccante a subire il danno che l'arma del difensore produce (senza
il Bonus Forza). Ovviamente il danno viene ``attutito'' dall'eventuale
armatura posseduta.

\es{Pugno contro Spada Bastarda. \persA\ para con la Spada un attacco
  di \persB\ con un pugno. \persB\ subisce un danno di 12 (pari al
  lancio di 2d8+2) al braccio con il quale effettua l'attacco. Avendo
  un'armatura di cuoio perde 9 (12 - 3) punti struttura al braccio e
  altrettanti punti vita Naturalmente anche l'armatura ne risente
  perdendo anch'essa punti struttura come indicato nel paragrafo
  ``Punti Struttura delle Armature'' a pagina \pageref{psarmature}.}

\nb{\`E opportuno fare attenzione al fatto che, se non viene dichiarata
  l'azione multipla come spiegato a pagina \pageref{azionemultipla},
  \textbf{si ha diritto ad una sola azione per round}, e che quindi se
  si \`e per esempio attaccato, in quel round non ci si pu\`o
  difendere da altri attacchi.
  Stesso discorso vale se si \`e difeso.}

\es {Un PG chiamato \persA\ attacca un PNG chiamato \persB\ 
senza dichiarare l'azione multipla, \persA\ non potr\`a difendersi
in quel round da eventuali attacchi di \persC\ o \persD\ e dovr\`a
inevitabilmente subire senza possibilit\`a di difendersi.}

\begin{table*}

  \centering
  {\sc\Large Riassunto del Combattimento\par}\bigskip
  \normalsize 
  L'ordine dell'azione \`e determinato dal risultato del \textbf{Tiro Iniziativa (TI)}

  Chi attacca realizza un \textbf{TPC}

  Chi difende realizza un \textbf{TPD}

  \bigskip

  \footnotesize
  \parbox[t]{6cm}{
    {\large\sc\centering Chi vince l'iniziativa pu\`o:}\bigskip
    \hrule
    \begin{itemize}
      \itemsep -3pt
    \item Lanciare un incantesimo
    \item Attaccare:
      \begin{enumerate}
        \itemsep -3pt
      \item Con arma
      \item Senza arma
      \end{enumerate}
    \item Attendere l'attacco dell'avversario\\ beneficiando di un bonus
      di +5 alla difesa
    \end{itemize}
    } \hspace{2cm}
  \parbox[t]{6cm} {
    {\large\sc\centering Chi perde l'iniziativa pu\`o:} \bigskip

    \hrule
    \begin{itemize}
      \itemsep -3pt
    \item Parare: 
      \begin{enumerate}
        \itemsep -3pt
      \item Con arma
      \item Senza arma
      \end{enumerate}
    \item Schivare 
    \item Divincolarsi
    \item Subire l'attacco e contrattaccare 
    \item Subire l'attacco e lanciare un
      incantesimo
    \end{itemize}
    } 

\end{table*}
\fi

{\raggedright \section{Perdita dei PV e dei PSC}}
\label{perditapvpsc}
  
\`E stato detto che se una qualsiasi parte del corpo viene coinvolta
in uno scontro, pu\`o essere soggetta alla perdita dei Punti Struttura
Corpo. E si \`e anche accennato al fatto che oltre ad essi vengono
persi un pari numero di Punti Vita.  Vediamo ora di spiegare bene
quale relazione esiste fra loro.

Ogni parte del corpo dei personaggi ha un numero di Punti Struttura
(PSC) che indica la sua ``resistenza'' agli urti, agli scontri, ecc.
Ogni volta che una parte del corpo viene colpita perde Punti
Struttura, ma oltre alla parte interessata, \`e anche il corpo nella
sua totalit\`a che ``subisce il danno''. Per questo motivo ogni
qualvolta si perdono Punti Struttura si perdono anche Punti Vita.

Pu\`o accadere infatti che si subiscano danni non abbastanza gravi da
far arrivare a 0 una qualsiasi parte del corpo; ma \`e possibile che i
danni globali subiti siano talmente tanti che i PV scendano abbastanza
in ``basso'' da causare in ogni caso la morte del personaggio oppure
da creargli problemi di concentrazione, di spostamento, ecc.

\`E possibile inoltre
che non vengano inflitti danni a nessuna parte del corpo in
particolare, ma che vengano comunque persi dei PV. Questo pu\`o
avvenire in seguito a incantesimi, malattie, veleni ecc. cio\`e in
seguito ad attacchi non localizzati.

{\raggedright \subsection{Perdita Punti Struttura Corpo (PSC)}}
\label{perditapsc}
Se una parte del corpo arriva a 0 Punti Struttura sar\`a
completamente inutilizzabile, sar\`a altres\`{\i} inutilizzabile la
parte del corpo da essa direttamente dipendente e dovr\`a essere
realizzato un TR a difficolt\`a 25 contro lo svenimento. 

Per esempio, se il braccio raggiunge 0 punti struttura, diverranno
inutilizzabili anche l'avambraccio e la mano; se \`e la gamba a
raggiungere 0 punti struttura anche il piede sar\`a inutilizzabile,
ecc. Se i Punti Struttura scendono in un sol colpo a meno di 0 la
parte viene amputata e con essa le parti direttamente dipendenti
(anche in questo caso si dovr\`a effettuare un TR contro svenimento
a difficolt\`a 25).

\`E evidente che se a raggiungere 0 Punti Struttura \`e una
\textbf{parte vitale (testa, collo, torace, addome)}, le cose si
mettono un po' male per il personaggio, perch\'e significa che sta
rischiando la morte. In questo caso infatti bisogner\`a effettuare un
TR a difficolt\`a 30 per restare in vita, e anche qualora esso venisse
realizzato si entrerebbe comunque in coma.

Se una parte vitale scende al di sotto di 0 Punti Struttura, il
personaggio muore automaticamente senza possibilit\`a di TR.

Per una dose di maggior realismo \`e opportuno che il Master faccia
realizzare un TR a Diff. 20 contro lo svenimento ogni qualvolta la
perdita dei punti struttura risulti particolarmente ingente in seguito
ad un attacco, anche qualora la parte interessata non raggiunga 0
punti struttura.

{\raggedright \subsection{Perdita Punti Vita (PV)}} 
\label{perditapv}

Oltre a rischiare la morte per la perdita di Punti Struttura Corpo, si
rischia di morire anche per la perdita dei Punti Vita.
Infatti, la perdita dei PV comporta un graduale indebolimento fisico e
psichico del vostro personaggio fino a portarlo alla morte. Con la
diminuzione dei PV pertanto, si hanno dei malus (come da tabella
``Malus e Difficolt\`a dei TR in relazione ai PV persi'') a tutte le
azioni.

Giunti a 3 PV il PG sviene automaticamente. Raggiungendo 1 PV il
personaggio entra in coma. \textbf{Se i PV scendono a zero o meno il PG muore.} Se
viene perso in un sol colpo un numero di punti vita pari almeno a 1/5
(un quinto) del totale bisogna effettuare un TR a difficolt\`a
stabilita dal Master, per evitare lo svenimento.

\pinup{mazza_ferrata.eps}{Mazza ferrata}

\`E quindi importante, ai fini del gioco e del realismo, che ci si ricordi
che ogni volta che si perdono PSC si perde anche un pari numero di PV.

Come nel caso dei punti fisico e dei punti mente quando si perde un
determinato numero di Punti Vita rispetto al totale, occorre
effettuare un TR a difficolt\`a variabile, come indicato in tabella,
per evitare lo svenimento. Con la perdita dei PV si hanno anche dei
malus a tutti i tiri (esclusi i Tiri Incremento) come indicato in
tabella \ref{malustr}.

\begin{table}[b]
\begin{radtable}{Malus e Difficolt\`a dei TR\\in relazione ai PV persi}{|l|c|c|}
Frazione di PV persi& Difficolt\`a&Malus\\ 
rispetto al totale & TR & ai Tiri \\ \hline\hline
da 1/5 a 2/5& 10& -1 \\ \hline
da oltre 2/5 a 3/5& 15& -2 \\ \hline
da oltre 3/5 a 4/5& 25& -3 \\ \hline
oltre 4/5& 35& -4 \\ \hline
\end{radtable}
\caption{Malus TR per PV persi}
\label{malustr}
\end{table}

Rifarsi all'esempio nel Paragrafo Stancarsi a pagina \ref{stancarsi}

\iffullversion
\subsection{Tipologia del Danno} 
\label{tipodanno}

La Tipologie del Danno indicano la ``natura'' del danno inflitto. Esse
sono:

\begin{itemize}
  \itemsep -6pt
\item Distorsione (DIS)
\item Lussazione (LUS)
\item Ustione (UST)
\item Ferita (FER)
\item Emorragia (EMO)
\item Frattura (FRA)
\item Emorragia Interna (EMI)
\end{itemize}

Esse dipendono dal Tipo di Colpo Inferto dell'arma utilizzata indicato
nella casella TCI nella tabella relativa alle armi. Qualora le TCI
indicate per un'arma dovessero essere pi\`u di una, \`e il Master a
decidere quella pi\`u opportuna in relazione all'attacco portato. Per
ogni TCI sono indicate pi\`u Tipologie di Danno, per cui, anche in
questo caso, il Master sceglie quella pi\`u opportuna in relazione
all'entit\`a del danno inflitto.

Considerate anche che ogni arma pu\`o produrre dei \textbf{Danni
  Addizionali} (vedi paragrafo a pagina \pageref{danniaddizionali}).

\begin{radtable}{Tabella Tipologia di Danno}{|l|l|}
TCI& TDA \\ \hline\hline
Punta& FER - EMO \\ \hline
Taglio& FER - EMO - FRA \\ \hline
Botta& DIS - LUS - FRA - EMI \\ \hline
\end{radtable}

\es{Plinius subisce 10 PSC/PV di danno da Taglio per via di un attacco di Spada
Bastarda. Il Master ritiene che la TDA pi\`u opportuna sia FER.}

\subsection{Danni Addizionali}\label{danniaddizionali}

Ogni arma utilizzata infligge una diversa tipologia di Danno ed
inoltre infligge un Danno addizionale (proporzionale al Danno inflitto), per
ogni round successivo a quello in cui si \`e subito il danno. 

Per i Danni Addizionali valgono le Tipologie di danno indicate nel
paragrafo Tipologia del Danno. Anche in questo caso \`e il Master a
scegliere la TDA per i Danni Addizionali in relazione all'entit\`a del
danno.

\`E possibile, inoltre, che anche attacchi particolari quali fiamme,
gelo, elettricit\`a e acido causino danni supplementari al PG. Questo
accade quando si lanciano incantesimi che causano un danno di entit\`a
sufficiente a produrre danni addizionali come spiegato qui sotto.

In questi casi la Tipologia di Danno \`e Ustione.

Per altri tipi di incantesimi che provocano danno il Master decide la
tipologia di Danno basandosi sui TCI e TDA delle Armi. Normalmente I
Danni Addizionali sono uguali ad un numero di PSC/PV pari alla met\`a
del danno netto subito meno 5 arrotondando per difetto. Cio\`e:
$$\frac{Danno\,Netto\,Subito}{2} - 5$$


\es{Plinius subisce 12 PSC/PV di danno netto da Taglio alla gamba per via di un
attacco di Spada Bastarda. A partire dal round successivo perder\`a un numero
di PSC/PV pari a: (12 / 2) - 5 = 1. Il Master decide che la TDA per il danno
addizionale \`e EMOrragia}

L'amputazione di una qualsiasi parte del corpo comporta in ogni caso
la perdita di almeno 1 PV di danno a round. Il protrarsi dei Danni
Addizionali pu\`o essere interrotto con l'abilit\`a Pronto
soccorso (vedi).

Finch\'e la ferita non viene curata con l'abilit\`a Medicina o
magicamente, oppure fino a quando la ferita non \`e rimarginata del
tutto sar\`a necessario rinnovare periodicamente la medicazione fatta
con il Pronto Soccorso. In caso contrario, il protrarsi dei danni
addizionali riprender\`a a scorrere. 

I PV e PSC persi a causa dei danni addizionali, se curati con
l'abilit\`a Medicina o Pronto soccorso (vedi), vengono recuperati al
ritmo di 1 ogni 24 ore circa.  
\fi

\subsection{Recupero Punti Vita e Punti Struttura Corpo}

I PV e I PSC possono essere recuperati mediante \textbf{incantesimi,
  erbe medicinali o mediante il trascorrere del tempo}.

Nei primi due casi il recupero dipende rispettivamente dalla potenza
dell'incantesimo e dal tipo di erba medicinale utilizzata, per cui ce
ne occuperemo nei paragrafi ad essi dedicati.

I PV e i PSC persi possono essere recuperati, quindi, col trascorrere
del tempo, in un numero di giorni pari al danno subito (PSC/PV) pi\`u
una costante che dipende dalla tipologia del danno (TDA), per due
ossia:

$$(Danno\,Subito+Costante\,TDA)\times 2$$

 Le tipologie di danno (TDA) e le loro costanti sono: 
\begin{radtable}{TDA e Costanti}{|l|c|}
Distorsioni& 1\\ \hline
Lussazioni& 5 \\ \hline
Ustioni& 10 \\ \hline
Ferite - Emorragie& 15 \\ \hline
Fratture& 20\\ \hline 
Emorragie Interne& 25\\ \hline
\end{radtable}

\es{Se vi siete procurati una distorsione al polso che vi ha fatto 2
  (PSC/PV) di danno, vi ristabilirete completamente in (2 danno + 5
  costante) x 2 = 14 giorni.
  
  Se avete subito una ferita che vi ha procurato un danno da 6 PSC/PV,
  per ristabilirvi occorreranno: (6 danno + 15 costante) x 2 = 42
  giorni.}

Se le ferite vengono curate con l'abilit\`a Medicina il tempo di
guarigione diminuisce (Vedi abilit\`a Medicina).

{\raggedright \section{Stancarsi durante un combattimento}}

Il combattimento, come tutte le azioni, determina la perdita di PF e
PM. Il numero di punti persi viene determinato dal Master al termine
dello scontro (o durante una fase intermedia del combattimento se
questo \`e molto lungo) in base alla durata del combattimento e alla
sua intensit\`a (numero di azioni per round, azioni in movimento,
ecc.). 

Orientativamente si pu\`o conteggiare 1 PF e 1 PM per ogni
azione di attacco o di difesa. Inoltre, alcuni colpi speciali compresi
nelle Arti Marziali determinano una perdita immediata di PM e PF.

\section{Le Armature} 

L'armatura \`e il tipo di protezione che indossate
per proteggervi passivamente dagli attacchi avversari.  

Potete scegliere fra i tipi indicati nella ``Descrizione delle
armature''.  Ogni materiale ha una determinata protezione
(\textbf{PP}) che indica il valore da sottrarre ai danni inferti dal
vostro nemico, una certa resistenza a subire danni (\textbf{PS}) ed
anche un proprio \textbf{peso}.

{\raggedright \subsection{Punti struttura delle Armature}}
\label{psarmature}

Ogni parte dell'armatura ha un certo numero di Punti Struttura (PS)
dipendenti dal tipo di armatura ed un certo numero di Punti Protezione
(PP) pari ad 1/5 (arrotondato per eccesso) dei PS.

Assorbendo i colpi, l'armatura, oltre a proteggere il PG, viene
gradualmente danneggiata perdendo PS e di conseguenza PP.

Se il danno inflitto \`e inferiore o uguale ai PP
l'armatura perde 1 PS. 

Se il danno inflitto \`e invece maggiore dei
PP, la parte dell'armatura colpita perde un numero di PS pari al danno
inflitto meno i PP dell'armatura stessa. In linea di massima perde
quindi lo stesso numero di punti struttura che perde il PG. 

\es{Un colpo di spada bastarda colpisce il torace di un PG protetto da
armatura di Piastre (50 PS e 10 PP). Il danno inflitto \`e 15.
l'armatura perde 5 punti struttura (cos\`{\i} come il torace del PG),
perch\'e 15 (danno inflitto)- 10 (punti protezione) = 5. Il suo valore
di protezione (PP) scende a 9 perch\'e perdendo 5 PS, questi
scendono a 45 e 45/5=9}
\iffullversion
\es{\persA\ viene colpito all'addome da un attacco di pugnale (1d6 di
  danno + Bonus Forza). Egli all'addome indossa un'armatura di cuoio
  (15 PS e 3 PP). Il danno inflitto dal pugnale \`e 5. l'armatura
  perde 2 PS (perch\'e 5 - 3 PP = 2), va a 13 PS ma i suoi PP restano 3
  perch\'e 13/5 = 2,6 che arrotondato per eccesso \`e uguale a 3}


\es{\persA\ \`e protetto da un'armatura di maglia (35 PS e 7 PP) e
  viene colpito al braccio da un pugno (1d6/2 + Bonus Forza di danno)
  subendo 3 punti di danno.  Poich\'e 3 \`e minore dei PP cio\`e 7,
  \persA\ resta illeso ma la sua armatura perde comunque 1 PS}
\fi
\subsection{Descrizione delle Armature} 

Potete costruire il tipo di armatura che volete, anche composta da
materiali diversi nelle varie parti del corpo, ma dovete fare molta
attenzione al peso totale dell'armatura, perch\'e questo vi da un
malus a tutte le azioni in movimento ed ai tiri iniziativa secondo la
tabella ``Malus per il peso dell'armatura'', a pagina
\pageref{maluspesoarmatura}.

\arm{Armatura di Cuoio}{3}{15} Questa armatura \`e la pi\`u versatile in
quanto pu\`o coprire tutto il corpo. Essa tuttavia non offre grande
protezione. Ogni parte dell'armatura \`e stretta al corpo con delle
stringhe.
\vfill
\begin{radtable}{}{|l|c|c|}
Parte del corpo& Peso (Kg)& Prezzo \\ \hline\hline
Testa& 0.5 & 2S \\ \hline
Collo& 0.3 & 1S \\ \hline
Torace& 5.4 & 5S \\ \hline
Addome& 3.2 & 3S \\ \hline
Braccia& 0.4 + 0.4 & 3S + 3S \\ \hline
Avambracci& 0.2 + 0.2 & 2S + 2S \\ \hline
Mani& 0.1 + 0.1& 2S + 2S \\ \hline
Cosce& 0.5 + 0.5& 4S + 4S\\ \hline
Gambe& 0.2 + 0.2& 3S + 3S\\ \hline
Piedi& 0.2 + 0.2& 2S + 2S \\ \hline\hline
\textbf{Totale}& \textbf{12.6}&\textbf{43S} \\ \hline
\end{radtable}

\iffullversion
\arm{Armatura di Osso/Legno}{5}{25} \`E un'armatura non molto elaborata. \`E
costituita da pezzi di legno o d'osso lavorati e legati da strisce di
pelle. A causa della sua scarsa flessibilit\`a non protegge tutte le
parti del corpo.

\begin{radtable}{}{|l|c|c|}
Parte del corpo&Peso (Kg)&Prezzo \\ \hline\hline
Testa&1&6 S \\ \hline
Collo&-&- \\ \hline
Torace&6&4 S \\ \hline
Addome&-&- \\ \hline
Braccia&0.5 + 0.5&2 S + 2 S \\ \hline
Avambracci&0.4 + 0.4&2 S + 2 S \\ \hline
Mani&-&- \\ \hline
Cosce&0.6 + 0.6&3 S + 3 S \\ \hline
Gambe&0.5 + 0.5&3 S + 3 S\\ \hline 
Piedi&-&- \\ \hline\hline
\textbf{Totale}&\textbf{11}&\textbf{30 S}\\ \hline
\end{radtable}
\fi

\arm{Armatura di Maglie}{7}{35} La cotta di maglie \`e composta da anelli di
metallo intersecati a formare una fitta rete. Pu\`o proteggere tutte
le parti del corpo.

\begin{radtable}{}{|l|c|c|}
Parte del corpo&Peso (Kg)&Prezzo \\ \hline
Testa&0.8&4 S \\ \hline
Collo&0.5&2 S \\ \hline
Torace&6.2&13 S\\ \hline
Addome&3.5&9 S \\ \hline
Braccia&0.7 + 0.7&6 S + 6 S \\ \hline
Avambracci&0.5 + 0.5&4 S + 4 S \\ \hline
Mani&0.3 + 0.3&7 S + 7 S \\ \hline
Cosce&0.8 + 0.8&8 S + 8 S \\ \hline
Gambe&0.5 + 0.5&6 S + 6 S \\ \hline
Piedi&0.3 + 0.3&5 S + 5 S \\ \hline\hline
\textbf{Totale}&\textbf{17.2}&\textbf{108 S}\\ \hline
\end{radtable}

\iffullversion
\arm{Armatura di Scaglie}{8}{40} \`E composta da rombi di metallo fissati su
supporti di cuoio. Offre un'ottima protezione, ma soltanto su alcune
parti del corpo.
\goodbreak
\begin{radtable}{}{|l|c|c|}
Parte del corpo&Peso (Kg)&Prezzo \\ \hline
Testa&-&- \\ \hline
Collo&-&- \\ \hline
Torace&7&22 S \\ \hline
Addome&3.6&16 S \\ \hline
Braccia&0.8 + 0.8&8 S + 8 S \\ \hline
Avambracci&-&- \\ \hline
Mani&-&- \\ \hline
Cosce &0.9 + 0.9&9 S + 9 S \\ \hline
Gambe&-&- \\ \hline
Piedi&-&- \\ \hline\hline
\textbf{Totale}&\textbf{14}&\textbf{102 S}\\ \hline
\end{radtable}

\arm{Armatura di Bande}{9}{45} l'armatura di bande \`e costruita con strisce
metalliche orizzontali bullonate fra loro. Copre poche parti del
corpo.

\begin{radtable}{}{|l|c|c|}
Parte del corpo&Peso (Kg)&Prezzo \\ \hline
Testa&2.2&7 S \\ \hline
Collo&-&- \\ \hline
Torace&9&35 S \\ \hline
Addome&4& 22 S \\ \hline
Braccia&-&- \\ \hline
Avambracci&-&- \\ \hline
Mani&-&- \\ \hline
Cosce&-&- \\ \hline
Gambe&-&- \\ \hline
Piedi&-&- \\ \hline\hline
\textbf{Totale}&\textbf{15.2}&\textbf{64 S}\\ \hline
\end{radtable}
\fi

\arm{Armatura di Piastre}{10}{50} 

\`E la classica armatura medievale costruita di metallo battuto. 

Copre tutte le parti del corpo tranne mani, collo e piedi.  

Nelle battaglie viene indossata sopra la cotta di maglie. In tal caso
si parla di armatura completa o da cavaliere. 

Se le si indossano entrambe, per montare su una cavalcatura
occorrer\`a essere aiutati, poich\'e il peso supera i 44 Kg.

La celata (visiera) ostacola la visibilit\`a con un malus all'OSS
variabile da -1 a -5, ma d\`a un pari malus al TPC mirato agli occhi.

\begin{radtable}{}{|l|c|c|}
Parte del corpo&Peso (Kg)&Prezzo \\ \hline
Testa&2.6&14 S \\ \hline
Collo&1.2&6 S \\ \hline
Torace&10&6 C\\ \hline
Addome&4.2&5 C \\ \hline
Braccia&1.2 + 1.2&2 C + 2 C \\ \hline
Avambracci&1 +1&1 C + 1 C \\ \hline
Mani&-&- \\ \hline
Cosce&1.3 + 1.3&3 C + 3 C \\ \hline
Gambe&1.2 + 1.2&2 C + 2 C \\ \hline
Piedi&-&- \\ \hline\hline
\textbf{Totale}&\textbf{27.4}&\textbf{290 S} \\ \hline
\end{radtable}

\`E possibile perci\`o equipaggiarsi con un'armatura totalmente di
piastre che vi dar\`a un malus di -5; con una completamente di cuoio (malus
-2); sovrapporre diverse armature cumulando il loro peso; oppure indossare una
armatura composita, che preveda cio\`e parti di diverse 
\iffullversion armature, come mostra
l'esempio seguente.

\es{Xeres vuole costruirsi un'armatura composita e decide di suddividerla
in questo modo: 

\begin{tabular}{|lll}
Testa& Piastre& Peso 2.6  \\
Collo& Maglia& Peso 0.8 \\
Torace& Bande& Peso 9 \\
Addome& Piastre& Peso 4.2 \\
Braccia& Scaglie& Peso 0.8 + 0.8 \\
Avambracci& Piastre& Peso 1 + 1 \\
Mani& Cuoio& Peso 0.1 + 0.1 \\
Cosce& Scaglie& Peso 0.9 + 0.9\\ 
Gambe& Maglie& Peso 0.5 + 0.5 \\
Piedi& Cuoio& Peso 0.2 + 0.2 \\
\end{tabular}

Il Peso Totale di quest`armatura
\`e pertanto pari a: 23,6 Kg. Secondo la tabella sottostante, il malus ai
suoi tiri sar\`a pari a $-4$.}
\else
armature.
\fi

Tenete presente che il peso dell'armatura che indossate rientra nel
Peso Trasportabile dal vostro PG, e che quindi subirete un ulteriore
malus a tutte le azioni in movimento ed ai tiri iniziativa qualora
questa dovesse superarlo.

Evitate di costruirvi armature troppo pesanti anche alla luce del
fatto che dovrete portarvi appresso tutto il vostro equipaggiamento e
potrebbe diventare troppo penalizzante per voi. 

\textbf{Anche se l'armatura viene danneggiata e raggiunge 0 PS, il peso non
diminuisce.}

\iffullversion
\`E possibile inoltre far costruire armature ``di
qualit\`a'' che possono essere pi\`u resistenti (maggior rapporto
fra PS e PP, per esempio 6 ogni 1 anzich\'e 5 ogni 1) o che diano
minor ingombro (minor peso per ogni parte del corpo). 

Il prezzo di
queste armature pu\`o lievitare fino ad oltre 4 volte il prezzo
originale e la loro reperibilit\`a \`e molto bassa. 

Il peso delle armature pu\`o essere ridotto magicamente. Bench\'e
l'armatura sia meno ingombrante o venga alleggerita con la magia il
malus alle azioni in movimento e ai tiri iniziativa non sar\`a mai
inferiore ad 1/2 arrotondato per eccesso del malus originario.

Cos\`{\i} una armatura di Piastre ``leggera'' avr\`a un malus minimo
di - 3 (5 diviso 2 arrotondato per eccesso).
\fi

\begin{table}
\begin{radtable}{Malus Per Il Peso dell'Armatura}{|l|c|}
Peso totale dell'armatura& Malus ai Tiri \\ \hline\hline
da 6 kg a 10 kg& -1 \\ \hline
da oltre 10 kg a 15 kg& -2 \\ \hline
da oltre 15 kg a 20 kg& -3 \\ \hline
da oltre 20 kg a 25 kg& -4 \\ \hline
Da oltre 25 kg a 30 kg& -5 \\ \hline
\end{radtable}
\caption{Malus per il Peso dell'Armatura}
\label{maluspesoarmatura}
\end{table}

Nella scheda del PG troverete accanto alle caselle dove indicare i PSC
le caselle dove riportare i PS e i PP dell'armatura.  Avrete cos\`{\i}
le caselle: PPS (Punti Protezione della parte Sinistra). PPD (Punti
Protezione della parte Destra). PSS (Punti Struttura della parte
Sinistra).  PSD (Punti Struttura della parte Destra).

\iffullversion
\subsubsection{Armature per i giganti} 

Poich\'e i giganti sono alti in media il doppio di un uomo, anche le
armature, a parit\`a di protezione, saranno pi\`u pesanti.

Ogni pezzo di armatura gigante, di qualunque tipo e materiale, pesa
\textbf{il doppio di quello normale}, perci\`o per conoscerne il
peso in kg deve essere moltiplicato per 2 il peso indicato nelle
tabelle precedenti. Un'armatura di piastre completa peser\`a in
questo modo, per un gigante 54.8 kg, una di cuoio completa 25.2 ecc.

Per quanto riguarda la tabella per il malus delle armature, per i
giganti vale quanto segue:

\begin{radtable}{Malus Per Il Peso dell'Armatura Gigante}{|l|c|}
Peso totale dell'armatura& Malus ai Tiri \\ \hline\hline
da 12 a 20 kg& -1 \\ \hline
da oltre 20 a 30& -2\\ \hline 
da oltre 30 a 40& -3\\ \hline
da oltre 40 a 50& -4\\ \hline
da oltre 50 a 60& -5\\ \hline
\end{radtable}
\fi

{\raggedright \section{Le Arti Marziali e Le Maestrie nelle armi}}

Le Arti Marziali e le Maestrie nelle armi costituiscono una parte
fondamentale di Radix Malorum

Qualunque personaggio, nei limiti imposti dalle razze e dalla classe
sociale, pu\`o aver imparato nel corso della sua vita ad utilizzare
armi o a combattere a mani nude.

Nel Capitolo dedicato al Combattimento abbiamo gi\`a accennato cosa
si \`e in grado di fare con le Abilit\`a di Combattimento; ora
vediamo di descrivere cosa significa aver imparato un'Arte Marziale
o essere diventati Maestri d'arma. 

Le abilit\`a di Combattimento, le Specifiche e le Maestrie vengono
spesso insegnate nell'Esercito.

\subsection{L'Arte Marziale} 
\label{artemarziale}

L'Arte Marziale (AM) \`e costituita da un insieme di
\textbf{Abilit\`a di Specializzazione} (vedi pagina
\pageref{tabspecializzazioni}) sui vari \textbf{Colpi Permessi} nel
combattimento Corpo a Corpo. Si tratta di abilit\`a che
\textbf{forniscono ulteriori Bonus al TPC o al TPD} (secondo la
tabella Specializzazioni) \textbf{ed al Danno}.

\iffullversion
Queste abilit\`a, denominate \textbf{Specifiche di Arte Marziale}
sono:

\begin{description}
\item{\bf Pugno} Conferisce un bonus al
TPC ed al Danno secondo la tabella Specializzazioni a tutti i colpi
inferti con l'arto superiore (colpo a mano aperta, pugno, gomitata,
spallata)
\item{\bf Calcio} Conferisce un bonus al TPC ed al Danno come
da tabella Specializzazioni a tutti i colpi inferti con l'arto
inferiore (calcio, ginocchiata, colpi di stinco),

\item{\bf Parata}
Conferisce un bonus al TPD per tutte le parate effettuate a mani nude,
secondo la Tabella Specializzazioni.

\item{\bf Schivata} Conferisce un
bonus al TPD per le schivate come da tabella Specializzazioni.

\item{\bf Presa} Conferisce un bonus al TPC ed al Danno secondo la
tabella Specializzazioni a tutte le mosse che consentono di
immobilizzare una parte del corpo dell'avversario e successivamente di
infliggere danni.

\item{\bf Divincolarsi} Conferisce un bonus al TPD
secondo la tabella specializzazioni ai movimenti che permettono di
liberarsi dalle prese se queste hanno avuto successo (cio\`e se si
\`e stati immobilizzati). 

\item{\bf Proiezione} Conferisce un bonus al
TPC ed al Danno secondo la tabella specializzazioni a tutte le
tecniche che possono infliggere dei danni all'avversario scagliandolo
contro oggetti, pareti, pavimenti ecc.

\item{\bf Caduta} Conferisce un
bonus ai tentativi di non subire danno in seguito ad una proiezione
secondo la tabella Specializzazioni. Conferisce inoltre un bonus
all'abilit\`a Cadere secondo la Tabella Specializzazioni. 

\end{description}

Tutte le Specifiche di AM usufruiscono del \textbf{Bonus Agilit\`a}.
Ci\`o significa che il TOT di ogni specifica \`e dato dal VAL che ad
essa viene attribuito al momento della creazione del personaggio pi\`u
il BON AGI.

\es{Markus ha un TOT di 20 in CAC, e un TOT di 15 nella specifica
Pugno, che corrisponde, secondo la tabella Specializzazioni ad un
Bonus al Danno e al TPC pari a +4. Markus indossa un'armatura di
cuoio.  Vuole sferrare un pugno al suo avversario. Il suo TPC \`e
pari a: 1d20+1 + 20 (abilit\`a CAC) - 2 (Malus Armatura) + 4 (Bonus
specifica Pugno)). Se il suo attacco andasse a segno, Markus
infliggerebbe oltre al normale danno del pugno (ossia 1d4 + BON FOR)
anche il +4 della specifica Pugno}

Possono essere attribuiti dei VAL
anche alle specifiche di AM esattamente come alle altre abilit\`a.

Al momento della creazione del
personaggio, ricordiamo, il TOT di queste non pu\`o per\`o
superare il 12. 

Per poter attribuire almeno 1 VAL alle Specifiche di
Arte Marziale bisogna avere almeno un TOT di 12 nell'abilit\`a Corpo
a Corpo. 

\subsection{La Maestria} 

La Maestria (MAE) in una o pi\`u armi consiste in una serie di
Abilit\`a di Specializzazione basate sull'uso di armi specifiche; il
Bonus che esse conferiscono si somma al TPC o al TPD per l'abilit\`a
di combattimento corrispondente al tipo di arma.

Si pu\`o quindi avere una Maestria in Spada Bastarda, il cui bonus
si somma ai TPC o ai TPD (nel caso di parata con l'arma) effettuati
sull'abilit\`a Armi da Taglio Lunghe quando si usa la Spada Bastarda
(ma non, ad esempio, quando si usa la Sciabola Elfica).

Ogni maestria fornisce anche un bonus al Danno per l'arma in cui si ha
la Maestria, secondo la tabella Specializzazioni.

Si pu\`o essere maestri di qualsiasi arma indicata nella Tabella
Armi a pagina \pageref{tabarmi}.

Le Maestrie usufruiscono del Bonus della Caratteristica cui fa
riferimento l'abilit\`a di combattimento relativa al tipo di arma di
cui si \`e maestri. Per esempio: la Spada bastarda \`e una ATL
pertanto la maestria usufruisce del BON AGI, l'Arco Lungo \`e un'ADT e
pertanto la maestria in quest`arma usufruisce del BON OSS, ecc. Quindi
il TOT nella Maestria \`e uguale al VAL + Bonus Caratteristica
Correlata.

\es{Markus ha un TOT di 18 in ADT (Armi Da Tiro) e un TOT di 15 (bonus
  +4 come Specializzazione) nella MAEstria in Arco Elfico.  Il suo TPC
  quando usa l'arco Elfico \`e pari a 1d20+18+4, ma il suo TPC per
  qualunque altro tipo di ADT \`e soltanto 1d20+18, cio\`e non
  pu\`o usufruire del bonus della MAEstria. Quando usa l'arco
  Elfico, il danno che infligge \`e pari a 3d6+2 (Danno normale
  dell'arco Elfico) + 4 (Bonus della Maestria)}

\es{Titius ha un TOT di 15 in ATL e un TOT di 12 (Bonus +3 come
  Tabella Specializzazioni) nella Maestria in Spada Bastarda (che
  ricordiamo ha un BP di +6) ed indossa un'armatura di cuoio. Vuole
  difendersi dall'attacco del suo avversario che ha realizzato 31 al
  TPC. Il suo TPD quando usa la Spada Bastarda \`e pari a 1d20 + 15
  (ATL) + 6 (Bonus Parata della Spada) - 2 (Malus dell'armatura) + 3
  (Bonus Maestria in Spada Bastarda)}

Per poter attribuire almeno 1 VAL alle Maestrie bisogna avere almeno
12 di TOT nell'abilit\`a di combattimento riferita all'arma di cui
si \`e maestri (Spada bastarda = ATL, Arco Lungo = ADT, Arco elfico
= ADT, ecc.).

\nb{Al momento della creazione del personaggio non si possono avere
  pi\`u di 3 maestrie}

\nb{Il TOT delle Maestrie nelle armi non pu\`o inizialmente superare
  il 12}

\nb{Non si pu\`o avere pi\`u di una
maestria per una stessa arma} 

\section{Tiri Incremento} 

Il Tiro Incremento su ognuna delle specifiche e sulle maestrie va
effettuato nei modi descritti nel paragrafo dedicato al miglioramento,
a pagina \pageref{miglioramento} tenendo presente che ogni specifica
ha D1 come codice di difficolt\`a (vedi pagina
\pageref{tabdifficolta}).

Il TOT della Specifica di arte marziale pu\`o superare il TOT
dell'abilit\`a di combattimento correlata (ADT, ATL, CAC, ecc.).

\nb{Il Master non deve concedere tiri incremento su abilit\`a
  utilizzate in concomitanza di un bonus dovuto ad incantesimi e
  simili}

\section{Tecniche Speciali}

Sia le Maestrie che l'arte Marziale comprendono inoltre una serie di
\textbf{Tecniche Speciali} che devono essere \textbf{apprese
  singolarmente} durante il gioco, nei luoghi e nei modi descritti in
altre parti di questo Manuale, attraverso l'insegnamento o lo studio
(non gli si possono attribuire dei VAL al momento della creazione del
personaggio).

Il Miglioramento funziona allo stesso modo di tutte le altre
abilit\`a, il codice di difficolt\`a \`e indicato per ogni
Tecnica Speciale.

A determinati valori di TOT raggiunti corrisponde un Bonus (come da
Tabella Specializzazioni a pagina \pageref{tabspecializzazioni}) che
viene utilizzato nelle modalit\`a specificate in \textbf{ogni colpo}
di ciascuna Tecnica Speciale.

Ogni tecnica speciale \`e basata su una
delle Specifiche di AM o MAE (cio\`e PUGno, CALcio, Arco Elfico,
ecc.)  e sulla CONCentrazione. 

Le tecniche comportano una spesa in PM,
PF, PE. 

Le tecniche speciali possono essere imparate,
solo durante il gioco e devono essere insegnate da maestri
specializzati o comunque da tutti coloro (PG o PNG) che hanno in
quella particolare tecnica almeno un TOT di 12 (come spiegato
nell'abilit\`a Insegnare). 

Nell'Arcipelago esistono sei principali Accademie di Arti Marziali,
dove vengono insegnate, oltre alle Abilit\`a di combattimento, alle
Specifiche e alle Maestrie, le corrispondenti Tecniche Speciali (vedi
pagina \pageref{accademiemarziali}, ``Accademie di Arti Marziali'')


\nb{Non \`e possibile imparare una Tecnica Speciale se non si hanno
le corrispondenti specifiche di AM o MAE (tranne dove esplicitamente
indicato)}  

Le Tecniche Speciali non sono correlate a nessuna
caratteristica, pertanto il TOT della tecnica speciale non include il
bonus di caratteristica, ma \`e uguale al VAL. nonostante ci\`o il
Tiro incremento usufruisce del Bonus relativo alla Caratteristica
correlata alla Maestria o alla Tecnica speciale cui fanno riferimento.

Le Tecniche Speciali hanno effetto solo per il round in cui vengono
utilizzate, a meno che non sia indicato diversamente. 

Affinch\'e la tecnica abbia successo \`e sufficiente realizzare un
TCONC a difficolt\`a 20 cui seguir\`a, se richiesto dalla tecnica, un
TPC. Se il TCONC ha successo vengono spesi i PM, PF, PE richiesti
dalla tecnica, anche qualora il TPC non abbia esito positivo.

Pi\`u tecniche possono essere portate contemporaneamente (es. Carica +
Danno Massimo + Finta) realizzando 1 solo TPC (a cui vanno sottratti
\textbf{tutti} i malus) ma un TCONC separato per ciascuna tecnica.
Saranno eseguite correttamente solo le tecniche per le quali il TCONC
ha avuto successo. Naturalmente, se il TPC non riesce,
\textbf{nessuna} delle tecniche speciali ha effetto.

\subsubsection{Tecniche speciali del Pugno} 

Possono essere imparate da tutti coloro che abbiano almeno
un TOT di 1 nella specifica Pugno.

\newcommand{\mae}[6] {
\item{\bf #1 $\Diamond$ D#2 --} 
  #3 
  \begin{itemize}
    \itemsep -3pt 
  \item \makebox[3em][l]{\bf PF:}#4
  \item\makebox[3em][l]{\bf PM:}#5
  \item \makebox[3em][l]{\bf PE:}#6
  \end{itemize}
  }

\begin{description}
  
  \mae{Pugno Sfondante}{1}{Permette di sferrare un colpo atto a
    infrangere materia \textbf{inerte} per un totale di punti
    struttura pari al \textbf{Danno inflitto x Bonus} della Tecnica.
    Necessita di TPC} {Bonus della Tecnica}{1}{Bonus della Tecnica}
  
  \es{Kurasay Hidenaga ha trovato uno scrigno di legno chiuso con un
    lucchetto che non riesce a scassinare.  Preso dall'ira decide di
    sfondare la cassa. Kurasay ha 16 in CONC, 14 nella specifica di AM
    PUGno (bonus +3) e 14 nella Tecnica, pari ad un bonus di +3.
    Effettua il TCONC e totalizza 23, il suo pugno va a bersaglio ed
    infligge cos\`{\i} 3 (1d4)+5 (BON FOR)+ 3 (Bonus della
    Specifica Pugno) x 3 (bonus tecnica) = 33. Il Master stabilisce
    che la cassa ha 20 PS; poich\'e il danno \`e superiore, la cassa
    va in frantumi. Kurasay perde 3 PF, 1 PM e 3 PE.}
  
  \mae{Pugno a distanza}{3}{Permette di colpire un determinato
    bersaglio alla distanza di un numero di metri pari al TOT della
    Tecnica. Si pu\`o infliggere il normale danno o le altre mosse
    speciali tipiche del Pugno (utilizzandole nelle modalit\`a
    descritte).  Necessita di TPC.}  {Bonus della tecnica}{4}{1 per
    metro}

  \es{Masamune Nakamura viene
    sbeffeggiato a causa della sua appariscente armatura da un avventore
    di una taverna. Costui \`e seduto ad un tavolo che dista circa 5
    metri da Masamune. Egli decide di colpirlo con un pugno a distanza,
    Masamune ha un TOT di 9 nella tecnica; potrebbe perci\`o colpire un
    bersaglio distante fino a 9 metri. Masamune ha 15 in CONC e ottiene 12
    col dado, totalizzando 27 al TCONC. Masamune effettua il suo attacco e
    riesce a colpire il bersaglio. Masamune perde 2 PF (pari al bonus di 9
    TOT tecnica), 4 PM e 5 PE}

  \mae{Cecit\`a}{2}{Quando si colpisce l'avversario alle tempie
  (realizzando un Tiro Mirato) con questa tecnica, lo si render\`a
  completamente cieco. La cecit\`a perdura per un numero di giorni
  pari al TOT della Tecnica.  Non infligge altri danni}{1}{1}{2
  $\times$ ogni giorno di cecit\`a.}
  
  \es{Mufasa intende accecare il suo avversario prima che inizi la
    caccia al tesoro e deve perci\`o colpirlo alle tempie. Mufasa ha
    13 in CONC ed effettua il TCONC totalizzando 21. Effettua poi un
    attacco mirato alle tempie riuscendo a colpire il suo avversario.
    Mufasa ha un TOT di 6 nella tecnica perci\`o il suo avversario
    rester\`a cieco per 6 giorni.  Mufasa perde 1 PF, 1 PM e 12 PE}
  
  \mae{Inibitore}{2}{Questa tecnica
    causa all'avversario colpito un handicap di tipo fisico.  Il bersaglio
    deve essere colpito alla nuca (realizzando un Tiro Mirato). Con questa
    tecnica si possono dimezzare i punteggi di FOR e AGI per un numero di
    round pari al TOT della tecnica.  Questa tecnica non infligge
    altri danni.}{6}{6}{12}
  
  \es{Olaf vuole limitare i movimenti
    del suo nemico applicando questa tecnica, nella quale ha un TOT di 6.
    Olaf ha 9 in CONC, effettua il TCONC e totalizza 20. Tenta poi di
    colpire il nemico alla nuca. Il suo attacco mirato va a bersaglio. Il
    suo avversario subir\`a un dimezzamento di FOR e AGI per 6 round.
    Olaf perde: 6 PF, 6 PM e 12 PE 6}
  
  \mae{Esplosivo}{3}{Questa tecnica \`e utilizzabile solo sugli arti
    (realizzando un Tiro Mirato) e produce il danno del pugno sotto
    forma di una piccola esplosione.  Ci\`o comporta l'ulteriore
    perdita di un numero di PSC nella parte interessata e conseguenti PV
    ogni round pari al Bonus della Tecnica (da trattare come un danno
    Addizionale per emorragia.)}{15}{3}{15}
  
  \es{Masamune intende utilizzare questa tecnica contro l'orco con cui
    sta duellando. Masamune ha un TOT di 14 nella specifica di AM
    Pugno (Bonus +3) ed un TOT di 16 nella tecnica (con un relativo
    bonus di +4). Effettua il TCONC totalizzando 22. Il successivo
    attacco mirato al braccio destro ha successo. Masamune infligge 1
    (1d4) + 5 (BON FOR ) + 3 (Bonus Specifica Pugno) = 9 PSC/PV.
    L'orco perder\`a inoltre 4 (Bonus Tecnica) PSC/PV per round.
    Masamune perde 12 PF (3 per 4 round), 3 PM e 4 (PE).}
  
  \mae{Danno Ritardato}{1}{Questa mossa permette di ritardare
    l'effetto della tecnica portata di un numero massimo di ore pari
    al TOT della tecnica. \`E pi\`u utile se utilizzata in
    combinazione con altre tecniche speciali}{2}{2}{4}
  
  \es{Joseph vuole infliggere dei danni al suo avversario ma desidera
    che subisca tali danni soltanto successivamente. Egli ha un TOT di
    9 nella tecnica. Il TCONC ha avuto successo cos\`{\i} come il
    successivo attacco. Joseph potrebbe ritardare il danno fino a 9
    ore, ma decide che la vittima subir\`a tale danno dopo 3 round.
    Joseph perde 2 PF, 1 PM e 4 PE }
\end{description}
  
\subsubsection{Tecniche speciali del Calcio} Possono
essere imparate da tutti coloro che abbiano almeno un TOT di 1 nella
specifica Calcio

\pinupbig{ascia.eps}{Ascia bastarda}{tb}

\begin{description}
  \mae{Calcio sfondante}{1}{Il funzionamento \`e analogo a quello
    della tecnica pugno sfondante}{Bonus della Tecnica}{1}{Bonus della
    Tecnica}
  
  \mae{Calcio rotante}{3}{Permette di colpire tutti gli avversari
    intorno al PG fino alla distanza di 50 cm + 5 cm ogni punto
    Tecnica.  Il TPC \`e penalizzato di -5}{6}{1}{Bonus della
    Tecnica}
  
  \es{Takashi \`e accerchiato da 3 lupi bianchi e decide di tentare
    un calcio rotante.  Takashi ha un TOT di 12 nella tecnica, 20 in
    CAC, 9 nella specifica Calcio e non indossa armatura. Pu\`o
    quindi colpire avversari attorno a lui sino a 110 cm di distanza
    (50 cm +5 cm x 12). I lupi si trovano a circa un metro Takashi
    realizza il TCONC. Il suo TPC \`e pari a 30: 20 (CAC) + 2 (Bonus
    specifica Calcio) + 13 (1d20) - 5 (malus al TPC della tecnica).
    Solo uno dei lupi riesce a schivare mentre gli altri due subiranno
    il danno del calcio. Takashi perde 6 PF, 1 PM e 2 PE (perch\'e il
    bonus della tecnica per un TOT di 9 \`e uguale a 2)}
  
  \mae{Calcio a distanza}{3}{Il funzionamento \`e analogo a quello
    della tecnica Pugno a Distanza}{Bonus della tecnica}{4}{1 punto
    per metro}
\end{description}

\subsubsection{Tecniche speciali della Presa} 
Possono essere imparate da tutti coloro che abbiano almeno un
TOT di 1 nella specifica Presa. 

\begin{description}
  \mae{Spacca ossa}{2}{Con tale
    mossa si colpiscono soltanto le ossa dell'avversario mirando a
    provocare delle fratture.  Al danno della presa va sommato anche il
    Bonus della Tecnica. La vittima effettua un TR difficolt\`a 20 +
    Bonus della tecnica. Se il TR ha successo subir\`a solo il danno
    della presa. Necessita di TPC.}{Bonus della tecnica}{1}{Bonus
    della tecnica}

  \es{Martin vuole spezzare il braccio al suo avversario
    con l'utilizzo di questa tecnica. Egli ha un TOT di 13 nella tecnica
    ed un relativo Bonus di +3. Il suo TCONC ha successo cos\`{\i} come il
    successivo attacco.  Egli ha la possibilit\`a di sommare al danno
    della presa il Bonus di + 3 se l'avversario non realizza il TR, in
    questo caso a difficolt\`a 23. l'avversario effettua il TR e
    totalizza 21, subendo oltre al danno normale anche i 3 punti
    aggiuntivi. Martin perde 3 PF, 1 PM e 3 PE} 
  
  \mae{Stordente}{2}{Tale presa esercitata al capo dell'avversario (realizzando un Tiro
    Mirato) per almeno 2 round, lo porta allo svenimento per mancata
    ossigenazione del cervello. Il Tiro Resistenza della vittima avr\`a
    difficolt\`a 20 + Bonus della Tecnica.  Per ogni round successivo ai
    due richiesti in cui si tiene la presa, la difficolt\`a del TR
    aumenter\`a di +1} {2}{2}{Bonus Tecnica + numero di round}
  
  \es{ Martin intende far svenire il suo nemico con questa tecnica. Egli
    ha un TOT di 17 e quindi un Bonus di +4. Realizza il suo TCONC. Il
    successivo attacco mirato va a segno e Martin blocca l'avversario e
    riesce a tenere la presa per i successivi 2 round. l'avversario deve
    realizzare un TR a difficolt\`a 24 o svenire.  Tira e totalizza
    25, perci\`o non perde i sensi. Martin tuttavia continua a tenere
    la presa per altri 3 round. l'avversario dovr\`a cos\`{\i}
    realizzare un TR a difficolt\`a 27 o svenire. Tira e totalizza 26
    perdendo i sensi.  Martin perde 2 PF, 2 PM e 7 PE}
  
  \mae{Strappacarne}{3}{Si colpisce direttamente la muscolatura
    dell'avversario strappando i muscoli dalla loro sede (se si
    vuole indirizzare l'attacco verso un particolare muscolo
    occorrer\`a un Tiro Mirato).
    
    Infligge il danno da Presa e implica un Tiro Resistenza a
    difficolt\`a 20 + Bonus della tecnica.
    
    Se il TR fallisce il muscolo viene divelto e la vittima perde un
    numero ulteriore di PSC per round nella parte interessata (e
    relativi PV) per dissanguamento pari al Bonus della tecnica. Tali
    danni vanno trattati come Danni Addizionali da Emorragia. 

    La strappacarne non pu\`o essere effettuata su parti del corpo
    coperte da armature rigide (piastre, bande, legno, osso).

    Necessita di TPC} { 8 }{ 4 }{ 5 + Bonus tecnica}
  
  \es{Takashi vuole strappare il bicipite al suo avversario. Egli ha un
    TOT di 15 e quindi un relativo bonus di +4 nella tecnica. Takashi
    realizza il TCONC ed il suo successivo attacco mirato ha successo.
    l'avversario deve realizzare un TR a difficolt\`a 24 per non
    subire oltre al danno da presa l'ulteriore danno di +4 per round.
    Takashi perde 8 PF, 4 PM e 9 PE}
  
  \mae{Paralizzante}{3}{Tale presa paralizza l'avversario
    impedendogli qualsiasi movimento. I suoi effetti perdurano per un
    numero di minuti pari al TOT della tecnica dopo che la tecnica \`e
    stata realizzata. La vittima effettua un TR a difficolt\`a 20 + il
    bonus della tecnica. Se il TR ha successo eviter\`a gli effetti
    della tecnica. Necessita di TPC.} {8}{6}{10} 
  
  \es{Tolip vuole
    paralizzare una pecora per tosarla con facilit\`a. Egli ha un TOT di
    19 nella tecnica ed un relativo bonus di +4. Realizza il TCONC e con
    il successivo attacco blocca la pecora. Questa dovr\`a realizzare un
    TR pari a 24 per evitare di restare paralizzata per 19 minuti. Tolip
    perde 8 PF, 6 PM e 10 PE}
  
\end{description}

\subsubsection{Tecniche speciali di Proiezione}
Possono essere imparate da tutti coloro che abbiano almeno un TOT di 1
nella specifica Proiezione. 

\begin{description}
  \mae{Lancio}{2}{Proietta l'avversario per un numero di metri pari al
  Bonus della Tecnica. Se l'avversario \`e fermo infligge, oltre al
  danno normale, 1d4 PV/PSC pi\'u un numero di PV/PSC pari bonus della
  tecnica. Se l'avversario \`e in Carica (vedi ``Tecniche Speciali
  delle Armi'') si somma al danno anche la sua FORza.  Necessita di
  TPC.}{6}{3}{3 + Bonus Tecnica}
  
  \es{Thalas il Gigante intende proiettare lontano il suo avversario.
    Egli ha un TOT di 20 nella tecnica e quindi un bonus di + 5.
    Thalas realizza il TCONC, il successivo attacco ha successo. La
    vittima subir\`a oltre al danno da proiezione, 3 (1d4+1) + 5
    (bonus tecnica) = 8 PSC/PV di danno. Thalas perde 6 PF, 3 PM e 8
    PE}
  
  \mae{Proiezione a sfera}{3}{Proietta tutti coloro che distano
    dall'attaccante 50 cm + 5cm per ogni punto Tecnica infliggendo ad
    ognuno il Danno normale. Il TPC \`e penalizzato di -5. Vedi
    l'esempio relativo alla tecnica Calcio Rotante.}{6}{1}{Bonus
    tecnica}
\end{description}

\subsubsection{Tecniche speciali di Schivata} 
Possono essere imparate da tutti coloro che abbiano almeno un
TOT di 1 nella specifica Schivata. 
\begin{description}
  \mae{Schivare Frecce}{1}{Permette di schivare frecce, proiettili e
    simili. Il Bonus al TPD schivata di questi ultimi \`e pari al
    TOT della Tecnica} {1}{5}{5 + Bonus tecnica}
  
  \es{Kurasay ha davanti a s\'e un elfo, armato di arco, che ha deciso di
    attaccarlo.  Kurasay ha 20 in CAC, 15 nella specifica di CAC
    Schivata (Bonus di + 4) e 12 nella Tecnica speciale Schivare
    Frecce (Bonus di + 3). Il suo avversario ha totalizzato al 25 al
    TPC, pertanto la difficolt\`a di schivare la freccia \`e pari
    a 45. Kurasay realizza il TCONC e prova a schivare la freccia
    scagliata dall'elfo. Lancia il dado ed ottiene 10. Il suo tiro per
    schivare (TPD), pertanto sar\`a pari a: 20 (CAC) + 4 (bonus
    della specifica Schivata) + 12 (TOT della tecnica Schivare Frecce)
    + 10 (risultato del d20) = 46. Kurasay riesce a schivare la
    freccia e perde 1 PF, 5 PM e 8 PE.}
\end{description}


\subsubsection{Tecniche Speciali di Parata}  Possono essere imparate da tutti coloro che abbiano almeno
un TOT di 1 nella specifica di AM Parata.

\begin{description}
  \mae{Parare frecce}{1}{Permette di parare frecce, proiettili e simili
    quando non si utilizza l'arma. Il Bonus al TPD Parata di questi
    ultimi \`e uguale al TOT della Tecnica. Vedi esempio relativo a
    Schivare Frecce.} {1}{5}{5 + Bonus Tecnica}
\end{description}

\subsubsection{Tecniche speciali delle Armi} 
Sono delle Tecniche Speciali riferite alle Armi.  Ogni Tecnica deve
essere riferita in particolare ad \textbf{una} delle armi di cui si
\`e Maestri.

Potete imparare Tecniche speciali con lo stesso nome riferite ad armi
diverse. Per esempio potete apprendere una Tecnica Carica riferita
alla Spada Bastarda o una Tecnica Carica riferita all'ascia Bipenne,
oppure conoscere due Tecniche Carica, riferite una allo Spadone a due
Mani, l'altra alla Sciabola Elfica.

\begin{description}
  \mae{Parare frecce}{1}{Permette di parare frecce, proiettili e simili
    quando si utilizza l'arma in cui si ha la Maestria. Il Bonus al
    TPD parata contro i proiettili \`e uguale al TOT della Tecnica.}
  {1}{5}{5 + Bonus Tecnica}
  
  \es{Caius ha un TOT di 15 in ATL, un TOT di 6 nella Maestria in
    Spada Bastarda (Bonus +2) ed un TOT di 15 (Bonus +4) nella Tecnica
    Parare Frecce riferita alla Spada Bastarda. Veste un'armatura di
    piastre.  Il suo avversario gli ha scagliato contro una freccia ed
    il suo TPC era pari a 30 per cui il TPD di Caius dovr\`a essere
    almeno pari a 50. Caius realizza il TCONC. Lancia il dado per il
    TPD e ottiene 15.  Il suo TPD (per Parare la freccia con la
    sua Spada Bastarda) \`e pari a: 15 (risultato di 1d20) + 15
    (ATL) + 4 (Bonus Specifica Spada Bastarda) + 6 (Bonus Parata della
    Spada Bastarda) - 5 (Malus Armatura) + 15 (TOT della Tecnica
    Parare frecce) = 50. Riesce a parare e perde 1 PF, 5 PM e 9 PE}
  
  \mae{Attacco circolare}{3}{Colpisce \textbf{tutti} i bersagli che si
    trovano ad una distanza dall'attaccante pari (o minore) alla
    lunghezza dell'arma + 5 cm per ogni punto Tecnica, infliggendo ad
    ognuno il Danno normale.  Il TPC \`e penalizzato di -5. Questa
    tecnica non pu\`o riferirsi alle ADL, ADT, ART e ADF. Vedi esempio
    relativo a Calcio Rotante.}{6}{1}{Bonus tecnica}
  
\mae{Lanciare arma}{2}{Si pu\`o lanciare la propria arma, anche se
    non \`e un'arma da lancio, con la gittata di un pugnale. Il malus
    al TPC \`e pari a -20 +TOT della Tecnica. \textbf{Al danno non
      viene sommato il BONus FOR dell'attaccante}. Non pu\`o riferirsi
    alle ADL, alle ADF, alle ADT e all'ART.}{3}{2}{3}

  \es{L'avversario di Loth si prepara a
    scagliare una freccia e Loth sa che \`e troppo
    lontano perch\'e possa raggiungerlo prima che
    l'avversario attacchi. Decide perci\`o di lanciare
    la propria spada. Loth ha un TOT di 12 nella
    Maestria in Spada Bastarda (bonus +3) ed ha un TOT
    di 17 nella Tecnica, pertanto il suo malus al TPC
    \`e pari a 20 + 17(TOT della tecnica) = -3. Loth
    realizza il TCONC ed effettua il TPC, penalizzato di
    -3, che va a segno. Loth lancia 2d8+2 per infliggere
    e ottiene 6, infliggendo 6 + 3 (bonus maestria in
    Spada Bastarda) = 9 PSC/PV di danno al suo
    avversario. Loth perde 2 PF, 3 PM e 3 PE.}

  \mae{Carica}{1}{Dopo una corsa di minimo 3 metri si colpisce
    l'avversario ponendo sul colpo tutta la propria spinta. Si
    infligge il danno normale dell'arma + la FOR dell'attaccante. Se
    si sta cavalcando si aggiunger\`a, al danno normale, la FOR
    della cavalcatura. Il Malus al TPC \`e -5 + il bonus relativo
    alla Tecnica. Se la carica viene schivata si ha un malus al tiro
    iniziativa per la successiva azione di -5 + il bonus relativo alla
    tecnica. Questa tecnica non pu\`o essere riferita alle ADL, alle
    ADT, alle ADF, e all'ART.} {8}{2} {Bonus della Tecnica}

  \es{William Wicker tenta di caricare
    il suo avversario che lo ha fatto cadere dal cavallo.
    Egli ha un TOT di 11 nella tecnica con un Bonus di +3, ed
    ha un TOT di 15 nella maestria in Spada Bastarda (bonus
    +4).  Il suo TCONC ha successo.
    Wicker prende la rincorsa
    ed effettua il TPC con un malus di - 5 + (+3 Bonus della
    Tecnica) = -2.  Colpisce l'avversario, che non \`e
    riuscito a schivare. William stava impugnando una spada
    bastarda ed ha 12 in FORza. Infligger\`a cos\`{\i}: 2d8
    + 2 (spada bastarda a 2 mani) + 4 (Bonus Maestria in Spada
    Bastarda) + 12 (FOR) per un totale di 29 PSC/PV.  Se
    l'avversario avesse schivato il suo attacco, la sua
    successiva iniziativa sarebbe stata penalizzata di -5 +
    (+3 bonus della Tecnica) = -2. In ogni caso William perde
    8 PF, 2 PM e 3 PE.} 
  
  
  
  \mae{Lanciare/Tirare bendato}{2}{Permette di lanciare alla cieca, al
    buio o bendati qualsiasi ADL o di scagliare qualsiasi proiettile o
    simile con un'ADT. Al malus di -20 per lanciare/tirare al buio
    (come da tabella ``Modifiche al combattimento'') si somma come
    bonus il TOT della tecnica.}  {0}{10}{3 + Bonus tecnica}
  
  \es{Alice sta gareggiando con l'arco in una fiera di paese e, da
    bendato, deve colpire un bersaglio. Ha un TOT di 19 nella tecnica
    (bonus di +4). Il suo malus al TPC \`e perci\`o pari a: -20
    (malus per tirare al buio) + 19 (TOT nella tecnica) = -1. Il suo
    TCONC ha successo ed il suo TPC sar\`a penalizzato di -1. Alice
    perde 0 PF, 10 PM e 7 PE.}
  
  \mae{Disarmare}{1}{Permette di disarmare l'avversario. Si effettua
    il normale TPC indirizzato sull'arma dell'avversario e si somma il
    Bonus della Tecnica. Se il TPC ha esito positivo (ossia se
    l'avversario non riesce a parare o a schivare) la vittima perde
    l'arma.  La tecnica non infligge alcun danno}{2}{3}{2 + Bonus
    Tecnica.}
  
  \es{Takashi vuole disarmare il suo avversario; ha un TOT di 15 nella
    Tecnica con un bonus di +4.  Il suo TCONC ha successo.  Takashi
    effettua il suo TPC sull'arma dell'avversario sommando al normale
    TPC anche il Bonus della Tecnica, cio\`e +4. l'avversario non
    riesce a schivare e perde l'arma.  Takashi invece perde 2 PF, 3 PM
    e 6 PE.}
\end{description}

\pinup{sciabola_elfica_2.eps}{Sciabola Elfica}

\subsection{Altre Tecniche Speciali} 

Oltre alle tecniche gi\`a descritte vi sono poi delle altre tecniche
speciali che hanno qualche differenza di funzionamento rispetto alle
tecniche speciali gi\`a descritte.

\subsubsection{Danno Massimo}

La tecnica Danno Massimo non \`e riferita n\`e a nessuna Specifica
di Arte Marziale, n\`e a nessuna Maestria.

Deve essere il giocatore a specificare per quale delle specifiche di
Arte Marziale o per quale arma di Maestria vuole impararla.  

Pu\`o per esempio imparare la Tecnica Danno Massimo riferendola al
Pugno, oppure impararla per il Calcio, oppure imparare due Tecniche di
Danno Massimo e riferirle una al Pugno ed una alla Spada Bastarda,
ecc.

\begin{description}
  \mae{Danno Massimo}{1}{Questa tecnica infligge il massimo del danno
    possibile con un certo colpo.  Si applica un malus di -5 al TPC
    pi\`u il bonus (dato dalla tabella Specializzazioni) della
    Tecnica. \`E possibile avere una tecnica di Danno Massimo per ogni
    specifica di CAC e per ogni MAE} { 3 }{ 3 }{ 10}
  
  \es{
    Titius, che non indossa armatura, ha 18 in FORza (BON +3),
    un TOT di15 in CAC, 10 in Pugno (bonus +3), e conosce la
    tecnica Danno Massimo per il Pugno con un TOT di 13. Decide
    di utilizzarla ed effettua il TCONC, che riesce. Il suo TPC
    \`e pari a 1d20+1 (dado per il TPC del Pugno) +15 (CAC) +3
    (Bonus Specifica PUGno) -5 (malus Danno Massimo) + 3 (Bonus
    Tecnica) = 1d20+17. Se il colpo va a segno il danno inflitto
    \`e pari al massimo valore che si pu\`o ottenere con
    1d4+1 + 3 (BONus FOR) + 3 (bonus al Danno per il PUGno)
    quindi 5+3+3=11 PSC/PV. Titius perde 3 PF, 3 PM e 10 PE.
    }
\end{description}

\subsubsection{Tecniche non riferite a nessuna specifica di Arte
  Marziale o Maestria}

Le altre tecniche particolari di cui vi parliamo sono le Tecniche
Speciali che non vanno riferite a nessuna Specifica di Arte Marziale o
Maestria.  Queste infatti possono essere apprese anche da coloro che
non conoscono le Specifiche di AM o che hanno un TOT di 0 nelle
abilit\`a di Combattimento.  Nel paragrafo relativo ad ognuna di
esse \`e indicata la caratteristica che conferisce loro il Bonus per
il Tiro Incremento
   
\begin{description}

  \mae{Pelle d'acciaio}{1}{COS--- Tale tecnica di contrazione muscolare
    permette di attutire i colpi subiti per un valore pari al bonus
    della Tecnica. La tecnica dura solamente un Round} {1}{3}{Bonus
    tecnica}
  
  \es{Caius ha un TOT di Pelle d'acciaio pari a 10 (Bonus +3, quindi
    pu\`o sottrarre al danno subito 3 punti).  Subisce un colpo al
    torace che gli infliggerebbe, superando l'armatura, 5 Punti di
    Danno.  Caius decide di tentare di usare la Pelle d'acciaio. Il
    suo TCONC riesce, la Tecnica Speciale gli consente quindi di
    subire 2 punti di danno anzich\'e 5.  Caius perde 1 PF, 3 PM e 3
    PE.}
  
  \mae{Meditazione}{1}{CONC--- Tale abilit\`a permette di aumentare
    temporaneamente, per il periodo della meditazione, l'abilit\`a
    Allarme di un valore pari al bonus della tecnica, riposando ma
    restando vigili.  Si recuperano 2 PF, 2 PM e 2 PE all'ora, per un
    numero massimo di ore giornaliere pari al TOT della tecnica.
    Durante la meditazione non si possono utilizzare altre abilit\`a
    se non Allarme. Qualsiasi altra azione interrompe la
    meditazione.}{1}{4}{10 + Bonus Tecnica}

  \es{Masamune ha un TOT di 12 nella tecnica e +3 di bonus e decide di
    meditare anzich\'e dormire perch\'e \`e da solo e, siccome non ha
    nessuno che possa fare la guardia, con la meditazione pu\`o
    restare all'erta. Il suo TCONC ha esito positivo.  Egli
    usufruir\`a di un bonus di +3 all'abilit\`a Allarme per tutto
    il tempo della meditazione. Perde, per l'utilizzo della Tecnica
    Speciale: 1 PF, 4 PM e 13 PE. Ma, poich\'e medita per 12 ore,
    recupera 24 PF, 24 PM e 24 PE.}
  
  \mae{Azioni Multiple}{1}{AGI--- Conferisce un Bonus alle azioni
    multiple pari al Bonus della Tecnica.}{1}{1}{Bonus della Tecnica.}
  
  \es{ Masamune vuole effettuare due attacchi in un solo round. Egli
    ha un TOT di 10 nella Tecnica che corrisponde al bonus di +3. Il
    malus normale per le 2 azioni \`e pari a -5. Masamune realizza
    il TCONC.  Il suo malus ad ogni azione diventa -5 (malus standard)
    + 3 (bonus della tecnica) = -2.  Masamune perde 1 PF, 1 PM e 3
    PE.}

  
  \mae{Volont\`a di Ferro}{1}{CONC--- Conferisce un Bonus al TV pari
    al Bonus della Tecnica. La tecnica ha effetto per un solo round.}
  {1}{5}{Bonus della Tecnica.}
  
  \es{Masamune ha 13 in questa tecnica, con relativo bonus di +3.
    Realizza il TCONC.  Al successivo TV usufruir\`a del bonus di
    +3.  Masamune perde 1 PM, 5 PM e 3 PE.}
  
  \mae{Finta}{1}{AGI--- Chi utilizza questa tecnica conferisce un malus
    al TPD dell'avversario pari al bonus della Tecnica secondo la
    tabella Specializzazioni.}  {3}{3}{Bonus della tecnica.}
  
  \es{Takashi, in uno scontro vuole disorientare il suo avversario.
    Egli ha 9 nella tecnica con un bonus di +2.  Il suo TCONC ha esito
    positivo.  Al TPD contro l'attacco di Takashi, il suo avversario
    sar\`a penalizzato di un malus di -2.  Takashi perde 2 PF, 3 PM e
    2 PE.}

\end{description}

\subsubsection{Nuove tecniche speciali}

Possono essere create dal Master altre Tecniche Speciali.
Per la loro regolamentazione il Master pu\`o agire per
analogia con le Tecniche Speciali esistenti.

\subsection{Accademie di Arti Marziali}
\label{accademiemarziali} 

Le Accademie di Arti Marziali sono delle scuole nelle quali si
perfezionano le proprie conoscenze nell'arte del Combattimento, sia a
mani nude che con le armi.  Ognuna di queste accademie \`e
``specializzata'' in determinate specifiche e nelle relative Tecniche
Speciali.  Le sei principali accademie dell'Arcipelago sono:

\subsubsection{Accademia del Tempio di Otomo}  Specializzata in: Presa,
Proiezione, Divincolarsi, Schivata, Caduta e nell'utilizzo di ATL e
AIA e nelle relative Tecniche Speciali.  Il Sommo Maestro \`e Kazuto
Urushimaru.  La Sede di questa Accademia \`e il Monte Tama nelle
Terre Orientali. 

\subsubsection{Accademia di Sakamoto}  
Specializzata in Pugno, Calcio, Parata, Schivata e nell'utilizzo di
ATL, ADT, AIA e ABC e nelle relative Tecniche Speciali.  Il Sommo
Maestro \`e Shinji Kobayashi.  La sede di questa Accademia \`e
nella citt\`a di Sokata, nelle Terre Orientali.

\subsubsection{Accademia di Kella}
Specializzata in Proiezione, Schivata e Caduta e nell'utilizzo di ABL,
ATC, ADT, ADL e nelle relative Tecniche Speciali.  Il Sommo Maestro
\`e Roydan Grenfeld.  La sede di questa Accademia \`e nei pressi
della citt\`a di Kella sul ``Suolo'' di Umbrosa.

\subsubsection{Accademia di Gowr-nhen}  
Specializzata in Pugno, Parata, Schivata, ABL, ABC, SCD, ATL, AIA e
nelle relative Tecniche Speciali.  Il Sommo Maestro \`e Browndo
Chapel.  La sede di questa Accademia \`e nella citt\`a di Mowark a
Gennoria.

\subsubsection{Accademia di Thyesi} Specializzata in Calcio, Pugno,
Schivata, Parata e nell'utilizzo di SCD, ABC, ATL e nelle relative
Tecniche Speciali.  Il Sommo Maestro \`e Kevin Nysh.  La Sede di
questa accademia \`e nella foresta di Thyesi, nelle Terre Levanti.

\subsubsection{Accademia di Norda}  
Specializzata in Presa, Divincolarsi, Proiezione, Schivata, Caduta e
Pugno e nelle relative Tecniche Speciali.  Il Sommo Maestro \`e
Gowen McKinney.  La sede di questa accademia \`e nella citt\`a di
Norda, nelle Terre del Nord.

Solitamente ogni Sommo Maestro \`e originario della Terra in cui
l'accademia \`e dislocata.

I Sommi Maestri scelgono direttamente il loro successore.  Alle
accademie \`e ammesso chiunque superi la difficile prova d'ammissione.
Le Accademie non fanno richiesta di alcuna retta per la frequentazione
dei corsi, ma accettano donazioni e possono assegnare ai loro adepti
mansioni non retribuite.

\nb{Tutte le accademie insegnano le Tecniche speciali non riferite a
  nessuna specifica di AM o Maestria}

{\raggedright \section{Esempio di combattimento}}

Adesso che sappiamo come funzionano le Abilit\`a di Combattimento
possiamo descrivere dettagliatamente cosa \`e successo durante il
combattimento tra Takashi e William Wicker descritto all'inizio del
Capitolo.

\es{Takashi Horuru \`e un signore
delle terre orientali, ha 20 in AGI e 20 in FOR, ma indossa
un'armatura composita di cuoio e piastre che gli d\`a un malus di
-4 alle azioni in movimento e all'iniziativa.  In compenso ha un TOT
di 20 in Armi da Taglio Lunghe (ATL), un TOT di 20 in Corpo a Corpo
(CAC), un TOT di 17 nella specifica di CAC Schivata, un TOT di 17
nella specifica di CAC Proiezione ed un TOT di 20 in acrobazia.

William Wicker \`e un allevatore di Terranova, ha 20 in AGI, 13 in
COS e 13 in FOR, ma indossa un'armatura di cuoio che gli d\`a un
malus di -2 alle azioni in movimento e all'iniziativa.  Ha un TOT di
16 in ATL e di un TOT di 7 nella Maestria in spada bastarda (bonus
+2).  Il combattimento si svolge in un solo round.  Al tiro iniziativa
Wicker totalizza un 24 dato da: 20 (AGI) + 11 (1d20) - 2 (armatura) -
5 (penalit\`a della spada bastarda).  Takashi realizza 23: 20 (AGI)
+ 11 (risultato di 1d20) -4 (armatura) - 4 (penalit\`a della
katana).  

Wicker vince l'iniziativa e decide di attaccare totalizzando
al TPC 23: 16 (ATL) + 7 (risultato di 1d20) + 2 (MAEstria in spada
bastarda) - 2 (armatura).  

Takashi compie due azioni in un solo round per sorprendere
l'avversario.  In questo caso egli subisce un malus di -5 per ciascuna
delle due azioni.  Come prima azione decide di schivare l'attacco
ottenendo un punteggio al suo TPD schivare di 30: 20 (CAC) + 4
(Specifica di AM Schivata) + 5 (bonus dato dall'abilit\`a Acrobazia
a 20) + 10 (1d20) - 5 (malus per 2 azioni) - 4 (armatura di piastre).
Riesce cos\`{\i} ad evitare il colpo.  A questo punto ha a
disposizione la seconda azione che potr\`a compiere senza che
l'avversario abbia la possibilit\`a di reagire.

Sceglie di attaccare effettuando una Proiezione e come TPC
ottiene 25 dato da: 20 (CAC) + 4 (Specifica di AM proiezione) + 10
(1d20) - 4 (armatura) -5 (malus per due azioni).  l'attacco riesce e
l'avversario viene catapultato al suolo subendo un danno da botta di 2
(1d4+1) + 5 (BON FOR) + 4 (Bonus della Specifica PROiezione) = 11
PSC/PV.  
Il Master determina che Wicker cade sul braccio destro.
Poich\'e Wicker indossa un'armatura di cuoio, il danno netto che
subisce al braccio \`e 11 - 3 (PP Armatura di cuoio) = 8 PV/PSC.  Il
Master, data l'entit\`a del danno (8 PSC su 10 del Braccio per COS
13) decide che la Botta ha provocato una Lussazione e che Wicker deve
realizzare un TR a difficolt\`a 25 per evitare di rimanere stordito.
Wicker, che ha 13 in RES, tira il Dado e ottiene 3 totalizzando 16.

L'orientale pu\`o ora facilmente fiondarsi sull'avversario e
convincerlo ad arrendersi.  \`E da notare come Takashi abbia lanciato
l'iniziativa con il Malus per la sua Katana ed abbia comunque
effettuato l'attacco a mani nude.
}

\fi
%%% Local Variables: 
%%% mode: latex
%%% TeX-master: "manual"
%%% End: