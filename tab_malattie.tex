{\setlength{\tabcolsep}{0.25em}
\centering

{\Large\sc Malattie}
\label{tabmalattie}

\footnotesize
\begin{longtable}{|p{2.1cm}|l|p{0.8cm}|p{0.8cm}|l|p{2.5cm}|p{4cm}|p{2.2cm}|}
  \par
  \hline
  Nome & \raggedright Durata & \raggedright TR Effetti & \raggedright TR Morte & \raggedright DIFF & \raggedright Cause & \raggedright Effetti & \raggedright Cura\tabularnewline \hline\hline
  \endfirsthead
  \hline
  Nome & \raggedright Durata & \raggedright TR Effetti & \raggedright TR Morte & \raggedright DIFF & \raggedright Cause & \raggedright Effetti & \raggedright Cura\tabularnewline \hline\hline
  \endhead
  \raggedright Anchilostomiasi & \raggedright Cronica & \raggedright 20 & \raggedright - & \raggedright 25 & \raggedright Parassita che entra attraverso la pelle da origine infetta.  & \raggedright Emorragie polmonari, Anemia, Letargia Generale (-5 INT, CONC, AGI, COS, FOR) & \raggedright Infuso di Erbe Vermifughe\tabularnewline \hline
  \raggedright Colera & \raggedright 1 Mese & \raggedright 25 & \raggedright 25 & \raggedright 25 & \raggedright Scarse condizioni igieniche. Aumenta la possibilit\`a di infettarsi dopo calamit\`a naturali & \raggedright Vomito, Sudori freddi, crampi muscolari, diarrea, febbre (-5 ogni caratteristica) & \raggedright Reintegrazione dei liquidi persi. Infusi, impacchi caldi.\tabularnewline \hline
  \raggedright Denutrizione & \raggedright Cronica & \raggedright 20 & \raggedright 15 & \raggedright 10 & \raggedright Dieta non equilibrata & \raggedright Scorbuto, Beri Beri, Rachitismo, Pellagra, con sintomi differenti. -3 alle caratteristiche & \raggedright Dieta equilibrata\tabularnewline \hline
  \raggedright Dissenteria & \raggedright Cronica & \raggedright 25 & \raggedright 23 & \raggedright 28 & \raggedright Contatto con Acqua inquinata, escrementi di individui infetti, mosche & \raggedright Febbre alta, Sudorazione abbondante con disidratazione, feci liquide e sanguinolente (-10 COS, -5 FOR) & \raggedright Infusi e molti liquidi \tabularnewline \hline
  \raggedright Lebbra & \raggedright 10 mesi & \raggedright 12 & \raggedright 35 & \raggedright 35 & \raggedright Contatto con individui infetti & \raggedright Lesioni gravi alla pelle e ai nervi, anestesia e perdita di colore della pelle, poi piaghe purulente, cecit\`a, mutilazione spontanea di estremit\`a (vedi fantoccio); -2 a FOR, AGI, BEL, COS ogni mese. Quando una di queste \`e 0, serve un TR contro morte & \raggedright Non esiste cura. Le attenzioni del medico danno +3 al TR contro morte; il decorso pu\`o essere fermato.\tabularnewline \hline
  \raggedright Malaria & \raggedright 1d100 mesi & \raggedright 25 & \raggedright 20 & \raggedright 23 & \raggedright Puntura della Zanzara anofele femmina in zone paludose & \raggedright Febbri ricorrenti, Brividi di Freddo accompagnati da sudorazione e forti tremori (-1 tutte le caratteristiche) & \raggedright Chinino (si ricava dalla corteccia della China)\tabularnewline \hline
  \raggedright Malattie Veneree & \raggedright Croniche & \raggedright 35 & \raggedright 15 & \raggedright 30 & \raggedright Rapporti sessuali con individui infetti & \raggedright Infiammazioni locali, pustole, febbre. Alcune malattie portano alla demenza progressiva. (-5 alle azioni in movimento, -1 int ogni 6 mesi) & \raggedright Medicazione locale con unguenti appositi\tabularnewline \hline
  \raggedright Male Freddo & \raggedright 1 settimana & \raggedright 35 & \raggedright In seguito    al danno & \raggedright 25 & \raggedright Solo REUBEN. Contatto della pelle di un Reuben con l'acqua per oltre 1 minuto, ingestione di acqua & \raggedright Febbre bassa, comparsa di 1d6 pustole gonfie d'acqua sulla zona colpita, o in 1d4 di zone in caso di ingestione. Se le pustole sono toccate si lacerano determinando 1 PF; -1 alla CONC per il prurito & \raggedright Asciugatura, svuotamento delle pustole; dieta di cibi secchi per una quantit\`a pari all'acqua ingerita\tabularnewline \hline
  \raggedright Mussiadura & \raggedright 1d6 giorni & \raggedright 30 & \raggedright 20 & \raggedright 35 & \raggedright Morso di Ragno & \raggedright Febbre, Brividi, Vomito, dolore alle giunture, macchie sulla pelle. Se il morso avviene sul viso si ha -1d6 in bellezza. Se non curato in tempo pu\`o costringere all'amputazione dell'arto colpito & \raggedright Erbe, impacchi, pomate. Incisione per far defluire il nero\tabularnewline \hline
  \raggedright Peste & \raggedright 2 settimane & \raggedright 35 & \raggedright 35 & \raggedright 35 & \raggedright Pulci e cimici dei topi & \raggedright Bubboni, febbre elevata, pelle arrossata, sonnolenza, delirio; emorragia che porta alla morte (.-1 ad ogni caratteristica al giorno) & \raggedright Pitzy-Adoree\tabularnewline \hline
  \raggedright Poliomielite & \raggedright Cronica & \raggedright 25 & \raggedright - & \raggedright 28 & \raggedright Acqua Inquinata & \raggedright Paralisi dell'arto colpito (fantoccio per sapere quale) & \raggedright Impacchi caldi sui muscoli (+5 ai TR)\tabularnewline \hline
  \raggedright Febbri (raffreddori, influenze, ecc.) & \raggedright 1d20 gg & \raggedright 35 & \raggedright - & \raggedright 25 & \raggedright Freddo, contatto con persone malate & \raggedright -1d6 IN CONC, FOR e COS,  INT & \raggedright Infusi, erbe, ambiente riscaldato.\tabularnewline \hline
  \caption{Malattie}{Tabella Malattie}\tabularnewline
\end{longtable}
}

{\setlength{\tabcolsep}{0.25em}
\centering

}

%%% Local Variables: 
%%% mode: latex
%%% TeX-master: "manual"
%%% End: 
