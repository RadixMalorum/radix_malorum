\chapter{Concludendo...}

\newenvironment{faq}{\begin{description}}{\end{description}}
\newcommand{\Question}[1]{\medskip \bf\item{D:} #1\par}
\newcommand{\Answer}[1]{\it \item{R:} #1}

{\raggedright \section{Suggerimenti per il Master}}

Per ricompensarvi della pazienza che avete dimostrato di avere nel
leggere il regolamento ed offrirvi volontari come Master, vi
offriamo una piccola raccolta di suggerimenti, affinch\'e
possiate avere un aiuto nel gestire al meglio le situazioni in cui si
troveranno i PG.

I suggerimenti non sono altro che le risposte alle domande pi\`u
comuni poste dai playtesters a cui abbiamo chiesto di provare Radix
Malorum, per verificarne la giocabilit\`a, il realismo, la chiarezza
e l'innovazione.

Dobbiamo ammettere che la maggior parte di essi si \`e dimostrata
competente e attenta a ci\`o che abbiamo loro mostrato.

Vogliamo perci\`o ringraziare questi amici che hanno dedicato il loro
tempo e la loro passione per il gioco di ruolo al nostro Radix
Malorum, che speriamo possa essere, in futuro, fonte di numerose
occasioni di divertimento.

\vspace{1cm}

\begin{faq}
  \Question{Come posso utilizzare il mondo di Radix Malorum per
    ambientarci le mie avventure?}
  
  \Answer{Il contenuto dei primi due capitoli \`e necessario al
    Master per poter introdurre i propri giocatori nel mondo di Radix
    Malorum. Leggete attentamente questa parte del manuale poich\'e in
    essa sono rinvenibili molti spunti per creare delle avventure
    nelle terre conosciute dell'Arcipelago.
    
    \`E sufficiente avere un po' di fantasia e di buona volont\`a.
    d'altronde il compito del Buon Master \`e quello di creare delle
    storie e muoverne i fili per far divertire sia i giocatori che se
    stesso! Ecco alcuni esempi su come impostare delle avventure
    basandosi sugli spunti rintracciabili nell'ambientazione.

  
    \es{I PG vengono invitati da un facoltoso elfo umbra a partecipare
      ad una battuta di caccia al Krenay a nord di Kella. Essi
      potrebbero anche imbattersi casualmente in un avamposto di
      briganti orchi costruito clandestinamente in territorio umbra e
      di conseguenza trovarsi costretti a combatterli}
    
    \es{I PG vengono assoldati da un ricco mercante d'arte che offre
      loro una lauta ricompensa in cambio di un antico cimelio
      custutodito nella tomba di un famoso eroe nano, sepolto nella
      catena del Monteferro.}
    
    \es{I PG vengono assoldati come mercenari da un possessore chire
      per proteggere un carico destinato ad Umbrosa. Sfortunatamente
      la nave si imbatte nei temibili pirati di Kowa, presso le terre
      orientali.}  }
  
  \Question{Come posso gestire un gruppo di PG difficilmente
    ``integrabili'' tra loro?}  \Answer{Cercate di essere flessibili
    nell'applicare le regole di creazione del personaggio, poich\'e
    la casualit\`a portata dai dadi potrebbe generare
    personaggi assolutamente ingiocabili: potete apportare delle
    piccole ed arbitrarie variazioni al risultato del lancio dei dadi,
    ma senza esagerare.
    
    Premete poi perch\'e i giocatori costruiscano, magari
    in collaborazione tra loro, il \textbf{background} dei propri PG, per
    semplificare la caratterizzazione dei personaggi, il conseguente
    divertimento e facilitare l'integrazione dei PG nel gruppo.
    Ricordate che l'interpretazione \`e il fulcro del Gioco di
    Ruolo, per cui cercate di non esagerare con le avventure del tipo
    ``Vai, Ammazza il Mostro, Arraffa il Tesoro e Fuggi''.
  
    Bench\'e il procedimento di creazione del PG
    possa richiedere del tempo, ricordate che dovrete affrontarlo una
    tantum, cio\`e solo all'inizio dell'avventura o nel caso di
    morte del personaggio}
  
  \Question{Quali sono le unit\`a di misura utilizzate in Radix?}
  
  \Answer{Le unit\`a di misura da prendere in considerazione sono,
    per ovvia comodit\`a dei giocatori, quelle utilizzate
    nell'occidente non anglossassone e cio\`e: il Metro, il
    Chilogrammo, il Litro ecc.
    
    Se ritenete che non siano abbastanza ``coreografiche'' potete
    inventarne delle altre, piedi (30 cm), braccia (90 cm), leghe (1.1
    km), navi (30 m), giganti (3.5 m), e adattarle alla vostra
    avventura}
  
  \Question{I generi descritti nel manuale possono avere prezzi
    diversi a seconda della situazione?}
  
  \Answer{I prezzi elencati nel manuale sono indicativi e quindi
    suscettibili di variazioni a causa delle pi\`u svariate
    circostanze, quali perigliosit\`a delle vie commerciali, periodi
    di carestia e via dicendo (a discrezione del Master), oppure a
    seconda della qualit\`a delle merci: nel Manuale sono indicati i
    prezzi di oggetti di media qualit\`a.
    
    Aumentate arbitrariamente il prezzo all'aumentare della
    qualit\`a.  }
  
  \Question{Come \`e possibile che esista un solo sistema monetario
    in un sistema complesso come l'Arcipelago?}  

  \Answer{ Bench\'e
    ogni nazione abbia una propria valuta monetaria interna, la
    necessit\`a di facilitare i commerci ha persuaso le autorita
    politiche ad adottare una valuta commerciale comune, accettata
    solitamente in tutte le terre (ovviamente il Master pu\`o far
    scontrare i PG con le difficolt\`a di operare in una nazione che
    abbia improvvisamente negato corso legale alla valuta
    commerciale).
    
    Inoltre, per i giocatori \`e pi\`u semplice gestire un solo
    sistema monetario (decimale!) che provare a districarsi tra
    valute, tassi di cambio e provvigioni!}
  
  \Question{Come si pu\`o rendere pi\`u rapido il combattimento?}
  
  \Answer{Il capitolo relativo al combattimento ha creato maggiori
    difficolt\`a ai nostri playtester.
    
    Per semplificarvi la vita potete seguire alcune regole:

\begin{itemize}    
    \item Date tempo ai PG, ma anche a voi Master, per impratichirvi delle
    numerose regole che servono (almeno speriamo!) per rendere
    realistici gli scontri, introducendo poche regole per volta;
  
    \item Accrescete la difficolt\`a dei combattimenti in maniera
    graduale, e prima di intraprendere la vostra prima avventura
    simulate dei combattimenti tra i PG dei vostri giocatori e alcuni
    PNG che potete creare alla bisogna;
    
  \item Fate calcolare le perdite di PV/PSC, dei PS e dei PP alla fine
    dell'azione di ogni PG dal giocatore che lo impersona, mentre voi
    Master gestite il combattimento con un altro personaggio.
  
    \item Fate calcolare ai giocatori in anticipo e una volta per tutte i
    punteggi di base Iniziativa, TPC e TPD in funzione delle armi e
    delle armature che usano di solito, in modo da non doverle
    ricalcolare ad ogni azione.
\end{itemize}
    
    Vi accorgerete, cos\`{\i} facendo, che in men che non si dica la
    velocit\`a dei combattimenti (realistici!)  risulter\`a
    pi\`u che accettabile e tale da non dover compromettere il
    divertimento con tediosi tempi morti}
  
  \Question{Come si pu\`o rendere pi\`u rapida la creazione degli
    incantesimi?}
  
  \Answer{L'aver voluto ideare un sistema di creazione immediata degli
    incantesimi, che rendesse il mago libero dagli schemi delle magie
    prefissate da memorizzare, dimenticare e rimemorizzare, ha
    introdotto un piccolo aumento della complessit\`a di alcune
    regole.
    
    Il rallentamento nella costruzione di un incantesimo ad hoc \`e
    tuttavia superabile con un po' di pratica ed esercizio.
    
    Vedrete che dopo qualche tempo di inevitabile rallentamento per
    districarsi fra le Tabelle di Creazione degli incantesimi, i PG
    diverranno rapidissimi. Anche in questo caso vale quanto detto per
    il combattimento:

\begin{itemize}    
    \item Consigliate ai vostri giocatori di creare l'incantesimo di cui
    abbisognano mentre voi gestite una situazione con un altro PG in
    modo da ridurre le perdite di tempo e lo stesso dicasi per la
    sottrazione dei PM, PF e PE.
    
    \item Fate segnare sulla scheda del PG i parametri degli incantesimi che
    vengono lanciati pi\`u frequentemente, in modo tale che ogni
    giocatore abbia il suo ``Grimorio'' personalizzato da consultare e
    modificare rapidamente.
    
    \item Fate utilizzare ai giocatori pi\`u pigri, che non vogliono
    cimentarsi nella creazione degli incantesimi personalizzati, il
    compendio dei pi\`u noti incantesimi preconfezionati delle varie
    Scuole e Liste che trovate nel manuale al capitolo \ref{scuole}}
\end{itemize}  

  \Question{In che categoria funzionale, fra Attacco e Alterazione, si
    colloca un incantesimo che arreca danno spostando oggetti?}
  
  \Question{Un incantesimo che scagli un macigno contro un avversario
    \`e di Danno mirato (Attacco) o di Spostamento in 3 dimensioni
    (Alterazione)?}
    
  \Answer{Questo problema \`e risolvibile tenendo d'occhio la
    finalit\`a effettiva dell'incantesimo.
    
    Con uno \textbf{Spostamento in 3 dimensioni} si
    sposter\`a, ad una data velocit\`a, un macigno che dovr\`a
    essere scagliato per colpire il bersaglio utilizzando una
    abilit\`a di combattimento del PG (in questo caso AO) e che
    infligger\`a un danno pari ad un numero di dadi stabilito
    arbitrariamente dal Master, in relazione alla grandezza e forma
    del masso, pi\`u il BON FOR dell'incantesimo.
    
    Con un \textbf{Danno mirato} il masso sar\`a
    semplicemente l'effetto coreografico, l'immagine, il modo di
    manifestarsi dello stesso, che a prescindere dalla sua forma
    infligger\`a un danno calcolato con il lancio di un certo numero
    di d6 (calcolati nella DB), indipendenti dalla velocita o dalla
    grandezza del macigno.
  
    In generale, verificate quale \`e il risultato dell'incantesimo,
    spogliandolo di tutta la ``coreografia'' che pu\`o mascherarne
    l'effetto reale}

\end{faq}

{\raggedright \section{La regola del Buon Senso}}

Il consiglio forse pi\`u importante che ci azzardiamo a dare ad ogni
Master (e non solo per Radix Malorum) \`e quello di usare il proprio
buon senso per adattare le regole ad una situazione non contemplata
nel manuale. La libert\`a con cui si muovono i PG potrebbe, inoltre,
far deviare l'avventura dai binari che il Master si \`e prefissato.
Questi perci\`o dovr\`a far fronte a situazioni imprevedlbili
sfruttando abilit\`a, fantasia, esperienza e \textbf{buon senso}.
Ricordate che lo scopo principale del gioco \`e il divertimento,
perci\`o non ingabbiate i PG in trame che riducano eccessivamente il
loro arbitrio, e concedete loro delle pause quando lo ritenete
necessario affinch\'e l'avventura non diventi pesante e noiosa.
\vfill
\clearpage

{\raggedright \section{Glossario delle Abbreviazioni}}

\begin{description}
\itemsep -4pt
\item{\bf ABC} Armi da Botta Corte 
\item{\bf ABL} Armi da Botta Lunghe 
\item{\bf AdC} Abilit\`a di Combattimento
\item{\bf ADF} Armi Da Fuoco 
\item{\bf ADL} Armi Da Lancio 
\item{\bf ADT} Armi Da Tiro 
\item{\bf AGI} Agilit\`a 
\item{\bf AIA} Armi In Asta 
\item{\bf AM} Arte Marziale
\item{\bf AO} Armi Occasionali 
\item{\bf ART} Artiglieria 
\item{\bf ATC} Armi da Taglio Corte 
\item{\bf ATL} Armi da Taglio Lunghe 
\item{\bf BEL} Bellezza 
\item{\bf BON} Bonus 
\item{\bf BL} Bonus Libro 
\item{\bf BP} Bonus Parata 
\item{\bf BR} Bonus Ricerca 
\item{\bf BU} Bonus Utilizzo 
\item{\bf C} Corone 
\item{\bf CAC} Corpo A Corpo 
\item{\bf CAR} Carisma 
\item{\bf CAT} Caratteristica
\item{\bf CON} Conoscenza 
\item{\bf CONC, CNC} Concentrazione 
\item{\bf COS} Costituzione 
\item{\bf DA} Difficolt\`a Assoluta
\item{\bf DB} Difficolt\`a di Base 
\item{\bf DIFF} Difficolt\`a 
\item{\bf DR} Difficolt\`a Relativa
\item{\bf DT} Difficolt\`a di Tipo 
\item{\bf FI} Fattore Istruzione 
\item{\bf FOR} Forza 
\item{\bf INT} Intelligenza
\item{\bf K} Kulos 
\item{\bf MAE} Maestria 
\item{\bf OSS} Osservazione 
\item{\bf P} Piastre 
\item{\bf PA} Penalit\`a Arma 
\item{\bf PE} Punti Energia 
\item{\bf PF} Punti Fisico 
\item{\bf PG} Personaggio Giocante 
\item{\bf PK} Punti Karma 
\item{\bf PM} Punti Mente 
\item{\bf PNG} Personaggio Non Giocante 
\item{\bf PP} Punti Protezione 
\item{\bf PS} Punti Struttura
\item{\bf PSA} Punti Struttura Arma 
\item{\bf PSC} Punti Struttura Corpo 
\item{\bf PSI} Psiche 
\item{\bf PV} Punti Vita
\item{\bf RES} Resistenza 
\item{\bf S} Scudi 
\item{\bf SCD} Scudo 
\item{\bf TCI} Tipo di Colpo Inferto 
\item{\bf TCONC} Tiro Concentrazione
\item{\bf TDA} Tipologia di Danno 
\item{\bf TINT} Tiro Intelligenza 
\item{\bf TL} Tempo di Lancio 
\item{\bf TOSS} Tiro Osservazione 
\item{\bf TP} Tiro Psiche 
\item{\bf TPC} Tiro Per Colpire 
\item{\bf TPD} Tiro Per Difendersi
\item{\bf TR} Tiro Resistenza 
\item{\bf TV} Tiro Volont\`a 
\item{\bf VOL} Volont\`a
\end{description}
%%% Local Variables: 
%%% mode: latex
%%% TeX-master: "manual"
%%% End: 
