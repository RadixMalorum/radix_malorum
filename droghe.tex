{\sloppypar\raggedright \section{Erbe medicinali, veleni, malattie e droghe}}

Anche questi fattori rivestono notevole importanza ai fini del gioco.
I veleni rendono preoccupanti degli apparentemente innocui animaletti,
le erbe hanno molteplici utilizzi fra i quali quello medicinale e
curativo. Per quanto riguarda le malattie, che dire? Non penserete
forse di essere inattaccabili da germi, virus, batteri e cose di
questo tipo, vero? 

Per veleni, erbe medicinali e malattie consultate le relative tabelle.

Per la preparazione di Veleni e per l'utilizzazione delle piante
medicinali bisogna far riferimento all'abilit\`a Alchimia.

\subsection{Erbe medicinali} 

Nella Tabella a pagina \pageref{taberbe} ``Erbe Medicinali'' (Erbe
medicinali e ingredienti per veleni e droghe) troverete indicati:

\pinupbig{mortaio_big.eps}{}{thb}

\begin{description}
\item{\textbf{Nome}} dell'erba medicinale,
\item{\textbf{Clima}} e \textbf{Habitat} in cui pu\`o essere
  reperita,
\item{\textbf{Diff}} ossia, la difficolt\`a a cui dovrete effettuare
  il tiro sull'abilit\`a Conoscere Piante per trovarla,
\item{\textbf{Parte}} della pianta da utilizzare,
\item{\textbf{Uso}} e cio\`e la modalit\`a di somministrazione,
\item{\textbf{Costo}} qualora la si volesse comprare da un erborista
\item{\textbf{Effetto}} tipico della pianta
\end{description}



\subsection{Veleni}

Nella Tabella ``Veleni'' a pagina \pageref{tabveleni} troverete indicati: 

\begin{description}
\item{  \textbf{Nome}} del veleno,
\item{  \textbf{Origine}} ossia da quale fonte lo si pu\`o ottenere,
\item{ \textbf{Uso}} cio\`e il metodi di preparazione e le modalit\`a di
  somministrazione,
\item{\textbf{Diff}} che indica la difficolt\`a della sua preparazione avendo
  i componenti (per trovarli riferirsi alle abilit\`a
  corrispondenti),
\item{\textbf{Costo}} ossia il prezzo a cui pu\`o essere acquistato,
\item{\textbf{TR morte}} che indica la difficolt\`a del TR che il PG
  dovr\`a realizzare per evitare di morire nel tempo indicato fra
  parentesi,
\item{\textbf{TR coma}} che indica la difficolt\`a del TR che il PG
  dovr\`a realizzare per evitare di entrare in coma per il tempo
  indicato fra parentesi,
\item{\textbf{TR danni}} che indica la difficolt\`a del TR che il PG
  dovr\`a realizzare per evitare il prodursi degli effetti indicati
  fra parentesi.
\end{description}

Tutti i TR sono riferiti ad una dose.  Ogni dose aggiuntiva
aumenter\`a la Diff. del TR di 2.  

\nb{il TR da effettuare \`e sempre e solo uno. Il TR va confrontato
  con le difficolt\`a dei TR morte, coma e danni. Si subiranno
  soltanto gli effetti dei TR non realizzati. Se si otterr\`a un
  risultato superiore alla difficolt\`a di tutti i Tiri indicati non
  si subir\`a alcun effetto}

\es{Moris ha appena bevuto una tisana in cui sua moglie aveva versato
  della cicuta. Come da Tabella Veleni il TR Morte \`e pari a
  28, il TR Coma \`e pari a 30 ed il TR Danni \`e pari a 32. 
  
  Moris ha 18 in RES e 9 nell'abilit\`a Resistenza ai Veleni che gli
  conferisce un bonus di +2 al TR contro i veleni.
  
  Moris tira il dado e ottiene 10, totalizzando 30. Riesce cos\`{\i} 
  ad evitare sia la morte che il coma, ma non i Danni del veleno
  descritti nella tabella. Se avesse totalizzato, invece, almeno 32
  avrebbe evitato anche i Danni}

\subsection{Malattie} Nella Tabella Malattie a pagina \pageref{tabmalattie}
(Malattie) troverete indicati: 
\begin{description}

\item{\textbf{Nome}} della malattia,
\item{\textbf{Durata}} della malattia,
\item{\textbf{TR effetti}} che indica la difficolt\`a del TR per
  evitare di ammalarsi,
\item{\textbf{TR morte}} che indica la difficolt\`a del TR per
  evitare la morte in quelle malattie che la prevedono (esso va
  effettuato al termine della Durata),
\item{\textbf{Diff}} che indica la difficolt\`a del tiro Medicina
  per curarla,
\item{\textbf{Origine}} che indica le fonti e le modalit\`a di
  trasmissione,
\item{\textbf{Effetti}} che indica gli effetti della malattia e
\item{\textbf{Cura}} che indica il tipo di cure mediche necessarie per
  guarire
\end{description}

\subsection{Gravidanze} 

Possono rimanere gravide le femmine di tutte le razze se non usano
metodi anticoncezionali.

Il ventre gonfio impedisce di indossare abiti o armature se questi non
vengono prodotti su misura. L'AGI subisce un malus di -5.

La gravidanza pu\`o terminare naturalmente dopo un periodo di
gestazione tipico di ciascuna razza o artificialmente in seguito a
incidenti o infusi che provocano l'aborto.

Al momento del parto dovr\`a essere realizzato un TR. Se il risultato
\`e minore di 13 si morir\`a di parto; ottenendo un risultato minore
di 15 si avranno complicazioni, che potranno essere evitate se ad
assistere la partoriente \`e un medico che realizza un Tiro Medicina a
difficolt\`a 20.

\begin{description}
\item{\bf Umani} La probabilit\`a che una donna umana venga fecondata
  in seguito ad un rapporto \`e del 20\%. La durata della gestazione
  \`e 9 mesi.
\item{\bf Elfi} Gli elfi sono poco fecondi. La probabilit\`a che una
  femmina di razza elfica venga fecondata \`e del 2\%. La durata della
  gestazione \`e 15 mesi.
\item{\bf Nani} La probabilit\`a di fecondazione \`e del 2\%, quindi
  molto scarsa. La durata della gestazione \`e 7 mesi.
\item{\bf Gnomi} Gli gnomi sono molto prolifici. La probabilit\`a che
  una femmina venga fecondata \`e del 35\%. La mortalit\`a infantile
  \`e alta, cos\`{\i} come la frequenza dei parti plurigemellari. La
  durata della gestazione \`e 5 mesi.
\item{\bf Giganti} La probabilit\`a che una femmina venga fecondata
  \`e del 2\%. La durata della gestazione \`e 20 mesi.
\item{\bf Reuben} I Reuben producono dopo 5 mesi di gestazione un uovo
  che deve essere covato per i 2 mesi successivi. La probabilit\`a di
  fecondazione \`e del 5\%, ma entrambi i partner possono restare
  gravidi (i Reuben sono ermafroditi).
\end{description}

\subsection{Droghe}
\label{droghe}
  
Su Quadrantal, l'uso delle droghe \`e molto diffuso; esse, come i
veleni e le erbe medicinali, richiedono una preparazione particolare.

La difficolt\`a di preparazione della droga \`e determinata dagli
effetti che deve produrre.

Ogni droga ha, di norma, degli
effetti collaterali che si manifestano una volta esauriti gli effetti
principali. Vediamo ora quali sono gli Effetti Principali che una
droga pu\`o produrre, ed il loro Valore di difficolt\`a (VEP).

\begin{center}

\begin{tabular}{|c|p{4 cm}|}
\hline
  VEP& Effetti Principali \\ \hline\hline
  6& \raggedright Per ogni punto di incremento o decremento di una CAT
  fisica: OSS, AGI, COS, FOR. \tabularnewline \hline
  4& \raggedright Per ogni punto di incremento o decremento della
  CONC. \tabularnewline \hline
  10& \raggedright Per ogni punto di incremento o decremento dell'INT. \tabularnewline \hline
  2& \raggedright Per ogni punto di incremento o decremento di RES, VOL, PSI. \tabularnewline \hline
  4& \raggedright Per ogni ora o frazione di durata dell'incremento o del decremento. \tabularnewline \hline
  +1& \raggedright Per ogni +1 di bonus al TR contro gli Effetti Principali. \tabularnewline \hline
  -1& \raggedright Per ogni -1 di malus al TR contro gli Effetti Principali. \tabularnewline \hline
\end{tabular}
\end{center}

Per ottenere la difficolt\`a totale bisogna sommare fra loro i VEP dei
vari effetti principali.

\es{Ci\`o significa che se si desidera creare una droga che dia +3
  in FOR e +2 in CONC, per 2 ore si devono sommare: 18 per aumentare
  di 3 punti la FOR (6+6+6) 8 per aumentare di 2 punti la CONC (4+4) 8
  per perdurare gli effetti per 2 ore (4+4) La VEP totale sar\`a quindi pari
  a (18+8+8)=34}

Gli effetti collaterali hanno un valore variabile a seconda della loro
durata in giorni. La somma dei valori dei suddetti effetti collaterali
(VEC) deve essere \textbf{sottratta} dal VEP totale.

Tuttavia tale somma non pu\`o essere superiore alla met\`a dei VEP
totali.

Ecco gli effetti collaterali, il loro Codice ed il loro VEC.
\begin{center}
  \begin{tabular}{|c|c|p{4cm}|}
    \hline
    COD& VEC& Effetto Collaterale \\ \hline\hline
    A& 1& \raggedright Per ogni giorno di fobie e allucinazioni che causano $-1d10$ PSI \tabularnewline \hline
    B& 3& \raggedright Per ogni giorno di spossatezza che comporta $-1d6$ FOR e $-1d4$ AGI.  \tabularnewline \hline
    C& 1& \raggedright Per ogni giorno di emicrania che comporta $-1d6$ CONC  \tabularnewline \hline
    D& 2& \raggedright Per ogni giorno di diarrea, che comporta $-1$ COS, $-1$ FOR, $-1d4$ CONC  \tabularnewline \hline
    E& 3& \raggedright Per ogni giorno di difficolt\`a respiratorie che causano $-1d4$ COS, $-1d6$ CONC  \tabularnewline \hline
    F& 2& \raggedright Per ogni giorno di problemi cardiaci che comportano $-1d6$ COS \tabularnewline \hline
    G& 1& \raggedright Per ogni giorno di tremiti che comportano $-1d4$ AGI  \tabularnewline \hline
    H& 3& \raggedright Per ogni giorno di crisi epilettiche che
    comportano -1d10 AGI, -1d10 CONC  \tabularnewline \hline
  \end{tabular}
\end{center}

\es{Continuando l'esempio precedente: Potremmo scegliere una
  combinazione di effetti collaterali, la somma dei cui VEC sia
  massimo 17(dato che il VEP \`e uguale a 34).

Cos\`{\i} potremmo avere per esempio: B per
3 giorni (3 di VEC per numero di giorni =9), pi\`u F per 3 giorni (2 di VEC
per numero di giorni =6), pi\`u A per 2 giorni (1 di VEC per numero di giorni=2).
La somma dei VEC ottenuti (9+6+2), \`e in questo modo uguale a 17, ma avreste
potuto ottenere questo valore con altre combinazioni}

Gli effetti collaterali si manifestano, per il numero di giorni
previsto, contemporaneamente. I relativi malus sono cumulativi.

La somma dei VEP meno la somma dei VEC permette di ottenere la
difficolt\`a del TR che il PG deve realizzare per evitare gli effetti
(principali e collaterali) della droga. Per produrre una dose di droga
occorre realizzare un tiro Alchimia con difficolt\`a pari al TR avendo
a disposizione tutti gli ingredienti necessari.  Potrebbe essere
quindi necessario acquistare erbe o estratti, o cercare erbe e animali
adatti usando altre abilit\`a.

Si pu\`o aumentare o diminuire ulteriormente la difficolt\`a del
TR fino ad un massimo di 10 punti.

\pinupp{rajina.eps}{Ampolla di Rajina}{t}

Il TR non va effettuato nel caso di assunzione volontaria della droga.
Ogni TR \`e riferito ad una dose singola, la somministrazione di pi\`u
dosi aumenta di 2 la Diff. del TR per ogni dose aggiuntiva.

Qualora il TR non venga realizzato, si subiranno gli effetti
principali della droga e gli effetti collaterali.

Quando il TR del PG \`e inferiore, di 7 o pi\`u, rispetto alla Diff.
del TR, il PG diverr\`a dipendente e sar\`a costretto ad usare tale
droga settimanalmente. Per ogni settimana di astinenza egli avr\`a -1
a tutti i tiri. Gli effetti della dipendenza possono essere annullati
mediante cure e la difficolt\`a del relativo Tiro nell'abilit\`a
Medicina sar\`a pari al TR della droga.

In caso di assunzione di pi\`u dosi contemporaneamente, se il TR del
PG \`e inferiore di 10 o pi\`u, rispetto alla Diff. del TR, il PG
muore per overdose in 1d20 minuti (per le cure, l'overdose va
considerata come il coma).

Se si assumono pi\`u di 10 dosi di una droga in un lasso di tempo
inferiore ai 20 giorni, a prescindere dalla riuscita o meno dei TR, il
PG avr\`a assuefazione agli effetti principali della droga, che
diminuiranno di 1 punto per ogni ulteriore assunzione, fino a svanire
del tutto, mentre gli effetti collaterali si manifesteranno
ugualmente. 

Il PG perder\`a l'assuefazione smettendo di assumere droga per almeno
30 giorni.

Col sistema descritto, ed ovviamente con l'uso dell'abilit\`a Alchimia
potete creare qualsiasi droga, ma per comodit\`a indicheremo nella
Tabella \ref{tabdroghe} (Droghe) quelle pi\`u comuni nell'Arcipelago.
In questa tabella troverete indicati:

\begin{description}
\item{\textbf{Nome}}
\item{\textbf{Effetti principali}}
\item{\textbf{VEP}}
\item{\textbf{Effetti collaterali}}, indicati con i codici A-H
\item{\textbf{VEC}}
\item{\textbf{TR}} che indica la difficolt\`a a cui deve essere realizzato
  il Tiro in Alchimia e la Diff. del TR
\item{\textbf{Costo}}, cio\`e il prezzo, per dose, della droga gi\`a pronta
  (esso, solitamente \`e uguale al VEP totale in scudi)
\item{\textbf{Origine}}, che indica se la droga \`e ricavata
  principalmente da un vegetale o da un animale
\end{description}

\onecolumn
{
\setlength{\tabcolsep}{0.25em}
\centering

{\Large\sc Erbe Medicinali}
\label{taberbe}

\footnotesize
\begin{longtable}{*{7}{|l}|p{5cm}|}
  \par
  \hline
  Nome&Clima&Habitat&Diff&Parte&Uso&Costo&Effetto\tabularnewline \hline\hline
  \endfirsthead
  \hline
  Nome&Clima&Habitat&Diff&Parte&Uso&Costo&Effetto\tabularnewline \hline\hline
  \endhead

  Alhow&Temperato&Latifoglie&10&Linfa&Applicazione&8K&\raggedright Allontana tutti gli insetti stordendoli\tabularnewline \hline
  Soire&Freddo&Conifere&10&Foglia&Applicazione&4S&\raggedright Dimezza il tempo di guarigione di fratture, lussazioni, distorsioni, ecc.\tabularnewline \hline
  Jongh&Freddo&Fiumi&40&Muschio&Infuso&6S&\raggedright Permette di ricomporre ossa e articolazioni polverizzate. Cura 2d6 di fratture in 10+1d10 ore \tabularnewline \hline
  Now-Dha&Temperato&Brughiera&15&Foglia&Ingestione&3S&\raggedright Diminuisce 1d6 giorni il tempo di guarigione di fratture, lussazioni, distorsioni, ecc.\tabularnewline \hline
  Fah-Solow&Temperato&Coste&30&Legume&Ingestione&15S&\raggedright Ripara 2d6 di fratture in 2d6 ore \tabularnewline \hline
  Origa&Temperato&Montagne&20&Stelo&Applicazione&9K&\raggedright Mitiga i dolori da ustione diminuendo di 1 il malus alla CONC.\tabularnewline \hline
  Ogow&Temperato&Coste&10&Foglia&Applicazione&5S&\raggedright Dimezza il tempo di guarigione di un' ustione\tabularnewline \hline
  Limba&Temperato&Latifoglie&20&Tubero&Applicazione&9S&\raggedright Cura 2d10 PV da ustione al ritmo di 1 PV ogni 30 min.\tabularnewline \hline
  Kare-gantsiow&Temperato&Coste&20&Fiore&Applicazione&6S&\raggedright Guarisce in 1d6*10 minuti una ferita. Occorre rimanere immobili finch\'e il tempo non sar\`a trascorso o la ferita potr\`a riaprirsi (70\%)\tabularnewline \hline
  Arheiga&Qualunque&Mari&10&Tubero&Infuso&4S&\raggedright Dimezza il tempo di guarigione delle ferite\tabularnewline \hline
  Cow-Pethonee&Temperato&Praterie&30&Foglia&Applicazione&1C&\raggedright Guarisce una ferita in 1d4 minuti facendo recuperare 1d10 PV (1 PV ogni 30 minuti)\tabularnewline \hline
  Aphiow&Temperato&Praterie&30&Radice&Applicazione&6C&\raggedright Guarisce in 1d4 minuti una ferita facendo recuperare 5d10 PV (2 PV ogni 30 minuti)\tabularnewline \hline
  Pre-Siow&Temperato&Latifoglie&30&Polpa&Ingestione&3C&\raggedright Guarisce 3d8 PV in 1d20 min.\tabularnewline \hline
  Browsha&Tropicale&Deserti&15&Seme&Infuso&9S&\raggedright Guarisce 1d10 PV in 3d10 min\tabularnewline \hline
  Merhida&Artico&Ghiacciai&40&Foglia&Ingestione&40C&\raggedright Guarisce in 1d10 min qualunque ferita facendo recuperare tutti i PV in 1d6 ore\tabularnewline \hline
  Fowndun-tsusow&Freddo&Montagne&20&Radice&Ingestione&2C&\raggedright Cura 3d6 PV in 1d4 X 10 min\tabularnewline \hline
  Arho-dow-gho&Tropicale&Sottosuolo&15&Petalo&Infuso&9S&\raggedright Cura emorragie in 1d6 ore recuperando 2d6 PV\tabularnewline \hline
  Tomata&Temperato&Coste&20&Polpa&Ingestione&25S&\raggedright Cura 2d10 PV in 3d10 min\tabularnewline \hline
  Arhoshow&Tropicale&Jungla&50&Bacca&Ingestione&N.A.&\raggedright Resurrezione se ingerita 6 giorni prima della morte\tabularnewline \hline
  Bowrha&Freddo&Montagne&40&Fiore&Infuso&45C&\raggedright Fa uscire dal coma in 1d12 giorni\tabularnewline \hline
  Umbrosa&Freddo&Montagne&45&Foglia&Infuso&N.A.&\raggedright D\`a il 20\% di probabilit\`a di resurrezione\tabularnewline \hline
  Zrow-Pa&Temperato&Conifere&40&Foglia&Applicazione&7C&\raggedright Fa recuperare la vista in 1d10 giorni\tabularnewline \hline
  Zerhiow&Temperato&Mari&30&Fiore&Applicazione&5C&\raggedright Fa recuperare l'udito in 1d8 giorni\tabularnewline \hline
  Cixiree&Temperato&Brughiera&15&Legume&Ingestione&3S&\raggedright Cura i calcoli, l'acidit\`a di stomaco, la tenia e l'anchilostomiasi\tabularnewline \hline
  Fah&Temperato&Praterie&15&Legume&Ingestione&7S&\raggedright Guarisce le malattie della pelle in 1d20 giorni\tabularnewline \hline
  Pisty-Naga&Tropicale&Vulcani&40&Tubero&Infuso&60C&\raggedright Guarisce qualsiasi organo in 1d10 ore\tabularnewline \hline
  Urdh&Artico&Deserti&15&Tubero&Infuso&3S&\raggedright Guarisce infezioni e malattie bronchiali\tabularnewline \hline
  Othigow&Temperato&Latifoglie&30&Corteccia&Ingestione&4S&\raggedright Bonus +3 al TR contro malattie per 2d6 giorni\tabularnewline \hline
  Ohpinow&Freddo&Conifere&20&Seme&Infuso&3S&\raggedright Cura febbre, delirio, allucinazioni\tabularnewline \hline
  Stimhyosa&Tropicale&Praterie&20&Radice&Applicazione&7S&\raggedright Cura in 1d20+10 giorni le malattie esantematiche\tabularnewline \hline
  Pitzy-Adoree&Tropicale&Brughiera&30&Foglia&Applicazione&25C&\raggedright Cura in 1d6/2 mesi la peste (se si realizza un TR con un bonus di +5; d\`a un forte prurito per 1d6 giorni)\tabularnewline \hline
  Spuntow&Temperato&Fiumi&30&Bacca&Ingestione&15C&\raggedright Cura in 1d20 giorni il colera\tabularnewline \hline
  Brulha&Freddo&Fiumi&40&Tubero&Ingestione&45C&\raggedright Cura in 1d20 giorni la lebbra (se si realizza un TR con un bonus di +5)\tabularnewline \hline
  Sowlimonee&Temperato&Latifoglie&15&Frutto&Infuso&4S&\raggedright Cura malattie intestinali, diarrea, ecc.\tabularnewline \hline
  China&Tropicale&Montagna&20&Corteccia&Infuso&10C&\raggedright Cura la malaria in 1d20 giorni\tabularnewline \hline
  Mutha&Tropicale&Jungla&30&Foglia&Pasta&2C&\raggedright Ci si ricava il curaro (Vedi trabella veleni)\tabularnewline \hline
  Cicuta&Temperato&Collina&15&Foglia&Liquido&6S&\raggedright Ci si ricava la cicuta (vedi tabella Veleni)\tabularnewline \hline
  Amhakhyaow&Temperato&Latifoglie&25&Radice&Liquido&12S&\raggedright Ci si ricava il Makhow  (vedi tabella veleni)\tabularnewline \hline
  Zinhybiree&Tropicale&Praterie&20&Corteccia&Polvere&3S&\raggedright Ci si ricava il Zurpow (vedi tabella veleni)\tabularnewline \hline
  Faow-lanchow&Freddo&Coste&30&Fiore&Liquido&3C&\raggedright Ci si ricava il Nofa-Ul-has  (vedi tabella veleni)\tabularnewline \hline
  Mowdregow&Freddo&Montagna&20&Bacca&Polvere&1S&\raggedright Ci si ricava il Cloroformio (vedi tabella veleni)\tabularnewline \hline
  Prisowchee&Temperato&Praterie&15&Fiore&Liquido&1C&\raggedright Ingrediente per Ryhan (vedi tabella Droghe)\tabularnewline \hline
  Sahpeera&Freddo&Conifere&20&Polpa&Liquido&3C&\raggedright Ingrediente per Zakhada (vedi tabella Droghe)\tabularnewline \hline
  Kant-ramow&Temperato&Latifoglie&25&Fungo&Liquido&3C&\raggedright Ingrediente per Niedha (vedi tabella Droghe)\tabularnewline \hline
  Kucha&Tropicale&Jungla&20&Foglia&Polvere&25S&\raggedright Ingrediente per Erba Cucha (vedi tabella Droghe)\tabularnewline \hline
  Rajina&Freddo&Montagna&20&Fiore&Liquido&28S&\raggedright Ingrediente per Rajina (vedi tabella Droghe)\tabularnewline \hline
  Khan&Tropicale&Sottosuolo&20&Resina&Liquido&25S&\raggedright Ingrediente per Khan (vedi tabella Droghe)\tabularnewline \hline
  Khan&Tropicale&Sottosuolo&20&Foglia&Inalazione&10S&\raggedright Ingrediente per Khan (fumo) (vedi tabella Droghe)\tabularnewline \hline
  \caption{Erbe}{Erbe Medicinali e Ingredienti per Veleni e Droghe}\tabularnewline
\end{longtable}
}
%%% Local Variables: 
%%% mode: latex
%%% TeX-master: "manual"
%%% End: 

{\setlength{\tabcolsep}{0.25em}
\centering

{\Large\sc Veleni}
\label{tabveleni}

\footnotesize
\begin{longtable}{|l|p{2cm}|p{2.3cm}|l|l|p{1.1cm}|p{1.1cm}|p{5.0cm}|}
  \par
  \hline
  Nome & \raggedright Origine & \raggedright Uso & \raggedright Costo & \raggedright Diff & \raggedright TR\linebreak Morte & \raggedright TR\linebreak Coma & \raggedright TR/\linebreak Effetti\tabularnewline \hline\hline
  \endfirsthead
  \hline
  Nome & \raggedright Origine & \raggedright Uso & \raggedright Costo & \raggedright Diff & \raggedright TR Morte & \raggedright TR Coma & \raggedright TR/Effetti\tabularnewline \hline\hline
  \endhead

  Curaro & \raggedright Pianta & \raggedright Pasta, iniezione o ingestione & \raggedright 8C & \raggedright 25 & \raggedright 28 (1d6 round) & \raggedright 30 (1d10 mesi) & \raggedright 32/ (-5 COS, -5 CONC, -8 AGI, -7 FOR) per 1d6 giorni\tabularnewline \hline
  Vedova nera & \raggedright Ragno & \raggedright Pasta, iniezione  & \raggedright 10C & \raggedright 27 & \raggedright 28 (1d4 minuti) & \raggedright 30 (1d10 mesi) & \raggedright 32/ (-7 COS, -7 CONC, -10 AGI, -8 FOR) per 1d6 giorni\tabularnewline \hline
  Argia & \raggedright Ragno & \raggedright Pasta, iniezione o ingestione & \raggedright 20C & \raggedright 35 & \raggedright 30 (1d4 round) & \raggedright 32 (1d10 mesi) & \raggedright 35/ (-8 COS, -10 CONC, -12 AGI, -10 FOR) per 1d20 giorni con allucinazioni, crisi epilettiche, emorragia alle mucose\tabularnewline \hline
  Narco & \raggedright Scimmia, Pesce & \raggedright Liquido, iniezione & \raggedright 2C & \raggedright 27 & \raggedright - & \raggedright 15 (1d6 mesi) & \raggedright 25/ Addormenta in 1d4 round in 1d10 ore\tabularnewline \hline
  Makhow & \raggedright Radice & \raggedright Liquido, ingestione & \raggedright 3C & \raggedright 22 & \raggedright - & \raggedright - & \raggedright 28/ (-8 CONC, -8 OSS) per 1d10 giorni, (-1d20 PSI al ritmo di 2 al giorno). Porta gradualmente alla follia. \tabularnewline \hline
  Musca & \raggedright Insetto & \raggedright Pasta, iniezione & \raggedright 4c & \raggedright 25 & \raggedright 18 (1d6 giorni) & \raggedright 20 (1d4 mesi) & \raggedright 30/ (-5 COS, -5 CONC, -8 AGI, -7 FOR) per 2d6 giorni. Tremiti, febbre e dolori addominali.\tabularnewline \hline
  Zurpow & \raggedright Corteccia & \raggedright Polvere, inalazione & \raggedright 3S & \raggedright 18 & \raggedright - & \raggedright - & \raggedright 25/ (-12 OSS) Rende ciechi.\tabularnewline \hline
  Noko-dhow & \raggedright Insetto & \raggedright Liquido, ingestione & \raggedright 2S & \raggedright 15 & \raggedright - & \raggedright - & \raggedright 25/ (-7 CONC) per 1d4 giorni. Forte prurito inguinale, rende sterili nel 70\% dei casi.\tabularnewline \hline
  Peeberah & \raggedright Serpente & \raggedright Liquido, iniezione & \raggedright 2C & \raggedright 20 & \raggedright 20 (1d10 minuti) & \raggedright - & \raggedright 25/ (-15 AGI). Attacca i centri nervosi paralizzando gli arti per 1d20 giorni.\tabularnewline \hline
  Nofa-ul-has & \raggedright Pianta & \raggedright Pasta, iniezione o ingestione & \raggedright 3C & \raggedright 25 & \raggedright - & \raggedright - & \raggedright 30/ Inibisce la capacit\`a di mentire\tabularnewline \hline
  Kow-haow & \raggedright Ragno & \raggedright Liquido, iniezione o ingestione & \raggedright 25C & \raggedright 20 & \raggedright 18 (1d4 mesi) & \raggedright 20 (1d20 giorni) & \raggedright 25 - (-18 AGI) Paralizza per 1d6 mesi tutto il corpo esclusi la testa e il collo.\tabularnewline \hline
  Cloroformio & \raggedright Bacca & \raggedright Liquido, inalazione  & \raggedright 3S & \raggedright 18 & \raggedright - & \raggedright - & \raggedright 28/ Addormenta in 1d4 round per 2d6 ore\tabularnewline \hline
  Cicuta & \raggedright Pianta & \raggedright Liquido, ingestione & \raggedright 8S & \raggedright 20 & \raggedright 25 (1d6 round) & \raggedright 27 (1d6 mesi) & \raggedright 30/ (-3 COS, -5 CONC, -5 AGI, -5 FOR) per 1d10 giorni \tabularnewline \hline
  \caption{Veleni}{Tabella Veleni}\tabularnewline
\end{longtable}
}

%%% Local Variables: 
%%% mode: latex
%%% TeX-master: "manual"
%%% End: 

{\setlength{\tabcolsep}{0.25em}
\centering

{\Large\sc Malattie}
\label{tabmalattie}

\footnotesize
\begin{longtable}{|p{2.1cm}|l|p{0.8cm}|p{0.8cm}|l|p{2.5cm}|p{4cm}|p{2.2cm}|}
  \par
  \hline
  Nome & \raggedright Durata & \raggedright TR Effetti & \raggedright TR Morte & \raggedright DIFF & \raggedright Cause & \raggedright Effetti & \raggedright Cura\tabularnewline \hline\hline
  \endfirsthead
  \hline
  Nome & \raggedright Durata & \raggedright TR Effetti & \raggedright TR Morte & \raggedright DIFF & \raggedright Cause & \raggedright Effetti & \raggedright Cura\tabularnewline \hline\hline
  \endhead
  \raggedright Anchilostomiasi & \raggedright Cronica & \raggedright 20 & \raggedright - & \raggedright 25 & \raggedright Parassita che entra attraverso la pelle da origine infetta.  & \raggedright Emorragie polmonari, Anemia, Letargia Generale (-5 INT, CONC, AGI, COS, FOR) & \raggedright Infuso di Erbe Vermifughe\tabularnewline \hline
  \raggedright Colera & \raggedright 1 Mese & \raggedright 25 & \raggedright 25 & \raggedright 25 & \raggedright Scarse condizioni igieniche. Aumenta la possibilit\`a di infettarsi dopo calamit\`a naturali & \raggedright Vomito, Sudori freddi, crampi muscolari, diarrea, febbre (-5 ogni caratteristica) & \raggedright Reintegrazione dei liquidi persi. Infusi, impacchi caldi.\tabularnewline \hline
  \raggedright Denutrizione & \raggedright Cronica & \raggedright 20 & \raggedright 15 & \raggedright 10 & \raggedright Dieta non equilibrata & \raggedright Scorbuto, Beri Beri, Rachitismo, Pellagra, con sintomi differenti. -3 alle caratteristiche & \raggedright Dieta equilibrata\tabularnewline \hline
  \raggedright Dissenteria & \raggedright Cronica & \raggedright 25 & \raggedright 23 & \raggedright 28 & \raggedright Contatto con Acqua inquinata, escrementi di individui infetti, mosche & \raggedright Febbre alta, Sudorazione abbondante con disidratazione, feci liquide e sanguinolente (-10 COS, -5 FOR) & \raggedright Infusi e molti liquidi \tabularnewline \hline
  \raggedright Lebbra & \raggedright 10 mesi & \raggedright 12 & \raggedright 35 & \raggedright 35 & \raggedright Contatto con individui infetti & \raggedright Lesioni gravi alla pelle e ai nervi, anestesia e perdita di colore della pelle, poi piaghe purulente, cecit\`a, mutilazione spontanea di estremit\`a (vedi fantoccio); -2 a FOR, AGI, BEL, COS ogni mese. Quando una di queste \`e 0, serve un TR contro morte & \raggedright Non esiste cura. Le attenzioni del medico danno +3 al TR contro morte; il decorso pu\`o essere fermato.\tabularnewline \hline
  \raggedright Malaria & \raggedright 1d100 mesi & \raggedright 25 & \raggedright 20 & \raggedright 23 & \raggedright Puntura della Zanzara anofele femmina in zone paludose & \raggedright Febbri ricorrenti, Brividi di Freddo accompagnati da sudorazione e forti tremori (-1 tutte le caratteristiche) & \raggedright Chinino (si ricava dalla corteccia della China)\tabularnewline \hline
  \raggedright Malattie Veneree & \raggedright Croniche & \raggedright 35 & \raggedright 15 & \raggedright 30 & \raggedright Rapporti sessuali con individui infetti & \raggedright Infiammazioni locali, pustole, febbre. Alcune malattie portano alla demenza progressiva. (-5 alle azioni in movimento, -1 int ogni 6 mesi) & \raggedright Medicazione locale con unguenti appositi\tabularnewline \hline
  \raggedright Male Freddo & \raggedright 1 settimana & \raggedright 35 & \raggedright In seguito    al danno & \raggedright 25 & \raggedright Solo REUBEN. Contatto della pelle di un Reuben con l'acqua per oltre 1 minuto, ingestione di acqua & \raggedright Febbre bassa, comparsa di 1d6 pustole gonfie d'acqua sulla zona colpita, o in 1d4 di zone in caso di ingestione. Se le pustole sono toccate si lacerano determinando 1 PF; -1 alla CONC per il prurito & \raggedright Asciugatura, svuotamento delle pustole; dieta di cibi secchi per una quantit\`a pari all'acqua ingerita\tabularnewline \hline
  \raggedright Mussiadura & \raggedright 1d6 giorni & \raggedright 30 & \raggedright 20 & \raggedright 35 & \raggedright Morso di Ragno & \raggedright Febbre, Brividi, Vomito, dolore alle giunture, macchie sulla pelle. Se il morso avviene sul viso si ha -1d6 in bellezza. Se non curato in tempo pu\`o costringere all'amputazione dell'arto colpito & \raggedright Erbe, impacchi, pomate. Incisione per far defluire il nero\tabularnewline \hline
  \raggedright Peste & \raggedright 2 settimane & \raggedright 35 & \raggedright 35 & \raggedright 35 & \raggedright Pulci e cimici dei topi & \raggedright Bubboni, febbre elevata, pelle arrossata, sonnolenza, delirio; emorragia che porta alla morte (.-1 ad ogni caratteristica al giorno) & \raggedright Pitzy-Adoree\tabularnewline \hline
  \raggedright Poliomielite & \raggedright Cronica & \raggedright 25 & \raggedright - & \raggedright 28 & \raggedright Acqua Inquinata & \raggedright Paralisi dell'arto colpito (fantoccio per sapere quale) & \raggedright Impacchi caldi sui muscoli (+5 ai TR)\tabularnewline \hline
  \raggedright Febbri (raffreddori, influenze, ecc.) & \raggedright 1d20 gg & \raggedright 35 & \raggedright - & \raggedright 25 & \raggedright Freddo, contatto con persone malate & \raggedright -1d6 IN CONC, FOR e COS,  INT & \raggedright Infusi, erbe, ambiente riscaldato.\tabularnewline \hline
  \caption{Malattie}{Tabella Malattie}\tabularnewline
\end{longtable}
}

{\setlength{\tabcolsep}{0.25em}
\centering

}

%%% Local Variables: 
%%% mode: latex
%%% TeX-master: "manual"
%%% End: 

{\setlength{\tabcolsep}{0.25em}
\centering

{\Large\sc Droghe}
\label{tabdroghe}

\footnotesize
\begin{longtable}{|l|p{2.5cm}|l|p{2.5cm}|l|l|l|p{3.4cm}|l|}
  \par
  \hline
  Nome& \raggedright Effetti Principali& \raggedright VEP& \raggedright Effetti Collaterali& \raggedright VEC& \raggedright TR& \raggedright Costo& \raggedright REP \%& \raggedright Origine\tabularnewline \hline\hline
  \endfirsthead
  \hline
  Nome& \raggedright Effetti Principali& \raggedright VEP& \raggedright Effetti Collaterali& \raggedright VEC& \raggedright TR& \raggedright Costo& \raggedright REP \%& \raggedright Origine\tabularnewline \hline\hline
  \endhead                                                                
  Ry-han& \raggedright +10 RES per 3 ore& \raggedright 32& \raggedright -& \raggedright 0& \raggedright 32& \raggedright 32S& \raggedright 60\%& \raggedright Pianta\tabularnewline \hline
  Chuga& \raggedright +3 FOR +6 RES per 10 min& \raggedright 30& \raggedright E per 3 giorni A per 1g& \raggedright 5& \raggedright 25& \raggedright 30S& \raggedright 20\%& \raggedright Rinoceronte\tabularnewline \hline
  Cika-ya& \raggedright +5 AGI per 30 min& \raggedright 30& \raggedright D per 5 giorni G per 5 giorni& \raggedright 5& \raggedright 25& \raggedright 30S& \raggedright 75\%& \raggedright Insetto\tabularnewline \hline
  Zakada& \raggedright -10 PSI -3 CONC per 10 ore& \raggedright 72& \raggedright A per 9 giorni B per 9 giorni& \raggedright 44& \raggedright 28& \raggedright 72S& \raggedright 30\%& \raggedright Tubero\tabularnewline \hline
  Nye-dha& \raggedright -3 FOR -3 COS per 8 ore& \raggedright 68& \raggedright C per 12 giorni F per 8 giorni& \raggedright 34& \raggedright 34& \raggedright 68S& \raggedright 30\%& \raggedright Fungo\tabularnewline \hline
  Rajina& \raggedright +3 FOR +10 RES-10 PSI per 1 ora& \raggedright 62& \raggedright A per 10 giorni C per 10 giorni E per 2 giorni& \raggedright 27& \raggedright 35& \raggedright 62S& \raggedright 100 \% nelle Terre del Nord, 50 \% altrove& \raggedright Pianta\tabularnewline \hline
  Maghus& \raggedright +2 INT per 30 ore& \raggedright 140& \raggedright E per 10 giorni F per 20 giorni& \raggedright 80& \raggedright 60& \raggedright 14C& \raggedright 1\%& \raggedright Balena\tabularnewline \hline
  Khan& \raggedright +5 CONC +6 VOL per 2 ore& \raggedright 40& \raggedright A per 5 giorni & \raggedright 5& \raggedright 35& \raggedright 40S& \raggedright 100 \% nelle Terre del Nord, 50 \% altrove& \raggedright Resina\tabularnewline \hline
  Kucha& \raggedright +2 OSS per 1 ora& \raggedright 16& \raggedright C per 1 giorno & \raggedright 1& \raggedright 15& \raggedright 16S& \raggedright 100\% nella Confederazione del Sud. 80\% altrove& \raggedright Foglia\tabularnewline \hline
  Khan (fumo)& \raggedright -5 PSI per 1 ora. +1 diff. del TR& \raggedright 15& \raggedright -& \raggedright 0& \raggedright 15& \raggedright 14S& \raggedright 100 \% nelle Terre del Nord, 50 \% altrove& \raggedright Foglia\tabularnewline \hline
  \caption{Droghe}{Tabella Droghe}\tabularnewline
\end{longtable}
}

%%% Local Variables: 
%%% mode: latex
%%% TeX-master: "manual"
%%% End: 
\twocolumn


%%% Local Variables: 
%%% mode: latex
%%% TeX-master: "manual"
%%% End: 
