\chapter{Movimento e attivit\`a} 

\section{Il movimento}
\label{movimento}
\iffullversion

\`Esplorazioni, viaggi, perlustrazioni, ricognizioni o semplici
spostamenti rivestono un ruolo fondamentale in molte delle avventure
dei vostri PG: \`e quindi necessario che sappiate come muovervi nel
mondo incorporeo della fantasia.

\fi

\subsection{Camminare (C)}

Rappresenta il movimento base del vostro PG. In linea generale potrete
spostarvi camminando di $$\frac{FOR + AGI}{8}$$
metri per Round (1
round = 3 secondi), arrotondando per difetto. 

I giganti possono spostarsi camminando di $$\frac{FOR + AGI}{4}$$
metri per Round.

\es{Gunnar \`e un Nordico con 18 in FOR e 16 in AGI. Pu\`o
  spostarsi di: (18 + 16) / 8 = 34 / 8 = 4 metri per Round}

\iffullversion
\subsection{Correre}

Per stabilire la normale velocit\`a
di corsa \`e sufficiente moltiplicare per 4 la velocit\`a al passo.
$$\frac{FOR + AGI}{2}$$

\subsection{Marciare (M)} 

In alcune situazioni potrebbe rivelarsi di utile per il vostro PG
ricorrere alla marcia forzata. Nello spostarsi in marcia forzata la
velocit\`a non si calcola in metri per round, ma in km al giorno (con
un massimo giornaliero di 10 ore) come da Tabella di Movimento.

Ricordate che la velocit\`a della carovana \`e quella del mezzo pi\`u
lento!
\fi

{\sloppypar\raggedright \subsection{Cavalcature e mezzi di trasporto}}

I tre tipi di cavalcature pi\`u diffusi su Quadrantal sono i
bipedi (B), i quadrupedi (Q) e gli alati (A).  Nella Tabella di
Movimento riportiamo la loro andatura in km al giorno.

I mezzi di trasporto sono molteplici, dal calesse alla nave.

\begin{table*}[t]
  \begin{center}
    {\Large\sc Tabella di Movimento}\medskip
    
    \begin{tabular}{|l|c|c|c|c|c|}
      \hline
      Terreno& C(**)& M(**)& B& A& Q \\ \hline
      \hline
      Pianura&30&60&80&30&90\\ \hline
      Collina&23&45&60&23&68\\ \hline
      Montagna&10&20&27&5&30\\ \hline
      Bosco&20&40&53&20&60\\ \hline
      Palude&8&15&20&3&23\\ \hline
      Jungla&8&15&{*}&{*}&{*}\\ \hline
      Neve (da 1 a 30 cm)& 15&30&40&10&45\\ \hline 
      Neve ($>$ 30 cm)&8&15&20&3&23\\ \hline
      Deserti sabbiosi&15&30&40&15&45 \\ \hline
      Via Aria&-&-&-&500&- \\ \hline
    \end{tabular}
  \end{center}
  {\footnotesize(*) Non percorribile con nessuna cavalcatura 
  (**) Per i giganti i valori indicati
  nelle caselle C e M vanno raddoppiati}
  \caption{Tabella di Movimento}
\end{table*}

\subsection{Stancarsi}
\label{stancarsi}

Tutte le azioni di movimento (nonch\'e quelle di combattimento che
verranno descritte pi\`u avanti) costeranno fatica al vostro PG, che
di conseguenza perder\`a Punti Fisico.  

Considerando che ci si sposta con indosso il carico, in linea di
massima il Master pu\`o ispirarsi alla seguente tabella per stabilire
quanti sono i PF persi dai PG per le diverse andature:

\noindent
\begin{center}
  \begin{tabular}{|l|l|}
    \hline
    Movimento& PF / Tempo \\ \hline\hline
    Camminare& 2/ 1 ora\\ 
    &12/ 5 ore\\ 
    &30/ 10 ore \\ \hline
    Correre (corsa veloce)& 2/ 1 minuto\\
    &12/ 5 minuti\\
    &30/ 10 minuti \\ \hline
    Correre (corsa leggera)& 2/ 3 minuti\\
    &12/ 15 minuti\\
    &20/ 30 minuti \\ \hline
    Marciare& 3/ 1 ora\\
    &15/ 5 ore\\
    &35/ 10 ore \\ \hline
  \end{tabular}
\end{center}

Per la perdita di PF e PM durante il combattimento vedere il paragrafo
``Stancarsi durante un combattimento'' nel capitolo
\ref{combattimento} ``Il Combattimento''.

Anche le attivit\`a di tipo mentale come studiare, scrivere,
leggere, ecc.  fanno stancare, comportando la perdita di PM, in modo
variabile a seconda dell'attivit\`a.

Orientativamente si perde 1 PM all'ora per codice di difficolt\`a
(D1, D2.  D3) dell'abilit\`a studiata o utilizzata. 

\es{Studiare per 10 ore un'incantesimo di una Scuola di Magia (D3)
  comporta la perdita di 30 PM}

In seguito alla stanchezza mentale (PM) o fisica (PF) si hanno dei
malus ai tiri, rispettivamente, su azioni ``mentali'' (basate su INT,
CONC, CAR, CON, OSS) o ``fisiche'' (basate su FOR, COS, AGI) secondo
la seguente tabella.

\noindent
\begin{center}
\begin{tabular}{|l|c|l|}
\hline
Frazione di PF o PM& Difficolt\`a &Malus \\
persi sul totale & TR o TV & azioni \\ \hline\hline
da 1/5 a 2/5& 10& -1\\ \hline
da oltre 2/5 a 3/5& 15& -2 \\ \hline
da oltre 3/5 a 4/5& 25& -3\\ \hline
oltre 4/5& 35& -4 \\ \hline
\end{tabular}
\end{center}

Se si fallisce il TR (per i PF) o il TV (per i PM) alla difficolt\`a
indicata il personaggio sviene. Con 1 solo PM rimasto o un solo PF si
sviene automaticamente.

\es{Caius ha un totale di 50 PF. In seguito a delle azioni molto
  dispendiose perde 11 PF. Poich\'e perde pi\`u di 1/5 dei punti
  fisico totali (50/5=10) ha un malus di -1 a tutte le azioni in
  movimento e deve realizzare un TR a difficolt\`a 10 per non
  svenire.
  
  Poich\'e egli ha un valore di resistenza di 16, il tiro riesce
  automaticamente}

\subsection{Riposarsi} 

Il riposo o il sonno (o la meditazione, che troverete nel capitolo
\ref{combattimento}, nella parte relativa alle Arti Marziali)
permettono di recuperare i punti fisico (PF) ed i punti mente (PM)
persi.

Essi verranno recuperati al ritmo di \textbf{2 ogni ora col riposo} e
al ritmo di \textbf{4 ogni ora col sonno}. Bisogna fare attenzione al
fatto che le frazioni di ora non danno diritto al recupero di nessun
punto fisico o mente.  Se si dorme mezz'ora, per esempio \textbf{non}
verranno recuperati due punti fisico. Se si dorme un'ora e mezzo si
recupereranno solo i 4 punti fisico (di un'ora) e cos\`i via. Lo
stesso discorso vale per la meditazione e il riposo.

Va detto che il ogni caso non si pu\`o prescindere dal sonno; i PG
devono dormire obbligatoriamente, un certo numero di ore al giorno (in
media 7-8), in caso contrario subiranno dei malus per la stanchezza a
discrezione del Master (a meno che non si utilizzi la Tecnica Speciale
``Meditazione'').

Va considerato per\`o che i malus devono essere inflitti dal Master
solo qualora il PG abbia perso un numero considerevole di ore di
sonno.  Non dovrebbe essere attribuito alcun malus ad un PG che per
una notte dorme, per esempio, solo 5 ore.

\iffullversion
\section{Temperature}

\begin{table*}[t]
  \begin{center}
    {\Large\sc Temperature}\medskip
    
    \small
    \begin{tabular}{|c|c|l|} \hline 
      $^OC$&TR& Effetto\\ \hline
        \hline 
        meno di -50& 40& 1d6 PV di danno da freddo ogni minuto;-10 a tutte le azioni \\ \hline 
        da -50 a -30& 35& 1d6 PV di danno da freddo ogni 5 minuti; -7 a tutte le azioni \\ \hline 
        da -29 a -10& 30& 1d6 PV di danno da freddo ogni 15
        minuti; -4 a tutte le azioni \\ \hline 
        da -9 a 0& 25& 1d6 PV
        di danno da freddo ogni 30 minuti; -2 a tutte le azioni \\ 
        \hline
        da 1 a 5& 20& -1 a tutte le azioni da 6 a 10 15; -1 a
        tutte le azioni \\ \hline 
        da 11 a 15& 10& -1 a tutte le azioni
        \\ \hline 
        da 16 a 25& - &  \\ \hline 
        da 26 a 35& 10& -1 a
        tutte le azioni \\ \hline da 36 a 45& 15& -2 a tutte le azioni
        \\ \hline 
        da 46 a 55& 20& -4 a tutte le azioni \\ \hline
        da 56 a
        65& 25& -7 a tutte le azioni; 1d6 PV di danno ogni 15 minuti
        \\ \hline
        da 66 a 80& 30& -10 a tutte le azioni; 1d6 PV di
        danno ogni 5 minuti 
        \\ \hline pi\`u di 80&35& -15 a tutte le
        azioni; 1d6 PV di danno ogni minuto\\ \hline
      \end{tabular}
  \caption{Effetti delle Temperature}
  \end{center}
\end{table*}

Data la diversit\`a dei climi presenti nell'Arcipelago,
nonch\'e la possibilit\`a di imbattersi in incendi e gelate magiche e non,
viene qui di seguito riportata una tabella che indica le difficolt\`a dei
TR da effettuare per evitare di subire gli effetti dannosi dovuti alle temperature
estreme. Il TR va ripetuto ogni 30 minuti.

La difficolt\`a del TR \`e riferita ad un PG che non indossa nessun
indumento, Certi indumenti conferiscono un bonus al TR (vedi
Tab. \ref{tabvestiario} a pag \pageref{tabvestiario}), sia contro il
freddo che contro il caldo (gli abiti di lana, aiutando a mantenere
stabile la temperatura del corpo, offrono protezione anche contro il
caldo).

Sino a $15^o$ il Tiro Resistenza \`e contro il Freddo (TR Freddo),
al di sopra di tale temperatura, il TR \`e contro il caldo (TR
Caldo).
\fi

\section{Attivit\`a}

Le attivit\`a possibili sono molteplici: tutte quelle che sono
possibili per una persona, attuabili con la parola o con l'utilizzo
delle vostre abilit\`a, incontrano gli unici limiti nella
razionalit\`a vostra e del vostro Master.

\iffullversion
Il Master non dovrebbe comunque consentire ad un personaggio di
operare nella stessa attivit\`a per \textbf{pi\`u di un numero di ore
  consecutive pari alla met\`a della CONC}, e comunque non dovrebbe
consentire ai PG di svolgere attivit\`a (anche multiple) per pi\`u di
\textbf{10 ore al giorno}, salvo che per brevi periodi.

Di solito tutte le attivit\`a vengono attuate facendo uso delle
abilit\`a di cui disponete e quindi semplicemente applicando le
regole di questo manuale.

Un consiglio che ci sentiamo di darvi, tuttavia, \`e quello di non
aggrapparvi troppo saldamente alla pedissequa applicazione del
regolamento, sottovalutando l'importanza della recitazione del vostro
personaggio.

Il gioco di ruolo infatti non consiste soltanto nell'uccidere i nemici
a colpi di spada, lanciare palle di fuoco o bere pozioni magiche: il
gioco di ruolo \`e anche l'interazione con gli altri personaggi,
giocanti (PG) e non giocanti (PNG), lo studio di tattiche adeguate per
le azioni di gruppo e ---perch\'e no?--- il dialogo con l'oste che vuole
farvi pagare un bicchiere di idromele il doppio di quanto valga in
realt\`a!
\fi

\subsection{Il sistema monetario}

Il sistema monetario \`e costituito principalmente da 4 unit\`a: i
Kulos, gli Scudi, le Corone e le Piastre.  

Le unit\`a monetarie sono nel seguente rapporto: 

\begin{itemize}
\itemsep -6pt
\item 10 Kulos (K) = 1 Scudo (S) 
\item 10 Scudi = 1 Corona (C) 
\item 10 Corone = 1 Piastra (P)
\end{itemize}

\subsection{Equipaggiamento}

I prezzi degli oggetti che intendete acquistare e che faranno parte
del vostro equipaggiamento sono indicati nelle Tabella dei Prezzi in queste
pagine.

\begin{table}[p]
  \begin{center}
    {\Large\sc Vitto}\medskip

    \small\begin{tabular}{|l|r|l|}
      \hline
      Articolo& Prezzo \\ \hline
      \hline
      Acqua (5l)& 1k \\ \hline
      Birra (0.5l) & 3K \\ \hline
      Superalcolico (0.1l)& 3K \\ \hline
      Idromele (0.5l)& 5K \\ \hline
      Vino (75 cl)& 3K \\ \hline
      Phyluf Herhu (10 cl)&5K \\ \hline
      Pasto Frugale&1S \\ \hline
      Pasto Normale&2S \\ \hline
      Abbuffata&4S \\ \hline
      Banchetto&7S \\ \hline
      Razione Giornaliera (800 g)&7K \\ \hline
      Pane (1 kg)&1K \\ \hline
      Frutta e verdura (1 kg)&2K \\ \hline
      Carne (1 kg)&1S \\ \hline
      Pesce (1 kg)&1S \\ \hline
    \end{tabular}
    \caption{Vitto}
  \end{center}         
\end{table}

\begin{table}[p]
  \begin{center}
    {\Large\sc Vestiario}\medskip
    
    \small\begin{tabular}{|p{1.5cm}|r|p{1cm}|p{2.2cm}|}
      \hline
      Articolo& Prezzo& Peso & Note  \\ \hline
      \hline
      Stivali& 4S&  \raggedright 2 kg & 1 paio \\ \hline
      Pantaloni& 1S& \raggedright da 0.5 a 4 kg &\\ \hline
      Calzari& 2S& \raggedright 0.5 kg &\\ \hline
      Casacca& 1S& \raggedright da 0.5 a 2 kg &\\ \hline
      Camicia& 3S& \raggedright da 0.2 a 1 kg &\\ \hline
      Giacca& 5S& \raggedright da 1 a 2 kg & +1 TR Freddo \\ \hline
      Pelliccia& 16 S& \raggedright da 4 a 8 kg & +5 TR Freddo\\ \hline 
      Cappotto&12S& \raggedright da 2 a 6 kg & +3 TR Freddo \\ \hline
      Mantello& 4S& \raggedright da 0.2 a 3 kg & +2 TR Freddo \\ \hline
      Gilet& 3S& \raggedright 100 g &\\ \hline
    \end{tabular}
    \caption{Vestiario}
    \label{tabvestiario}
  \end{center}         
\end{table}



\iffullversion
\begin{table}[p]
  \begin{center}
    {\Large\sc Mezzi di Trasporto}\medskip
    
    \small\begin{tabular}{|l|r|}
      \hline
      Articolo& Prezzo\\ \hline
      \hline
      Mulo&4C\\ \hline
      Pony&7C\\ \hline
      Cavallo da soma&15C\\ \hline
      Cavallo da sella&30C\\ \hline
      Cavallo da tiro&25C\\ \hline
      Cavallo da corsa&50C\\ \hline
      Cavallo da guerra&75C\\ \hline
      Fortral o Nortral&20C\\ \hline
      Reptis&70C\\ \hline
      Cammello&25C\\ \hline
      Elefante&95C\\ \hline
      M\`uvara&25C\\ \hline
      Calesse&4C\\ \hline
      Carretto&2C\\ \hline
      Carro&6C\\ \hline
      Carro grande&9C\\ \hline
      Carro coperto&12C\\ \hline
      Carrozzone&27C\\ \hline
      Diligenza&70C\\ \hline
      Barca a remi&16C\\ \hline
      Barca a vela&29C\\ \hline
      Galea&370C\\ \hline
      Nave da carico&3900C\\ \hline
      Nave da guerra piccola&6000C\\ \hline
      Nave da guerra Grande&13000C\\ \hline
      Bahn-yere&18000C\\ \hline
      Nave da trasporto&6000C\\ \hline
      Chiatta&9C\\ \hline
      Battello fluviale&35C\\ \hline
     \end{tabular}
    \caption{Mezzi di Trasporto}
  \end{center}         
\end{table}    
\fi

\iffullversion
\begin{table}
  \begin{center}
    {\Large\sc Alloggio}\medskip

    \small\begin{tabular}{|l|r|l|}
      \hline
      Articolo& Prezzo \\ \hline
      \hline
      Posto Letto&5K \\ \hline
      Camera Singola&3S\\ \hline
      Camera Doppia&4S\\ \hline
      Stallaggio&6K\\ \hline
     \end{tabular}
    \caption{Alloggio}
  \end{center}         
\end{table}     
\fi

\begin{table*}
  \begin{center}
    {\Large\sc Equipaggiamento}\medskip
    
    \small\begin{tabular}{|l|r|p{1cm}|p{8cm}|}
      \hline
      Articolo& Prezzo& Peso & Note  \\ \hline
      \hline
      Finimenti& 1C& 2 kg& \\ \hline
      Sella&3C& 8 kg& \\ \hline
      Coperta& 1S& 3 kg& 250x170 cm  \\ \hline
      Sacco a pelo& 2S &3 kg& 210x100 cm \\ \hline
      Zaino& 3S& 500 g& 20kg carico \\ \hline
      Zaino intelaiato& 5S& 3 kg& 30kg carico \\ \hline
      Bisaccia& 1S& 100g& 4kg carico \\ \hline
      Acciarino& 1S& 20g&\\ \hline
      Torcia& 3K& 200g& Luce per 2 ore \\ \hline
      Lanterna ad olio& 12K& 50g& Luce per 8 ore (1 carica)\\ \hline
      Olio per lanterna (1l)& 2K& 1 kg& 2 cariche \\ \hline
      Chiodo da arrampicata& 3K& 20 g&\\ \hline
      Faretra& 15K& 500 g& 12 frecce o 30 dardi \\ \hline
      Freccia& 6K& 200 g&\\ \hline
      Dardo per balestra& 4K& 100 g&\\ \hline
      Corda (30 m)& 15K& 5 kg& Tenuta kg 200 \\ \hline
      Rampino& 6K& 1 kg&\\ \hline
      Corda rinforzata (15m)& 25K& 3 kg& Tenuta kg 400 \\ \hline
      Sacco& 1K& 200 g& 15 kg \\ \hline
      Fodero& 1S& 300 g&\\ \hline
      Telo cerato& 3S& 4 kg& 250x300 cm \\ \hline
      Tenda 4 posti& 15S& 8 kg&\\ \hline
      Tenda 2 posti& 1C& 4 kg&\\ \hline
      Borraccia (2l)& 5K& 200 g&\\ \hline
      Mortaio e pestello& 4K& 300 g&\\ \hline
      Tegami da campo& 1S& 3 kg&\\ \hline
      Sestante& 6C& 300 g& +5 orientamento (necessario un tiro Astronomia a 20)\\ \hline 
      Bussola& 5S& 50 g& +2 orientamento \\ \hline
      Cannocchiale&4C& 200 g& 3 ingrandimenti\par (+2 TOSS per avvistamenti a distanza) \\ \hline
      Attrezzi da scasso& 9S& 600 g& da +1 a +3 in Scassinare \\ \hline
      Kit pronto soccorso& 6S& 1 kg& +2 in Pronto Soccorso \\ \hline
      Martello& 6K& 2 kg&\\ \hline
      Piccozza& 8K& 4 kg&\\ \hline
     \end{tabular}
    \caption{Equipaggiamento}
    \label{tabequipaggiamento}
  \end{center}         
\end{table*} 

\iffullversion
\begin{table*}
  \begin{center}
    {\Large\sc Servizi di Trasporto}\medskip
    
    \small\begin{tabular}{|l|r|p{8cm}|}
      \hline
      Servizi& Prezzo & Note  \\ \hline
      \hline
      Attraversamento fluviale&3-12K& Da una sponda all'altra di un fiume \\ \hline
      Traghetto per isole limitrofe& 1-9S& A seconda della qualit\`a del servizio \\ \hline 
      Viaggio in diligenza& 3-10K& Per ogni 10 Km \\ \hline
      Ferrare un cavallo& 6S& \\ \hline
     \end{tabular}
    \caption{Trasporto}
  \end{center}         
\end{table*} 
\fi

%%% Local Variables: 
%%% mode: latex
%%% TeX-master: "manual"
%%% End: 
