%%% Local Variables: 
%%% mode: latex
%%% TeX-master: "manual"
%%% End: 

\chapter{Il Personaggio}
\label{personaggio}
\iffullversion
\section {Chi \`e il personaggio}

Il personaggio rappresenta il vostro alter-ego, colui che permette al
giocatore di muoversi ed agire in un mondo immaginario, quale \`e
quello in cui vi apprestate ad entrare.

La nascita del personaggio avviene attraverso il procedimento di
creazione esposto a pagina \pageref{creazione}. 

Prima di tutto, per\`o, \`e necessario sapere quali dadi
utilizzare.
\fi

\section{Tirare i dadi}
\label{dadi}
I dadi necessari per giocare a Radix Malorum sono:

\begin{itemize}
\itemsep -6pt
\item dado a 4 facce (d4)
\item dado a 6 facce (d6)
\item dado a 8 facce (d8)
\item dado a 10 facce (d10)
\item dado a 20 facce (d20)
\item dado a 100 facce o dado percentuale (d100), facilmente
  sostituibile da 2 dadi a 10 facce.
\end{itemize}

Il numero che \textbf{precede} la nomenclatura del \textbf{tipo} di
dado (d6, d8, ecc.) indica il \textbf{numero} di dadi di quel tipo che
devono essere lanciati, per poi \textbf{sommarne} i risultati.

Il \textbf{numero con il segno} che segue il numero ed il tipo del dado indica
un \textbf{punteggio fisso} che va \textbf{aggiunto o sottratto} alla somma dei
risultati ottenuti dai singoli dadi.

\es{\textbf{3d6} significa lanciare \textbf{3 dadi a 6 facce}, 5d8+2
  significa lanciare 5 dadi a 8 facce e sommare 2 al risultato.}

Il \textbf{d100} viene utilizzato per ottenere una gamma di valori
variabile da 1 (01) a 100 (00). Qualora non aveste 1 dado a 100 facce
(noi l'abbiamo visto soltanto in una foto in bianco e nero, in
fotocopia e a dieci metri di distanza e tuttora siamo scettici sulla
sua reale esistenza) utilizzate 2 dadi a 10 facce. Il primo dado
rappresenter\`a le unit\`a, il secondo le decine (o viceversa, basta
che lo decidiate \textbf{prima} di averli lanciati).

\es{Se lanciando i due dadi a 10 facce si ottiene 7 col primo
(quello delle unit\`a) e 6 col secondo (quello delle decine), il
risultato \`e 67}

\pinupbig{bambino.eps}{}{p}

Il tiro del dado a 20 facce \`e sempre e comunque \textbf{aperto};
ci\`o significa che si aggiunge il punteggio di un nuovo tiro ogni volta che
si ottiene 20 (col dado). Se tiro il dado e ottengo 20, dovr\`o
rilanciarlo. Se al secondo tiro otterr\`o per esempio 7
significher\`a che il valore ottenuto col dado sar\`a 27. Se invece al
secondo lancio otterr\`o un altro 20 sommer\`o ancora ed andr\`o
avanti.

\es{Sequenze aperte:
\begin{tabbing}
$1^o$ tiro 7   $\rightarrow$ \= $punteggio=7$\\
$1^o$ tiro 20,\> $2^o$ tiro 10 $\rightarrow$ \= $punteggio=30$\\
$1^o$ tiro 20,\> $2^o$ tiro 20, \> $3^o$ tiro 5 \\
               \>$\rightarrow$ $punteggio=45$\\
$1^o$ tiro 20,\> $2^o$ tiro 20, \>$3^o$ tiro 20, $4^o$ tiro 18 \\
               \>$\rightarrow$ $punteggio=78$\\
               \>(non state barando?) 
\end{tabbing}
}

Ottenendo 1 come risultato del tiro aperto del dado a 20 facce
(cio\`e 1 al primo tiro) si provoca, se non indicato diversamente,
un \textbf{Fallimento Catastrofico}.

{\raggedright \section{Il Fallimento Catastrofico}}
\label{fumble}

Consiste nel fallimento automatico dell'azione o di tutti i tiri
(Resistenza, Volont\`a, Psiche, Osservazione, ecc.) in occasione dei
quali viene realizzato. Provoca \textbf{effetti collaterali} a scelta
del Master, dannosi per il personaggio.

\iffullversion
Un fallimento catastrofico nell'uso di un'\textbf{arma} pu\`o
provocare la rottura della stessa, o cagionare danni al Personaggio
che la utilizza o ai suoi compagni.

Con un fallimento catastrofico ai \textbf{tiri Osservazione} si pu\`o
credere di vedere delle cose che in realt\`a non esistono, ecc.

Fanno eccezione a questa regola i tiri Concentrazione e i tiri per il
lancio degli incantesimi, per i quali il fallimento catastrofico
sottost\`a a regole specifiche.
\fi

\section{La Creazione}
\label{creazione}

Per creare un personaggio si percorrono i seguenti passi:
\begin{enumerate}
  \itemsep -6pt
\item Determinazione delle \textbf{caratteristiche}
\item Assegnazione delle \textbf{abilit\`a}
\item Costruzione del \textbf{background}
\end{enumerate}

A tal fine \`e necessario procurarsi almeno un dado a venti facce
(d20), uno a dieci facce (d10), uno a 8 facce (d8) e una \textbf{Scheda del
Personaggio}.

\`E comunque indispensabile, prima di ogni altra cosa, scegliere la
\textbf{razza e l'etnia} a cui il vostro personaggio (da ora
\textbf{PG}) apparterr\`a. Avete a disposizione tutte le razze e le
etnie descritte nel capitolo precedente; ricordate che al momento
della creazione il vostro personaggio dovr\`a rispettare i limiti
imposti dalle razze.

Scelto? Procediamo!

\section{Le Caratteristiche}

Le caratteristiche (\textbf{CAT}) sono una quantificazione numerica
degli attributi psicofisici del PG.  Nella vita reale tutti possediamo
queste caratteristiche, ma non perdiamo tempo ad attribuirgli un
valore. Vero \`e, per\`o, che c'\`e chi \`e pi\`u agile, chi
\`e pi\`u forte, ecc.

Ai fini del gioco, per\`o, siamo costretti ad attribuirgli un valore,
che viene stabilito con il seguente procedimento: lanciate 8 volte il
dado a 20 facce (8d20), ricordando che il tiro del d20 \`e \textbf{aperto}
e sommate i punteggi ottenuti.
Il numero cos\`{\i} ottenuto \`e il totale dei
\emph{Punti CAT}. Questo numero va annotato sulla scheda nell'apposita
casella.

Procedete ora a \textbf{dividere i Punti CAT} fra le
\textbf{Caratteristiche primarie} descritte nel prossimo paragrafo.
Potete dividere tali punti come meglio credete in relazione al tipo di
PG che avete in mente di interpretare ed ai limiti imposti dalla sua
razza e dalla sua etnia di appartenenza.

{\raggedright \subsection{Le Caratteristiche primarie (CAT)}}

Sono i parametri pi\`u importanti per il vostro personaggio, per gli
umani il loro punteggio pu\`o variare generalmente da un minimo di 3
ad un massimo di 20. Gli estremi sono diversi per ogni razza e sono
indicati nella descrizione di ognuna di esse.  Il punteggio della CAT
va indicato per ognuna di esse sulla Scheda del Personaggio, nella
casella ``VAL''.

\subsubsection{Forza (FOR)}

Rappresenta la... forza del vostro PG (vi aspettavate qualcosa di
diverso?), la sua potenza muscolare.  Questo punteggio pu\`o essere
suscettibile di modificazioni durante la vita del vostro PG, per
effetto di allenamento, di incantesimi, ecc. 
In caso di incrementi dovuti alla magia il valore potr\`a essere
superiore ai limiti previsti dalla razza. Non vi sono limiti ai
decrementi.

\pinup{ren.eps}{Ren, guerriero nordico}

\subsubsection{Costituzione (COS)}

Misura la vostra corporatura, la vostra stazza, la robustezza del
vostro corpo; anch'essa pu\`o variare nel corso della vita del
vostro PG, ma non pu\`o aumentare di oltre 1/3 del valore originario
con l'allenamento o l'incremento ``naturale'', mentre non esiste nessun
limite nel caso di incrementi magici. Non esistono altres\`{\i} limiti
in caso di decrementi di qualunque genere.

\subsubsection{Agilit\`a (AGI)}

Indica la vostra destrezza, capacit\`a di manipolazione, i vostri
riflessi, la vostra reattivit\`a. Come le altre pu\`o essere
modificata nel tempo, senza incontrare nessun limite, se non il
massimo previsto per ogni razza, salvo incrementi magici. Non vi sono
limiti ai decrementi.

\subsubsection{Osservazione (OSS)}

\`E sommariamente un indice dei vostri sensi. \`E utile quando dovete
scoprire trappole, origliare alle porte, sentire determinati odori,
tirare con l'arco, ecc. Pu\`o variare con il migliorare o col
peggiorare delle vostre capacit\`a sensoriali fermi restando i limiti
di razza, o per effetto della magia. Non vi sono limiti ai decrementi.

\subsubsection{Concentrazione (CONC)}

\`E un valore indicativo che rappresenta la vostra capacit\`a di
focalizzare l'attenzione, la vostra memoria. Questa caratteristica
pu\`o essere variata permanentemente con l'esercizio o per effetti
magici ma pu\`o anche essere modificata solo temporaneamente.
Pu\`o aumentare di +1 per ogni round speso a concentrarsi, fino ad un
massimo di 5 round consecutivi, o per effetto di droghe (Vedi paragrafo
``Droghe'' a pagina \pageref{droghe}). Non vi sono limiti ai
decrementi.

\subsubsection{Intelligenza (INT)}

Parametro indicante la genialit\`a, l'acume, la prontezza di spirito,
la rapidit\`a di apprendimento del vostro PG. Come le altre CAT
pu\`o aumentare nel corso della vita del PG (fermi restando i limiti
di razza) anche per effetti magici; pu\`o essere aumentata o diminuita
temporaneamente con l'uso di droghe, veleni e simili. Non vi sono
limiti ai decrementi.

\subsubsection{Carisma (CAR)}

Indica vostro ascendente sulle altre persone, la vostra capacit\`a di
influenzarle.  Questa caratteristica pu\`o variare nei limiti
previsti dalle razze, salvo effetti magici. Non vi sono limiti ai
decrementi.
\pinup{orcofaccia.eps}{Gor Sok, Orco Haroka}
\subsubsection{Bellezza (BEL)}

Esprime numericamente l'aspetto fisico del vostro PG, la sua
avvenenza, il suo fascino. Pu\`o variare in aumento di non pi\`u
di un decimo del valore iniziale, senza limiti in diminuzione, salvo
effetti magici.



\subsubsection {Il valore \%}

Per specificare ulteriormente le vostre caratteristiche, il Master
lancer\`a per ognuna di esse un dado a 100 facce (1d100). Il valore
ottenuto andr\`a indicato nella scheda del personaggio nella casella
con il simbolo \textbf{\%}. 

Durante il gioco questi valori crescono per effetto dei Tiri
Incremento (vedi paragrafo \ref{incremento} a pagina
\pageref{incremento}). Ogni qualvolta si raggiunge il valore 100, le
caratteristiche corrispondenti aumentano di un punto.

\es{Il vostro personaggio ha 15 in CARisma e un valore \% di 95.  Alla
  fine di una sessione particolarmente fruttuosa, incassate 7 punti
  \%. Il valore di CAR salir\`a a 16 e il suo valore \% diventer\`a 02
  (95+7=102 cio\`e +1 in CAR e 02 \%).}

\nb{Considerate che in ogni caso non potranno essere superati i limiti
  di razza.}

\filler{elfobarba.eps}

{\raggedright \subsubsection{Bonus e Malus delle CAT}}

Ad ogni valore delle caratteristiche primarie (e della CONoscenza)
corrisponde un certo \textbf{BONus} (o malus) che \`e di importanza
fondamentale per l'utilizzo di tutte le \textbf{abilit\`a}.  Bonus e
malus vanno annotati sulla Scheda del Personaggio, nella casella
\textbf{BON}. Nella tabella \ref{tabbonuscat} sono indicati i valori e
i rispettivi bonus e malus.

\begin{table}[hbt]
\centering

{\Large\sc Bonus/Malus\\delle CAraTteristiche\medskip}

\begin{tabular}{|l|c|}
\hline Valore della CAT & Bonus/Malus \\ \hline\hline
meno di 1 & -5+ (-1 per ogni\\
 & punto sotto 1) \\ \hline
1 & -5\\ \hline
2 & -4\\  \hline
3 &-3 \\  \hline
da 4 a 5& -2\\ \hline
da 6 a 8& -1 \\  \hline
da 9 a 12& 0 \\  \hline
da 13 a 15& +1 \\  \hline
da 16 a 17& +2 \\  \hline
18 &+3 \\  \hline
19 &+4 \\  \hline
20 &+5 \\  \hline
pi\`u di 20& +5 + (+1 per ogni\\
 &  punto sopra il 20)\\  \hline
\end{tabular}
\caption{Bonus/Malus delle Caratteristiche}
\label{tabbonuscat}
\end{table}

Un PG (o un PNG) che ha un valore di 0 in una caratteristica avr\`a un
malus corrispondente di $-6$, un personaggio con un valore di $-1$ avr\`a
$-7$ e cos\`{\i} via. Un personaggio con un valore di 21 ha invece un
bonus di $+6$, uno con un valore di 22 ha $+7$ ecc.

{\raggedright \subsubsection{Il Fattore Istruzione (FI) e la Classe Sociale}}

Il Fattore Istruzione indica il livello culturale del vostro PG ed
\`e determinato dalla \textbf{Classe Sociale} di appartenenza. 

La Classe Sociale indica la vostra estrazione sociale, la professione
svolta dai vostri genitori o l'ambiente in cui siete cresciuti. Per
sapere quale sia la vostra Classe Sociale di appartenenza dovete prima
di tutto aver scelto la vostra Razza e la vostra Etnia.  Fatto questo
avete due opportunit\`a:

\begin{itemize}
\item Decidere direttamente insieme al Master la vostra Classe
  Sociale.
\item Stabilirla mediante il lancio dei Dadi.
\end{itemize}

In questa seconda ipotesi dovete lanciare \textbf{il dado a 100 facce}

Apparterrete alla classe sociale i cui numeri indicati nella
\textbf{probabilit\`a di farvi parte} comprendono quello che avete
ottenuto con il lancio dei dadi.

\es{Se aveste ottenuto 76 con il d100, e scelto come Razza ed Etnia
  gli Gnomi Beedha, fareste parte della Classe Sociale degli
  Inventori; se foste degli Gnomi Kunetha, sareste degli artigiani, se
  foste degli Elfi Luxi, sareste un Benestante, ecc.}

Il valore decimale indicato sulla destra della Classe Sociale \`e il
Fattore Istruzione (\textbf{FI}) e deve essere trascritto sulla Scheda
del Personaggio, nell'apposita casella.

Questo valore \`e molto importante per la determinazione della
\textbf{CONoscenza} (vedi \S \ref{CONoscenza} a pagina
\pageref{CONoscenza}) e dei punti disponibili per le abilit\`a di
Classe Sociale (vedi paragrafo ``Le Abilit\`a! a pagina
\pageref{abilita})

{\raggedright \subsection{Le caratteristiche secondarie}}

I valori delle caratteristiche secondarie non sono stabiliti
direttamente dal giocatore, ma dipendono dai valori attribuiti
alle caratteristiche primarie o da fattori casuali.

\subsubsection{Resistenza (RES)}

Individua numericamente la capacit\`a di resistere a shock e danni
fisici evitando stordimenti, svenimenti, morte. Il suo valore si
ottiene facendo \textbf{la media fra FORza e COStituzione}
approssimando per difetto. Varia col variare delle Caratteristiche
primarie cui \`e correlata. $$RES=\frac{COS + FOR}{2}$$

\subsubsection{Volont\`a (VOL)}

\`E un parametro che indica la propria determinazione e la capacit\`a
di resistere ad attacchi mentali. Si ottiene facendo \textbf{la media
  fra Intelligenza e Concentrazione} approssimando per difetto.  Varia
col variare delle Caratteristiche primarie cui \`e correlata.
$$VOL=\frac{CONC + INT}{2}$$

\subsubsection{Psiche (PSI)}

Indica il vostro equilibrio mentale e la vostra capacit\`a di evitare
shock emotivi.  Si ottiene sommando il punteggio ottenuto tirando
\textbf{due volte il d8 e aggiungendo a questo valore 4}. La Psiche
pu\`o diminuire a causa di forti shock.
$$PSI=2d8 + 4$$

\subsubsection{Punti Fisico (PF)}

Rappresentano numericamente quanto il personaggio sia in grado di
resistere alla fatica fisica prima di doversi obbligatoriamente
riposare. I punti fisico diminuiranno in relazione alle azioni
compiute dal vostro PG.  

Il loro valore \`e dato dalla \textbf{somma di FORza e del
  doppio della COStituzione}; si recuperano col riposo come indicato
nel capitolo \ref{movimento} ``Movimento e Attivit\`a'', a pagina
\pageref{movimento}.  Il loro massimo varia con il variare delle
Caratteristiche primarie cui sono correlati.
$$PF=FOR + (2 \times COS)$$

\subsubsection{Punti Mente (PM)}

Stabiliscono il grado di resistenza all'affaticamento mentale. I Punti
Mente diminuiranno in relazione agli sforzi mentali del vostro PG.
Sono dati da \textbf{Concentrazione pi\`u due volte l'Intelligenza}.
Anch'essi possono essere recuperati col riposo. Il loro massimo varia
con il variare delle Caratteristiche primarie cui sono correlati.
$$PM=CoNC + (2 \times INT)$$

\subsubsection{Punti Energia (PE)}

Sono l'energia magica del vostro PG, da cui si attinger\`a per
lanciare un incantesimo o per realizzare particolari colpi speciali
nel combattimento. Per il loro utilizzo vedi i capitoli\ref{magia}
``La Magia'' e \ref{combattimento} ``Il Combattimento''.

Si ottengono lanciando 1d100 per tre volte e scegliendo il valore
maggiore, che andr\`a annotato sulla scheda del personaggio
nell'apposita casella.  I punti energia persi durante il gioco vengono
recuperati al ritmo di 4 all'ora.
$$PE=\max(1d100, 1d100, 1d100)$$

\subsubsection{Punti Vita (PV)}

Rappresentano a grandi linee la salute del personaggio, la sua
resistenza alle ferite. Inizialmente sono pari a \textbf{tre volte la
  costituzione pi\`u il bonus della caratteristica forza}. Vedi in
particolare paragrafo ``Perdita Punti vita.'' Il loro massimo varia
con il variare delle Caratteristiche primarie cui sono correlati.
$$PV=(3 \times COS) + BON FOR$$

\subsubsection{Punti Struttura Corpo (PSC)}

Essi indicano la resistenza di ogni singola parte del corpo. Sono
correlate alla COStituzione come da tabella \ref{tabpsc} ``Punti
Struttura per parte del Corpo'' a pagina \pageref{tabpsc}.  Vanno
indicati nella scheda nell'apposita casella ``PSC''. Quando il loro
punteggio diminuisce in seguito a danni fisici, il valore residuo va
indicato, nella casella relativa alla parte del corpo interessata, nel
Fantoccio disegnato sulla Scheda del Personaggio.

Per valori di COS superiori a 25 i PSC vengono determinati sommando fra
loro i PSC del valore di COS pi\`u alto possibile pi\`u la parte
residua. Per esempio i PSC di un personaggio che ha una COS pari a 27
sono pari alla somma dei PSC di COS 24 + COS 3, cos\`{\i} alla testa i
PSC saranno pari a 14 + 2 = 16, al collo 8 + 1 = 9, al torace 24 + 3 =
27, ecc. Vedi in particolare paragrafo ``Perdita punti struttura
corpo.'' a pagina \pageref{perditapvpsc}.

\begin{table*}[htb]
\begin{radtable}{Punti Struttura Per Parte Del Corpo}{|l||*{11}{c|}}
\footnotesize  COS&
\footnotesize Testa&
\footnotesize Collo&
\footnotesize Torace&
\footnotesize Braccio&
\footnotesize Avamb.&
\footnotesize Mano&
\footnotesize Addome&
\footnotesize Coscia&
\footnotesize Gamba&
\footnotesize Piede&
\footnotesize Coda\\ \hline\hline
\small
  3&2&1&3&2&2&1&2&3&3&2&2\\ \hline
  4&2&2&4&3&3&1&2&4&4&2&3\\ \hline
  5&3&2&5&4&4&2&3&5&5&3&4\\ \hline
  6&4&2&6&5&5&3&4&6&6&4&5\\ \hline
  7&4&3&7&6&6&3&4&7&7&4&6\\ \hline
  8&5&3&8&6&6&4&4&8&8&5&6\\ \hline
  9&5&3&9&7&7&4&5&9&9&5&7\\ \hline
  10&6&4&10&8&8&5&6&10&10&9&8\\ \hline
  11&7&4&11&9&9&5&6&11&10&7&9\\ \hline
  12&7&4&12&10&10&6&6&12&11&7&10\\ \hline
  13&8&4&13&10&10&6&7&13&12&8&10\\ \hline
  14&8&5&14&11&11&7&8&14&13&8&11\\ \hline
  15&9&5&15&12&12&8&8&15&14&9&12\\ \hline
  16&10&5&16&13&13&8&8&16&15&10&13\\ \hline
  17&10&6&17&14&14&9&9&17&16&10&14\\ \hline
  18&11&6&18&14&14&9&10&18&17&11&14\\ \hline
  19&11&6&19&15&15&10&10&19&18&11&15\\ \hline
  20&12&6&20&16&16&10&10&20&18&12&16\\ \hline
  21&13&7&21&17&17&11&11&21&19&13&17\\ \hline
  22&13&7&22&18&18&11&12&22&20&13&18\\ \hline
  23&14&7&23&18&18&12&12&23&21&14&18\\ \hline
  24&14&8&24&19&19&12&12&24&22&14&19\\ \hline
  25&15&8&25&20&20&13&13&25&23&15&20\\ \hline
\end{radtable}
\normalsize
\caption{Punti Struttura Corpo}
\label{tabpsc}
\end{table*}


\subsubsection{Et\`a}

Indica il numero di anni con i quali il vostro PG inizia la sua
avventura. Per calcolarla si tira 1d6 e si somma a questo valore 19.
In tal modo si pu\`o iniziare con un'et\`a variabile tra i 20 e i
25 anni.  Contrariamente a quanto di solito avviene nei giochi di
ruolo, in Radix Malorum, questa caratteristica ha un'importanza
rilevante perch\'e viene utilizzata per determinare la caratteristica
``CONoscenza''.

\filler{elfo1.eps}

{\raggedright \subsection{La Caratteristica ``Conoscenza''(CON)}}
\label{CONoscenza}

Nella scheda del personaggio, dopo tutte le caratteristiche primarie
\`e indicata una Caratteristica particolare: la \textbf{CONoscenza}.  Essa
esprime numericamente il vostro bagaglio culturale, il sapere
nozionistico. 

La CONoscenza ha delle peculiarit\`a comuni sia alle
Caratteristiche Primarie sia a quelle Secondarie. Come le primarie ha
una parte indicante un valore generale (VAL), uno indicante un valore
percentuale ed un bonus (BON), come le secondarie deriva dalle
caratteristiche primarie.  Essa infatti si calcola facendo la media
fra Et\`a ed Intelligenza e moltiplicando questo valore per il
Fattore Istruzione dato dalla Classe Sociale di appartenenza (vedi il
Capitolo ``Razze ed Etnie''). Le frazioni di punto (i numeri dopo la
virgola, per intenderci) rappresentano i \textbf{punti \%} da scrivere
nella relativa casella. Qualora non ci fossero frazioni di punto, nella
casella \% andr\`a scritto 00.
\iffullversion $$CON = \frac{ETA' + INT}{2} \times FI$$ \fi
Una particolarit\`a di questa caratteristica \`e che essa non ha
limiti di incremento ma pu\`o liberamente superare il 20, sia con
mezzi naturali (il miglioramento) sia con mezzi magici.

\subsection{Essere Destrimani o Mancini}

Per determinare se il vostro PG \`e destrimano (che usa
prevalentemente la mano destra), mancino (che usa prevalentemente la
sinistra) o ambidestro (che usa tutte e due le mani
indifferentemente), \`e sufficiente lanciare 1d100. Se il risultato
\`e compreso fra 0 e 60 il PG \`e destrimano.  Se il risultato
\`e compreso fra 61 e 90 il PG \`e mancino. Altrimenti \`e
ambidestro.

\nb{Le azioni effettuate con la mano destra per i mancini o con la
  sinistra per i destrimani subiscono un malus di $-5$. Gli ambidestri
  usano ambedue le mani senza malus.}

\subsection{Il Peso trasportabile}

Indica quanti chili di carico potete \textbf{trasportare senza subire malus}
dovuti al peso dello stesso.  Nel peso trasportabile rientrano tutti
gli oggetti che in un dato momento avete addosso: armatura, armi,
equipaggiamento standard, libri, ecc. Il suo valore \`e uguale a
1kg per ogni punto COS + 3 kg per ogni punto FOR sopra il 10.

$$PT = COS + (FOR - 10) \times 3$$

\es{``A'' \`e un PG con 13 in FOR e 15 in COS. Il suo Peso
  Trasportabile \`e di: 15 (punti COS) + 3 (punti FOR sopra il 10) x
  3 kg = 24 kg }

\es{``B'' \`e un PG con 20 in FOR e 20 in COS. Il suo peso
  trasportabile \`e di: 20 (punti COS) + 10 (punti FOR sopra il 10)
  x 3 kg = 50 kg }

Se questo valore viene superato, ossia il Carico \`e maggiore del
Peso Trasportabile il PG subir\`a \textbf{un malus a tutte le azioni in
movimento ed ai Tiri iniziativa}.  Il malus dovuto al peso
trasportabile \`e pari all'eccesso di carico diviso 4 arrotondato
per eccesso.  Peso trasportabile e Carico vanno indicati nelle
apposite caselle della scheda del personaggio.

Per i quadrupedi il Peso trasportabile \`e il doppio di quello di un
PG con le stesse FOR e COS.

\es{``B'' \`e un PG con 20 in FOR e 20 in COS. Il suo peso
  trasportabile \`e di 50 kg; trasporta un carico totale (armatura
  compresa) di 60 kg. Oltre al malus per l'armatura (vedi) subisce una
  ulteriore penalizzazione di -3 perch\'e 60-50=10 e 10/4 = 2.5, che
  arrotondato per eccesso d\`a 3}

\pinupp{candelab.eps}{}{b}

\subsection{Punti Karma}

Questa innovazione tecnica ha lo scopo di premiare la fedelt\`a
della vostra recitazione al carattere e alla morale del PG.  Tanto
meglio esprimerete il carattere, le inclinazioni e la morale del PG,
tanto pi\`u realistica risulter\`a l'interpretazione, e tanto maggiori
saranno i PK che riceverete dal Master.

Lo scopo principale di un gioco di ruolo, come sar\`a ripetuto pi\`u
volte in altre parti di questo manuale, \`e a nostro avviso la
recitazione. Non bisognerebbe infatti basare la vita del proprio PG
solamente sulla fortuna nel lancio dei dadi, ma entrare nei panni del
Personaggio e giocarlo come se esso vivesse veramente all'interno del
mondo creato dal Master.

\iffullversion 
Al momento della creazione, il PG ha 0 Punti Karma.
Alla fine di ogni sessione di gioco, il Master assegner\`a ad ogni
giocatore un numero di Punti Karma variabile fra -5 e +5. Un punteggio
negativo verr\`a assegnato a giocatori con scarsa capacit\`a di
immedesimazione, lo zero corrisponde ad una recitazione e
interpretazione sterile, senza infamia e senza lode. Punteggi positivi
verranno assegnati a giocatori con una buona interpretazione.

I PK assegnati si sommano (o si sottraggono) a quelli gi\'a presi
precedentemente, accumulandosi sessione dopo sessione.

I Punti Karma determinano, per il PG all'interno del gioco,
conseguenze \textbf{immediate} (Punti Aura), o effetti
\textbf{post-mortem} (reincarnazione).

PK e Aura vanno annotati sulla scheda del personaggio nelle apposite
caselle.

\subsubsection{Aura}

L'effetto immediato che i PK provocano, consiste nel far sviluppare
un'Aura al PG. Essa si concretizza nella fortuna che il ``destino''
gli concede. \textbf{L'Aura \`e dipendente dai PK nel senso che per ogni +20
PK si avr\`a diritto ad 1 Punto Aura.} I Punti Aura cos\`{\i}
accumulati potranno essere utilizzati dal giocatore per migliorare il
risultato di un'azione (compresi Tiri Volont\`a, Tiri Resistenza, Tiri
sulle Caratteristiche, ecc.) o per incrementare un danno inflitto. In
entrambi i casi bisogner\`a decidere prima del lancio dei dadi,
\textbf{se} utilizzare o meno i Punti Aura e \textbf{quanti}
utilizzarne.
\nb{I Punti Aura non possono essere utilizzati per migliorare i Tiri
  Incremento.}
I Punti Aura utilizzati verranno persi e potranno essere recuperati al
ritmo di 1 ogni 30 giorni. Prima di leggere gli esempi che seguiranno
\`e opportuno, per ben comprenderne il significato, leggere il
paragrafo \ref{azioni} ``Le Azioni'' ed il capitolo
\ref{combattimento} ``Il Combattimento''.

\es{Martin ha 63 PK, a cui corrispondono 3 Punti Aura, ed \`e
  inseguito da 3 predoni. Sfortunatamente la sua via di fuga \`e
  interrotta da un crepaccio.  Martin decide che se vuole salvarsi
  deve provare a saltarlo. Il crepaccio \`e largo 5 metri e la
  difficolt\`a di saltare tale distanza \`e 25. Martin ha un TOT
  di 10 nell'abilit\`a Saltare (5 + 5 BON AGI).
  
  Sapendo che il salto \`e particolarmente difficile decide di
  utilizzare tutti i suoi Punti Aura in questa azione, potendo
  cos\`{\i} migliorare il tiro del dado di 3 punti.  Lancia il dado ed
  ottiene 13, totalizzando complessivamente: 10 (TOT nell'abilit\`a
  Saltare) + 13 (risultato del d20) + 3 (Punti Aura utilizzati) = 26.
  Il risultato ottenuto \`e maggiore della difficolt\`a: Martin
  riesce a saltare il crepaccio ed a continuare a fuggire. Senza
  l'utilizzo della sua aura sarebbe caduto nel crepaccio perch\'e il 13
  realizzato col dado non gli sarebbe stato sufficiente.  }

\es{Masamune Nakamura, sta combattendo contro un orso molto grosso
  (COS 25) e sa che deve ucciderlo senza dargli la possibilit\`a di
  attaccare. Egli ha 89 PK che corrispondono a 4 Punti Aura. Decide di
  utilizzare 2 di questi 4 Punti Aura per aumentare il danno della sua
  spada bastarda che sta utilizzando a 2 mani.
  
  Il Danno da lui inflitto sar\`a cos\`{\i} pari a: 2d8 + 2 (danno
  della spada bastarda) + 5 (BON FOR di Masamune) + 2 (Punti Aura
  utilizzati).  Masamune ha colpito l'orso alla testa. Poich\'e
  l'animale ha 15 PSC alla testa \`e necessario (per essere sicuro
  di ucciderlo) che Masamune gli infligga 15 punti di danno. Tira i
  dadi per infliggere e ottiene: 9 (risultato di 2d8 + 2) + 5 (BON
  FOR) + 2 (Punti Aura) = 16. Il risultato \`e sufficiente ad
  uccidere l'orso sul colpo }

\subsubsection{Reincarnazione}

Qualora il vostro PG nonostante tutto, alla fine, dovesse soccombere,
\`e probabile che voi, giocatori incalliti, manifestiate il
desiderio di continuare a giocare, creandovi un nuovo personaggio.

Accade spesso ai ``vecchi'' giocatori di ruolo che, per non
abbandonare il deceduto amatissimo personaggio, lo ricreino uguale
adducendo come motivazione che si tratti del fratello gemello
scomparso, tenuto prigioniero, nascosto, ecc.

Capita inoltre che un giocatore si penta del personaggio che ha creato
e lo interpreti male di proposito, cercando di farlo morire al pi\`u
presto, finendo per rovinare l'avventura che il Master, con tanto
impegno, aveva preparato, per potersi creare il prima possibile un
nuovo personaggio.

Per evitare questi ``spiacevoli inconvenienti'', \`e stato
introdotto, in questo sistema di regole, il processo della
\textbf{reincarnazione}, che lega il vecchio personaggio deceduto al
nuovo che sta per nascere, nel quale i PK svolgono un ruolo
preminente.  

La reincarnazione non \`e una tappa obbligata nella storia di un PG.
Questo, una volta morto pu\`o avere molteplici destini: la sua anima
potrebbe per esempio essere conquistata dagli esseri del Limbo, degli
Inferi o dell'Empireo.  Tuttavia la reincarnazione rappresenta uno dei
destini pi\`u probabili.

Al momento della creazione di un nuovo PG i PK accumulati dal vecchio
PG possono essere spesi per migliorare le Caratteristiche o le
abilit\`a del nuovo PG, secondo un fattore di conversione specificato
nella seguente tabella.

\begin{center}
\begin{tabular}{|l|l|}
\hline
20 PK & 1 Punto CAT \\ \hline
20 PK & 1 Punto Psiche \\ \hline
5 PK & 1 Punto per le Abilit\`a \\ \hline
4 PK & 1 Punto Energia \\ \hline
2 PK & 1 Punto \% \\ \hline
\end{tabular}
\end{center}

Questa possibilit\`a di conversione \`e un incentivo per il giocatore
affinch\'e si dedichi alla recitazione e all'interpretazione del PG.
Nel caso in cui l'ammontare dei PK del vecchio PG dovesse essere
negativo, dovr\`a essere il Master a decidere (basandosi sempre sulla
tabella soprastante), quali penalit\`a attribuire al nuovo PG che il
giocatore si appresta a creare.

\es{ Dopo una lunga e movimentata vita, Pyp muore con un numero di
  punti Karma pari a +170. Il giocatore che lo recitava, decide per il
  suo nuovo PG, la seguente conversione dei Punti Karma: + 3 Punti CAT
  aggiuntivi (60 PK) = cio\`e aumenta di 3 punti il valore dei punti
  CAT che aveva precedentemente definito col lancio dei dadi. +10
  Punti ENErgia (40 PK)= cio\`e aumenta di 10 punti il valore dei
  Punti Energia determinato dal precedente lancio di dadi. +10 Punti
  per le Altre Abilit\`a (50 PK)= cio\`e aumenta di 10 punti i punti
  disponibili per le Altre Abilit\`a (gi\`a definiti nel modo
  descritto nel paragrafo dedicato alle Abilit\`a).  }

\es{Diegus, a causa della scarsa interpretazione del suo giocatore,
  muore con -60 PK. Il Master decide di penalizzare il nuovo
  personaggio del giocatore convertendo i suoi PK nel seguente modo:
  
  $-20 PK\rightarrow -1$ Punto Psiche. Diminuisce di 1 punto il valore
  dei Punti Psiche definiti col lancio dei dadi.

  $-40 PK\rightarrow -2$ Punti CAT (20 PK x 2). Toglie 2 punti
  alla somma dei punti CAT ottenuti dal lancio dei dadi.
  }
\fi

{\raggedright \subsection{I Tiri Resistenza, Volont\`a e Psiche}}

Quando il vostro personaggio \`e molto affaticato o subisce uno shock
psichico \`e opportuno effettuare dei Tiri, a difficolt\`a variabile in
funzione delle circostanze, che indicano se siete in grado di andare
avanti, dovete subire dei danni mentali o fisici, oppure se dovete
semplicemente riposarvi.

\subsubsection{Il Tiro Resistenza (TR)}

Viene effettuato in linea di massima in caso di affaticamento fisico,
per resistere al caldo ed al freddo, per resistere ad incantesimi di
tipo ``fisico'', quando il personaggio perde un numero elevato di
punti vita o punti struttura ed in ogni altro caso previsto dal
regolamento. 

Il Master determina la difficolt\`a del tiro se questa
non \`e fissata da una precisa regola (per classificare la
difficolt\`a vedere tabella Difficolt\`a nel paragrafo ``Le
azioni''), il giocatore tira \textbf{1d20 e somma il risultato al valore della
Caratteristica Secondaria: RESistenza}.  Normalmente la difficolt\`a
\`e Media (Diff. 20). Se il totale \`e maggiore o uguale alla
difficolt\`a il tiro riesce e il personaggio potr\`a evitare
spiacevoli conseguenze.

\subsubsection{Il Tiro Volont\`a (TV)}

Viene effettuato in linea di massima in caso di affaticamento mentale,
per resistere a incantesimi di tipo ``mentale'', quando il personaggio
deve fare determinate azioni ``controvoglia'', ed in ogni altro caso
previsto dal regolamento o dal Master.

Come per il TR il Master determina la difficolt\`a, ma questa volta
il giocatore \textbf{al risultato del tiro del d20 deve sommare il valore
della Caratteristica: Volont\`a}.  Anche in questo caso normalmente
la difficolt\`a \`e Media (Diff. 20).  Se il totale \`e maggiore
o uguale alla difficolt\`a, il tiro riesce.


\subsubsection{Il Tiro Psiche (TP)}

Viene effettuato quando il personaggio \`e sottoposto ad uno
\textbf{shock emotivo} come un grave lutto, uno spavento o una
situazione che pu\`o determinare paura. Il Master determina la
difficolt\`a; \textbf{il giocatore tira 1d20 e somma il risultato al
  valore di PSIche}. Se il totale \`e maggiore o uguale alla
difficolt\`a il tiro riesce e il personaggio evita lo shock.  In caso
contrario pu\`o trovarsi shockato e sar\`a il Master a determinare le
conseguenze. Potrebbe per esempio accadere di restare bloccati per un
certo numero di round o svenire.

Se lo shock \`e giudicato dal Master molto forte (difficolt\`a 30 o
pi\`u) e il tiro viene fallito, il personaggio perder\`a \textbf{un
  Punto Psiche ogni 2 punti di differenza} fra la difficolt\`a e il
risultato ottenuto (arrotondando per difetto) e dovr\`a essere
sottoposto a cure naturali o incantesimi per recuperarlo.
Orientativamente serve una settimana di cura da un medico per ogni
punto Psiche perso. Il medico dovr\`a realizzare Tiro Medicina a
difficolt\`a pari a 15 + i punti Psiche persi.

\es{Eric, un Nordico con un valore di 16 in Psiche, si sta guardando
  allo specchio quando alle sue spalle appare un Benair. A causa della
  mostruosit\`a dell'essere (BEL $-2$) e dell'apparizione inaspettata,
  il Master stabilisce che il TP deve essere effettuato con una
  difficolt\`a di 35. Eric, con il suo tiro totalizza 28 (16 + 12
  col d20). Egli rester\`a terrorizzato e perder\`a (35 - 28)/2 =
  3 Punti Psiche.}

\subsection{Altri tiri sulle Caratteristiche}

Per determinare se il personaggio \`e in grado o meno di compiere
una certa azione o di avere particolari idee, il Master potr\`a
chiedere al giocatore di effettuare un Tiro su una qualunque
Caratteristica primaria.

Il Master determina la difficolt\`a, il giocatore lancia 1d20 e somma
al risultato ottenuto il valore della Caratteristica utilizzata. Se il
risultato \`e maggiore o uguale alla difficolt\`a l'azione riesce.

Gli esempi pratici che seguono descrivono le situazioni tipo in cui
vi imbatterete durante le sessioni di gioco.

\subsubsection{Tiro Osservazione (TOSS)}

Il Master reputa che in una stanza ci sia un particolare di un quadro
particolarmente interessante ma difficile (difficolt\`a 25) da notare.
I PG effettuano un Tiro OSServazione (TOSS); solo coloro che
totalizzeranno almeno 25 (1d20+ caratteristica OSS) riusciranno a
scorgere tale particolare.

\subsubsection{Tiro Intelligenza (TINT)} 

Il Master reputa che ad uno dei personaggi possa venire in mente
qualcosa che il giocatore non ha pensato. In questo caso pu\`o
decidere di far realizzare un Tiro INTelligenza (TINT) e, se il tiro
riesce, di fornire tale informazione al PG.

\subsubsection{Tiro Concentrazione (TCONC)} 

Il PG si deve ricordare un particolare di un discorso, una formula,
ecc (che magari il giocatore non ricorda). Il Master fornir\`a
l'informazione soltanto nel caso che il Tiro CONCentrazione (TCONC) a
una difficolt\`a da lui stabilita riesca.

I TCONC vengono utilizzati soprattutto per gli incantesimi e le Arti
Marziali (vedi capitoli ``La Magia'' e ``Il Combattimento'').

Per una dose di maggior realismo pu\`o essere necessario che sia il
Master a realizzare tali tiri sulla Caratteristica di ogni PG.

\nb{In ogni caso (TP, TR, TV, TINT, TOSS, TCONC,  ecc.) se la difficolt\`a
  del tiro \`e minore o uguale al valore della caratteristica, la
  riuscita del tiro \`e automatica (cio\`e l'azione riesce anche
  senza tirare il dado) }

\section{Le Abilit\`a}
\label{abilita}

Il vostro PG, anche nella sua vita precedente l'inizio del gioco,
avr\`a certamente imparato a compiere numerose attivit\`a e sar\`a
addestrato in diverse discipline: le sue \textbf{Abilit\`a}.

Sulla scheda del Personaggio, accanto ad ogni abilit\`a di
Combattimento e Standard, sono presenti 5 caselle:

\begin{description}
\item{\bf D} Indica il codice di difficolt\`a, cio\`e la
  difficolt\`a di apprendimento di un'abilit\`a (D1, D2 o D3).
  Maggiore \`e il codice di difficolt\`a, pi\`u \`e complicato
  l'apprendimento.  Questa regola vale sia per la Creazione del
  personaggio che per i successivi miglioramenti durante il gioco.
  
  Al momento della Creazione del Personaggio, Il Codice di
  Difficolt\`a indica quanti punti bisogna utilizzare per incrementare
  il VAL di 1 quando dividete i punti delle Abilit\`a di Classe
  sociale e i punti per le Altre abilit\`a. In particolare si
  spender\`a un punto per aumentare il VAL di 1 punto in quelle
  abilit\`a con difficolt\`a D1 (Allarme, Anatomia, Archeologia, ecc);
  si spenderanno 2 punti per aumentare il VAL di 1 in quelle abilit\`a
  con difficolt\`a D2 (per es. Acrobazia) e si spenderanno 3 punti per
  aumentare il VAL di 1 in quelle abilit\`a con difficolt\`a D3 (per
  es. Medicina).
  
  Per il resto della vita del PG, il codice di difficolt\`a (D)
  specifica un malus per la realizzazione del Tiro incremento (vedi
  paragrafo \ref{miglioramento} a pagina \pageref{miglioramento} ``Il
  Miglioramento'').
  
\item{\bf CAT} Indica la caratteristica a cui \`e correlata
  l'abilit\`a. La caratteristica attribuisce all'abilit\`a, un
  bonus o un malus (BON) secondo la tabella BONUS e MALUS delle
  Caratteristiche.
  
\item{\bf VAL} In questa casella dovrete indicare il valore che
  intendete attribuire all'abilit\`a, sempre tenendo conto del
  codice di difficolt\`a. Se per esempio volete attribuire 12 al VAL
  in un'abilit\`a a D3 dovrete utilizzare 36 punti.
  
\item{\bf BON} In questa casella dovrete indicare il bonus o il
  malus della caratteristica correlata (CAT), cio\`e il numero che
  avete indicato sulla Scheda del Personaggio nella casella BON al
  fianco di ogni caratteristica.  Il bonus pu\`o essere utilizzato
  solo se avete attribuito almeno 1 al VAL dell'abilit\`a.
  
\item{\bf TOT} In questa casella dovrete riportare la somma di VAL e
  BON. Il TOT rappresenta la vostra capacit\`a di compiere azioni che
  richiedono quell'Abilit\`a:  \`e il valore che dovrete sommare al
  tiro del dado per la riuscita dell'azione che avete in mente di
  compiere.

\end{description}

Al momento della creazione del personaggio bisogna distinguere tra
quattro tipi di abilit\`a:

\begin{enumerate}
  \itemsep -6pt
\item Abilit\`a di base
\item Abilit\`a di razza
\item Abilit\`a di classe sociale
\item Altre abilit\`a; divise in: (A) Abilit\`a di combattimento e
  standard; (B) Abilit\`a magiche
\end{enumerate}

\filler{libro1.eps}

\subsection{Abilit\`a di base}

I punteggi delle Abilit\`a di base vengono attribuiti a tutti i
personaggi per il solo fatto di esistere. I VAL di base sono fissati
al momento della creazione.  Nella scheda del personaggio, tali valori
di base vengono indicati fra parentesi accanto all'abilit\`a stessa.
Queste abilit\`a con il loro VAL di base sono riportati nella
tabella \ref{tababibase} a pagina \pageref{tababibase}.

\begin{table}[t]
\begin{center}

{\Large\sc Abilit\`a di base\medskip}

\begin{tabular}{|l|c|}
\hline
\iffullversion Abilit\`a di base & VAL \\ \hline\hline
Anatomia & 2  \\ \hline
Bere & 1 \\ \hline \fi
Cadere & 1 \\ \hline
\iffullversion Controllo respirazione & 1 \\ \hline
Empatia & 2 \\ \hline \fi
Fare nodi & 2 \\ \hline
\iffullversion Fiutare & 2 \\ \hline \fi
Geografia & 1 \\ \hline
\iffullversion Individuare menzogne & 2 \\ \hline
Mitologia & 2 \\ \hline 
Muoversi al buio & 1 \\ \hline \fi
Muoversi in silenzio & 2 \\ \hline
Nascondersi & 2 \\ \hline
Raggirare & 2 \\ \hline
\iffullversion Religione & 2 \\ \hline
Riconoscere & 2 \\ \hline \fi
Saltare & 5 \\ \hline
Scalare & 5 \\ \hline
\iffullversion Sedurre & 2 \\ \hline
Sfondare porte & 1 \\ \hline
Storia Generale & 3 \\ \hline \fi
Trattare & 2 \\ \hline
Udire & 4 \\ \hline
\end{tabular}
\end{center}
\caption{Abilit\`a di base}
\label{tababibase}
\end{table}

\subsection{Abilit\`a di razza}

I VAL delle Abilit\`a di razza vengono attribuiti a seconda della
razza di appartenenza del PG. Sono fissati al momento della scelta
della razza e possono essere sommati a quelli eventualmente gi\`a
presenti nelle Abilit\`a di base, come indicato nel capitolo ``Razze
ed etnie'', nelle descrizioni delle razze.

\subsection{Abilit\`a di Classe sociale}

Le abilit\`a di Classe sociale vengono specificate nel momento in
cui viene determinata la classe sociale del personaggio. Sono elencate
nelle descrizioni delle Classi Sociali, nel capitolo ``Razze ed
etnie''; il loro VAL viene deciso dal giocatore distribuendo, fra
esse, un numero di punti che dipende dal Fattore Istruzione, secondo
la seguente tabella:

\begin{center}
\begin{tabular}{|c|c|}
\hline 
Fattore&Punti\\
Istruzione & disponibili \\ \hline\hline
0.1& 80 \\ \hline
 0.2& 75\\ \hline
 0.3& 70 \\ \hline
0.4& 65 \\ \hline
0.5& 60\\ \hline
 0.6& 55 \\ \hline
0.7& 50\\ \hline
 0.8& 45\\ \hline
\end{tabular}
\end{center}

\textbf{Tutti questi punti devono essere suddivisi fra TUTTE le abilit\`a di
Classe Sociale e soltanto fra queste, con un minimo di 1 ed un massimo
di 12 per VAL dell'abilit\`a}. Se fra le abilit\`a di classe
sociale vi sono delle abilit\`a che fanno parte delle abilit\`a di
razza o delle abilit\`a di base, il VAL che intendete attribuirgli
andr\`a sommato a quello gi\`a attribuito a queste ultime e in
ogni caso non si potr\`a superare il 12.  

\nb{Nel dividere i punti fate molta attenzione al codice di
  difficolt\`a di ogni abilit\`a}

\subsection{Altre abilit\`a}

Le altre abilit\`a sono a scelta del giocatore e dipendono da
ci\`o che il PG ha fatto nel suo passato. Il loro VAL viene deciso
dal giocatore dividendo i punti a sua disposizione.

Il numero di punti che avete a disposizione \`e pari al totale dei
Punti CAT diviso otto pi\`u il punteggio di CONoscenza, il tutto
moltiplicato per 6.

$$(\frac{Punti\ CAT}{8} + CON) \times 6$$

Potete assegnare questi punti ad abilit\`a diverse da quelle che vi
sono state attribuite in partenza (Abilit\`a di combattimento,
Standard, Abilit\`a magiche, Specifiche di Arte Marziale o Maestrie
nelle Armi) o usarli per incrementare i VAL di Abilit\`a base,
Abilit\`a di razza e di Classe sociale.

\nb{Anche in questo caso, nel dividere i punti fate
  molta attenzione al codice di difficolt\`a di ogni abilit\`a}

\bigskip
\es{Il vostro PG ha 128 Punti CAT ed ha 13 in CONoscenza. I punti
  aggiuntivi da suddividere saranno 128 diviso 8 = 16 pi\`u 13 (CON)
  = 29 per 6 = 174.}

\subsubsection{Abilit\`a di combattimento e Standard}

Ogni Abilit\`a di Combattimento pu\`o avere un VAL massimo di 12
ed ogni Abilit\`a Standard un VAL massimo di 15 (comprendendo la
somma dei VAL anche le Abilit\`a di base, di classe sociale e di
razza).  Un discorso particolare meritano le Specifiche di Arte
Marziale e le Maestrie il cui TOT iniziale non pu\`o superare il 12.
Esse verranno descritte pi\`u dettagliatamente nel capitolo relativo
al Combattimento.

\subsubsection{Abilit\`a magiche}

Le abilit\`a magiche non sono correlate a nessuna caratteristica, di
conseguenza non gli viene attribuito nessun bonus ed il VAL
corrisponde al TOT.

Le abilit\`a magiche si dividono in Scuole di Magia e Liste Magiche
(vedi il capitolo ``La Magia''). Fra le 2 Scuole di Magia che \`e
possibile aver appreso al momento della Creazione del personaggio (non
possono essere pi\`u di 2 e tantomeno dello stesso Regno, vedi il
capitolo ``La Magia''), potete attribuire una somma di VAL di massimo
8, ci\`o significa che se conoscete per esempio Stregoneria ed
Elementalismo potete attribuire alla prima 5 di VAL e alla seconda 3,
oppure ad una 2 e all'altra 6, oppure 4 e 4, ecc.

Ciascuna delle Liste Magiche ha, indipendentemente dai punti
attribuiti alle Scuole di Magia, un limite massimo di 8 al VAL,
ci\`o significa che potete attribuire 8 al VAL di entrambe le liste,
pur conoscendo anche le Scuole di Magia.

Le Scuole di Magia hanno codice di difficolt\`a D3. Le Liste Magiche
hanno D2. Per quanto riguarda la Scheda del Personaggio, troverete
indicate tutte le Scuole di Magia e le Liste Magiche nell'apposita 
sezione.

\begin{table*}
  \framebox{
    \parbox{0.95\hsize}{
      \small
      Alle \textbf{Abilit\`a di Base} vengono attribuiti dei VAL
      fissati, a tutti i Personaggi, prescindendo dalla razza di
      appartenenza. 
      \medskip\hrule\medskip
      Alle \textbf{Abilit\`a di Razza} vengono attribuiti dei VAL
      fissati a seconda della Razza di appartenenza. Alle Abilit\`a di
      Classe Sociale devono attribuirsi dei VAL utilizzando i Punti che
      dipendono dal Fattore Istruzione, facendo attenzione al Codice di
      Difficolt\`a ed attribuendo almeno 1 VAL ad ognuna delle abilit\`a
      indicate. 
      \medskip\hrule\medskip      
      Alle \textbf{Altre Abilit\`a} (che comprendono TUTTE le Abilit\`a
      di Combattimento, Standard e Magiche, comprese quelle gi\`a indicate
      nelle Abilit\`a di Base, di Razza e di Classe Sociale) devono
      attribuirsi dei VAL utilizzando i Punti ottenuti dalla formula (Punti
      CAT/8 + CON) x 6 facendo attenzione al Codice di Difficolt\`a delle
      Abilit\`a stesse. 
    }
  }
\caption{Riassunto delle Abilit\`a}
\end{table*}

\nb{\`E importante notare che alcune abilit\`a conferiscono un
  bonus ai tiri su altre abilit\`a o caratteristiche correlate.  In
  tal caso si parla di Abilit\`a di Specializzazione.}

\subsection{Abilit\`a di Specializzazione}

Le cosiddette Abilit\`a di Specializzazione sono quelle che, in
funzione del loro TOT, forniscono dei bonus all'utilizzo di altre
abilit\`a.  Questi bonus sono indicati nella Tabella Specializzazioni.

L'utilizzo delle abilit\`a di specializzazione verr\`a spiegato in
dettaglio quando parleremo delle Abilit\`a di Combattimento e delle
abilit\`a Standard.

Le abilit\`a di specializzazione sono molto pi\`u importanti di
quanto si pensi ed occupano un ruolo fondamentale quando dovete
utilizzare le Specifiche di Arte Marziale o di MAEstria.

In linea di massima se in un'abilit\`a (considerata di
specializzazione), avete un TOT di 16 avrete alle abilit\`a
collegate un BONus di +4, se avete un TOT di 12 avrete +3, ecc.

I bonus assegnati per ciascun valore del TOT sono riportati in tabella
\ref{tabspecializzazioni}

\begin{table}[hbt]
\centering

{\Large\sc Specializzazioni\medskip}

\begin{tabular}{|c|c|}
\hline
TOT dell'abilit\`a& Bonus \\ \hline\hline
1-4&+1\\ \hline
5-9& +2\\ \hline
10-14& +3\\ \hline
15-19& +4\\ \hline
20& +5\\ \hline
oltre 20& +5 +1 per ogni \\
&5 punti oltre 20 \\ \hline
\end{tabular}
\caption{Specializzazioni}
\label{tabspecializzazioni}
\end{table}

\subsection{Le Azioni}
\label{azioni}
Le Abilit\`a servono per far compiere al vostro PG determinate
azioni, la cui difficolt\`a \`e stabilita dal Master e
contrassegnata da un valore secondo la tabella seguente. 

\textbf{Per tentare l'azione bisogna tirare 1d20; l'azione ha successo se il
risultato del dado, sommato al TOT dell'abilit\`a che deve essere
usata per svolgere l'azione (sommando anche gli eventuali bonus o
malus), eguaglia o supera il valore di difficolt\`a}. 

Il grado di difficolt\`a \`e dato da particolari tabelle o viene
deciso dal Master, cos\`i come l'abilit\`a da usare per portare a
termine l'azione. Tuttavia, in linea di massima, si pu\`o attribuire ad
ogni difficolt\`a il valore riportato in tabella \ref{tabdifficolta}.

\begin{table}[ht]
  \centering

  {\Large\sc Difficolt\`a\medskip}

  \begin{tabular}{|l|c|}
    \hline
    Difficolt\`a (Diff.)& Valore\\ \hline\hline
    Banale& 5\\ \hline
    Facile& 10\\ \hline
    Normale& 15\\ \hline
    Medio& 20\\ \hline
    Difficile&25\\ \hline
    Molto Difficile& 30\\ \hline
    Difficilissimo& 35\\ \hline
    Improbabile& 40\\ \hline
    Quasi impossibile& 45\\ \hline
    Follia& 50\\ \hline
  \end{tabular}
  \caption{Difficolt\`a}
  \label{tabdifficolta}
\end{table}

Le difficolt\`a sono attribuite per azioni da compiersi in condizioni
\textbf{normali}.

Eventuali difficolt\`a aggiuntive dipendenti dal PG non fanno
aumentare la difficolt\`a oggettiva ma conferiscono dei Malus ai tiri
come indicato nel regolamento nelle varie circostanze.  

In questo modo quindi cadere da un'altezza di 6 metri (vedi abilit\`a
Cadere) avr\`a sempre e comunque una difficolt\`a di 30, qualsiasi
siano le condizioni in cui il PG si trova (problemi alle gambe, di
salute, di stanchezza, armatura, ecc.). Ci\`o che muta l'esito del
tiro sono i \textbf{bonus o i malus} di cui il personaggio dispone.

Per le \textbf{Azioni contrapposte} si dovranno confrontare i risultati
dei tiri tra di loro.  Solo l'agente che realizzer\`a il punteggio
maggiore riuscir\`a nell'azione; ad esempio: Raggirare contro
Individuare Menzogne; Azioni di Combattimento, ecc.

Un risultato di 1 al tiro di dado provoca il \textbf{fallimento catastrofico}
dell'azione.

\bigskip
\nb{Se la difficolt\`a dell'azione, stabilita dal Master, \`e minore o
  uguale al TOT dell'abilit\`a del personaggio, la riuscita
  dell'azione \`e automatica e non \`e necessario tirare il dado}

\bigskip
\es{Il PG ha un TOT di 15 in un'abilit\`a che ha intenzione di
  utilizzare. Il Master stabilisce che la difficolt\`a dell'azione
  \`e 10 (facile).  l'azione riesce automaticamente senza bisogno di
  tirare il dado perch\'e 10 \`e minore di 15.}  

\`E possibile utilizzare le abilit\`a anche qualora non si sia
attribuito ad esse nessun VAL (ossia quando si ha 0 nel TOT). In
questo caso verr\`a contato semplicemente il valore del dado.
  
Se il valore ottenuto col lancio del dado da 20 (\textbf{1d20}) supera
o eguaglia la \textbf{difficolt\`a oggettiva} determinata dal Master,
\textbf{l'azione ha successo}.
  
In questo caso, se l'azione riesce, il Master dovrebbe far segnare al
PG 1 punto sul VAL dell'abilit\`a usata.

Il Master pu\`o conferire degli ulteriori Bonus ai PG, per esempio
quando l'azione che si intende svolgere \`e abitudinaria, quando si
hanno degli attrezzi che possono aiutare a svolgerla, ecc.

Ricordate che per tutte le azioni in movimento, al tiro va sempre
sottratto il Malus per il peso dell'armatura (vedi il Capitolo
\ref{combattimento}, ``Il Combattimento'').

Per ``Azione in movimento'' si intende un'azione per cui occorra
spostarsi o effettuare movimenti rapidi, come un attacco o una difesa
in combattimento.

\subsection{Il Round}

Tutte le azioni si svolgono in un certo lasso di tempo. Quello pi\`u
breve si chiama \textbf{Round} e dura \textbf{circa 3 secondi}, per
cui 20 round equivalgono ad 1 minuto. 

\subsection{Abilit\`a Standard}

\iffullversion
\abi{Acrobazia}{2}{AGI} Permette di compiere evoluzioni, salti
mortali, equilibrismo, ecc. \`E anche un'abilit\`a di specializzazione
per Schivare (vedi il paragrafo Le Arti Marziali e le Maestrie nelle
armi nel capitolo ``Il Combattimento'') e le conferisce un bonus
secondo la tabella Specializzazioni. Una capriola ha difficolt\`a 10,
un salto mortale in avanti ha difficolt\`a 25, un ``flick''
all'indietro ha Diff. 20.

\es{Masamune Nakamura, signore delle terre orientali, vestito con
  armatura di piastre, ha un valore di 17 (12+ bonus AGI +5) ed
  intende compiere un salto mortale all'indietro, il Master decide che
  tale azione \`e difficile (Diff. 25). Masamune, avendo un -5 ai
  tiri sulle azioni in movimento per colpa dell'armatura di piastre,
  con 1d20 dovr\`a ottenere almeno 13. Tira ed ottiene 16. l'azione
  \`e riuscita. Se Masamune avesse ottenuto meno di 13, l'azione
  sarebbe fallita.  Se avesse tirato 1, la sua azione avrebbe potuto
  avere degli effetti catastrofici (cadere fratturandosi il braccio o
  peggio ecc.)}

\pinupp{alambicco.eps}{}{b}

\abi{Alchimia}{3}{CON} Permette di riconoscere, fabbricare e
manipolare composti per ottenere veleni, droghe, pozioni, e
medicinali. Per veleni, droghe ed erbe medicinali vedi le tabelle a
pagina \pageref{tabdroghe}.  Per la somministrazione delle erbe
medicinali l'azione ha difficolt\`a variabile come segue:

\begin{itemize}
  \itemsep -6pt
\item Ingestione: 5
\item Infuso : 10
\item Applicazione: 15
\end{itemize}

\es{Alice Ryan, un elfo Umbra di Malagana, ranger, ha un valore di 18
  (18 + bonus CON +0) ed ha bisogno di preparare una droga che
  permetta di resistere alla stanchezza (resistenza + 10 per tre ore),
  senza effetti collaterali. Come da Tabella Costruzione Droghe,
  l'azione ha Diff. 32. Ad Alice \`e sufficiente ottenere 14, tira ed
  ottiene 16, la sua droga \`e pronta. Se avesse ottenuto meno di
  14, la sua droga non avrebbe avuto effetto. se Alice avesse lanciato
  1, il suo preparato avrebbe potuto sortire effetti differenti (un
  veleno, una tisana contro i reumatismi ecc.)}

\es{Xeres Sum ha trovato un'erba che deve essere somministrata
  preparando un infuso.  Per riuscire a somministrarla correttamente
  deve realizzare un tiro Alchimia a difficolt\`a 10.}



\abi{Allarme}{1}{OSS} Consente di evitare di essere colti di
sorpresa. Per ragioni di realismo \`e pi\`u conveniente che il
tiro venga effettuato dal Master. 

Si pu\`o usare in contrapposizione a Muoversi in Silenzio, altrimenti
la difficolt\`a \`e decisa dal Master

\es{Kurasai Hidenaga, ninja delle terre orientali, sta cavalcando
  nella foresta e sta per cadere vittima di un agguato. Kurasai ha un
  valore di 13 (11+bonus OSS +2).  il Master decide che individuare
  l'agguato \`e normale (Diff. 15).  \`E sufficiente per Kurasai non
  ottenere 1 per riuscire. Il Master ottiene 3, Kurasai si accorge
  dell'esistenza di un pericolo (non di quale pericolo) e decide di
  fuggire al galoppo. Se il Master avesse lanciato 1 Kurasai non si
  sarebbe avveduto di nulla e sarebbe caduto in trappola}

\abi{Am\-mae\-stra\-re}{2}{CON} Attitudine ad ammaestrare gli animali. Per
la difficolt\`a di ammaestrare ogni animale vedi la Tabella
\ref{tabanimali} Alcuni animali non possono essere ammaestrati: per
es. insetti, pesci, ecc.

\es{Zantor Megresh, un cavaliere di Norda, decide di ammaestrare un
  cavallo. Zantor ha un TOT di 3. il Master decide la difficolt\`a
  in base all'animale da ammaestrare, come indicato nella tabella
  \ref{tabanimali}, per il cavallo sar\`a Media (Diff. 20), il che
  implica la necessit\`a di ottenere almeno 17 con il dado
  affinch\'e l'azione abbia esito positivo.
  
  Zantor tira ed ottiene 9, il suo tentativo non \`e andato a buon
  fine. Nel caso in cui il verdetto del dado fosse stato 1,
  probabilmente l'animale avrebbe sviluppato una notevole antipatia
  per Zantor}

\abi{Anatomia}{1}{CON} \textbf{2 di Base.} Indica la conoscenza
dell'anatomia umana e umanoide. \`E indispensabile per portare alcuni
colpi di arti marziali.  Indovinare l'esatta posizione del cuore ha
Diff. 10, della milza e del fegato 15, del pancreas 20. Maggiore \`e
la specificit\`a, pi\`u difficile sar\`a il Tiro.

\`E anche un'abilit\`a di Specializzazione per l'abilit\`a
Medicina e le conferisce un bonus, come da tabella Specializzazioni.

\es{Takashi Horuru, un Signore delle terre orientali, ha un TOT di 7
  (6+bonus CON. +1) e ha necessit\`a di effettuare un colpo mirato
  allo Tsubo Ji-hin che si trova esattamente alla base del setto
  nasale.  Per conoscere esattamente il punto dovr\`a effettuare un
  tiro anatomia difficile (Diff.: 25).  Dovr\`a ottenere quindi almeno
  18 con il dado. Il risultato ottenuto \`e invece 13; non ricorda
  quindi l'esatta dislocazione dello tsubo e non potr\`a quindi
  colpirlo nel punto desiderato. Se avesse ottenuto almeno 18 avrebbe
  invece avuto la possibilit\`a di effettuare il colpo mirato.
  Ottenendo invece 1 sarebbe stato convinto ad esempio che lo Tsubo si
  trovasse dietro la lingua, o sotto l'alluce, ecc.}

\abi{Archeologia}{1}{CON} Permette di ritrovare e riconoscere reperti
molto antichi o di civilt\`a scomparse. 

La difficolt\`a \`e in
rapporto all'antichit\`a dei reperti. Diff. 15 Fino a 400 anni prima
del ritrovamento, Diff. 20 Fino a 700 anni, Diff. 30 fino a 2000 anni,
Diff. 40 fino a 15000 anni, Diff. 50 fino a 50000 anni 

\es{Lars Sannhom, un nobile di Terranova, ha un TOT di 7 (5+ bonus CON
  +2), trova una vecchia spada e vuole capire a quale civilt\`a e a
  quale periodo risalga. 

  Il Master decide che questa operazione \`e
  Normale in quanto essa non \`e molto antica (Diff. 15). Sar\`a
  quindi sufficiente, affinch\'e l'azione abbia successo, realizzare
  un 8 sul dado. Lars ottiene 9, sufficiente al buon esito e scopre
  che la spada risale a circa 400 anni prima ed \`e di fattura
  elfica. 
  
  Non realizzando l'8 non sarebbe stato in grado di stabilire
  alcunch\'e, mentre con un fallimento catastrofico (1) avrebbe
  grossolanamente sbagliato la sua valutazione}

\abi{Architettura}{1}{CON} Permette di progettare edifici e
riconoscere lo stile e la fattura di quelli esistenti. 

Progettare una
capanna di legno ha difficolt\`a 10, una casa ad un piano 15, un
palazzo 25, un duomo di un certo valore artistico 30-35.

\es{Lars Sannhom ha un TOT di 19 (17+ BONus CON +2) e vuole capire lo
  in quale stile \`e stato costruito un antico palazzo. Per fare ci\`o
  il Master ha stabilito che l'azione \`e Molto Difficile (Diff. 30).
  Col dado ottiene 17 e scopre che la costruzione \`e di uno stile
  antichissimo, avente i connotati tipici di una setta eretica di adoratori
  di Baal. Con un 1 avrebbe potuto dire che si trattava di una chiesa
  sacra a Ryless e sicuramente di recente fattura.}

\abi{Astrologia}{2}{CON} Permette di stilare oroscopi e indovinare le
personalit\`a con larga approssimazione.  La probabilit\`a di
prevedere correttamente un evento \`e pari al TOT dell'abilit\`a in
percentuale. La difficolt\`a \`e sempre 20.

\es{Edwig Than Lufthas, un elfo Luxi di Bahuney, ha un TOT di 18 (15
  +Bonus CON +3) e vuole fare un oroscopo ad un viandante (Diff.  20).
  Edwig tira il dado ottenendo 12, per un totale di 30. La previsione
  avrebbe una probabilit\`a del 18\% di avverarsi.
  
  Il Master tira segretamente 1d100 ottenendo 7, che \`e minore di 18,
  facendo s\`i che la previsione di Edwig sia esatta.  Il Master sa
  che il futuro dell'uomo sar\`a roseo: egli incontrer\`a una
  ricchissima donna che si innamorer\`a di lui, perci\'o suggerir\`a
  ad Edwig di predire all'uomo fama e ricchezza.}

\abi{Astronomia}{1}{CON} Permette di calcolare direzioni e grandi
distanze esaminando le stelle e i corpi celesti. Sapere dove tramonta
il sole ha Diff. 5 , sapere dove si trovano le principali
costellazioni Diff.10, sapere i periodi in cui sono visibili
determinate costellazioni Diff. 20, calcolare grandi distanze con
l'aiuto di attrezzi Diff.  30, calcolare le distanze ad occhio nudo
Diff. 50.

Conferisce un bonus all'abilit\`a Orientamento come da tabella
Specializzazioni.

\es{Ulzath Savegolth, un elfo Luxi di Bahunei, ha un TOT di 17 (15 +
  bonus CON +2) viene rapito e trasportato bendato in un posto a lui
  sconosciuto. Riuscito a liberarsi vuol capire:

\begin{enumerate} 
  
\item a che distanza si trova dalla sua terra (Diff. 50 poich\'e non
  possiede strumenti in grado di calcolare le distanze e il Master sa
  che la sua terra dista 300 km)
  
\item verso quale direzione deve andare per tornarvi.
\end{enumerate}

Nel primo caso lancia 1d20 e ottiene 15, non abbastanza cio\`e per
superare la Diff. Se avesse lanciato un 1 avrebbe probabilmente
creduto di trovarsi in un continente sconosciuto.  Nel secondo caso,
avr\`a semplicemente un bonus di +4 all'abilit\`a Orientamento}

\abi{Balistica}{1}{CON} Permette di calcolare e prevedere le
traiettorie dei proiettili; utile per colpire bersagli non in linea di
tiro, es. dietro un ostacolo, da dentro una trincea senza sporgersi
ecc.  Per colpire il bersaglio sar\`a poi necessario un TPC (vedi
capitolo Il Combattimento) a difficolt\`a 20.  Se il tiro balistica
non viene realizzato la difficolt\`a del TPC sar\`a di 35. In caso di
Fallimento Catastrofico il successivo TPC sar\`a in ogni caso fuori
bersaglio.

Se il tiro Balistica ha successo, conferisce un bonus ad Artiglieria
secondo la tabella Specializzazioni.

\es{Masamune Nakamura ha un TOT di 15 (12 + bonus CON +3), vuole
  colpire un'avversario nascosto in una trincea e non vuole sporgersi
  dal suo nascondiglio. Il Master determina una Diff. di 30.  Masamune
  lancia 1d20 e ottiene 17 riuscendo a calcolare perfettamente la
  traiettoria che la freccia del suo arco dovr\`a compiere.
  Occorrer\`a poi un TPC a difficolt\`a standard (20). Se non fosse
  riuscito nel suo calcolo avrebbe effettuato il TPC a Diff. 35.  Nel
  caso avesse ottenuto 1 avrebbe calcolato una traiettoria
  completamente errata mancando il bersaglio qualunque punteggio
  avesse effettuato nel TPC. Se avesse voluto colpire con una
  catapulta, e fosse riuscito a calcolare la traiettoria del
  proiettile, avrebbe usufruito di un bonus +4 nel TPC.}

\abi{Barare}{2}{AGI} Facilita la vittoria nel gioco d'azzardo.  Il
baro pu\`o per\`o essere individuato con un tiro Osservazione a
difficolt\`a pari ad un tiro sull'abilit\`a Barare.  Conferisce un
bonus alla abilit\`a Giocare d'azzardo secondo la tabella
Specializzazioni ma se si decide di usare il bonus si pu\`o venire
scoperti.

\es{Pyp, uno gnomo Kunetha di Loydi Genya, ha un TOT di 20 (15 + bonus
  AGI + 5), usufruir\`a quindi di un bonus +5 all'abilit\`a
  giocare d'azzardo. Sta giocando a carte con un barista che ha 12 in
  Osservazione e decide di barare; al tiro Barare totalizza 35: 20 in
  barare + 15 (1d20). Il suo avversario lancia 1d20 su Osservazione
  ottenendo 3; l'imbroglio di Pyp non viene scoperto e tutti i tiri
  successivi su Giocare d'azzardo usufruiscono del bonus +5. Se il
  tiro Osservazione fosse riuscito (oppure Pyp avesse ottenuto un 1
  col dado), Pyp sarebbe stato smascherato}

\abi{Bere}{1}{COS} Resistenza all'alcool e ai liquori. Bere un
bicchiere di vino senza ubriacarsi ha difficolt\`a 10 cos\`{\i} come
un bicchierino di un superalcolico, bere un litro di vino ha Diff. 20,
un litro di idromele ($70^o$) Diff. 35. Se il tiro non viene
realizzato il PG avr\`a un malus a tutti i tiri pari alla differenza
tra la Diff. e il tiro, per 4d6 di ore. Se tale differenza
\`e maggiore di 10 si deve realizzare un TR a difficolt\`a 40 per
non cadere addormentati. Un Fallimento Catastrofico al Tiro Bere
pu\`o far entrare la vittima in coma etilico. 

Conferisce un bonus
all'abilit\`a Enologia come da Tabella Specializzazioni. 

\es{Thalas Sorokasaal, un gigante, nato a Mahn-Hesa, ha un TOT di 25
  (15 + bonus COS +10). Una sera alza un po' il gomito e si scola 2
  litri di idromele (un liquore molto alcolico: 70 gradi); il Master
  conferisce al Tiro una Diff. di 40. Thalas lancia 1d20 e ottiene 15.
  Riesce nel suo intento poich\'e 25 + 15 \`e uguale a 40. l'alcool
  che ha in circolo non gli ha fatto assolutamente alcun effetto. Con
  un tiro pari a 35 si sarebbe ubriacato (con malus di (40-35)=-5 a
  tutti i tiri per 4d6 ore). Se avesse ottenuto un 1 col dado, il suo
  organismo avrebbe reagito negativamente e sarebbe entrato in coma
  etilico (per un numero di giorni deciso dal Master in base alla
  difficolt\`a del tiro...)} 

\abi{Blaterare}{1}{CAR} Permette di
confondere l'interlocutore pronunciando rapidamente frasi prive di
senso.  Conferisce un bonus all'abilit\`a Raggirare secondo la
tabella Specializzazioni.
\fi

\abi{Borseggiare}{2}{AGI} Abilit\`a propria del ladro tagliaborse.
Consente di sottrarre oggetti ad una persona senza essere scoperti.
Rubare i soldi dal piattino di un cieco ha Diff. 5, scippare una
persona con una borsa bene in vista Diff. 15, rubare un sacchetto di
monete assicurato ad una catenella Diff. 30.

Se il tiro non raggiunge la difficolt\`a decisa dal Master, la vittima
potr\`a effettuare un TOSS di difficolt\`a pari al punteggio ottenuto
dal borseggiatore nel tiro per accorgersi del tentativo di furto.

\es{Pyp ha un TOT di 20 (15 +
bonus AGI +5), e decide di compiere un certo tipo di prelievo ad un
ricco signore, alle porte di Rylex. La sua vittima porta alla cintola
un sacchetto tintinnante, ma \`e legato strettamente con una piccola
catena (Diff. 30). Pyp lancia 1d20 e ottiene 12 trovandosi CASUALMENTE
in mano lo stesso tintinnante sacchetto, senza che il signore possa
notare alcunch\'e.  Se avesse ottenuto un punteggio totale inferiore a
30, la vittima avrebbe potuto effettuare un TOSS, a difficolt\`a
pari al Punteggio ottenuto da Pyp. Se Pyp avesse lanciato un 1, avrebbe
fallito miseramente e sarebbe stato scoperto.}


\abi{Cadere}{1}{AGI} \textbf{1 di base.} Capacit\`a di cadere da
grandi altezze senza farsi male. La difficolt\`a \`e data dalla
tabella seguente:

\smallskip
\begin{center}
  \begin{tabular}{|rl|r@{ metri }|}
    \hline
    \multicolumn{2}{|c|}{Diff.}& Altezza \\ \hline\hline
    5& Banale& 1.5 \\ \hline
    10& Facile& 2 \\ \hline
    15& Normale & 2.5 \\ \hline
    20& Medio& 3 \\ \hline
    25& Difficile& 4\\ \hline
    30& Molto Difficile& 6\\ \hline
    35& Difficilissimo& 8 \\ \hline
    40& Improbabile& 10\\ \hline
    45& Quasi impossibile& 15 \\ \hline
    50& Follia& 20  \\ \hline
  \end{tabular}
\end{center}
\smallskip

Se il tiro non supera la Diff., il PG subisce un danno di 1d6
ogni 2 metri nelle parti del corpo interessate nella caduta (1d4 di
numero di parti interessate), che verranno determinate dal Master
tirando sulla tabella \ref{tabfantoccio} ``Il Fantoccio'' nel capitolo
\ref{combattimento} ``Il Combattimento''. Un fallimento catastrofico
provoca il massimo possibile dei danni in 4 parti del corpo.

\es{William Wicker, un allevatore di Terranova, ha un
TOT di 12 (7 + bonus AGI +5). Cade dal fienile mentre accatasta la
biada, da un'altezza di 4 metri. Lancia il dado e ottiene 13. Il
suo atterraggio sar\`a perfetto ed egli non subir\`a danni. Se
avesse ottenuto, in totale, meno della Diff. avrebbe subito un danno
di 1d6 ogni 2 metri nelle parti del corpo interessate dalla caduta. Se
avesse ottenuto un 1 la caduta gli avrebbe provocato il massimo dei
danni possibile (in questo caso 12) in 4 parti del corpo.}

\iffullversion
\abi{Calcolare}{1}{INT} Saper fare di conto rapidamente e con
precisione. Pu\`o essere utilizzata per calcolare il rimbalzo di
certi incantesimi e per definire l'epicentro di un'esplosione ed il
numero di persone interessate ad essa. Fare una moltiplicazione di
numeri con due cifre (a mente) ha difficolt\`a 15. Calcolare il
rimbalzo di un incantesimo o l'epicentro della sua esplosione ha Diff.
20.

\abi{Cantare}{1}{CAR} Abilit\`a canora. Il PG sceglie la
difficolt\`a del brano che intende eseguire prima di effettuare il
tiro.  Una ninna nanna ha difficolt\`a 5, una canzone popolare 15,
un'opera lirica 30.

\abi{Carpenteria}{1}{CON} Padronanza delle tecniche di lavorazione del
legno e capacit\`a di costruire strutture con esso. Intagliare un
ciondolo ha Diff. 15, riparare un carro 20, scolpire una statua 25.

\es{Alice Ryan, un elfo umbra di Malagana, ha un TOT di 7 (2 + bonus
AGI +5); vuole intagliare un ciondolo per la sua amata Haryell. Il
Master decide che la difficolt\`a \`e normale (15).  Lancia 1d20 e
ottiene 9. Il ciondolo riesce perfettamente.}

\abi{Cartografia}{1}{CON}  Saper
leggere mappe e tracciare mappe di luoghi visitati o stimare
l'affidabilit\`a di mappe gi\`a tracciate. 

La difficolt\`a viene stabilita dal Master in relazione all'attenzione
riposta nell'osservazione del luogo visitato.

\es{Justin O'Connor, un commerciante di stoffe di Terranova, ha un TOT
  di 12 (10 + bonus CON +2) e deve tracciare una mappa delle terre
  orientali nella zona di Komaha, che ha da poco visitato. Il Master
  stabilisce che l'azione ha Diff. 20. Justin tira il dado e realizza
  un 8. La mappa riesce sufficientemente precisa. Se avesse realizzato
  un 1 la mappa sarebbe stata completamente sballata, confondendo
  probabilmente le idee a coloro che avessero provato a utilizzarla in
  seguito (-3 ai tiri su Orientamento e Geografia)}
\fi

\abi{Cavalcare Alati}{2}{AGI} Capacit\`a di condurre esseri volanti. La difficolt\`a standard
\`e 15. In caso di evoluzioni o manovre particolari \`e il Master
a decidere la difficolt\`a

\abi{Cavalcare Bipedi}{1}{AGI}
Capacit\`a di condurre bipedi.  Condurre un animale al passo ha
difficolt\`a 10, al trotto 15, al galoppo 20.

\abi{Cavalcare Quadrupedi}{1}{AGI} Capacit\`a di condurre
quadrupedi. Condurre un animale al passo ha difficolt\`a 10, al
trotto 15, al galoppo 20. 

\iffullversion
\abi{Conoscere Angeli}{1}{CON} Indica il livello di conoscenza degli
esseri dell'empireo (Vedi il Capitolo ``Le Scuole di Magia'' al
Paragrafo ``Gli Angeli'').
La difficolt\`a del tiro su questa abilit\`a dipende dalla potenza
del singolo angelo.
\fi

\abi{Conoscere Animali}{1}{CON} Indica il livello di conoscenza di
specie animali. (vedi la Tabella a pagina \pageref{tabanimali}). Per
ogni animale \`e indicata una diversa difficolt\`a del Tiro
Conoscere Animali.

\iffullversion
\abi{Conoscere Animali Magici}{1}{CON} Indica il livello di conoscenza
di Animali Magici. (Vedi il Capitolo ``Le Scuole di Magia'' al
Paragrafo ``Gli Animali Magici'').  La difficolt\`a del tiro su questa
abilit\`a dipende dalla potenza dell'animale magico in questione.

\abi{Conoscere Demoni}{1}{CON} Conoscenza degli esseri degli inferi
(Vedi il Capitolo ``Le Scuole di Magia'' al Paragrafo ``I Demoni'').
La difficolt\`a del tiro su questa abilit\`a dipende dalla potenza
del demone.

\abi{Conoscere Elementali}{1}{CON} Conoscenza degli Elementali (Vedi
il Capitolo ``Le Scuole di Magia'' al Paragrafo ``Gli Elementali'').
La difficolt\`a del tiro su questa abilit\`a dipende dalla potenza
degli elementali.

\abi{Conoscere Esseri del Limbo}{1}{CON} Conoscenza dei non-morti,
delle anime erranti, spiriti irrequieti abitanti il limbo. (Vedi il
Capitolo ``Le Scuole di Magia'' al Paragrafo ``Gli Esseri del
Limbo'').  La difficolt\`a di conoscere la creatura dipende dalla sua
potenza.


\abi{Conoscere Esseri Onirici}{1}{CON} Conoscenza delle creature della
Dimora Onirica (Vedi il Capitolo ``Le Scuole di Magia'' al Paragrafo ``Gli
Esseri Onirici'') La difficolt\`a di conoscere ogni Essere Onirico
dipende dalla sua potenza.
\fi

\abi{Conoscere Piante}{1}{CON} Conoscenza della botanica. La
difficolt\`a di trovare una pianta \`e indicata nella tabella Erbe a
pagina \pageref{taberbe}. Per ogni Pianta \`e indicata una diversa
difficolt\`a del Tiro Conoscere.

\iffullversion
\abi{Contatti}{1}{CAR} Conoscenza personale di persone influenti,
malavitosi, informatori o semplicemente amici nella propria o in altre
citt\`a. La difficolt\`a \`e decisa dal Master a seconda
dell'estrazione sociale e del background delle persone coinvolte. Per
uno schiavo delle terre orientali figlio di schiavi, la difficolt\`a
di conoscere di persona l'arcivescovo di Rylex ha Diff.  50, per un
ambasciatore Umbra a Terranova la difficolt\`a di conoscere il Duca di
Arethin \`e 20, ecc.

\abi{Contorsionismo}{2}{AGI}
Capacit\`a di controllare il proprio corpo, di aumentare gli angoli
di apertura delle articolazioni, di entrare in spazi molto stretti. La
difficolt\`a \`e decisa dal Master in rapporto alla corporatura di
chi la utilizza. Conferisce un bonus alla specifica di AM Divincolarsi
(vedi il paragrafo ``Le Arti Marziali e le Maestrie nelle Armi'' nel
capitolo ``Il Combattimento'') come da tabella Specializzazioni.

\abi{Controllo del battito cardiaco}{3}{CONC} Permette di variare la
frequenza del proprio battito cardiaco.  Diminuendolo si pu\`o
simulare uno stato di morte (Diff. 35), accelerandolo permette di
ottenere un bonus di +1 ai TPC, per ogni 4 punti oltre il 20. Ogni 5
round consecutivi di accelerazione occorrer\`a realizzare un TR a
difficolt\`a 20 + il bonus al TPC ottenuto, pena un numero di d6 PV
di danno netto pari a tale bonus. Un fallimento catastrofico provoca
una crisi cardiaca (svenimento, coma o morte).  

\es{Alice Ryan, elfo Umbra, con un TOT nell'abilit\`a pari a 18 (13+
  bonus CONC +5), durante una perlustrazione rimane bloccato dentro
  una caverna molto poco ampia; sa che i soccorsi arriveranno soltanto
  l'indomani, ma l'aria a disposizione non gli baster\`a.  Decide
  allora di provare ad entrare in catalessi rallentando il suo battito
  cardiaco. Il Master stima una difficolt\`a di 35; Alice tira 1d20
  ottenendo 17.
  
  Per il rotto della cuffia ce la fa; si sveglia la mattina dopo nel
  suo letto essendo riuscito a restare in vita fino all'arrivo dei
  soccorsi.  Se avesse ottenuto un punteggio inferiore non sarebbe
  riuscito nell'intento, il suo battito cardiaco sarebbe rimasto
  uguale e probabilmente sarebbe morto soffocato. Ottenendo un 1
  sarebbe sicuramente morto per una crisi cardiaca}

\es{Masamune Nakamura \`e coinvolto in un combattimento contro due
  guerrieri di una fazione avversa. Ha un TOT di 15 (11 +bonus CONC
  +4). Per aumentare la sua velocit\`a di movimento prova ad
  accelerare il suo battito cardiaco.  Tira il dado ed ottiene 15; in
  totale 30, quindi avr\`a a disposizione un +2 al TPC per 5 round.
  Dopo 5 round \`e ancora nel pieno del combattimento, quindi
  realizza un TR a difficolt\`a 20+2.  Lancia un dado e ottiene 8;
  essendo la sua resistenza pari a 20 il TR riesce. Dopo altri 3 round
  il combattimento si conclude; 
  
  Masamune riporta il suo battito cardiaco alla normalit\`a, quindi
  non effettua ulteriori TR. Se avesse fallito il primo tiro il suo
  battito cardiaco non avrebbe subito modifiche; se avesse totalizzato
  un 1 sarebbe stato vittima di un infarto. Se non avesse realizzato
  il TR avrebbe subito 2d6 PV di danno netto per problemi di
  circolazione sanguigna e il suo cuore sarebbe ritornato al ritmo
  normale}

\abi{Controllo respirazione}{2}{CONC} \textbf{1 di base.} Consente di variare
il ritmo di respirazione e di restare in apnea da fermi per un numero
di minuti pari a:
\bigskip
\begin{center}
  \begin{tabular}{|rl|r|}
    \hline
    \multicolumn{2}{|c|}{Difficolt\`a}& Tempo\\ \hline\hline
    5& Banale& 30 secondi \\ \hline
    10& Facile& 1 minuto \\ \hline
    15& Normale & 1.5 minuti \\ \hline
    20& Medio& 2 minuti \\ \hline
    25& Difficile& 3 minuti\\ \hline
    30& Molto Difficile& 4 minuti\\ \hline
    35& Difficilissimo& 5 minuti \\ \hline
    40& Improbabile& 7 minuti\\ \hline
    45& Quasi impossibile& 9 minuti \\ \hline
    50& Follia& 10 minuti  \\ \hline
  \end{tabular}
\end{center}
\bigskip

In movimento la durata \`e dimezzata. Per ogni round successivo al
tempo trascorso si effettuer\`a un TR a difficolt\`a 20 + il numero
dei round trascorsi. Il fallimento del TR porta allo svenimento.  Un
fallimento Catastrofico pu\`o portare alla morte.

\es{Takashi Horuru, Signore,
  con 10 nel TOT dell'abilit\`a (5+ bonus CONC+5), naviga da solo su
  una barca sul lago Yatsura; un rollio inaspettato fa cadere in acqua
  il suo Naginata. 
  
  Il lago in quel punto \`e profondo 10 metri e stima che ci voglia
  1 minuto per recuperarla, visto che il fondo del lago \`e fangoso;
  si toglie l'armatura, si tuffa in apnea totalizzando 10 col dado (20
  in totale) quindi ha 1 minuto di autonomia in movimento.  Realizza
  un tiro sull'abilit\`a Nuotare e risale senza problemi dopo un
  minuto con il suo Naginata. Se fosse dovuto restare per 1.5 minuti
  in movimento (con lo stesso tiro) non sarebbe riuscito a restare
  immerso e sarebbe dovuto ritornare in superficie, oppure avrebbe
  potuto tentare di ``forzare'' l'apnea per altri 10 round,
  realizzando un TR a difficolt\`a 30.

  Fallendolo, avrebbe perso conoscenza e sarebbe morto annegato. Tirando
  1 col dado sarebbe morto sul colpo per una sincope.}

\abi{Criptografia}{2}{INT} Conoscenza di codici e linguaggi
segreti. Se utilizzata con successo permette di usufruire di aiuti e
tracce per decodificare i messaggi o di codificare con efficienza un
messaggio. Per decodificare un messaggio senza la chiave bisogna
realizzare un tiro con difficolt\`a pari al punteggio ottenuto da
chi ha codificato il messaggio. 

Chiavi incomplete possono fornire dei
bonus. 

Il tempo necessario a decifrare un messaggio, in minuti,
\`e pari alla difficolt\`a.

\es{Nimh, bardo Levante di
  Dyshu-Thandy, ha un TOT di 7 (2+ Bonus INT +5), riceve uno strano
  messaggio dopo il suo ultimo concerto. Prova a decifrarlo tirando 12
  col dado, in totale 19. Non capisce niente, in quanto colui che ha
  crittografato il messaggio aveva ottenuto 25. 
  
  Pochi minuti dopo riceve
  un biglietto anonimo con le istruzioni per decodificarlo; purtroppo la
  chiave \`e incompleta e non gli consente di decifrarlo direttamente
  ma gli conferisce un bonus di +10 al tiro successivo; Nimh totalizza 9
  con il dado ottenendo in tutto 7+9+10=26. Il contenuto del messaggio
  \`e scoperto. Se avesse ottenuto 1 col dado avrebbe ricavato
  informazioni completamente sbagliate.}

\abi{Cucinare}{1}{CON} Abilit\`a culinaria. Abbrustolire una
bruschetta ha difficolt\`a 5, un'insalata difficolt\`a 10,
arrostire un animale difficolt\`a 15, una ricetta degna di uno chef
30.

\es{Ty Neen, gnoma di Loydy-Genya, vuole preparare un succulento pasto
  a base di carote e cavoli per il suo fidanzato Maursh. La gnoma ha
  10 (9+1 bonus CON) nell'abilit\`a. Il Master decide che la
  difficolt\`a di questa azione \`e 35 (avete idea di quanto sia
  difficile preparare un SUCCULENTO pasto a base di CAROTE e CAVOLI?).
  Tira il dado e ottiene 15. l'azione non riesce perch\'e 10+15=25. Il
  risultato \`e qualcosa di praticamente immangiabile.

  Ty ripiega allora su un ricco piatto di frutta fresca. Se avesse
  ottenuto 1, avrebbe fatto correre al suo Maursh il rischio di una
  intossicazione alimentare}

\abi{Danzare}{1}{AGI}  Attitudine alla danza. Muoversi
a tempo di musica senza conoscere i passi ha difficolt\`a 5. In
generale c'\`e una difficolt\`a particolare per ogni tipo di
ballo.
\begin{itemize}
\itemsep -3pt
\item Lango, Danza Levante, Diff. 10
\item Sentago, Danza Umbra, Diff. 15 
\item Folka, Danza Luxi, Diff.  15
\item  Kaltzer, Danza Novese, Diff. 20 
\item Pockenk\`ol, Danza gnomica, Diff. 30
\end{itemize}

\es{Seinen Laftinais, elfo Luxi, conosce ad una festa una splendida
  dama; per sedurla la invita a ballare una Folka (nota danza Luxi).
  Seinen ha 12 in danzare (7+5 bonus AGI). Il Master sa che la
  difficolt\`a di questo ballo \`e 15. Seinen tira e ottiene 5,
  facendo una dignitosa figura.
  
  Se avesse tirato 1, avrebbe sbagliato i passi, confondendoli con
  quelli di un Kaltzer (diffusissimo ballo di origine Novese), oppure
  sarebbe cascato a terra. In entrambi i casi avrebbe suscitato
  l'ilarit\`a generale}

\abi{Empatia}{3}{OSS} \textbf{2 di base} Permette di capire i
sentimenti e gli stati d'animo dell'interlocutore. Pu\`o essere usato
come azione contrapposta ad un tiro Recitare o Raggirare, se la
vittima cerca di mascherare le proprie emozioni. In condizioni normali
il tiro \`e sempre a difficolt\`a 15.

Conferisce un bonus come da tabella Specializzazioni all'abilit\`a
Individuare Menzogne.

\abi{Enologia}{1}{CON} Conoscenza dei vini e degli alcolici e
delle loro modalit\`a di produzione. Riconoscere se si tratta di
vino, birra o superalcolici ha difficolt\`a 5. Distinguere un tipo
di vino da un altro ha difficolt\`a 10. Dare informazioni sulle loro
modalit\`a di produzione ha difficolt\`a 20. Riconoscere le giuste
proporzioni dei diversi vini in un cocktail ha difficolt\`a 35.
\fi

\pinupp{birra.eps}{}{b}

\abi{Fare Nodi}{1}{AGI} \textbf{2 di base.} Capacit\`a di annodare ed
intrecciare funi e simili. Un nodo per scarpe ha difficolt\`a 5, un
cappio difficolt\`a 10, legature (per es. per imbarcazioni) Diff.  20.

\abi{Fiutare}{2}{OSS} \textbf{2 di base.} Permette di individuare gli
odori e di riconoscerli. Attraverso l'uso continuato di questa
abilit\`a \`e possibile seguire tracce olfattive.

Riconoscere un odore particolarmente evidente ha Diff. 5, distinguere
due odori in mezzo ad altri Diff. 20, annusare un oggetto e seguire la
traccia per scovarne il padrone Diff. 35.

\iffullversion
\abi{Galateo}{1}{CAR} Conoscenza delle regole di comportamento prescritte nell'ambiente
nobiliare e delle formalit\`a necessarie a dare una buona
impressione di se stessi. Se realizzate un 10 vi prenderanno per un
campagnolo rozzo, con 20 riuscirete a fare bella figura anche in una
corte importante, con 35 tutti staranno a guardare con ammirazione il
vostro portamento.
\fi

\abi{Geografia}{1}{CON} \textbf{1 di base} Conoscenza della morfologia
del territorio e dei tratti economico-politici dell'Arcipelago.
Conoscere la posizione della propria citt\`a rispetto alla regione di
appartenenza ha difficolt\`a 10. Conoscere l'altezza del monte pi\`u
alto della propria regione 20. Sapere l'esatto numero di abitanti di
una citt\`a lontana ha difficolt\`a 25.

\iffullversion \abi{Geologia}{2}{CON} Conoscenza delle tipologie dei
minerali, delle pietre e dei terreni. Riconoscere un terreno fertile
ha Diff. 10, Classificare i vari strati di un carotaggio Diff. 20,
stimare correttamente la produttivit\`a di un giacimento minerario a
prima vista Diff. 30.

\abi{Giocare d'azzardo}{2}{INT} Abilit\`a che permette di
conoscere le regole, le probabilit\`a di vincita di un gioco
d'azzardo.  Si fa un confronto fra i tiri sull'abilit\`a dei
giocatori d'azzardo.

\es{Pyp, uno gnomo di Loydi Genya, ha un
  punteggio in giocare d'azzardo di 15 (10 + 5 Bonus INTelligenza) e
  sfida, al gioco delle Tre Carte, Martin, un elfo Umbra di Malagana che
  ha un punteggio di 8 (3 + 5 BONus INT). Martin scommette 1 Corona che
  riuscir\`a ad indovinare dove andr\`a a finire la carta Rossa. 
  
  Pyp tira ed ottiene 32 (17 col dado + 15), Martin col dado non va
  oltre 12 e quindi totalizza 20. La carta che ha indicato era quella
  sbagliata e Pyp pu\`o tranquillamente prendersi la Corona che si
  \`e guadagnato}

\abi{Giocoleria}{2}{AGI} Abilit\`a che consente di manipolare sfere,
clavette, coltelli, piatti, e gli oggetti pi\`u disparati. Roteare 2
oggetti ha Diff. 5, 3 oggetti Diff.  10. 

Per ogni 2 oggetti aggiuntivi la Diff. aumenta di 5. Conferisce un
bonus all'abilit\`a ADL come da tabella Specializzazioni.

\abi{Giu\-rispru\-denza}{1}{CON} Conoscenza delle leggi e delle
disposizioni atte a salvaguardare l'ordine pubblico e i rapporti tra
gli individui. Conoscere le normali regole di convivenza ha Diff. 5,
conoscere le principali leggi del proprio paese ha Diff. 15. Conoscere
il diritto del proprio paese ha Diff. 20. Sapersi muovere nei meandri
di leggi e sentenze anche di paesi stranieri ha Diff. 35.

\abi{Imitare}{2}{OSS} Saper imitare i versi degli animali e dei suoni
pi\`u comuni.  Imitare lo scricchiolio di una porta ha Diff.  5, saper
imitare un'animale domestico ha Diff.10, riprodurre la voce di una
persona Diff. 20, imitare il ruggito di un animale feroce (leone per
es.) Diff. 30. Pu\`o essere usata in contrapposizione con l'abilit\`a
Riconoscere o Conoscere animali.

\abi{Incoccare Veloce}{2}{AGI} Consente di non perdere il round per
incoccare una freccia sull'arco o sulla balestra. La difficolt\`a \`e
sempre 20, tranne che nei casi specificatamente indicati nelle note
della Tabella Armi.

\es{Alice Ryan ha appena scoccato una freccia ma il suo nemico
  continua ad avanzare. Alice vuole scoccare un'altra freccia senza
  perdere un round per incoccarla. Alice ha un TOT di 12 in
  quest'Abilit\`a.  Tira il dado e ottiene 9 totalizzando 21.
  Alice riesce nel suo intento e incocca la freccia senza perdere il
  round}

\abi{Individuare Menzogne}{3}{OSS} \textbf{2 di base} Permette di
capire se l'interlocutore sta mentendo, effettuando un confronto con il
tiro Raggirare dell'interlocutore. Se l'interlocutore dice la verit\`a
(quindi non usa l'abilit\`a Raggirare) il Master dovr\`a far eseguire
il tiro a Diff. 20.

\abi{Individuare Passaggi Segreti}{2}{OSS} Consente di scoprire porte
e passaggi nascosti. La difficolt\`a \`e stabilita dal Master a
seconda del tipo di passaggio segreto.
\fi

\abi{Individuare Tracce}{2}{OSS} Capacit\`a di scoprire tracce,
impronte, seguire piste, ecc.  Notare le impronte sul fango ha Diff.
5, individuare il passaggio di qualcuno attraverso i rami spezzati in
un bosco Diff. 20.

\abi{Individuare Trappole}{2}{OSS} Arte dello scoprire trappole
meccaniche e non, eccettuate quelle magiche. Si fa un confronto con il
tiro nell'abilit\`a Trappole con cui la trappola \`e stata
costruita, meno 5. Con un fallimento catastrofico si \`e sicuri che
non ci siano trappole.

\es{Alice Ryan ha un TOT di 13
  nell'abilit\`a e vuole sapere se alla porta che si appresta ad
  aprire \`e collegata una trappola. Il Master sa che la porta
  nasconde una trappola e sa che il suo costruttore aveva totalizzato
  nel suo tiro nell'abilit\`a Trappole, un punteggio di 30.  La
  difficolt\`a di individuarla \`e 30 - 5 ossia 25. Alice tira il
  dado e ottiene 15 e totalizza 28 (15 + 13). 
  
  Alice scopre che alla porta \`e collegato un filo che fa scattare
  una balestra. Se non avesse superato la difficolt\`a non avrebbe
  potuto dire se ci fossero state trappole o meno. Se avesse tirato 1
  sarebbe stato sicuro che non ci fossero state trappole}

\iffullversion
\abi{Ingegneria}{2}{CON} Abilit\`a nel progettare strutture,
meccanismi, costruzioni, ecc. Progettare una molletta ha Diff. 5,
progettare una catapulta Diff. 20, progettare un motore a vapore Diff.
50.

\filler{penna.eps}

\abi{Insegnare}{1}{CAR} Abilit\`a che consente di insegnare altre
abilit\`a e discipline.  Chi insegna effettua un tiro sull'abilit\`a
Insegnare a difficolt\`a pari a (25 - la differenza tra il TOT
dell'abilit\`a insegnata dell'insegnante e il TOT dell'abilit\`a
corrispondente dell'allievo), usufruendo di un bonus (o malus) al tiro
pari al BONus INTelligenza dell'allievo.

La riuscita del tiro determina l'apprendimento della materia insegnata
con tempi \textbf{dimezzati} rispetto a quanto si avrebbe senza insegnante
(quindi un Tiro Incremento ogni 15-20 ore).

\es{ Masamune ha un TOT di 13 nell'abilit\`a Insegnare ed un TOT di
  13 nell'abilit\`a Trappole.  Vuole insegnare a Takashi (BON INT di
  +5) un po' di trucchi del mestiere in quest\`abilit\`a visto che
  quest'ultimo non \`e molto bravo (TOT di 6). La differenza fra i
  TOT delle loro abilit\`a \`e pertanto di 7 (13- 6) La
  difficolt\`a del Tiro insegnare di Masamune \`e perci\`o pari
  a: 25 - 7 = 18.  Masamune tira il dado e ottiene 4. Totalizza
  perci\`o 13 (TOT dell'abilit\`a insegnare) + 5 (BON INT di
  Takashi) + 4 (risultato del d20) = 22. l'azione ha successo
  
  Dopo 20 ore di insegnamento Takashi avr\`a diritto ad un Tiro
  Incremento nell'abilit\`a Trappole con un bonus pari al Bonus come
  da tabella specializzazioni per l'abilit\`a Insegnare di Masamune,
  che in questo caso \`e pari a +3}

Nei casi in cui l'allievo debba effettuare un tiro per la ``verifica''
dell'apprendimento (come per esempio nella magia), usufruisce per quei
tiri di un bonus pari al BONus dell'abilit\`a Insegnare
dell'insegnante come da tabella Specializzazioni, se a costui riesce
il Tiro Insegnare.

Il fallimento fa s\`{\i} che i tempi di apprendimento siano pari a
quelli di studio senza insegnante.  \textbf{Nel caso in cui si insegni un
incantesimo o un Colpo Speciale di Arte Marziale, il fallimento del
tiro Insegnare causa il mancato apprendimento dell'incantesimo o del
Colpo Speciale.}

Un fallimento catastrofico comporta il \textbf{raddoppio} del tempo necessario
all'apprendimento e a un malus di -5 ai tiri incremento (o verifica
dell'apprendimento, v. ``La Magia'') delle nozioni apprese.

Non si pu\`o insegnare un'abilit\`a ad un allievo che abbia un
TOT maggiore o uguale al TOT dell'abilit\`a dell'insegnante. Per
esempio se l'insegnante ha un TOT di 16 nell'abilit\`a che intende
insegnare, l'alunno non pu\`o avere pi\`u di 15. Non si pu\`o
insegnare un'abilit\`a in cui si ha un TOT minore di 12.

\abi{Intimidire}{1}{CAR}  Abilit\`a che permette di incutere
timore anche non motivato nelle persone. Si utilizza effettuando un
confronto con il TV dell'interlocutore.

\es{Stephan de la Lune ha un
TOT di 15 nell'abilit\`a e vuole far paura al piccolo Pyp che ha 15
in volont\`a.  Tira il dado ed ottiene 10 totalizzando 25. Pyp
effettua un TV totalizzando 22 (15 VOL + 7 di dado). Stephan riesce
nel suo intento, intimorendo il povero Pyp.}

\abi{Ipnotismo}{3}{CON} Capacit\`a di influenzare un soggetto
attraverso l'ipnosi. Consente di impartire una serie di comandi se si
vince un confronto tra il Tiro Ipnotismo e il TV della vittima.

Non si possono far eseguire alla vittima azioni contro la sua morale o
la sua etica o contro il suo istinto di sopravvivenza. \`E necessario,
per tentare il tiro, focalizzare completamente l'attenzione del
soggetto su di s\'e per almeno 3 minuti consecutivi.

\`E possibile impartire ordini post-ipnotici collegati a
eventi particolari. Per eliminare gli effetti dell'ipnosi \`e
necessaria un'altra ipnosi.

\es{Vicious von Vinicious, un Nobile di Terranova, ha un TOT di 17
  nell'abilit\`a. Vuole ipnotizzare un'avvenente cameriera della
  sua locanda preferita in modo che ogni qualvolta egli pronunci la
  frase ``Il Solito'', lei gli si conceda (e Vicious sa che questo non
  \`e contro la morale di lei!). La ragazza ha 10 in Volont\`a.
  Vicious tira ed ottiene 25 (20 + 5 col dado, in virt\`u del tiro
  aperto) totalizzando 42. La ragazza realizza 19 col dado,
  totalizzando 29 al TV, punteggio non sufficiente ad evitare il
  comando post-ipnotico.
  
  Vicious da ora potr\`a chiedere e ottenere ``il solito'' ogni volta
  che lo desidera.}
  
\abi{Lavorare Metalli}{1}{CON} Abilit\`a nel lavorare i metalli.
Costruire un ferro di cavallo ha Diff. 10, la lama di un pugnale ha
Diff.  15, un'armatura di piastre Diff.  25, costruire un infisso in
metallo di grandi dimensioni Diff. 35.

\abi{Lavorare Pietre}{1}{CON}  Abilit\`a nel lavorare le pietre e i minerali. Scolpire il
gesso ha Diff. 10, il marmo Diff. 20, tagliare un diamante Diff.  30.

\pinup{pinza.eps}{}

\abi{Letteratura}{1}{CON}  Abilit\`a che indica la conoscenza di
testi, poesie, brani teatrali e simili. Indica anche la capacit\`a
di scriverli.  Il valore letterario dell'opera scritta \`e tanto
maggiore quanto maggiore risulta il Tiro nell'abilit\`a.
\fi

\abi{Lingua (Levante-Madre-Straniera)}{1}{CON} Conoscenza di un
particolare linguaggio scritto e parlato. Parlare una lingua
aiutandosi coi gesti e conoscere le principali parole e i verbi, senza
per\`o saperli coniugare,ha Diff.  5, parlare con una certa scioltezza
ma con la propria inflessione dialettale e saper scrivere
correttamente Diff.  15, parlare forbitamente e senza accento Diff.
25.

La lingua Levante \`e conosciuta da tutte le razze ed \`e usata come
lingua commerciale.

Per Lingua madre si intende la lingua ufficiale del proprio paese
d'origine.

Per lingua straniera si intende una lingua diversa dalla lingua madre
e dal Levante. 

Nella scheda sono indicate altre abilit\`a generiche di lingua che
dovranno essere associate alle altre lingue eventualmente apprese.

\abi{Medicina}{3}{CON} Abilit\`a medica: conoscenza
e capacit\`a di diagnosi e cura di malattie, fratture e danni fisici
di qualsiasi tipo ecc. La difficolt\`a di curare danni fisici \`e
pari al numero di PV o PSC persi pi\`u una costante che dipende
dalla tipologia del danno (TDA), come segue in tabella.

\bigskip
\noindent
\begin{center}
\begin{tabular}{|l|c|}
\hline
  Tipologia del Danno (TDA)& Difficolt\`a \\ \hline\hline
  Distorsioni - Lussazioni & 5 \\ \hline
  Ustioni& 10 \\ \hline
  Ferite - Emorragie& 15 \\ \hline
  Fratture& 20 \\ \hline
  Emorragie Interne& 25 \\ \hline
\end{tabular}
\end{center}
\bigskip

Le cure prestate fanno recuperare al personaggio malato i PV
persi in un numero di giorni pari alla difficolt\`a del Tiro Medicina
e di interrompere completamente la perdita di PV dovuta ai danni
addizionali (vedi paragrafo ``Danni Addizionali''). Ogni Tipologia di
DAnno necessita di un Tiro Medicina.

I danni addizionali vanno curati per primi. 

Un fallimento Catastrofico procura una ulteriore perdita di PSC e PV
pari alla difficolt\`a del Tiro Medicina diviso 5 (approssimando per
difetto).

Le difficolt\`a di curare malattie \`e riportata nella tabella Malattie a pagina
\pageref{tabmalattie}.

\es{Martin, un elfo Umbra, ha subito una ferita da 11 PSC alla gamba,
  che gli procura anche una perdita di 1 PV per round a causa
  dell'emorragia (danno addizionale). Roland van der Reitz, un nobile
  novese, con un TOT di 16 in Medicina, tenta di curarlo.
  
  Poich\'e le sue cure arrivano subito la difficolt\`a del tiro per
  curare l'emorragia (danno addizionale) sar\`a pari a: 15
  (tipologia di danno: emorragia) + 1 (PV persi) = 16. Poich\'e la
  difficolt\`a \`e uguale al TOT nell'abilit\`a Medicina di
  Roland, l'azione ha successo automatico e quindi si impedisce il
  prodursi di ulteriori danni addizionali (stabilizzazione). 
  
  Il Punto vita perso verr\`a recuperato in un giorno.
  
  Per curare la ferita, invece, la difficolt\`a \`e pari a: 15
  (Tipologia di danno: ferite) + 11 (PSC persi) = 26. Roland tira il
  dado e ottiene 10 totalizzando 26.  l'azione ha successo e Martin
  potr\`a recuperare i PV e i PSC persi in 26 giorni. 
  
  Se avesse realizzato un Fallimento Catastrofico (1 col dado) avrebbe
  causato alla vittima ulteriori 26 (Diff. del tiro) / 5 = 5 PSC di
  danno.} 

\iffullversion
\abi{Mitologia}{1}{CON} \textbf{2 di base.} Conoscenza dei
miti dell'antichit\`a. Difficolt\`a 15 per un mito risalente fino
a 400 anni prima, Diff. 20 fino a 700 anni, Diff. 30 fino a 2000 anni,
Diff.  40 fino a 15000 anni, Diff. 50 fino a 50000 anni.

\abi{Muoversi al buio}{2}{OSS} \textbf{1 di base.} Capacit\`a di
muoversi nell'oscurit\`a evitando eventuali ostacoli. Muoversi al buio
in un posto conosciuto Diff. 10, in un posto visto poche volte 20, in
un posto mai visto 30. Conferisce un bonus al Combattimento al buio
secondo la tabella Specializzazioni. 
\fi

\abi{Muoversi in silenzio}{2}{AGI} \textbf{2 di base.} Capacit\`a di
muoversi senza produrre rumore.  Si usa in contrapposizione con
l'abilit\`a Udire o Allarme.

\iffullversion
\abi{Musica}{2}{CON} Capacit\`a di leggere, scrivere e comporre musica e
dirigere una banda.  Leggere semplici melodie ha Diff. 10, leggere
partiture pi\`u complesse a prima vista Diff.  20, dirigere un
orchestra con molti elementi Diff. 30. L'importanza di un'opera
composta \`e tanto maggiore quanto pi\`u alto \`e il tiro realizzato
sull'abilit\`a.

\pinup{nimh.eps}{Nimh, bardo levante}
\fi

\abi{Nascondersi}{2}{AGI} Abilit\`a di
trovare ed utilizzare nascondigli per s\`e o per oggetti. Si
utilizza in contrapposizione al TOSS dell'osservatore.

\abi{Navigare}{1}{CON} Capacit\`a di condurre mezzi natanti.
Condurre una barca a remi Diff. 5, condurre una barca a vela Diff. 15,
condurre una piccola imbarcazione a pi\`u alberi Diff. 25, condurre
una nave Bahn-Yera Diff. 30

\abi{Nuotare}{1}{AGI} Capacit\`a di nuotare e di muoversi
nell'acqua. Stare a galla ha Diff. 5, nuotare nelle acque calme ha
Diff. 10, in acque mosse Diff. 20, in un mare in burrasca Diff. 30.
Per gare di nuoto si usano in contrapposizione le abilit\`a dei
concorrenti.

\iffullversion
\abi{Occultismo}{3}{CON}  Conoscenza dei riti, usi e
costumi esoterici. Conoscenza di antichi testi magici e loro possibile
dislocazione. Conoscere l'esistenza delle Scuole della propria terra
d'origine Diff. 10, conoscere l'esistenza delle Torri della Magia
Diff. 15, la loro dislocazione 30.

\abi{Oratoria}{2}{CAR} Capacit\`a di parlare in pubblico ed
influenzare gli auditori. L'azione ha successo se si effettua un tiro
abilit\`a con un punteggio superiore al TV realizzato dall'ascoltatore/i.
Conferisce un bonus all'abilit\`a Raggirare come da tabella
Specializzazioni.

\es{Hudibras Arethin, duca di Terranova, \`e a colloquio con due
  marchesi che vuole portare in guerra come alleati. Decide di
  convincerli con le sue doti di oratore. Hudibras ha un TOT di 17
  nell'abilit\`a (12 + bonus CAR +5) tira il dado ed ottiene 13 per un
  punteggio totale di 30. Il marchese di Tulhe ha 18 in VOLont\`a,
  tira il dado ed ottiene 14 totalizzando 32.
  
  Poich\'e il suo TV \`e
  maggiore del Tiro in Oratoria di Hudibras, egli non riterr\`a
  soddisfacenti le motivazioni addotte dal Duca per convincerlo e decide
  di dichiararsi neutrale. Il marchese di Montague ha 16 in Volont\`a,
  tira il dado e realizza 3 totalizzando 19. Egli resta favorevolmente
  impressionato dal discorso del Duca, dichiarandosi favorevole ad
  un'alleanza.}

\abi{Orienta\-mento}{2}{OSS} Capacit\`a di
trovare sempre la direzione voluta attraverso l'osservazione
dell'ambiente circostante. Sapere dov'\`e un punto cardinale ha
Diff. 10, orientarsi in una fitta foresta Diff. 20, orientarsi al
chiuso in posti sconosciuti ha Diff. 30.

\abi{Pre\-sti\-di\-gi\-ta\-zio\-ne}{2}{AGI} Abilit\`a manuale che consente di fare giochi di prestigio e
far sparire piccoli oggetti. Conferisce un bonus come da tabella
Specializzazioni in Borseggiare. Si utilizza in contrapposizione con
un Tiro Osservazione.

\es{Martin, saltimbanco di Umbrosa, decide di procurarsi i soldi per
  la cena allestendo un piccolo spettacolo.  Martin ha un TOT di 6 in
  quest'abilit\`a (1 + BON AGI +5), tira il dado ed ottiene 19,
  per un totale di 25.
  
  I due uomini al tavolo in cui si tiene l'esibizione effettuano un
  TOSS a Diff. 25 per avvedersi del trucco. Il primo ha 13 in OSS e
  tirando il dado ottiene 16 totalizzando 29; l'altro ha 19 in OSS e
  lanciando il dado ottiene 2, totalizzando 21. Il primo nota che
  Martin ha nascosto una carta nella manica della sua giacca, il
  secondo invece non si accorge di nulla e decide di ricompensare
  Martin con uno scudo.
  
  Poich\'e Martin sa che uno scudo non gli baster\`a per la cena,
  decide di borseggiare uno degli avventori e, grazie al suo 6 in
  prestidigitazione, usufruir\`a di un bonus di +2 all'abilit\`a
  Borseggiare.}
\fi

\abi{Prontezza}{1}{AGI} Consente di non perdere il round per sguainare
l'arma. La Diff. \`e sempre 20.

\es{Plumbrok, un orco Pran-horu,
  deve affrontare una guardia a cui ha vomitato addosso. La sua arma
  \`e per\`o ancora nel fodero. Poich\'e altrimenti perderebbe un
  round per poterla sguainare, decide di utilizzare l'abilit\`a
  Prontezza. Plumbrok ha un TOT di 9 in prontezza (3 + BON AGI +6). Tira
  il dado ed ottiene 12, totalizzando 21. Riesce ad avere la spada in
  mano senza perdere il round e pu\`o subito fronteggiare il suo
  avversario}

\iffullversion
\abi{Pronto soccorso}{1}{CON} Permette di prestare
il primo soccorso ad un ferito o ad un malato. Questa abilit\`a
permette di interrompere gli effetti dei Danni Addizionali per un
numero di ore pari al TOT dell'abilit\`a di chi la utilizza. Le
difficolt\`a sono quelle dell'abilit\`a Medicina.  Un fallimento
catastrofico causa 1d6 ulteriore di danni addizionali per i round
successivi.

\es{Martin ha subito un'ustione sul braccio che gli ha
  provocato dei danni addizionali pari a 1 PSC (e conseguente PV) per
  round. Prima che Xeres possa intervenire sono passati 5 round per cui
  la perdita totale ammonta a 5 PSC (PV). Xeres ha in pronto soccorso un
  TOT di 13.  La difficolt\`a del tiro \`e pari a: 10 (Costante per
  l'ustione) + 5 (PSC/PV persi prima dell'intervento) = 15. Tira il dado
  e ottiene 6, totalizzando 19.
  
  Il prodursi dei danni addizionali viene
  interrotto per 13 ore. Dopo tale periodo Xeres dovr\`a rinnovare la
  medicazione.}
\fi

\abi{Raggirare}{2}{CAR}  Permette di circuire le
persone per far valere le proprie ragioni e di mentire senza essere
scoperti. Si usa in contrapposizione con l'abilit\`a Individuare
Menzogne.

\iffullversion
\abi{Recitare}{1}{CAR}  Permette di eseguire brani
teatrali o di improvvisare secondo un canovaccio.  La difficolt\`a
dipende dall'opera che si intende interpretare. D\`a un bonus
all'abilit\`a Raggirare secondo la tabella Specializzazioni.

\abi{Religione}{1}{CON} \textbf{2 di base.} Conoscenza della propria o
delle altre religioni. Conoscere i principali dogmi ha Diff. 5,
conoscere sommariamente i testi sacri Diff. 10, sapere come si svolge
un particolare rito Diff. 20, conoscere alla perfezione un passo di un
testo sacro Diff. 30.
\fi

\abi{Resistenza ai Veleni}{1}{COS}
Capacit\`a di resistere ai veleni. Conferisce un bonus al TR contro
i veleni, come da tabella Specializzazioni.

\abi{Resistenza al Caldo}{1}{COS} 
Capacit\`a di resistere al caldo. Conferisce un
bonus al TR per il caldo, come da tabella Specializzazioni.

\abi{Resistenza al Dolore}{2}{COS} Capacit\`a di resistere al dolore,
alle torture ecc.  Conferisce un bonus al TV contro le Torture e al TR
contro lo svenimento in seguito a traumi, come da tabella
Specializzazioni.

\abi{Resistenza al Freddo}{1}{COS} Capacit\`a di resistere al
freddo. Conferisce un bonus al TR per il freddo, come da tabella
Specializzazioni.

\abi{Resistenza alla Fatica}{1}{COS}
Resistenza alla fatica fisica. Sottrae al numero di punti fisico persi
durante un'azione faticosa (corsa, combattimenti prolungati ecc.)
un numero di punti pari al bonus calcolato con la tabella
Specializzazioni.  

\es{Dopo una lunga marcia in montagna, Justin, un
  Druido di Umbrosa, si \`e stancato molto ed il Master stabilisce una
  perdita di 15 PF.  Poich\'e egli ha nell'abilit\`a Resistenza alla
  Fatica un TOT di 16 (15 + BON COS +1), potr\`a sottrarre 4 dal
  totale dei PF persi stabiliti dal Master, perdendo in realt\`a
  soltanto (15 - 4)=11 PF}

\abi{Resistenza alle Droghe}{1}{COS} 
Capacit\`a di resistere alle droghe. Conferisce un bonus al TR per
l'assunzione delle droghe, come da tabella Specializzazioni.

\iffullversion
\abi{Riconoscere}{1}{OSS} \textbf{2 di base.}  Abilit\`a nel
riconoscere persone anche se travestite o camuffate. Si usa in
contrapposizione con l'abilit\`a Travestirsi o Imitare.
\fi

\abi{Saltare}{1}{AGI} \textbf{5 di base.} Consente di saltare in lungo
o in alto, come da tabella.  

I giganti sommano +5 al tiro di dado.

Nella tabella \`e indicata l'altezza dei piedi da terra.

\noindent
\begin{center}
\begin{tabular}{|rl|c|c|}
\hline
\multicolumn{2}{|c|}{Difficolt\`a} & Lungo& Alto \\ \hline
5  &Banale& 1m& 50cm \\ \hline
10 & Facile& 2m& 75cm\\ \hline
15 & Normale& 3m& 100cm \\ \hline
20 & Medio& 4m& 125cm\\ \hline
25 & Difficile& 5m& 150cm\\ \hline
30 & Molto Difficile& 6m& 175cm \\ \hline
35 & Difficilissimo& 7m& 200cm \\ \hline
40 & Improbabile& 8m& 225cm\\ \hline
45 & Quasi impossibile& 9m& 250cm \\ \hline
50 & Follia& 10m& 275cm \\ \hline
\end{tabular}
\end{center}

\es{Xeres dei Sum, elfo umbra, deve saltare un fossato di 4 metri.
  Egli ha un TOT di 12 nell'abilit\`a.  Tira il dado ed ottiene 10 per
  un totale di 22.  Poich\'e saltare 4 metri in lungo ha Diff. 20 egli
  riesce nel suo intento. Se avesse ottenuto meno di 20 sarebbe caduto
  dentro il fosso}

\es{Takashi Horuru, Signore delle Terre Orientali deve fare una gara
  di salto in lungo e vuole perci\`o saltare il pi\`u lontano
  possibile. Egli ha un TOT di 10 nell'abilit\`a, tira il dado ed
  ottiene 19 totalizzando 29 e compiendo un salto di quasi 6 metri. Il
  suo avversario dovr\`a saltare almeno 6 metri se vorr\`a
  superarlo}

\abi{Scalare}{1}{AGI} (5 di base) Abilit\`a di
arrampicata libera. L'uso di rampini, chiodi, corde ecc. conferisce
bonus a discrezione del Master.  Scalare una roccia non molto ripida
ha Diff.  10, scalare una parete con pochi appigli Diff. 20, scalare una
parete verticale Diff. 35, scalare una parete pressoch\'e liscia
Diff. 40.

\abi{Scassinare}{2}{AGI} Capacit\`a di aprire porte,
lucchetti, casseforti, bauli... senza chiavi. Scassinare la serratura
di un piccolo lucchetto ha Diff.  10, la serratura di una porta Diff.
20, una cassaforte Diff. 30 o pi\`u.  L'uso di attrezzi da scasso e
oggetti simili conferisce un bonus da + 1 a +3 a seconda della
qualit\`a degli attrezzi.

\iffullversion 
\abi{Scoutismo}{2}{OSS} Capacit\`a di adattarsi alla
vita selvatica. Consente di accendere fuochi, trovare ripari,
mandare o lasciare messaggi tramite segni convenzionali utilizzando
arbusti, pietre, fumo, ecc.  Consente anche di comprendere
segnalazioni lasciate da altri.

Accendere un fuoco con una pietra focaia ha Diff. 5, lasciare e
comprendere indicazioni sulla direzione da seguire ha Diff.  10,
comprendere e utilizzare codici comuni Diff. 20, codici particolari o
poco conosciuti Diff.  30.

\abi{Sedurre}{1}{CAR} \textbf{2 di base.} Attitudine alla seduzione. La Diff.
\`e stabilita arbitrariamente dal Master.

\abi{Sfondare porte}{1}{FOR} \textbf{1 di base.} Capacit\`a di buttare gi\`u
porte, barricate, ostacoli, ecc.  Sfondare una porta di legno marcio
ha Diff. 5, sfondare una porta di legno Diff. 15, una porta di legno
con montanti di metallo Diff. 20, una porta in metallo Diff. 35.

Un Fallimento Catastrofico causa a chi ha tentato l'azione un danno ad
arbitrio del Master.
\fi

\abi{Sollevare Pesi}{1}{FOR} Tecnica di sollevamento pesi sulla testa.
Consente di sollevare sulla testa un peso in kg pari alla media fra
FOR e COS pi\`u il bonus FOR, per un moltiplicatore (Mol.)
variabile

\noindent
\begin{center}
\begin{tabular}{|rl|c|}
\hline
\multicolumn{2}{|c|}{Difficolt\`a}&\rule{0pt}{0.5cm}$\frac{FOR+COS}{2}+BF$ x \\ \hline\hline
5 & Banale & 2 \\ \hline
10 & Facile & 4\\ \hline 
15 & Normale & 6 \\ \hline
20 & Medio & 8 \\ \hline
25 & Difficile & 10 \\ \hline
30 & Molto Difficile & 11\\ \hline 
35 & Difficilissimo & 12 \\ \hline
40 & Improbabile & 13 \\ \hline
45 & Quasi impossibile & 14 \\ \hline
50 & Follia & 15 \\ \hline
\end{tabular}
\end{center}

\es{Thalas Sorokasaal, un gigante di
  Mahn-hesa, vuole sollevare sulla testa un peso di 350 kg. Egli ha un
  TOT di 15 in quest'abilit\`a.  Tira il dado ed ottiene 15,
  totalizzando 30. Poich\'e Thalas ha 25 in FOR e 25 in COS, sar\`a in
  grado con questo tiro di sollevare un peso pari a: {[}(25 + 25) / 2 +
  10{]} x 11 (moltiplicatore) = ossia 35 x 11 = 385 Kg. Thalas riesce
  perfettamente nel suo intento}

\iffullversion
\abi{Storia (Generale, Locale)}{1}{CON} Conoscenza della storia delle
terre conosciute (Generale), o del proprio o di altri paesi (Locale).

Conoscere gli avvenimenti principali Diff. 10, ricordarsi nomi, luoghi
e date precise Diff. 20, conoscere particolari poco conosciuti Diff.
30. La specializzazione nella storia di un determinato paese (Storia
Locale) viene considerata come Abilit\`a di Specializzazione nei
tiri Storia del paese stesso.

\es{Lars Sannhom ha un TOT di 15 in Storia (generale) e 12 in Storia
Novese.  Un suo alunno gli chiede in quale anno \`e stata sancita la
pace fra elfi e levanti. 

Il Master stabilisce che la difficolt\`a
\`e 10. Lars non ha bisogno di alcun tiro per poter rispondere in
quanto la difficolt\`a \`e minore del TOT dell'abilit\`a. Un
altro alunno gli chiede, invece, come si chiamava il primo figlio
naturale del primo Duca di Arethin. Il Master stabilisce che la
difficolt\`a \`e 30. Poich\'e si parla di storia Novese, Lars
pu\`o sommare al suo tiro in storia la specializzazione in Storia
Novese, che \`e di +3. Lars tira il dado ed ottiene 15 totalizzando
12 + 3 + 15 = 30. Lars riesce a rispondere all'imbarazzante quesito.}

\abi{Storia dell'arte}{1}{CON} Conoscenza delle opere d'arte, degli
autori e della loro storia. La difficolt\`a \`e stabilita dal
Master per ogni singola opera d'arte. Pi\`u l'opera \`e famosa,
pi\`u bassa \`e la difficolt\`a.

\abi{Suonare}{1}{CAR} Saper suonare un particolare strumento. Suonare
una ninna-nanna ha Diff. 5, una canzone popolare Diff. 15, un'opera
sinfonica Diff.30.  Per attribuire VAL a questa abilit\`a, bisogna
attribuire almeno 1 VAL all'abilit\`a Musica. Per saper suonare
pi\`u strumenti occorre una abilit\`a distinta Suonare per ognuno
degli strumenti.

\abi{Tattica militare}{3}{CON} Conoscenza delle pi\`u usate tattiche
di guerra. Conferisce un bonus come da tabella Specializzazioni ai
propri subalterni durante tutte le Azioni di combattimento.  Per
usufruire del bonus bisogner\`a realizzare un tiro a difficolt\`a 20.
Un fallimento Catastrofico provoca la trasformazione del Bonus in un
Malus.

\es{Masamune Nakamura, Signore
delle terre orientali, comanda un gruppo di 6 guerrieri. Viene
assalito da un gruppo di briganti. Egli ha un TOT di 10 in
quest'abilit\`a. I suoi sottoposti usufruiranno di un bonus di +3
a tutte le azioni di combattimento}

\abi{Torture}{2}{CON} Conoscenza delle tecniche di tortura usate per
far confessare un soggetto.  Conferisce un malus al TV della vittima
per il primo minuto come da tabella Specializzazioni. Per ogni minuto
successivo si aggiunge un ulteriore malus di 1 fino ad un massimo di
4. La difficolt\`a del TV \`e normalmente 20 (Media). 
Un fallimento catastrofico pu\`o provocare la morte del torturato.

\es{Don Patrizio, inquisitore della chiesa rileana, deve
far confessare un uomo accusato di eresia. Egli ha un TOT di 15 in
Torture per cui il TV della vittima sar\`a penalizzato di 4 punti.
L'eretico ha 13 in VOL, tira il dado ed ottiene 19 totalizzando 32.
Nonostante il malus di -4, riesce a resistere poich\'e 32 - 4 = 28 che
\`e maggiore di 20.

Don Patrizio prosegue la tortura per altri 4
minuti dopodich\'e chiede all'eretico di confessare. Oltre al -4
precedente, l'eretico avr\`a un ulteriore malus di -4 (-1 x 4
minuti) per un totale di -8. Tira il dado e ottiene 13, totalizzando
26. Poich\'e 26 - 8 = 18 \`e minore di 20, l'eretico dovr\`a
confessare la sua eresia.}

\abi{Trappole}{2}{AGI} Abilit\`a di saper costruire trappole e saperle
disinnescare. Per disinnescare una trappola bisogna realizzare un tiro
in Trappole superiore al Tiro in Trappole del costruttore. Un
fallimento catastrofico nel costruire una trappola fa s\`i che il
costruttore non si renda conto che la trappola non funziona. Un
Fallimento Catastrofico nel disinnescare la trappola, la fa scattare
automaticamente.  Conferisce un bonus come da tabella Specializzazioni
all'abilit\`a Individuare Trappole.

\es{Kurasay Hidenaga, delle
Terre Orientali, vuole costruire una trappola per difendere
l'accampamento. Predispone un filo che quando viene teso provoca la
caduta di un tronco legato con funi a 2 alberi. Kurasay ha un TOT di 9
in quest\`abilit\`a, tira il dado ed ottiene 16, totalizzando 25.
La sua trappola \`e predisposta a regola d'arte. Se qualcuno la
volesse disinnescare dovrebbe realizzare almeno 26 nel Tiro in
Trappole. Se Kurasay avesse tirato un 1 avrebbe ottenuto una trappola
non funzionante ma non se ne sarebbe accorto}

\es{Alice Ryan, ranger Umbra, ha un TOT di 17 in quest'abilit\`a, ed
  ha individuato (anche grazie al bonus di +4 conferitogli ad
  Individuare Trappole), una trappola nascosta nella serratura di una
  porta. Decide di disinnescarla.  Il Master sa che chi ha piazzato la
  trappola aveva totalizzato 30 nel Tiro Trappole.  Alice tira il dado
  ed ottiene 14, totalizzando 31. Poich\'e 31 \`e maggiore di 30, la
  trappola \`e disinnescata. Se avesse tirato un 1 avrebbe fatto
  scattare la trappola} \fi

\abi{Trattare}{2}{CAR} Saper contrattare sul prezzo o sulle condizioni
di scambio di qualcosa. Si usano in contrapposizione le abilit\`a
Trattare dei contraenti, chi realizza il risultato maggiore riuscir\`a
a convincere l'altro della bont\`a della sua offerta.

\iffullversion
\abi{Travestirsi}{1}{CAR} Capacit\`a di camuffare le proprie
sembianze. Si usa in contrapposizione con l'abilit\`a Riconoscere.
\fi

\abi{Udire}{1}{OSS} \textbf{4 di base.} Indica la capacit\`a di udire
suoni, rumori, ecc.  Udire il pianto di un bambino nella stanza
accanto ha Diff. 5, udire un discorso origliando dietro una porta
Diff. 20, se il discorso dietro la porta \`e sussurrato ha Diff. 30,
udire un discorso bisbigliato in una taverna affollata che si tiene ad
un tavolo a dieci metri dal PG ha Diff. 50. 

Pu\`o essere usata in contrapposizione con l'abilit\`a Muoversi in
Silenzio.

\iffullversion
\abi{Valutare}{2}{CON} Valutare Antiquariato, Metalli, Oggetti d'arte,
Oggetti Magici, Pietre. Conoscenza del valore degli oggetti in
questione. Per poter utilizzare questa abilit\`a bisogna conoscere
l'oggetto in questione (eseguendo un tiro sulla relativa abilit\`a
Conoscere).  

Per conoscere l'antiquariato si utilizza l'abilit\`a Archeologia.  

Per Metalli e Pietre, l'abilit\`a Geologia.

Per valutare gli Oggetti Magici deve prima essere effettuato con
successo uno studio approfondito sugli incantesimi presenti sugli
oggetti e sui materiali che lo compongono. 

Per gli oggetti d'arte l'abilit\`a di conoscenza \`e Storia
dell'arte.

La difficolt\`a del tiro Valutare \`e normalmente 20.

\es{Justin O'Connor, un architetto novese, ha trovato un'antica
  sella elfica.  Esaminando l'oggetto con l'abilit\`a Archeologia ha
  stabilito che essa risale ha 200 anni prima. Justin ha un TOT 13
  nell'abilit\`a Valutare Antiquariato.  Tira il dado ed ottiene 9
  totalizzando 22.  Scopre che la sella vale circa 8 corone}

\abi{Ventriloquio}{2}{CAR} Capacit\`a di emettere suoni e articolarli
tenendo le labbra chiuse. La difficolt\`a \`e solitamente 20.

\subsection{Nuove Abilit\`a}

Il Master \`e libero di creare alla bisogna nuove abilit\`a facendo
attenzione che non siano troppo generiche, o troppo specializzate,
assegnando loro un adeguato codice di Difficolt\`a (D1, D2, D3).

\fi

\nb{Per Abilit\`a di Combattimento e Abilit\`a Magiche vedi capitoli
``Il Combattimento'', ``La Magia'' e ``Le Scuole di Magia''.}

\iffullversion
\pinupbig{pallottoliere_big.eps}{}{bth}

\section{Il Background}
  
La parte pi\`u rilevante del background del vostro PG \`e gi\`a stata
determinata al momento della scelta della razza e della classe sociale
di appartenenza. Dovreste comunque fare un altro piccolo sforzo.

Per rendere il gioco pi\`u piacevole e realistico e
permettere al Master di conoscere la vita passata del vostro PG \`e
consigliabile redigere la parte della scheda detta: ``Background e
tratti morali del personaggio''.  In questa parte dovrete
schematicamente descrivere l'infanzia e la giovinezza del vostro PG
spiegando razionalmente anche come avete appreso le abilit\`a da voi
scelte durante la creazione.  Servir\`a a dare un tocco di classe al
vostro PG. 

Nello spazio bianco nella parte posteriore della Scheda del
Personaggio, si pu\`o disegnare l'aspetto del PG o quello che vi
pare, anche carote, caffettiere, panini all'olio e non. Potete pure
utilizzarlo per le note di gioco (a voi la scelta). 

Al momento della
partenza il PG non possiede null'altro che l'abito e tutto ci\`o che
il Master ritiene opportuno concedergli (in relazione alla Classe
Sociale di appartenenza). Tutto il resto dell'equipaggiamento dovrete
procurarvelo durante la prima sessione di gioco che di solito si
svolger\`a faccia a faccia col Master e servir\`a a porre le basi
dell'avventura che vi preparerete a vivere.

\section{Il Miglioramento}
\label{miglioramento}
Il miglioramento pu\`o avvenire sia per le \textbf{Caratteristiche} che per
le \textbf{Abilit\`a}. Esso pu\`o derivare sia dall'esperienza pratica che
da quella teorica data da un insegnante (vedi abilit\`a insegnare) o
dallo studio di un testo.

\label{incremento}
Per migliorare le abilit\`a tirate 1d20 (\textbf{Tiro Incremento}) su
indicazione del Master. Se ottenete col dado un risultato maggiore del
TOT dell'abilit\`a che dovete migliorare, avete migliorato di 1 punto
la vostra abilit\`a. Tuttavia, al tiro del dado vanno sommati:


\begin{enumerate}
\item i bonus delle caratteristiche: Bonus INT per le abilit\`a di
  INT e CON e per le Abilit\`a Magiche. Bonus AGI, CONC, FOR, COS,
  BEL, CAR per le abilit\`a corrispondenti.  Per le Tecniche
  Speciali di Arte Marziale e Maestria nelle Armi la caratteristica
  che conferisce il bonus al Tiro Incremento \`e indicata per ogni
  tecnica nei paragrafi corrispondenti
\item i malus per il codice di difficolt\`a (D1, D2, D3)
  dell'abilit\`a. I malus sono definiti dalla seguente tabella.
  
  \begin{center}
    \begin{tabular}{|l|r|}
      \hline
      Codice di Difficolt\`a & Malus \\ \hline\hline
      D1& 0 \\ \hline
      D2& -3 \\ \hline
      D3& -5 \\ \hline
    \end{tabular}
  \end{center}

\item eventuali Bonus derivanti dall'insegnamento (vedi abilit\`a
  Insegnare)
\end{enumerate}

Ricordate che \`e il Master a stabilire l'abilit\`a in cui avete
la possibilit\`a di migliorare. Il Master dovr\`a basarsi sul
reale utilizzo della stessa e dovr\`a stare attento al numero di
tiri incremento concessi. 

In parole povere il Master pu\`o concedere un tiro incremento soltanto
in un'abilit\`a che sia stata \textbf{utilizzata o imparata con
  successo}, usata in circostanze \textbf{particolari} od in maniera
particolarmente \textbf{geniale}. \textbf{Non} pu\`o concedere un tiro
incremento su un'abilit\`a \textbf{non} utilizzata per niente.

Oltre a questo, si pu\`o assegnare un Tiro Incremento ogni 35-40 ore
(anche non consecutive) di studio o allenamento \textbf{proficuo} in una
determinata abilit\`a (ossia se l'avete studiata bene, siete stati
dei bravi allievi e l'avete proprio imparata!). 

Se l'abilit\`a \`e insegnata da un maestro il Tiro Incremento
potr\`a essere effettuato dopo 15-20 ore di studio. 

L'allenamento o lo studio \textbf{non sono proficui} se l'azione ha un
valore di difficolt\`a \textbf{inferiore} a quello dell'abilit\`a che
si sta utilizzando.

\es{Titius ha un TOT di 15 nell'abilit\`a Armi da Taglio Lunghe
  (ATL); si allena in un'azione che ha difficolt\`a 10. Non ha
  diritto a nessun tiro incremento indipendentemente dal numero di ore
  spese}

Questo discorso vale per tutte le abilit\`a: Standard, di
Combattimento, per l'incremento nelle Abilit\`a Magiche, per le
specializzazioni.  

\pinupbig{occhio_big.eps}{}{bt}

Schematicamente, per incrementare di un punto
un'abilit\`a si deve quindi: 
\begin{enumerate}
  \itemsep -6pt
\item tirare 1d20
\item sommare il bonus della caratteristica corrispondente
  all'abilit\`a che si intende migliorare
\item sottrarre il malus dato dalla difficolt\`a dell'abilit\`a
\item sommare il Bonus dato dall'insegnamento
\item verificare che il risultato ottenuto sia \textbf{maggiore} del
  TOT dell'abilit\`a. Es.
\end{enumerate}

\es{Nimh \`e un levante che ha un TOT di 15
  (12 + 3 Bonus Carisma) in Suonare. In una serata in un'osteria tenta
  di improvvisare su una canzone popolare suonata da dei musici.  La
  difficolt\`a \`e 25.  Nimh realizza 12 col dado riuscendo
  cos\`{\i} nell'azione. Il Master reputa l'azione proficua e concede
  al giocatore 1 tiro incremento}



\es{Xeres dei Sum \`e un elfo umbra studioso di elementalismo e ha un
  TOT di 18. Egli studia l'incantesimo ``Incendio'' che ha una
  difficolt\`a relativa di 46 da un libro che gli d\`a un bonus di
  +10. Per impararlo gli occorrono 46 ore di studio. Il Master
  effettua un tiro ``nascosto'', ottiene 19 e stabilisce che Xeres ha
  imparato l'incantesimo. Per accertarsene, quest'ultimo dovr\`a
  provarlo almeno una volta con successo. Fino ad allora non potr\`a
  usufruire di un eventuale Tiro Incremento. 
  
  Xeres lancia l'incantesimo, che produce normalmente i suoi effetti,
  dimostrando la riuscita dell'apprendimento. Dato il numero di ore
  impiegate nello studio proficuo il Master stabilisce che Xeres abbia
  diritto a 1 Tiro Incremento.
}



Il miglioramento delle \textbf{Caratteristiche} procede insieme al
miglioramento nelle Abilit\`a.  Ogni Tiro Incremento riuscito, infatti
fa aumentare la caratteristica Conoscenza e la Caratteristica
corrispondente all'abilit\`a migliorata, di 1 punto percentuale per
livello di difficolt\`a (D1, D2, D3) dell'abilit\`a migliorata.

\nb{Ricordatevi che quando il valore \% supera i 100, la caratteristica
aumenta di un punto e il valore \% diminuisce di 100.}

\es{ Hudibras, duca Novese,
  effettua con successo un tiro incremento sull'abilit\`a Empatia.
  Poich\'e questa \`e un'abilit\`a a difficolt\`a 3 (D3), la
  caratteristica Conoscenza e la caratteristica Osservazione
  (corrispondente all'abilit\`a) aumentano di 3 punti \%.
  
  Hudibras aveva un valore di 17 in CONoscenza, pi\`u 98 punti \%, per cui
  dopo l'aggiunta dei 3 punti dovuta ai tiri incremento realizzati avr\`a 101 punti \%, quindi 
  il suo punteggio in conoscenza sar\`a 18, pi\`u 01 punti \%.}



Il miglioramento delle caratteristiche pu\`o anche essere deciso dal
Master con l'assegnazione diretta di punti percentuali.
Orientativamente si aumenta di 1 punto \% ogni 5 ore circa di
allenamento \textbf{mirato}.

\fi
%%% Local Variables: 
%%% mode: latex
%%% TeX-master: "manual"
%%% End: 
