\documentclass[10pt,italian,onecolumn,twoside,xdvi,openany,a4paper]{book}
\begin{document}
      {\large Licenza d'uso e distribuzione di Radix Malorum}

      \textit{versione 1.0, aprile 2003}
\footnotesize

      \begin{enumerate} \item L'``uso personale'' e la distribuzione
      totale o parziale del materiale contenuto in quest'opera (da qui
      in poi {\bf Radix Malorum}) sono consentite e gratuite per
      ``scopi non commerciali'', fermo restando l'obbligo per il
      distributore e l'utilizzatore di rispettare questa Licenza in
      toto e distribuire integralmente ({\rm verbatim}) il testo di
      questa Licenza in maniera chiara e visibile, insieme a ciascuna
      copia integrale o parziale di {\bf Radix Malorum}.

      \item Gli autori si riservano di revocare la licenza d'uso e
      distribuzione agli utilizzatori e ai distributori che si
      trovassero in contravvenzione di almeno uno degli articoli di
      questa Licenza.

      \item Qualunque utilizzo o distribuzione per ``scopi
      commerciali'' di {\bf Radix Malorum} deve essere esplicitamente
      concordato con gli autori, nelle forme che verranno di volta in
      volta ritenute opportune dagli autori stessi.

      \item Per uso o distribuzione per ``scopi commerciali'' si
      intende qualunque situazione o condizione d'uso che pu\`o
      portare un guadagno, anche indiretto, in termini economici, per
      l'utente o il distributore. Il rimborso spese per il supporto,
      la stampa e la distribuzione non sono da considerare
      ``guadagno'' nella valutazione della ``scopo commerciale''. Per
      uso o distribuzione per ``scopo non commerciale'' si intende
      qualunque situazione o condizione d'uso che non rientra nello
      ``scopo commerciale''. Per ``uso personale'' si intende
      qualunque situazione in cui una singola persona fisica utilizza
      {\bf Radix Malorum} per s\'e, in toto o in parte, senza
      distribuirlo o cederlo a terzi.

      \item Per ``distribuzione'' si intende l'atto con il quale si
      consente a terzi di accedere ad una copia parziale o totale di
      {\bf Radix Malorum}, attraverso qualsiasi mezzo. Sono esempi di
      distribuzione: la trasmissione radiotelevisiva; la duplicazione,
      la stampa, la cessione o la condivisione di un supporto digitale
      o analogico contenente parti di {\bf Radix Malorum}; la
      pubblicazione su una rete informatica o affini di parti di {\bf
      Radix Malorum} che ne consenta la consultazione o la copia a
      pi\'u di una singola persona. Ogni supporto analogico o digitale
      contenente {\bf Radix Malorum} deve contenere una copia
      integrale di questa Licenza.

      \item Gli organizzatori di manifestazioni aperte al pubblico in
      cui si utilizza, si presenta o si distribuisce {\bf Radix
      Malorum}, tra cui tornei, fiere, meeting, ecc., sono vivamente
      pregati (anche se non formalmente obbligati) di contattare
      anticipatamente gli autori, o i loro delegati, per informarli
      circa le modalit\`a di utilizzo e distribuzione di {\bf Radix
      Malorum}, fermi restando i rimanenti articoli della presente
      Licenza.

      \item Tutte le informazioni ufficiali su {\bf Radix Malorum},
      ivi incluse le modalit\`a per contattare direttamente o
      indirettamente gli autori di {\bf Radix Malorum} e le
      informazioni relative alla presente Licenza e alle sue
      successive modifiche, sono accessibili al sito internet
      ufficiale, che pu\'o essere raggiunto via World Wide Web
      all'indirizzo:

      {\tt http://www.radixmalorum.it}

      \item La distribuzione di versioni modificate di {\bf Radix
      Malorum} dovr\`a essere esplicitamente concordata con gli
      autori.

      \item Gli autori di {\bf Radix Malorum} sono (in ordine
      alfabetico): Juri Orr\`u, Nicola Orr\`u, Sergio Picciau e Carlo
      Rosas.

      \end{enumerate} 

\end{document}