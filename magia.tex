\iffullversion
\onecolumn
\fi
\chapter{La Magia}
\label{magia}

\iffullversion
\vfill
\begin{racconto}
  
  I maghi si riunirono nella stanza della stella, erano l\`{\i} per
  distruggere Indariel e per disperdere la Sfera della Creazione.
  Howen di Terranova prese la parola:
  
  --Chi sar\`a il Raggio della Stella? Chi sieder\`a al centro?--
  
  Nessuno si mosse o disse nulla, come prescriveva la norma. Howen
  riprese:

  --Due sono i prescelti e uno di essi briller\`a del potere dei Dodici--
  
  All'unisono, gli undici maghi votanti dissero:
  
  --Cos\`{\i} \`e scritto e cos\`{\i} sar\`a, il potere di tutti in un unico
  fuoco--
  
  Detto questo Howen continu\`o:

  --La mano destra per Rogan e la sinistra per Shalaran--
  
  Tre per Rogan e otto per Shalaran.
  
  --Cos\`{\i} \`e deciso-- sentenzi\`o Howen --Shalaran il Raggio e
  Rogan la dodicesima punta. Cos\`{\i} vuole la Madre--
  
  Gli arcimaghi insieme risposero:
  
  --Cos\`{\i} vuole la Madre--
  
  Tutti si misero ai loro posti: i dodici sulle punte della stella e
  Shalaran al centro. I famigli ai piedi dei padroni emisero il
  proprio verso. Shalaran disse con voce mesta:
  
  --Che i famigli rendano l'anima alla madre senza rimpianti--
  
  Seguirono lunghi attimi di muta preghiera. Poi ognuno tracci\`o
  con un dito, nell'aria, sopra il capo del proprio famiglio, --Gheow--
  Morte nell'antica lingua delle Rune. E i famigli si assopirono
  nell'eternit\`a.
  
  Quanti di loro sarebbero sopravvissuti al Rito di Alhnora? Tutto era
  pronto.  I tredici pronunciarono nello stesso istante le mistiche
  parole, un salmodiare che doveva durare ininterrotto per dodici ore.
  
  Allo scoccare della dodicesima ora le aure dei maghi si staccarono
  da essi e confluirono tutte su Shalaran.  Questi salmodi\`o per
  altre tre ore, mentre gli altri assistevano passivamente, assorti in
  una magica trance ipnotica.
  
  Poi un raggio verde si sprigion\`o da Shalaran e si innalz\`o
  fino alla sommit\`a della torre, dove l'energia si concentr\`o
  per poi dipanarsi in cerchi concentrici in ogni parte delle terre
  conosciute col fine ultimo di privare del suo potere il malvagio
  Indariel e di allontanare le sfere della creazione.
  
  E cos\`{\i} fu. Al termine del rito solo due persone sopravvissero:
  Rogan, che poi si tolse la vita per raggiungere la sua amata
  spentasi nel Rito, e Antiochio, l'Eterno, la cui et\`a gi\`a
  allora nessuno conosceva.
  
  La leggenda narra che l'anima di Rogan e della sua amata Elisabetta
  vaghino ancora senza riposo attorno alla torre''

\bigskip
{\raggedleft\hfill\rm ---Cronache di un Tempo Passato. Libro VII, cap. X, p.3960 dell'Oracolo di Giassa}

\end{racconto}
\pinupbig{maga.eps}{}{p}
\twocolumn
\fi

{\raggedright \section{Architettura del Sapere Magico}}

Il sapere magico \`e diviso in \textbf{sei Scuole} di Magia
fondamentali, raggruppate a formare \textbf{tre Regni},
ognuno dei quali risulta essere in equilibrio interno.

Esistono inoltre altre \textbf{due Liste Magiche}, dette
dell'\textbf{Utilit\`a} e della \textbf{Salvaguardia}, accessibili
agli adepti di qualunque scuola, e perfino ai non adepti.

\textbf{Ogni regno comprende due scuole}, tra di loro complementari. I tre
Regni sono:

\begin{description}
\item{\bf Vimohr}: costituito dalla scuola di \textbf{Necromanzia}
  (sovrana del potere sulla morte) e dalla scuola di
  \textbf{Esistenza} (sovrana del potere sulla vita);
\item{\bf Memath}: costituito dalle scuole di \textbf{Illusionismo}
  (dominatrice dei poteri della mente) ed \textbf{Elementalismo}
  (che controlla i poteri dei Quattro Elementi);
\item{\bf Deimst}: dato dall'unione della scuola di
  \textbf{Stregoneria} (sovrana dei poteri degli Esseri Terreni) e
  dalla scuola di \textbf{Demonologia} (sovrana dei poteri degli
  Esseri degli Inferi).
\end{description}

Vimohr, Memath sono i regni pi\`u antichi, solo qualche centinaio di
anni dopo venne scoperta l'esistenza di Deimst.

\iffullversion
\begin{racconto}
  ``Agli albori del tempo la magia era una, essa comprendeva tutte le
  branche dell'occulto e per scatenarla era sufficiente la pronuncia
  di mistiche parole. Ma l'uomo bramava di pi\`u, voleva l'oro,
  voleva l'eternit\`a e allora guerra, morte, distruzione. E la
  terra pianse per trecento giorni e altrettante notti e tutto fu
  distrutto dai suoi tremiti e dalle sue lacrime di fuoco. Ed il mondo
  rinacque di nuovo puro, ma il passato fu perso nel bene e nel male.
  La magia venne cos\`{\i} spartita e il potere degli antichi maghi si
  perse nei venti''

\bigskip
\hfill{\rm\raggedleft ---Trattato ragionato teorico sulla magia, \S 1.1, a cura dell'Oracolo di Giassa}
\bigskip
\end{racconto}

\fi
\subsection{I Princ\`{\i}pi}

Per poter esercitare la magia occorre padroneggiarne i principi:

\begin{enumerate}
\item La magia \`e un potere bipolare, di creazione e distruzione
  ininterrotta, per cui tutto si pu\`o creare, modificare e infine
  distruggere.
  
\item La magia non \`e n\`e buona n\`e cattiva: essa semplicemente
  \`e, per cui la bont\`a o la cattiveria sono trasmesse alla
  magia dall'usufruitore.
  
\item La magia viene utilizzata con la canalizzazione delle energie
  dei Venti della Terra, per cui tutto si crea e si distrugge con la
  padronanza del Vento.
  
\item I Venti della Terra sono in perenne movimento, a spirale dal
  centro della terra verso l'esterno. Essi quindi salgono a spirale e
  scendono a cascata.
  
\item Colui che usa la magia e non domina a sufficienza il Vento
  pu\`o supplirvi con l'energia interna, per cui solo i predestinati
  operano nella Magia.

\end{enumerate}

{\raggedright \section{Le organizzazioni della Magia}}

Ad organizzare e diffondere le conoscenze magiche sono preposti due
organismi: \textbf{il Conclave}, ufficiale e riconosciuto, \textbf{e
  l'Anticonclave}, clandestino e illegale.

\subsection{Il Conclave}

Il Conclave \`e un'istituzione antichissima, pubblica, convocata una
tantum ad opera di uno degli arcimaghi (le massime autorit\`a del
Conclave), che si tiene nella Torre dell'Isola del Gran Consiglio nel
Mare Interno.

\iffullversion
La torre \`e protetta da una potente barriera, inattaccabile
sia dalle armi che dalla magia, disattivabile solo dal pi\`u vecchio degli
arcimaghi: Antiochio l'Eterno, o da sette arcimaghi insieme. 

All'interno, la torre \`e dotata di un'ampia sala a forma di stella
a dodici punte, in essa vi sono tredici posti per gli Arcimaghi,
ma raramente tutti vengono occupati.
\fi

\subsubsection{Gli Arcimaghi} 

Le pi\`u alte autorit\`a del Conclave sono gli Arcimaghi. Essi
sono preposti all'amministrazione e alla politica dell'organizzazione.
Rappresentano anche i pi\`u grandi depositari del sapere magico
nelle terre conosciute.

\iffullversion
Tanti e tanti furono nel corso dei secoli gli arcimaghi che si
successero alla guida del Conclave. I loro nomi sono registrati
nell'Albo Magno del Conclave, ubicato presso la Sala dei Grandi, nella
Torre dell'Isola del Gran Consiglio.

La carica di Arcimago spetta di diritto ai \textbf{Rettori} delle sei
Accademie magiche pi\`u importanti e ai sei \textbf{Signori delle
  Torri} della magia. 

Solitamente i nuovi Arcimaghi vengono scelti fra gli aventi titolo.
Titolo richiesto per divenire Arcimago \`e aver completato almeno un
Regno di Magia (cio\`e avere VAL 20 in 2 scuole complementari).
Tuttavia anche questa regola pu\`o essere disattesa qualora Arcimago
o Torre ritengano meritevole una persona senza titolo di successione.

\subsubsection{I Rettori}

I Rettori sono direttamente responsabili dell'amministrazione del
Conclave.  Decidono a maggioranza ed il voto del Sommo Rettore,
nominato ogni 2 anni e rieleggibile, vale doppio.

Le decisioni prese vengono comunicate ai signori delle Torri che
possono, con la maggioranza relativa dei voti, porre il veto alle
decisioni.  Alle riunioni dei Signori partecipa il Guardiano della
Torre del Gran Consiglio, anch'egli con diritto di voto.

I Rettori nominano i loro successori quando vogliono ritirarsi, oppure
li designano mediante testamento. Qualora il rettore muoia senza aver
designato un successore, il nuovo rettore viene nominato dal
Guardiano, sentiti i Signori delle Torri.

Attualmente i Rettori delle 6 principali Accademie sono: 

\begin{description}
\item{\bf Rehan Astante}, umano Levante, all'accademia di Mut ad
  Umbrosa
\item{\bf Justin O'Connor}, umano Novese, all'accademia di Barlina a
  Terranova
\item{\bf Amon Denon}, umano Novese, all'accademia di Silays a Bahuney
\item{\bf Tamiya Shimoburo}, umana Orientale, all'accademia di Seemur
  nella Terra di Levante
\item{\bf Connor McPental}, umano Nordico, all'Accademia della Khare
  Centrale nella Confederazione del Sud
\item{\bf Ulzath Savegolth}, elfo Luxi, all'accademia di Finnan nelle
  Terre del Nord
\end{description}

\pinup{occhio_small.eps}{}

\subsubsection{I Signori delle Torri}

I Signori delle Torri non hanno un compito determinato all'interno
dell'organizzazione, se si eccettua il rito di Alhnora.  Essi hanno
tuttavia diritto di veto sulle decisioni dei Rettori.

I Signori delle Torri nominano direttamente il loro successore, quando
vogliono abdicare, o lo scelgono mediante testamento fra gli
apprendisti che ammettono all'interno della Torre della Magia. Questa
regola pu\`o tuttavia essere disattesa ed il successore pu\`o essere scelto
altrove.

Quando un Signore muore senza aver nominato un successore, la Torre
stessa designa magicamente diversi successori Candidati, che si
sfidano in un torneo dinanzi alla Torre. Solo il vincitore
acquisir\`a il titolo di Signore.

Quando un Signore della Torre muore o abdica, 100 dei suoi PE vanno a
sommarsi ai PE del successore, mentre i restanti si disperdono nei
venti della Terra.

La nomina di Guardiano della Torre del Gran Consiglio avviene
ufficialmente nello stesso modo delle altre torri, ma a memoria d'uomo
non si ricorda che un solo Guardiano, Antiochio l'Eterno.

In assenza di Antiochio, la torre \`e custodita da un suo aiutante di
fiducia.  

Il famiglio di Antiochio \`e una piccola tartaruga di cui nessuno
conosce n\`e il nome, n\`e l'et\`a.

I Signori delle Torri sono:

\begin{description}
\item{\bf Alice Ryan}, elfo Umbra: Torre della Necromanzia, situata
  nell'isola di Norda. Il suo famiglio \`e un'aquila reale chiamata
  Kyra.
  
\item{\bf Xeres Sum}, elfo Umbra: Torre dell'Elementalismo, situata
  nell'isola di Kush. Il suo famiglio \`e il falco Tarx.
  
\item{\bf Seinen Laftinais}, elfo Luxi: Torre dell'Illusionismo,
  situata nell'isola di Bahuney. Il suo famiglio \`e il gatto
  Haryell.
  
\item{\bf Vicious Von Vinicious}, Novese: Torre della
  Stregoneria, situata nell'isola di Malagana. Il suo famiglio \`e
  un boa constrictor chiamato Loomie.
  
\item{\bf Masamune Nakamura}, Orientale: Torre della
  Demonologia, situata nell'isola di Loydi-Genya. Il suo famiglio
  \`e una tigre delle nevi di nome Tora.
  
\item{\bf Grande Aquila}, umano Nomade: Torre dell'Esistenza, situata
  nell'isola di Dyshu-Thandy. Il suo famiglio \`e un topo di nome
  Ettore.
\end{description}

Nessuno conosce l'esatta dislocazione delle Torri della Magia. 

Presso le Torri della Magia i venti della Terra sono Immani.

{\raggedright \subsubsection{Il rito di Alhnora}}

Solo una volta nella storia fu necessario compiere il Sacro Rito di
Alhnora, capace di dissolvere qualunque forza magica; fu quando
Indariel, l'elfo Umbra divenuto Signore di Malgram, si impadron\`{\i} 
della Sfera della Creazione, custodita a quel tempo nel grande Albero
della Vita degli Elfi,

Il Rito ha il 5\% di probabilit\`a di riuscita per ogni Arcimago
presente al suo compimento.  Il Rito pu\`o comunque essere effettuato
solo una volta ogni 10 anni. Se il rito fallisce, per gli arcimaghi
che lo hanno tentato vi \`e il 90\% di probabilit\`a di impazzire e il
70\% di probabilit\`a di morire fra mille tormenti.

Se il rito riesce, le probabilit\`a di impazzire si riducono al 10\%
mentre quelle di morire al 5\%.

Compiuto il rito l'Arcimago deve riposare per un mese senza poter
lanciare alcun incantesimo.


\subsection{Le leggi dell'uso}

Ad un certo punto nel corso della storia gli arcimaghi sentirono la
necessit\`a di regolamentare l'uso della Magia per evitare scempi da
parte di maghi irresponsabili.

Dettarono cos\`{\i} le Leggi dell'Uso, da cui poi verranno tratte le
Leggi di ogni singola scuola descritte in seguito.

Le Leggi dell'Uso sono: 

\begin{enumerate}
  \itemsep -6pt
\item Non modificare gli eventi passati 
\item Non lasciar libere le entit\`a evocate 
\item Obbedire agli Arcimaghi 
\end{enumerate}


Essere membri del Conclave, e quindi rispettare le Leggi dell'Uso, d\`a
la possibilit\`a all'80\% di trovare vitto e alloggio per 3 giorni
presso la dimora di un signore locale.
\fi

\subsection{L'Anticonclave}

\iffullversion
Quando il Conclave eman\`o le ``leggi dell'uso'' per regolamentare
l'utilizzo della magia, un folto gruppo di usufruitori si oppose
fermamente affermando che all'origine la magia era libera e che tale
doveva restare senza nessuna gabbia ad imprigionarla. I pi\`u
estremisti cercarono addirittura di distruggere il Conclave assediando
la Torre del Gran Consiglio, ma fu inutile.

I dodici Arcimaghi (poich\'e i rivoltosi erano guidati dall'Arcimago
dei Reuben) sconfissero quasi tutti i componenti delle forze nemiche.
Cos\`{\i} termin\`o la prima guerra della magia (Sui cui dettagli vi
rimandiamo al libro V delle ``Cronache di un tempo passato''
dell'Oracolo di Giassa).

I pochi sopravvissuti, riuniti sotto il comando di Eghay, figlio
dell'Arcimago ribelle, si organizzarono in un'associazione che \`e
tutt'oggi chiamata ``Anticonclave'', la quale offre rifugio ai maghi
che ritengono che il rispetto delle ``leggi dell'uso'' sia un limite
all'equilibrio naturale.
\fi

L'Anticonclave agisce in segreto infiltrandosi in tutte le classi
sociali per gettare nel disordine tutte le contrade. Chiunque infranga
le ``leggi dell'uso'' in una grande citt\`a ha il 40\% di probabilit\`a
di essere contattato dalle spie dell'Anticonclave che gli offriranno rifugio
e, nel caso venga ritenuto capace e idoneo, di essere membro dell'Anticonclave.
L'Anticonclave offre libero apprendimento e uso illimitato della magia, ma
punisce il doppio gioco con la morte.

Far parte dell'Anticonclave consente di conoscere i suoi rappresentanti
presso le citt\`a pi\`u importanti e i rifugi in esse presenti. Per il
rifugio vengono chieste da 1 a 6 Corone offerte per la Causa.

\clearpage    
\section{Le basi della magia} 
\label{basimagia}

\iffullversion
\begin{racconto}
  --Bene, Yoimoi, pare che tu abbia capito bene. Sei molto
  intelligente!-- E\-scla\-m\`o il Maestro Lotharielen, evidentemente
  soddisfatto, regalando un affascinante sorriso alla giovane ed
  attraente allieva umana.
  
  --Ora mostra a tutta la platea quello che hai imparato--
  
  La timida fanciulla dai capelli corvini, visibilmente imbarazzata,
  nascose la testa nel cappuccio della sua veste per evitare i
  sorrisetti maliziosi dei suoi compagni di corso e attese che il
  silenzio cadesse nella grande aula. Dopo qualche manciata di secondi
  pronunci\`o sottovoce la formula e trasal\`{\i}, tirando fuori la
  testa dal cappuccio. Poi sgran\`o gli occhi stupita e squadr\`o il
  bel viso del suo Maestro, cercando di scorgerne magicamente i
  pensieri.
  
  --Dunque?-- Seinen Lotharielen continuava a sorridere,
  maliziosamente come al solito.
  
  --Lei... Lei... Sta pensando... di...--

  --...di?-- la voce del Maestro si fece pi\`u premurosa. 
  
  Yoimoi deglut\`{\i} --... di mangiare carne di pernice!--
  sentenzi\`o alla fine, sospirando.
  
  --Complimenti! Sei riuscita ad attivare un nuovo incantesimo! Puoi
  restare dopo l'orario di lezione? Ti insegner\`o qualche trucco.--
  
  Yoimoi torn\`o al suo posto con le guance in fiamme, un po' per la
  fatica, molto pi\`u per l'imbarazzo. I compagni e le compagne di
  corso soffocarono i risolini ironici tra le pieghe delle tuniche.
  
  Un dongiovanni come Seinen Lotharielen dei Laftinais non poteva
  perdere una occasione cos\`{\i} ghiotta! Un elfo Luxi che di fronte
  ad un pezzo di figliola come Yoimoi pensa a mangiare? Ma quando mai?
  Forse pensava di banchettare con un altro tipo di carne...
  
  Yoimoi rimase dopo le lezioni.
\end{racconto}
\fi

\subsection{Gli incantesimi}

L'incantesimo \`e un'operazione di tipo magico che consente di
modificare la realt\`a fisica circostante.  Solo coloro che hanno
attribuito almeno 1 VAL in un un'abilit\`a magica, come descritto
nel Paragrafo ``Le Abilit\`a'' a pagina \pageref{abilita}, possono
lanciare incantesimi. Ogni PG pu\`o tentare di lanciare un
incantesimo anche se non l'ha mai studiato o appreso in precedenza,
basandosi semplicemente sulle sue conoscenze delle Scuole e delle
Liste magiche, in quanto \textbf{gli incantesimi vengono valutati, come le
Azioni, in termini di difficolt\`a}.

Il modo di manifestazione, l'apparire esterno dell'incantesimo non
\`e influente ai fini del gioco, ma pu\`o costituire un ottimo
spunto per l'interpretazione se si giustifica l'effetto
dell'incantesimo in relazione alla Scuola di Magia utilizzata, al suo
dominio e ai suoi limiti e peculiarit\`a.  Esso dipende da
molteplici fattori, ma soprattutto dalla fantasia del giocatore e del
Master.

Il Master pu\`o ritenere impossibile un incantesimo se non ne viene
giustificato l'effetto e la relazione alla scuola. 

Ogni incantesimo \`e caratterizzato da una \textbf{Difficolt\`a
  Assoluta (DA)} cio\`e da una difficolt\`a oggettiva (espressa da
un numero, come le difficolt\`a delle azioni) indipendente dalla
Scuola, e da una \textbf{Difficolt\`a Relativa (DR)}, calcolata in
funzione della Scuola. DA e DR verranno spiegate in dettaglio al
paragrafo ``Calcolo della DA e della DR'' a pagina \pageref{dadr}.

\subsection{Lanciare un incantesimo}
\label{lanciareincantesimo}

Prima di tentare di lanciare l'incantesimo occorre concentrarsi
(realizzando un TCONC a difficolt\`a 20). Se questo fallisce,
l'azione per quel round viene persa. Lanciare un 1 col dado in questo
caso \textbf{non} porta ad un fallimento catastrofico.

\iffullversion
Si pu\`o ottenere un bonus fino a +5 al TCONC aspettando 1 round per
ogni punto di bonus prima di tentare il tiro. Se il TCONC ha successo,
l'incantesimo riesce ottenendo, con un tiro sull'abilit\`a Magica
(1d20 + VAL della Scuola o Lista), \textbf{un valore maggiore o uguale
  alla DR dell'incantesimo}, esattamente come avverrebbe per
un'azione qualsiasi.
\fi

Se questo tiro non riesce si ottiene un Fallimento, tanto pi\`u
disastroso quanto maggiore \`e la differenza tra la DR
dell'incantesimo e il tiro sull'abilit\`a Magica\iffullversion, secondo la tabella
Fallimenti Catastrofici per la Magia a pagina
\pageref{fumblemagia}.\else.\fi

Lanciare un 1 col dado pertanto non porta sempre e direttamente ad un
fallimento catastrofico.

Il Fallimento Catastrofico Specifico \`e differente per ogni
Categoria Funzionale.  Ogni fallimento Catastrofico Specifico \`e
indicato nella relativa Categoria Funzionale.

Se volete dotare il gioco di una maggior dose di realismo, consigliate
ai giocatori di pronunciare le formule magiche.

\iffullversion
\pinup{mortaio.eps}

Alcuni maestri delle scuole di magia insegnano a attivare gli incantesimi
in modi differenti dalla parola (suonando uno strumento, danzando, scrivendo,
ecc.).


\begin{table}[t]
  \begin{radtable}{Fallimenti Catastrofici per la magia}{|l|l|}
    Differenza&Effetto\\ \hline\hline
    1&Nessun effetto\\ \hline
    2&Tabella incantesimi 1\\ \hline
    3&Sfinimento e Svenimento\\ \hline
    4&Tabella incantesimi 2\\ \hline
    5&Fallimento specifico\\ \hline
    6&Fallimento specifico\\ \hline
    7&Tabella incantesimi 3\\ \hline
    8&Coma 1d20 giorni\\ \hline
    9&Handicap permanente\\ \hline
    10&Morte\\ \hline
    10+&Morte\\ \hline
  \end{radtable}

  \footnotesize

  Se la differenza \`e 3 il mago perder\`a tutti i PM e i PF tranne 1, svenendo
  immediatamente. Se la differenza \`e 8 colui che tenta l'incantesimo entra
  in coma. Se la differenza \`e 9, l'handicap permanente si determina tirando
  1d10 sulla tabella che segue:
  
  Se la differenza \`e 2, 4 o 7, si ottiene l'effetto di un incantesimo ``a
  caso'', scelto tirando 1d20 sulla tabella corrispondente.

  \begin{radtable}{Handicap}{|l|l|}
    1d10&Handicap\\ \hline
    1&Sordit\`a (-20 in udire)\\ \hline
    2&Cecit\`a \\ \hline
    3&Zoppia (-5 AGI)\\ \hline
    4&Mutismo\\ \hline
    5&Elefantiasi (-3 AGI, -2 BEL)\\ \hline
    6&Artrosi (-5 AGI)\\ \hline
    7&Asma (-3 COS, -2 CONC)\\ \hline
    8&Singhiozzo (-5 CONC)\\ \hline
    9&Paresi (-3 CAR, -2 BEL)\\ \hline
    10&Ipertricosi (-5 BEL)\\ \hline
  \end{radtable}
  \caption{Fallimenti catastrofici per la magia}
  \label{fumblemagia}
\end{table}
  
\begin{table*}[p]
\parbox[t]{7.5cm} {
  \begin{radtable}{Tabella Incantesimi 1}{|c|l|p{4.2cm}|}
    1d20&Scuola & Incantesimi\\ \hline\hline
    1&Nec&Tocco mortale I\\ \hline
    2&Esi&Antishock\\ \hline
    3&Nec&Illusione di morte\\ \hline
    4&Nec&\raggedright Protezione Esseri del Limbo I\tabularnewline \hline
    5&Str&Ululato\\ \hline
    6&Str&Parlare agli animali\\ \hline
    7&Dem&Macellaio\\ \hline
    8&Ele&Parlare agli Elementi\\ \hline
    9&Esi&Purificazione\\ \hline
    10&Ill&Protezione Incubi I\\ \hline
    11&Nec&Fuoco fatuo\\ \hline
    12&Nec&Ombra\\ \hline
    13&Nec&Tocco mortale II\\ \hline
    14&Str&Torpedine\\ \hline
    15&Dem&Filamento adesivo\\ \hline
    16&Dem&Pelle robusta\\ \hline
    17&Dem&Protezione dai Demoni I\\ \hline
    18&Ele&Protezione Elementali I\\ \hline
    19&Ill&Psicosi\\ \hline
    20&Nec&Parere Spiritico\\ \hline
  \end{radtable}
}
\hfil
\parbox[t]{7.5cm}{
  \begin{radtable}{Tabella Incantesimi 2}{|c|l|p{4.2cm}|}
    1d20&Scuola&Incantesimi\\ \hline\hline
    1&Dem&Sparizione\\ \hline
    2&Dem&Violinista\\ \hline
    3&Ele&Controllo Acqua\\ \hline
    4&Ele&Evocazione Elementali I\\ \hline
    5&Ill&Ipnosi I\\ \hline
    6&Ill&Telecinesi\\ \hline
    7&Nec&\raggedright Protezione Esseri del Limbo III\tabularnewline \hline
    8&Str&Spugna\\ \hline
    9&Str&Fuga\\ \hline
    10&Str&Furto del corpo\\ \hline
    11&Str&Malia\\ \hline
    12&Dem&Colonna di fuoco\\ \hline
    13&Esi&Protezione Angeli III\\ \hline
    14&Esi&Sfera di immunit\`a\\ \hline
    15&Esi&Trasferimento vitale\\ \hline
    16&Ill&Evoca Essere Onirico II\\ \hline
    17&Ill&Istigazione al suicidio\\ \hline
    18&Ill&Tempesta di frecce\\ \hline
    19&Nec&Dito II\\ \hline
    20&Nec&\raggedright Evoca/Crea Esseri del Limbo II\tabularnewline \hline
  \end{radtable}
}
\par
  \begin{radtable}{Tabella Incantesimi 3}{|c|l|p{4cm}|}
    1d20&Scuola&Incantesimi \\ \hline\hline
    1&Ill&Scudo Psichico\\ \hline
    2&Nec&Urlo\\ \hline
    3&Ill&Scomparsa totale\\ \hline
    4&Ill&Sdoppiamento\\ \hline
    5&Nec&Morte II\\ \hline
    6&Str&Scossa Tellurica\\ \hline
    7&Dem&Evoca Arioch\\ \hline
    8&Ill&Sonno profondo\\ \hline
    9&Str&Capanna\\ \hline
    10&Dem&Anima diabolica\\ \hline
   11&Esi&Guarigione III\\ \hline
    12&Ill&Ipnosi III\\ \hline
    13&Nec&Dismissione II\\ \hline
    14&Ele&Dimensioni Pietra II\\ \hline
    15&Str&Rospo\\ \hline
    16&Nec&Dismissione III\\ \hline
    17&Ele&Meteora\\ \hline
    18&Ele&Pietrificazione II\\ \hline
    19&Dem&Fiamma demoniaca\\ \hline
    20&Nec&Onda Mortale\\ \hline            
  \end{radtable}
  \caption{Tabelle Incantesimi Casuali per Fallimento Catastrofico}
\end{table*}  


{\raggedright \subsection{Lanciare incantesimi in gruppo}}

Pi\`u usufruitori di magia possono ritenere utile lanciare un
incantesimo in gruppo, cooperativamente. In tal caso dovranno:

\begin{enumerate}
\item Lanciare \textbf{lo stesso} incantesimo \textbf{contemporaneamente}
\item Effettuare ciascuno il TCONC
\item Effettuare ciascuno il tiro sull'abilit\`a magica con un bonus
  pari a +2 per ogni mago oltre il primo che collabora al lancio
  dell'incantesimo (se i maghi sono 2 il bonus \`e +2, se sono 3 il
  bonus \`e +4 e cos\`i via).
\end{enumerate}

Se anche uno solo dei maghi fallisce il Tiro Concentrazione
l'incantesimo non parte; se anche un solo mago fallisce il tiro
sull'abilit\`a magica, l'incantesimo non ha effetto e tutti coloro che
partecipano subiscono gli effetti del fallimento o dei fallimenti.

Il target dell'incantesimo o l'area interessata devono trovarsi ad
una distanza inferiore o uguale alla gittata da tutti i maghi che
lanciano l'incantesimo.

In caso di riuscita o fallimento dell'incantesimo, i PM, PF, PE
richiesti verranno suddivisi in parti uguali tra i partecipanti.

Per effettuare l'incantesimo in gruppo i personaggi devono mettersi
d'accordo in anticipo, designando il \textbf{leader}, che \`e il mago
che effettuer\`a, se necessario, i TV nell' Evocazione e prender\`a le
altre decisioni sull'incantesimo.
\fi

\subsection{Tempo di Lancio (TL)} 

Il numero di round necessari per completare l'attivazione di un
incantesimo, detto Tempo di Lancio (TL), \`e solitamente uguale ad
1/6 della DR, salvo condizioni particolari, con un minimo di 1 round.

Il Tiro Iniziativa viene effettuato soltanto per il round durante il
quale l'incantesimo ha effetto, cio\`e l'ultimo round del tempo di
lancio. Nei round precedenti il round di attivazione, il tentativo di
effettuare altre azioni comporta la realizzazione di un TCONC a
difficolt\`a variabile a seconda della complessit\`a dell'azione
per evitare di perdere l'incantesimo.

Mantenere la concentrazione mentre si cammina comporta la realizzazione
del TCONC a difficolt\`a 5, mentre si corre a difficolt\`a 15, mentre
si effettua un'azione di combattimento a difficolt\`a 35. 

Se si subiscono dei danni il TCONC \`e a difficolt\`a 30.

\nb{Non \`e possibile compiere alcuna azione nel round di attivazione
  dell'incantesimo}

\subsection{Consumo degli incantesimi} 

\textbf{Per attivare un incantesimo ci si stanca.} La stanchezza \`e
espressa in termini di PM (Punti Mente), PF (Punti Fisico). Ogni
incantesimo prevede inoltre un consumo di PE (Punti Energia) necessari
all'incantesimo per svolgere i suoi effetti.

I PM spesi sono pari alla DR dell'incantesimo. 

I PF spesi sono 1/3 della DR. 

I PE spesi sono pari alla DA. 

Sia che l'incantesimo riesca sia che fallisca devono essere sottratti
i PM, PF e PE.

Se i PM e i PF del personaggio non 
sono sufficienti, l'incantesimo non pu\`o
essere neppure tentato. Per quanto riguarda i PE,
invece...

{\raggedright \subsection{Prelevare l'energia magica dall'esterno}}

Se i PE non sono sufficienti all'attivazione dell'incantesimo o non li
si vuole utilizzare per qualunque motivo, essi possono essere
prelevati dall'esterno.

Su Quadrantal l'energia si pu\`o attingere dai cosiddetti ``venti
della terra'', canali energetici che provengono dal centro del
pianeta, e che vengono emanati in quantit\`a variabile attraverso la
superficie. La canalizzazione dell'energia magica avviene realizzando
un tiro a difficolt\`a 25 su una qualsiasi delle Scuole di Magia
conosciute.

\iffullversion
Ricordiamo che Utilit\`a e Salvaguardia non sono considerate Scuole,
bens\`{\i} Liste, perci\`o se si conoscono solamente liste non \`e
possibile canalizzare l'energia esterna.

Se il tiro ha esito positivo si canalizza (ossia si rende disponibile
per l'incantesimo) l'energia desiderata sino al massimo valore di PE
contenuto nei venti.

Se il tiro ha esito negativo la quantit\`a di energia canalizzata
\`e determinata casualmente a seconda dell'intensit\`a del vento
nella zona come da tabella ``I venti della terra'' a pagina
\pageref{venti}

\goodbreak
\begin{radtable}{I Venti Della Terra}{|c|c|l|}\label{venti}
PE& Numero di Dadi& Intensit\`a\\ \hline\hline 
1-10& 1d10& Debole\\ \hline
1-20& 1d20& Leggero \\ \hline
3-30& 3d10& Medio \\ \hline
4-40& 4d10& Forte \\ \hline
6-60& 6d10& Fortissimo \\ \hline
8-80& 8d10& Potente\\ \hline
11-110& 1d100+10& Molto potente \\ \hline
12-210& 2d100+10& Immane \\ \hline
\end{radtable}

In particolari luoghi del pianeta i venti potranno essere anche
pi\`u potenti (a discrezione del Master).  Inoltre i venti saranno
inesistenti in mare e su qualsiasi specchio d'acqua pi\`u profondo
di 10 metri. 
\fi

Generalmente i venti sono \textbf{medi}. Sono leggeri in zone
vicine al mare e dove l'influsso della terra \`e limitato da qualche
fattore; sono forti sottoterra; da fortissimi a molto potenti in
``zone consacrate'' alla terra, immani nelle zone runiche o nelle
Torri della Magia. L'energia canalizzata con questo procedimento
potr\`a essere quindi inferiore (in tal caso si potr\`a supplire
attingendo dai propri PE o impiegando i round successivi per attingere
ulteriore energia dai venti, ripetendo quindi il tiro per
canalizzare), superiore (in tal caso se non dovesse essere scaricata
nel tempo e nel modo descritto al paragrafo ``Scaricare l'Energia
Residua'' a pagina \pageref{scaricare} si subiranno le conseguenze dovute al sovraccarico come da
tabella ``Effetti dell'Energia Residua''), o uguale (e in tal caso
nessun problema). 

Per canalizzare i venti occorre avere almeno una mano libera rivolta
verso il suolo. Un adepto di una scuola di magia \`e capace di stimare
senza fatica la potenza dei venti nel punto in cui si trova (ossia
senza bisogno di alcun tiro). 

A questo punto non resta che lanciare l'incantesimo.

\nb{Non \`e possibile reintegrare i propri
PE assorbendoli dai venti}

\subsection{Scaricare l'energia residua}
\label{scaricare}
Per lo scarico dell'energia residua dopo che si \`e lanciato l'incantesimo,
occorre effettuare un tiro a difficolt\`a 20 su una delle Scuole di
magia e poggiare la mano sull'oggetto in cui si intende scaricare.  A
tal fine sono idonei gli oggetti che hanno un contatto diretto con il
terreno poich\'e l'energia scaricata scende a cascata verso il centro
del pianeta.

L'energia deve essere scaricata o utilizzata nel 
round successivo al lancio dell'incantesimo.

Sono validi quindi: muri, pavimenti al pianterreno,
rocce, ecc. Non funzionali sono i soffitti, i pavimenti non al
pianterreno, le porte se sollevate da terra, ecc. 

Gli oggetti su cui
si pu\`o scaricare l'energia inoltre devono essere inanimati e
quindi non possono essere utilizzati gli esseri viventi. L'energia
residua pu\`o anche essere scaricata pronunciando altri incantesimi
nei round immediatamente successivi.

\iffullversion
\subsection{Conservare l'energia magica}

L'energia in eccesso pu\`o essere conservata (scaricandola) in
``contenitori'' magici fatti di materiali particolari e naturali, gli
stessi che devono essere usati per costruire gli oggetti magici. I
``contenitori'' sono generalmente rari e costosi; i pi\`u scadenti
contengono 15-30 PE, i pi\`u costosi e capienti possono facilmente
superare i 1000 PE.

\nb{Oggetti creati magicamente non possono contenere PE.}

\begin{radtable}{I Contenitori Di Energia}{|l|l|} 
Oggetto& PE (MAX) \\ \hline\hline
Guscio di Tsyu-betch& 70 PE / kg \\ \hline
Coralli& 15 PE / kg \\ \hline
Diamanti naturali& 20000 PE/ kg \\ \hline
Scaglie di Drago Adulto& 300 PE / kg \\ \hline
Myrtium& 2500 PE/ kg \\ \hline
Tela di Argia& 40 PE / kg \\ \hline
Dente di Drago Adulto& 3000 PE / kg \\ \hline
Corno di Unicorno& 3000 PE/ kg \\ \hline
Stelle marine Kalareyna& 2000 PE / kg \\ \hline
\end{radtable}

Per scaricare su tali materiali \`e necessario un tiro a
difficolt\`a 15 sulla scuola di magia.  Per poter utilizzare i punti
energia conservati in questi materiali \`e sufficiente avere una
mano a contatto con essi.
\fi

\es{Seinen, Illusionista Luxi con un VAL
di 20, si trova in prossimit\`a delle rovine di Giassa. Vuole lanciare un
incantesimo che richiede 35 PE. Seinen stima la potenza dei venti, accorgendosi
che sono Forti (4d10). Prova quindi a controllarli. Tira per controllare e ottiene
11: 20 (VAL) + 11 (1d20) = 31, che \`e maggiore di 25. Seinen riesce a ricavare
dai venti tutta l'energia che gli serve per attivare l'incantesimo (con un massimo
di 40). 

\iffullversion
Se avesse ottenuto col dado meno di 5, la quantit\`a di energia ricavata
sarebbe stata casuale (il Master avrebbe tirato 4d10), quindi Seinen si sarebbe
potuto trovare in uno di questi tre casi: 
\begin{itemize}
\item i 4d10 totalizzano 35. Per caso, l'energia
canalizzata \`e pari a quella necessaria. L'incantesimo pu\`o essere immediatamente tentato.

\item i 4d10 totalizzano 25 (minore di
35). Seinen deve prelevare l'energia mancante dall'interno (cio\`e dai suoi
punti energia) oppure tentare ancora di controllare i venti nel round successivo.
Il tiro per il controllo viene ripetuto, cumulando i 25 PE alla nuova canalizzazione.
Il procedimento pu\`o essere ripetuto per pi\`u round consecutivi. 

\item i 4d10 totalizzano 40 (maggiore di 35). Seinen ha accumulato 5 PE in pi\`u rispetto
a quelli necessari. Nel round successivo al lancio dell'incantesimo dovr\`a
quindi necessariamente scaricare quella in eccesso. Ci\`o pu\`o avvenire:
effettuando un nuovo incantesimo, scaricando l'energia a terra (Diff. 20), scaricandola
in un contenitore magico (Diff. 15)
\end{itemize}
\fi
} 

\subsubsection{Effetti dell'energia residua}
\label{energiaresidua}
 In caso
di mancato scaricamento dell'energia residua, si possono presentare numerosi
effetti collaterali a seconda della quantit\`a presente all'interno del corpo.
Secondo quanto segue

\begin{radtable}{Effetti dell'Energia Residua}{|r@{ PE }|p{4cm}|}
  1-3 &\raggedright Cambiamento del colore degli occhi, dei capelli e simili. \tabularnewline \hline
  4-10 &\raggedright Perdita dei capelli, occhi infossati, e simili. -1 BEL \tabularnewline \hline
  11-20 &\raggedright Pelle squamata, labbro leporino, dita palmate, gengiviti. -3 BEL. \tabularnewline \hline
  21-40 &\raggedright Asma, tosse perenne, (-2 COS), distrofia ad un arto
  (da -3 a -6 AGI e FOR), amnesie parziali e simili. \tabularnewline \hline
  41-100 &\raggedright Follia (si va a 0 PSI), paralisi
  agli arti (da -5 a -10 AGI e FOR), demenze leggere (-1 INT). \tabularnewline \hline
  101-200 &\raggedright Mutismo,
  cecit\`a, paralisi totali, gravi demenze (da -3 a -10 INT) \tabularnewline \hline
  Oltre 200 &\raggedright Coma 1d6 anni o morte \tabularnewline \hline
\end{radtable}

\iffullversion
Gli effetti possono anche apparire gradualmente ed a discrezione
del Master potranno anche essere cumulativi.

Il Master dovrebbe adattare l'aspetto ``coreografico'' degli effetti
negativi alla Scuola utilizzata, ad esempio pietrificando la gamba ad
un Elementalista (-5 AGI), facendo spuntare le corna ad uno Stregone o
le zanne ad un Demonologo (-1 BEL), rendendo un Necromante magro e
pallido come un morto (-2 COS), ecc.

\es{Martin, Elfo Umbra con VAL 10 in Stregoneria cerca di controllare
  i venti, per effettuare un incantesimo che richiede 40 PE, dopo
  averne stimato la potenza (venti potenti: 8d10). Tira il dado e
  ottiene 9, totalizzando 19. Il controllo fallisce. Il Master tira
  8d10 ottenendo 51. Martin effettua con successo l'incantesimo, e gli
  restano da scaricare 11 PE. Martin tenta di scaricare a terra, tira
  il dado ottenendo 3, totalizzando 13. Il punteggio \`e
  insufficiente. Il Master consulta la tabella ``Effetti dell' energia
  Residua'' decidendo che Martin avr\`a un malus di -2 all'AGI dovuto
  al rallentamento degli stimoli nervosi (come avviene al Bradipo).}
\fi

\section{Costruire gli incantesimi} 

\subsection{Categorie Funzionali}

Gli incantesimi, oltre a far parte di una \textbf{Scuola di Magia} o di
una \textbf{Lista}, sono ``catalogati'' in \textbf{Categorie
Funzionali} a seconda di quale \`e la loro \textbf{Funzione}.

Le categorie funzionali sono:

\begin{itemize}\itemsep -6pt
\item Alterazione
\item Attacco
\item Controllo
\item Difesa
\item Evocazione
\item Informazione
\end{itemize}

\textbf{La difficolt\`a} dell'incantesimo \textbf{dipende dalla sua funzione.}
Questo spiega l'importanza della classificazioni degli incantesimi 
per \textbf{Categoria Funzionale} e \textbf{Tipo}.

La Difficolt\`a Assoluta (DA) degli incantesimi \textbf{non dipende dalla
Scuola}, ma soltanto dalla \textbf{Funzione}

\subsection{Potenzialit\`a e DR}

Oltre a dipendere dalla funzione, la difficolt\`a di \textbf{lancio}
dell'incantesimo (Difficolt\`a Relativa, DR) dipende anche dalla
Scuola.

Ad ogni scuola viene associato un valore per ognuna delle Categorie
Funzionali; tali valori costituiscono le \textbf{Potenzialit\`a}
della scuola. Maggiore \`e la potenzialit\`a di una Scuola in una
Categoria, pi\`u \`e semplice per gli adepti di quella scuola
realizzare incantesimi che appartengano a quella Categoria.

Sottraendo la Potenzialit\`a alla DA si ottiene la DR
dell'incantesimo. \textbf{A parit\`a di effetto} (quindi di DA), ad
una \textbf{Potenzialit\`a maggiore} corrisponde una
\textbf{Difficolt\`a Relativa DR minore}.

Ad ogni Tipo sono correlati due valori che concorrono a determinare
con precisione la DA, chiamati \textbf{Difficolt\`a di Tipo (DT)} e
\textbf{Difficolt\`a di Base (DB)}.

\subsection{Difficolt\`a di Tipo (DT)} 

\`E un valore \textbf{costante per ogni Tipo} d'incantesimo. In alcuni casi
(cio\`e quando qualcuno tenta di fare un incantesimo il cui tipo non
\`e stato specificato) pu\`o essere necessario che il Master
stabilisca una nuova DT usando come metro di paragone le DT indicate
nelle regole.

Troverete le DT dei tipi indicati in dettaglio nei paragrafi relativi
ad ogni Categoria Funzionale.

\es{Vicious vuole creare un incantesimo di Levitazione. La categoria
  funzionale relativa \`e Alterazione e il Tipo \`e ``spostamento
  verticale'', che ha una DT pari a 17.}

\subsection{Difficolt\`a di Base (DB)} 

\`E il valore pi\`u importante, poich\'e stabilisce la ``potenza''
degli effetti dell'incantesimo. La DB \`e calcolata sommando fra
loro i parametri caratteristici del tipo di incantesimo a seconda
della potenza che gli si vuole attribuire.

I diversi parametri necessari a calcolare la DB sono indicati in
dettaglio, per ogni Tipo, nei paragrafi delle Categorie Funzionali.

\es{Continuando l'esempio precedente: la DB prevede dei valori che
  vanno sommati fra di loro fino ad ottenere l'effetto desiderato.
  Vicious deve sollevare un peso di 80 kg (s\'e stesso) pi\`u 5 kg di
  equipaggiamento e vuole spostarsi ad una velocit\`a di 100 km/h.
  Egli deve quindi sommare 3 (1 ogni 30 kg di massa vivente) + 1
  (1 per ogni 100 kg di massa non vivente) + 10 (1 per ogni 10 km/h di
  velocit\`a), la DB quindi \`e pari a 14.}

\subsection{Modificatori Standard} 

Indicano la ``portata'' dell'incantesimo, e possono essere applicati
praticamente a tutti gli incantesimi (se non specificato
diversamente). 

I Modificatori \textbf{si possono imparare solo se vengono
studiati da libri o spiegati da persone che li conoscono}, 
eccetto quelli seguenti, conosciuti dal PG fin dall'inizio, 
a patto che abbia almeno 1 VAL in una Scuola o Lista magica:

\begin{itemize}\itemsep -6pt
\item il Modificatore di Campo limitato ad 1 Target
\item il Modificatore alla Gittata limitato a 7 metri
\item il Modificatore alla Durata illimitato
\item il Modificatore Bonus/Malus al TV/\-TR/\-TP/\-TOSS.  
\end{itemize}

Ecco spiegati in dettaglio i Modificatori standard.

\subsubsection{Gittata} 
Definisce la distanza massima del punto iniziale di azione
dell'incantesimo, ovvero la distanza massima degli oggetti su cui
l'incantesimo avr\`a effetto (Target) da colui che attiva
l'incantesimo.

Il modificatore vale: \textbf{-1 punto per gli incantesimi a gittata
  nulla (\textit{a contatto}), zero punti fino ad 1 metro, +1 punto per
  ogni 3 metri oltre il primo.}

\iffullversion
\subsubsection{Azione lungo la gittata} 
\`E possibile per alcuni tipi di incantesimo fare in modo che
l'incantesimo agisca, oltre che nel punto di arrivo, \textbf{anche
  lungo la traiettoria} dal mago al punto di arrivo stesso.  Il
modificatore \`e pari a \textbf{3 punti}.

In tal caso si pu\`o scegliere se fargli seguire una traiettoria
\textbf{``conica''} (dalle mani del mago l'incantesimo parte
``espandendosi'' progressivamente fino a raggiungere l'area colpita) o
\textbf{``cilindrica''} (l'incantesimo si ``espande'' immediatamente e
percorre la traiettoria colpendo una ``striscia'' di larghezza pari al
doppio del Campo d'azione).

\subsubsection{Rimbalzi} Se
l'incantesimo ha azione lungo la gittata \`e possibile \textbf{farlo
rimbalzare} ``a specchio'' su oggetti inanimati per un numero limitato
di volte. 

I bersagli possono essere colpiti pi\`u volte (e subiscono
il danno pi\`u volte) a seconda del rimbalzo.

Per calcolare il rimbalzo pu\`o essere necessario un tiro
sull'abilit\`a Calcolare con difficolt\`a stabilita dal Master
di volta in volta.

Questo modificatore vale \textbf{1 punto ogni rimbalzo} fino ad un
massimo di 5 rimbalzi e si pu\`o applicare soltanto agli incantesimi
di Attacco.
\fi

\subsubsection{Campo} 

Definisce \textbf{il numero di oggetti (Target)} sui quali
l'incantesimo ha effetto \textbf{oppure l'area circolare}, con centro
nel punto iniziale di azione, sulla quale l'incantesimo agisce. 

Il
modificatore \`e pari a \textbf{2 punti per ogni Target} oppure a \textbf{1 punto
per ogni metro di raggio}, oppure \textbf{6 punti ogni d6 di Target}

\nb{Le armature, le armi, l'equipaggiamento indossato da target viventi
  non devono essere considerati target a s\'e stanti: il target \`e
  colui che le indossa, e gli incantesimi devono essere applicati ad
  esso. Per applicare un incantesimo ad un'arma, un'armatura, ecc.
  queste non devono essere indossate o in uso.}

\subsubsection{Durata} 
Il modificatore \`e pari a \textbf{1 punto ogni 4 minuti oltre i
  primi 3}.

\`E possibile tenere l'incantesimo \textbf{``a concentrazione''}
(l'incantesimo opera fino a quando non ne viene tentato un altro o
viene compiuta un'azione che determina la perdita della
concentrazione); in tal caso il modificatore \`e pari a \textbf{5
  punti}. Quando la concentrazione si interrompe, l'incantesimo
perdura altri tre minuti.

\`E possibile rendere l'incantesimo \textbf{Permanente} facendolo
perdurare per tutto il tempo desiderato: in tal caso il modificatore
\`e pari a \textbf{15 punti}. 

Una volta sciolta la permanenza, l'effetto dell'incantesimo termina
esattamente come gli incantesimi con durata prestabilita.

Applicando la permanenza ad un incantesimo si ha la perdita permanente
di un numero di PE pari ad 1/8 dei PE richiesti per quell'incantesimo.

\nb{Per ``PE permanenti'' si intende che devono essere
sottratti dai PE del personaggio e che non possono essere recuperati
finch\'e la Permanenza non viene sciolta.}

La Permanenza pu\`o sciogliersi in seguito ad un evento qualunque a
scelta del mago.

La Permanenza si dissolve comunque alla morte del mago.

\iffullversion
Un caso particolare di permanenza si ha quando l'incantesimo \`e
reso permanente ma i suoi effetti si manifestano in modo non continuo,
cessando temporaneamente in certi periodi, a discrezione del mago (ad
esempio, una trasformazione che avviene solo durante le notti di luna
piena, o solo durante il giorno o solo quando il mago lo desideri,
ecc.). 

Questo tipo di permanenza \`e detto \textbf{Permanenza Ciclica} e il
modificatore \`e pari a \textbf{20 punti}.  La perdita di PE
permanenti \`e sempre 1/8. La permanenza ciclica si dissolve alla
morte del mago.


\subsubsection{Bonus/malus al TV/TR/TP/TOSS}

Normalmente il TR/TV/TP/TOSS che il target deve effettuare per
limitare o evitare gli effetti dell'incantesimo \`e \textbf{uguale
  alla DB + 10} (salvo dove specificatamente indicato nella DB).

In certi casi pu\`o essere applicato un modificatore al
TV/TR/TP/TOSS del bersaglio o del mago: il modificatore \`e pari a:
\textbf{+1 ogni punto di aumento} della difficolt\`a del TV/TR/TOSS
e \textbf{-1 per ogni punto di diminuzione} della difficolt\`a del
TV/TR/TP/TOSS.

Il modificatore pu\`o valere \textbf{al massimo 5 punti} in
aumento o diminuzione.


\subsubsection{Azione nel tempo}
L'incantesimo pu\`o non agire subito, ma svolgere la sua azione
gradualmente (e costantemente) nel corso del tempo. In quei casi il
valore del modificatore \`e indicato nella seguente tabella: 

\begin{radtable}{Azione nel Tempo}{|l|c|}
da 1 a 6 ore (1d6 ore)& 0 \\ \hline
da 6 a 24 ore (6d4 ore)& -1 \\ \hline
da 1 a 6 giorni (1d6giorni)& -3 \\ \hline
da 7 a 28 giorni (1d4 settimane)& -5\\ \hline
 oltre 28 giorni& -7 \\ \hline
\end{radtable}

\es{Vicious desidera che gli effetti del suo incantesimo si producano
  gradualmente in 1d6 giorni. Il modificatore ha un valore di -3}

\fi

\subsection{Calcolo della DA e della DR} 
\label{dadr}
Per calcolare la DA \`e sufficiente sommare fra loro: 

\begin{itemize}\itemsep -6pt
\item Il Valore della \textbf{DT}
\item Il Valore della \textbf{DB}
\item I Valori di tutti i \textbf{Modificatori} standard (tenendo
  presente che possono anche essere negativi).
\end{itemize}

Dopo aver calcolato la DA \`e semplicissimo calcolare anche la DR,
essa \`e infatti pari alla \textbf{DA meno la Potenzialit\`a} di
ciascuna scuola nella Categoria Funzionale di cui fa parte
l'incantesimo.

In pratica, gli incantesimi che producono gli stessi effetti hanno la
stessa difficolt\`a assoluta (\textbf{DA}) in tutte le scuole, ma le
difficolt\`a relative (\textbf{DR}) possono essere molto diverse.

Riassumendo, la DA
di un incantesimo \`e data da: 
$$DA = DT + DB + Modificatori\;Standard$$ 
Mentre la DR \`e uguale a: 
$$DR = DA - Potenzialita$$
della Scuola utilizzata.

\iffullversion
{\raggedright \subsection{Riduzione del Tempo di Lancio}}

\`E possibile ridurre il Tempo di Lancio (TL) necessario per attivare
l'incantesimo. In tal caso il Tiro sull'abilit\`a Magica per il
lancio dell'incantesimo sar\`a penalizzato di un malus di \textbf{-5
  per ogni round in meno} rispetto al Tempo di Lancio previsto.
Perci\`o se si vuole lanciare un incantesimo che necessita di un TL
di 3 round in 1 round, si avr\`a un malus pari a -10. DA, DR e tutti
gli altri parametri dipendenti da esse rimangono inalterati.
\fi

\subsection{Il Differimento} \`E
possibile posticipare il momento dell'attivazione dell'incantesimo e
farlo coincidere con un evento particolare a scelta dell'usufruitore.

Per fare ci\`o il Tiro sull'abilit\`a Magica viene penalizzato di
un \textbf{malus di -10} al momento del lancio. Il consumo dei PM, dei
PF e dei PE va applicato al momento del lancio dell'incantesimo e non
della sua attivazione.

L'attivazione dell'incantesimo avviene all'inizio del Round in cui
l'evento si verifica.

L'evento pu\`o consistere in una \textbf{manifestazione di
  volont\`a} dell'usufruitore o di un altro essere scelto
dall'usufruitore, \textbf{in un gesto}, o nel \textbf{verificarsi di
  determinate condizioni} anche indipendenti dalla sua volont\`a.

L'usufruitore, rendendo un incantesimo differito, perder\`a inoltre
un numero di \textbf{PE permanenti pari ad 1/10 dei PE necessari al
  lancio} dell'Incantesimo, fino al momento dell'attivazione
dell'incantesimo.

Se il mago muore, l'incantesimo differito si dissolve senza effetti.

\iffullversion
\subsection{Indovinare un incantesimo}

Un mago pu\`o riuscire ad indovinare in 1 round quale sar\`a
l'effetto dell'incantesimo che un altro mago sta pronunciando. Per
fare ci\`o \`e necessario realizzare \textbf{un tiro sulla Scuola}
a cui appartiene l'incantesimo da indovinare ad una
\textbf{difficolt\`a pari alla DR} dello stesso. Il mago pu\`o
compiere senza malus altre azioni nello stesso round.

Un mago che abbia almeno \textbf{VAL 1} in una scuola di magia \`e
in grado di capire immediatamente se un incantesimo pronunciato da un
altro appartenga o meno a quella scuola.

Un mago che ha \textbf{VAL 0} nell'abilit\`a Magica corrispondente
ad una Scuola di Magia \textbf{non pu\`o tentare di indovinare} gli
incantesimi di quella Scuola.

Gli incantesimi delle liste sono individuabili effettuando un tiro su
un'abilit\`a magica qualsiasi a scelta del giocatore. 

Se il tiro fallisce l'incantesimo non viene individuato. Un fallimento
catastrofico (1 col dado) fa scambiare l'incantesimo per un altro. 
Il Master pu\`o determinare ``l'entit\`a della svista'' rifacendosi al
Fallimento catastrofico per gli Incantesimi a pagina \pageref{fumblemagia}.
\fi

\subsection{Annullare un incantesimo}

Per annullare un incantesimo occorre effettuare un
\textbf{controincantesimo}. Il controincantesimo \`e a tutti gli
effetti un incantesimo che ha \textbf{la stessa DA, lo stesso tipo e
  la stessa Categoria Funzionale dell'incantesimo da annullare}.
Pu\`o per\`o appartenere ad un'altra scuola: in quel caso la sua
DR dovr\`a essere ricalcolata in base alle Potenzialit\`a della
Scuola utilizzata dal mago che effettua il controincantesimo.

Il controincantesimo viene lanciato con un tiro sull'abilit\`a
magica, se questo ha successo il controincantesimo annulla gli effetti
del precedente incantesimo.

Possono essere lanciati controincantesimi solo su incantesimi
che hanno durata non immediata, a concentrazione o permanente.

Il fallimento del controincantesimo ha le stesse modalit\`a e gli stessi effetti
del fallimento dei normali incantesimi.

Il consumo del controincantesimo in PM, PF, PE e il Tempo di Lancio, 
sono pari a quell di un incantesimo normale di pari DR.

\subsection{Apprendimento degli incantesimi e ricerca magica}

\label{ricercamagica}
Si attribuisce \textbf{un bonus al lancio degli incantesimi se li si
  studia}.

Un incantesimo pu\`o essere:

\subsubsection{Studiato da un libro o insegnato} In
tal caso viene conferito al tiro per il lancio dell'incantesimo un
bonus normalmente \textbf{compreso tra +1 e +10}, a seconda della qualit\`a
del libro. 

Questo bonus viene detto \textbf{Bonus Libro (BL)}. Lo studio del
libro richiede un numero di ore pari alla DR dell'incantesimo. 

Per verificare l'apprendimento il Master effettua un tiro ``nascosto''
sulla abilit\`a nella Scuola di Magia del personaggio (aggiungendo
al tiro il BON INT e il BL): ottenendo un punteggio maggiore o uguale
alla DR l'apprendimento ha successo; se l'apprendimento fallisce, il
bonus per quell'incantesimo cambia di segno, diventando un malus. 

Se l'incantesimo viene insegnato, al tiro di verifica
dell'apprendimento \textbf{si somma il bonus dell'abilit\`a Insegnare} del
maestro (da +1 a +5 secondo la tabella Specializzazioni).

Insegnando un incantesimo si conferisce a chi apprende un bonus pari
(o, a scelta, inferiore) a quello ``posseduto''.

\iffullversion
\subsubsection{Studiato con ricerca} Viene conferito un bonus al lancio
dell'incantesimo pari a +1 per ogni 20 ore di ricerca $-2 \times BON\;INT$. 

Il Bonus dato dalla ricerca magica \`e cumulativo fino a un massimo di
+10, quindi una ricerca pu\`o essere iniziata, interrotta, poi
ripresa in seguito per ottenere nuovi punti di bonus. 

Per scrivere l'incantesimo su un libro \`e necessario il doppio del
tempo. Il libro pu\`o essere ceduto ad altri, e conferisce, se
studiato, un bonus pari a quello della ricerca, meno 2.

\`E possibile ``proseguire'' lo studio di un incantesimo imparato da
un libro per ``migliorare'' il libro, a patto che il bonus totale non
superi +10. In entrambi i casi, al termine di ogni sessione di ricerca
bisogna effettuare un Tiro per l'apprendimento analogo a quello
illustrato precedentemente.

Il successo del Tiro conferisce il bonus dovuto allo studio.  Il suo
fallimento indica che la ricerca o lo studio devono essere ripetuti.

Il test consiste nel lancio effettivo dell'incantesimo, sommando al
tiro del dado il bonus della ricerca. 

Il fallimento del test \`e soggetto alle regole standard del
fallimento del lancio dell'incantesimo.

\nb{Ogni incantesimo \`e caratterizzato da tutti suoi parametri, che
  concorrono a determinare DT, DB e Modificatori Standard.  Cambiarne
  anche uno soltanto comporta la realizzazione di un incantesimo
  differente e quindi la ricerca deve ripartire da zero}

I Valori specificati nel calcolo della DA si devono per\`o considerare
come valori massimi. Al momento del lancio dell'incantesimo il mago
pu\`o decidere di \textbf{limitarne} gli effetti a piacimento
riducendo i valori, senza che il suo sia considerato un nuovo
incantesimo.  Il consumo di PM, PF, PE, e il TL per\`o non si
riducono.

Per costruire un incantesimo (con ricerca o improvvisazione) l'adepto
deve conoscere \textbf{tutti} i modificatori che fanno parte
dell'incantesimo stesso.

Per lanciare un incantesimo studiato da un libro l'adepto
pu\`o non conoscerne tutti i modificatori, ma in tal caso
non ricever\`a dal Master Bonus all'Utilizzo, e non pu\`o
effettuare ulteriori ricerche.
\fi

\subsection{Bonus all'Utilizzo}

Chi utilizza un incantesimo ripetutamente, con successo e
fruttuosamente, ricever\`a dal Master dei bonus al lancio fra loro
cumulabili fino ad un massimo di +10.

\nb{I bonus dello Studio dai
Libri (BL), Studio con Ricerca (BR) e Utilizzo (BU) sono fra loro
cumulabili, fino ad un massimo di +30}

\iffullversion
\subsection{Incantare oggetti} Un
oggetto normale pu\`o essere reso magico ``installandovi'' degli
incantesimi.  L'oggetto deve essere in grado di contenere PE per far
partire gli incantesimi, e deve essere quindi parzialmente o
totalmente costituito da materiale in grado di conservare i PE.

 Il tempo necessario per incantare un oggetto \`e pari a 30 ore di
ricerca $(-2 \times BON\;INT)$ per ogni punto di DR dell'incantesimo. 

Se l'incantesimo \`e gi\`a stato ricercato, studiato, insegnato,
utilizzato con successo, cio\`e si hanno dei BR, BL, BU, al numero
di ore necessarie ad incantare l'oggetto vanno sottratte 10 ore per
punto di Bonus. 

Comunque, in nessun caso si pu\`o scendere sotto le 20 ore
totali di ricerca.

 Il mago che incanta l'oggetto sceglie anche il modo
di attivare gli incantesimi. Tipicamente, basta un TV a difficolt\`a
10 tenendo in mano l'oggetto magico. 

Nel momento in cui viene innescato l'incantesimo installato
nell'oggetto magico, questo si svuota dei PE dell'incantesimo,
recuperandoli al ritmo di 4 all'ora. L'incantesimo si attiva sempre in
un round.

Ogni oggetto magico pu\`o essere
incantato con pi\`u incantesimi. \`E necessario per\`o che il
numero dei PE totali contenuti nell'oggetto sia maggiore o uguale del
totale dei PE richiesti dagli incantesimi. Per inserire pi\`u
``cariche'' dello stesso incantesimo non \`e necessario ripetere la
ricerca.  \`E sufficiente investire 1 ora per ogni punto di DR
dell'incantesimo e accertarsi che il numero di PE sia sufficiente. Il
procedimento deve essere ripetuto per ognuna delle ``cariche''. 

Al termine del procedimento si deve effettuare la verifica di buona
riuscita: l'incantamento ha avuto successo se un tiro sulla Scuola di
magia + BR, BL, BU, eguaglia o supera la DR dell'incantesimo.

L'incantesimo potr\`a essere lanciato nuovamente solo dopo che
l'oggetto ha recuperato per intero i PE spesi, destinati a quella particolare
``carica''.

\nb{I PE possono essere reintegrati nell'oggetto anche col procedimento di controllo
dei venti della terra e successivo scaricamento all'interno
dell'oggetto} 

\es{Edwig Than Lufthas, elfo Umbra esperto di
Stregoneria (14), vuole incantare il suo bastone di legno con: 2
Incantesimi di ``Freccia Incantata'' (Attacco/Danno mirato), che ha
DA=25, quindi DR=25-9=16, PE=25.  1 Incantesimo ``Volare''
(Alterazione/Spostamento 3D) che ha DA=40, quindi DR=40-20=20, PE=40.

In ``Volare'' Edwig ha gi\`a un bonus di +3 dovuto a ricerca magica
precedente e utilizzo frequente dell'incantesimo, mentre non conosce
``Freccia Incantata''. 

Edwig ha 20 in INT, quindi il bonus INT \`e
+5, e perci\`o, per ogni punto di DR, gli servono 20 ore. In totale
il bastone deve contenere 25 (primo dardo) +25 (secondo dardo) +40
(volare)=90 PE. Edwig traffica un po' per mercati, infine spende tutti
i suoi risparmi per comprare un frammento di scaglia di drago capace
di inglobare 90 PE. Inserisce quindi il frammento nel pomello del
bastone. Per incantare il bastone inserendoci il primo dardo sono
necessarie quindi 16x20 ore=320 ore, cio\`e poco pi\`u di un mese
di lavoro. Al termine del lavoro il bastone viene
collaudato.  Edwig tira e ottiene un 5. La ricerca ha avuto successo
perch\'e 14 (Scuola) +5 (Dado) = 19, che \`e maggiore della DR. Edwig
decide che il primo dardo parte puntando il bastone verso l'obiettivo e
gridando ``Fuori uno!''. 

Per incantare il secondo dardo, che \`e uguale al primo, bastano 16
ore e un nuovo tiro, che Edwig realizza senza difficolt\`a (con un 8).
Nel giro di meno di 2 giorni anche il secondo incantesimo \`e
installato nel bastone. Edwig decide che il secondo dardo parte
puntando il bastone verso l'obiettivo e battendo il piede destro per
terra. In questo modo pu\`o lanciare i due dardi contemporaneamente.
Chiunque entrasse in possesso del bastone dovrebbe conoscere questi
metodi per poter utilizzare gli incantesimi.

Per il Volare sarebbero necessarie 20x20 ore=400 ore, ma avendo
gi\`a un bonus di +3, ne servono 10x3=30 in meno: in totale 370 ore
(altre cinque settimane abbondanti). Al termine della procedura Edwig
``tira'' e ottiene un 4. 14+4=18, che \`e minore di 20.

L'inserimento di ``Volare'' sul bastone fallisce. Le 370 ore sono
state ``sprecate''. Edwig, frustratissimo, tenta di ripetere
l'esperimento. 

Dopo \textbf{altre} 370 ore di lavoro, tira ottenendo 12.  14 + 12 =
26, che \`e maggiore di 20. Stavolta sul bastone c'\`e l'incantesimo.
Edwig decide che il modo di attivare l'incantesimo consiste,
semplicemente, nel concentrarsi (TCONC a 20).}

\subsection{Armi Magiche}
 Per Arma Magica, o Arma Incantata, si intende un'arma sulla
quale sia stato installato almeno un incantesimo di qualunque DR.
\fi
%\newcommand{\dt}[1]{\noindent\framebox[\linewidth][l]{\small\tt DT= \parbox[t]{6cm}{#1}}}
%\newcommand{\db}[1]{\noindent\framebox[\linewidth][l]{\small\tt DB= \parbox[t]{6cm}{#1}}}

\newcommand{\dt}[1]{\smallskip\noindent {\raggedright\small\bf DT= \parbox[t]{6.5cm}{\it #1}}\par}
\newcommand{\db}[1]{\smallskip\noindent {\raggedright\small\bf DB= \parbox[t]{6.5cm}{\it #1}}\par}

\section{Alterazione} 

Questa categoria include tutti gli incantesimi che permettono di
muovere o modificare oggetti, esseri e persone variandone le
capacit\`a, senza infliggere direttamente danni o produrre
meccanismi di difesa (armature, cura di ferite, ecc.).

\nb {Il Master deve classificare gli incantesimi volti a infliggere
  danni diretti (o quasi) nella categoria funzionale Attacco, e gli
  incantesimi atti a produrre armature e difese nella categoria
  funzionale Difesa.}

Il fallimento catastrofico specifico di questa categoria consiste nei
fallimenti automatici dei TP, dei TR e dei TV, nell'inversione del
segno delle variazioni (ad es. da un bonus di +3 si passa ad un malus
di -3), nella ribellione delle creature create; in genere
all'inversione degli effetti dell'incantesimo.

\subsection{Tipi di Alterazione}

\subsubsection{Creazione oggetti} Possono essere creati dal nulla
oggetti di forma, dimensioni, complessit\`a variabile, anche organici ma
non animati. 

Al termine della durata dell'incantesimo l'oggetto creato scompare.

Se un oggetto creato permanentemente viene distrutto (es. bruciato o
mangiato), i PE permanenti vengono persi definitivamente.

Il cibo e le bevande, per poter nutrire, devono essere creati
permanentemente.

Con un TOSS pari alla DB+10 chiunque potr\`a identificare l'oggetto
come il prodotto di un incantesimo.


\dt{19}

\db{1 punto ogni 30 kg di Massa Non vivente\\ +
Complessit\`a}

\es{Maya deve attraversare un lago e decide di creare
una zattera. La DT \`e pari a 19, la DB \`e da calcolare nel modo
seguente: la zattera pesa circa 100 kg. Sono necessari perci\`o 4
punti (1 ogni 30 kg). La complessit\`a della creazione \`e
stabilita dal Master nella misura di 5 (oggetto semplice fatto di
legno e corda e poco lavorato). La DB ammonta a 4+5=9. I modificatori
sono invece dati da: tempo stimato da Maya per l'utilizzo della
zattera e cio\`e 15 minuti (0 fino a 3 minuti, 1 per ogni 4 minuti
per i 12 minuti successivi = 3 punti).  La zattera si materializza ad
1 metro (0 punti per la gittata). Il target \`e uno (1 oggetto=2
punti). La DA \`e quindi pari a: 19 (DT)+9 (DB)+5 (modificatori) =
30}

\iffullversion
\subsubsection{Creazione esseri Viventi} Gli esseri viventi creati non sono
vincolati in alcun modo al loro creatore, il quale viene da loro visto come un padre.

Gli incantesimi che trasformano oggetti in esseri viventi rientrano in
questa categoria. 

Gli esseri viventi creati non possono procreare. 

La creazione di un essere vivente \`e sempre permanente; la durata
dell'incantesimo non deve essere calcolata. La perdita dei PE
permanenti deve comunque essere calcolata, e i PE permanenti non
potranno pi\`u essere recuperati.

In questa categoria rientra anche l'animazione di oggetti non viventi
non dotati di poteri magici.

Le caratteristiche non possono avere valori minori di 1 e
maggiori della DB dell'incantesimo + 5. 

Le caratteristiche da considerare sono le 8 primarie; la CON
dell'essere creato \`e 0.

\dt{29}

\db{1 punto ogni 30 kg di Massa Vivente\\ + 1 punto ogni 4 Punti
Caratteristica} 

\es{Mothra ha necessit\`a di una cavalcatura e decide
di creare un fortral. La DT \`e 29, la DB \`e cos\`{\i} calcolata:
poich\'e il fortral pesa circa 180 kg sono necessari 6 punti (1 ogni 30
kg). Inoltre Mothra vuole attribuire al fortral 52 punti
caratteristica, che comportano una spesa di 13 punti (1 ogni 4 punti
caratteristica, 52/4=13).  

La DB ammonta a 6+13=19. I modificatori
sono: 1 target (1 animale creato) = 2 punti. La DA \`e pari a 29
(DT) + 19 (DB) + 2 (Modificatori) = 50} 
\fi

\subsubsection{Spostamento}
  Permette lo spostamento visibile degli oggetti con traiettorie variabili e
sotto il controllo del mago. Si pu\`o far s\`{\i} che lo
spostamento rientri sotto il controllo dell'essere su cui viene fatto
l'incantesimo. 

Gli oggetti spostati magicamente vengono usati dal mago
con le sue normali abilit\`a (ad esempio, un carro spostato
magicamente va guidato dal mago con la sua abilit\`a Guidare Carri),
armi comprese. Il Danno determinato da oggetti che si classificano tra
le Armi Occasionali (AO) \`e stabilito dal Master. 

Ai Danni
eventualmente determinati da armi e oggetti in movimento si somma il
BON FOR degli oggetti.  La FOR degli oggetti spostati \`e pari alla
DB dell'incantesimo. 

Gli oggetti si possono far muovere su un piano,
in alto e in basso (come nella levitazione) o combinando i due
movimenti per ottenere uno spostamento su 3 dimensioni. 

Il mago perde
il controllo sul target spostato se questo scompare dalla sua vista.
In tal caso l'oggetto interrompe il suo movimento, bloccandosi. 

Il
mago pu\`o, al momento dell'incantesimo, ``cedere'' (senza poterlo
riprendere in seguito) il controllo dell'oggetto a chiunque si trovi
ad una distanza massima pari alla gittata. 

\dt{16 (Spostamento sul piano)}

\dt{17 (Spostamento verticale)} 

\dt{19 (Spostamento su 3 dimensioni)}

\db{1 punto ogni 30 kg di Massa Vivente\\ + 1 punto ogni 100
kg di Massa Non vivente\\ + 1 punto ogni 10 km/h di Velocit\`a}

\es{Toshi deve attraversare un burrone alto 60 m e decide di farlo volando
(spostamento in 3d). La DT \`e 19. La DB \`e pari a 3 (poich\'e
Toshi pesa 70 kg) +1 (l'equipaggiamento di Toshi, che pesa meno di 100
kg!) + 1 (Toshi decide di spostarsi a 10 km/h).  La DB \`e
perci\`o pari a 3+1+1=5. I modificatori sono: 1 target (Toshi) +2;
gittata nulla (a contatto) -1; durata (fino a 3 minuti) +0. La DA
ammonta a 19 (DT) + 5 (DB) +1 (Modificatori) = 25}

\subsubsection{Teletrasporto} Per il teletrasporto (trasferimento a distanza
immediato) \`e necessario conoscere o indicare con precisione il
punto di arrivo. 

Il target soggetto per la prima volta ad un
teletrasporto deve effettuare un TP pari a 25 o restare shockato (-10
su tutti i tiri) per 1d6 di minuti. 

Esistono tre tipi di teletrasporto
(con diverse DT) che hanno la stessa DB. 

\db{1 punto ogni 30 kg di
Massa Vivente\\ +1 punto ogni 100 kg di Massa Non Vivente\\ + 5 punti ogni
potenza di 10 della Distanza di Spostamento in m}

\subsubsection{Teletrasporto Standard}
 Il target pu\`o essere trasportato in un punto conosciuto
dal mago, ad esempio un posto che il mago ha visitato. Il target,
pu\`o subire danni a discrezione del Master qualora nel luogo di
arrivo ci siano ostacoli o pericoli. Non si pu\`o teletrasportare un
target contro la sua volont\`a.

\dt{19}

\subsubsection{Teletrasporto Intelligente}  
Con un incantesimo pi\`u difficile si pu\`o creare
un teletrasporto ``intelligente'' che inibisca tali
danni trasportando il target nel luogo sicuro pi\`u vicino al punto
designato. 

\dt{24}

\subsubsection{Teletrasporto Semintelligente} 

Si pu\`o costruire un incantesimo ``se\-mi\-in\-tel\-li\-gen\-te'', che inibisca
i danni, in cambio di una limitazione stabilita dal giocatore e dal
Master o dal regolamento (es. ci deve essere un oggetto di
propriet\`a del mago nel punto di arrivo, ecc.).

Se nel punto di arrivo il target pu\`o subire danni, egli viene
trasportato nel luogo sicuro pi\`u vicino al punto designato oppure
pu\`o decidere che il teletrasporto non funzioni affatto.

\dt{21}

\es{Gopher vuole
teletrasportare Sarah alla di lei residenza, pur non essendoci mai
stato. 

La \textbf{DB} \`e: \textbf{+2} (Sarah pesa meno di 60 kg) \textbf{+1} (equipaggiamento
di Sarah) \textbf{+20} (la casa di Sarah dista 10 km = 10.000 metri
cio\`e 4 zeri dopo l'uno, quindi 4x5)=\textbf{23}. 

I \textbf{modificatori} da
calcolare sono: +2 (1 target: Sarah); -1 (gittata nulla). Se Gopher
utilizzasse un teletrasporto standard, la DT sarebbe pari a 19 e la DA
pari a 19 (DT) +23 (DB) +1 (Modificatori) = 43. La materializzazione
nella casa sarebbe per\`o casuale, per cui Sarah potrebbe trovarsi a
6 metri dal pavimento oppure materializzarsi all'interno di un oggetto
subendo i conseguenti danni. 

Se utilizzasse un teletrasporto
intelligente, la DT sarebbe \textbf{24} e la DA 24+23+1=48; la
materializzazione sarebbe al livello del pavimento e in un luogo
sicuro. 

Se utilizzasse il teletrasporto semintelligente, questo
funzionerebbe come quello intelligente a patto che nell'abitazione di
Sarah ci sia, per esempio, un oggetto precedentemente toccato da
Gopher (ad. Es. un vaso regalato da Gopher a Sarah anni prima). In
questo caso la DT \`e 21 e la DA \`e 21+23+1=45}

\subsubsection{Modifica dell'anatomia}
Le parti del corpo di un essere possono essere variate per modificare
o creare organi che svolgono funzioni particolari (branchie, ali,
ecc.). 

La modifica dell'anatomia non pu\`o infliggere danni al target, nemmeno
indirettamente. 

Se il target non \`e predisposto (non conosce gli effetti
dell'incantesimo) deve effettuare un TP a difficolt\`a pari alla DB
+ 10 o restare shockato (-10 a tutti i tiri) per 1d6 minuti.

Se la trasformazione implica anche una variazione delle
dimensioni, si deve sommare anche la difficolt\`a di base della
variazione come per Variazione delle Dimensioni.

\dt{15}
 
\db{1 punto ogni 30 kg di Massa Vivente\\ + Complessit\`a\\ + 2 punti
  ogni unit\`a di Fattore Dimensione}

\es{Fletcher si trova in una grotta la cui unica uscita \`e un
  sifone colmo d'acqua, egli ritiene opportuno, per evitare rischi
  inutili, di dotarsi di un paio di branchie per poter respirare
  sott`acqua. 

  La DT dell'incantesimo e 15, la DB \`e data da: 1
  (massa vivente coinvolta inferiore a 30 Kg) +10 (complessit\`a del
  nuovo organo)=11. 

  I modificatori da calcolare sono: +2 (1 target:
  Fletcher), -1 (gittata nulla) +5 (Fletcher intende mantenere
  l'incantesimo a concentrazione) =6.  La DA \`e pari a: 15 (DT)
  +11 (DB) +6 (modificatori)=32}

\subsubsection{Modifica dell'anatomia per
generare armi o Aumento del danno delle armi} 

Le parti del corpo dei viventi possono esser modificate per produrre
armi o rendere pi\`u efficaci gli attacchi a mani nude (CAC).

Le armi cos\`{\i} ottenute vengono utilizzate come armi normali, con
le normali Abilit\`a di Combattimento.

Se un arto del target viene modificato per ottenere un'arma da
utilizzare con un'abilit\`a diversa da CAC, al danno inflitto dall'arma
\textbf{non si deve sommare il danno base di CAC}.

Con lo stesso tipo di incantesimo si pu\`o \textbf{aumentare} il danno
inflitto da armi esistenti.

\dt{20} 

\db{1 punto ogni 30 kg di Massa\\ + 5 punti per ogni d6 di Danno che l'arma pu\`o infliggere} 

\es{Turok si trova a combattere a mani nude contro una guardia
  vescovile.  Decide di modificare l'anatomia del suo braccio per
  dotarlo di una escrescenza ossea da utilizzare come una spada. La DT
  \`e pari a 20.  La DB \`e pari a: 1 (massa vivente entro i
  trenta Kg) +15 (3d6 di danno, 5 per ogni d6) =16. I modificatori
  sono: +2 (1 target: Turok), -1 (gittata nulla). La DA ammonta a: 20
  (DT) +16 (DB) +1 (modificatori)=37. Mentre l'incantesimo \`e attivo,
  il braccio di Turok infligger\`a il danno di una spada, utilizzata
  con l'Abilit\`a di Combattimento ATL}

\subsubsection{Variazione propriet\`a fisiche} 

Le propriet\`a fisiche degli oggetti e degli esseri (peso, colore,
densit\`a, trasparenza, durezza, ecc.) possono essere
variate \textbf{senza cambiamento di forma e di dimensioni.}

Se un target assume una consistenza che impedisce di essere attaccato
fisicamente, lo stesso target non potr\`a attaccare n\`e utilizzare
poteri magici o incantesimi mentre si trova nello stato alterato.

\dt{15} 

\db{1 punto ogni 30 kg di Massa Vivente\\ + 1 punto ogni 100 kg di Massa
  non vivente\\ + Complessit\`a}

\es{Bownlar decide di evadere dalla cella in cui
\`e stato rinchiuso assumendo forma gassosa. La DT \`e 15. La DB
\`e pari a: 3 (massa vivente 90 Kg: Bownlar), +1 (massa non vivente
fino a 100 Kg: l'equipaggiamento) +10 (complessit\`a della forma
gassosa)= 14. I modificatori sono: +2 (1 target: Bownlar) +2 (durata
11 minuti) -1 (gittata nulla)=3. La DA \`e: 15 (DT) +14 (DB) +3
(modificatori)= 32}

\iffullversion
\subsubsection{Variazione caratteristiche fisiche}

Apporta variazioni a una o pi\`u caratteristiche fisiche
\textbf{(FOR, COS, AGI, OSS, BEL)} in aumento o in diminuzione per un
totale di Punti Caratteristica indicato nella DB.

Pu\`o essere effettuata una sola variazione per volta sulla stessa
caratteristica.  

\nb{Il Master pu\`o lasciare la possibilit\`a al giocatore di variare
anche INT e CONC, impedendo per\`o (per ragioni di giocabilit\`a)
di rendere permanenti queste variazioni.}

\dt{19}
\db{6 punti ogni Punto Caratteristica di variazione}

\es{Alvin desidera analizzare accuratamente l'ambiente
  circostante e decide di incrementare la sua OSS di 3 punti. La DT
  \`e 19. La DB \`e: 18 (6x3 punti variazione caratteristica)=
  18: I modificatori sono: +2 (1 target: Alvin) -1 (gittata nulla)= 1.
  La DA \`e: 19 (DT) +18(DB) +1 (modificatori)= 38}
\fi

\subsubsection{Variazione di abilit\`a fisica}

Consente di modificare un'abilit\`a che \textbf{non} sia basata su
INT, CON o CONC con un bonus o un malus. Il numero dei punti
variazione abilit\`a necessari per incrementare (o decrementare) di
1 il TOT dell'abilit\`a \`e pari alla difficolt\`a (D1, D2, D3)
dell'abilit\`a da variare. Pu\`o essere effettuata una sola
variazione alla volta sulla stessa abilit\`a.

Le Abilit\`a di Combattimento si devono considerare a D2 invece che
D1. Ogni punto di variazione di abilit\`a di combattimento far\`a
aumentare di 2 punti la DB.

\dt{18}

\db{1 punto ogni Punto Abilit\`a di variazione}

\es{Elween vuole vincere assolutamente
la gara di tiro al bersaglio organizzata nella sua cittadina. Decide
di incrementare il VAL di ADT di 5 punti. La DT \`e 18. La DB \`e:
10 (2x5 punti variazione abilit\`a)= 10. I modificatori sono: +2 (1
target: Elween), -1 (gittata nulla) +3 (durata 15 minuti)= 4. La DA
\`e pari a: 18 (DT) +10 (DB) +4 (modificatori) = 32}

\subsubsection{Variazione delle dimensioni} 

Le dimensioni di uno o pi\`u target possono essere variate in
aumento o in diminuzione di un Fattore Dimensione (FD) specificato e
maggiore di 1. In aumento di dimensioni, la perdita di AGI per gli
esseri \`e uguale a -5 x (FD-1); il guadagno in FOR e COS \`e
uguale a +3 x (FD-1).

Per la diminuzione di dimensioni si guadagnano +5 x (FD-1) punti in
AGI e si perdono -3 x (FD-1) punti in FOR e COS. PV, RES e PF cambiano
di conseguenza, fino ad un minimo di 0. 

\dt{17} 
\db{2 punti ogni unit\`a di Fattore Dimensione\\ + 1 punto ogni 30 kg di Massa Vivente\\
+ 1 punto ogni 100 kg di Massa Non Vivente}

\es{Paddy vuole intimorire suo fratello diventando 2 volte pi\`u
  grosso. La DT \`e 17. La DB \`e pari a: 4 (2x2 unita di FD) +2
  (60 Kg d massa vivente: Paddy pesa 57 Kg)= 6.
  
  I modificatori sono: +2 (1 target: Paddy) -1 (gittata nulla)= 1. La
  DA \`e pari a: 17 (DT) +6 (DB) +1 (modificatori)= 24. La sua AGI
  diminuisce di 5 x (2 - 1)= 5 punti, la sua FOR e la sua COS
  aumentano di 3 punti}

\subsubsection{Altri tipi di alterazione} 

La magia di Alterazione \`e quella che presenta
pi\`u sfumature, perci\`o risulta difficile catalogare tutti i
tipi di incantesimo di questa classe. In tali casi deve intervenire il
Master per assegnare un valore di DB e uno di DT, basandosi comunque
sugli altri tipi di alterazione. Il resto dei calcoli viene effettuato
di conseguenza. 

{\raggedright \subsection{Spiegazione dei parametri della DB}}
\subsubsection{Complessit\`a}

\`E un numero determinato dal Master che indica quanto \`e
``complicato'' l'oggetto da modificare o da creare o quanto \`e
``radicale'' la trasformazione.

Ad esempio: vale 1 nel caso della creazione di oggetti di forma
semplice fatti di un solo materiale come un cubo di ferro o una pozza
d'acqua, 5 per oggetti semplici composti da pi\`u materiali (ad
esempio una sella rudimentale con finimenti) o per oggetti di un solo
materiale ma di forma pi\`u complessa (ad esempio un grimaldello),
10 in caso di oggetti complessi contenenti meccanismi o di forma
particolare (clessidre, sculture grezze), 15 per macchine complete,
oggetti artistici, sculture molto dettagliate.

Nel cambiamento delle propriet\`a fisiche specifica ``quanto'' si
trasforma il target: il cambiamento di colore ha complessit\`a 1,
quello di odore ha 2; quello di consistenza (come Forma Aria di
Elementalismo) ha 10 ecc.  Cos\`{\i} pure per le trasformazioni
anatomiche: 1 cambia la ``rugosit\`a'' della pelle; 5 attribuisce
agli organi nuove propriet\`a come impermeabilit\`a della pelle
ecc., 10 crea nuovi organi (ali, branchie, tentacoli, braccia,
ecc.) 15 pu\`o cambiare completamente la forma del target (es.
trasforma un uomo in un delfino o in un grosso uccello) senza
cambiarne dimensioni e massa.

\subsubsection{Fattore di dimensione (FD)}
\`E il moltiplicatore o divisore delle dimensioni originarie. Per
raddoppiare o dimezzare le dimensioni di un target il Fattore di
Dimensione \`e 2. Per ingrandire o rimpicciolire un oggetto 3 volte
si applica un Fattore di Dimensione pari a 3, e cos\`{\i} via. Il
Fattore di Dimensione deve essere intero e maggiore di 1.

\iffullversion
\subsubsection{Punti variazione caratteristica} 

\`E il numero di punti totali di variazione dei valori delle
caratteristiche del target.

Per esempio: per costruire un incantesimo che conferisce +2 al CAR e
+4 alla BEL occorrono 6 punti variazione Caratteristica. Per costruire
un incantesimo che diminuisce di 2 punti la FOR e di 2 punti la COS
occorrono 4 punti variazione Caratteristica.
\fi

\subsubsection{Punti variazione abilit\`a}

Determina l'ammontare delle variazioni in aumento o in diminuzione al
punteggio di una abilit\`a.  Il numero dei punti variazione
abilit\`a necessari per incrementare (o decrementare) di 1 il TOT
dell'abilit\`a \`e pari alla difficolt\`a dell'abilit\`a da
variare. Per conferire +5 (o -5) ad una abilit\`a a D1 sono
necessari 5 punti; per conferire +10 (o -10) ad una abilit\`a a D3
sono necessari 30 Punti Variazione Abilit\`a. 

\nb{Ricordiamo che le Abilit\`a di Combattimento fanno aumentare di 2
  punti la DB per ogni punto di variazione;  vanno quindi trattate come
  se avessero codice di difficolt\`a D2}

\subsubsection{Massa vivente}
Determina il peso massimo in kg dei target viventi (o delle parti di
essi) sottoposti all'incantesimo.

\subsubsection{Massa non vivente} Determina
il peso massimo in kg dei target non viventi (oggetti inanimati)
soggetti all'incantesimo. Include vestiti ed equipaggiamento.

\subsubsection{Distanza massima teletrasporto} Per gli incantesimi di
teletrasporto specifica la distanza
massima dello spostamento dal punto in cui viene lanciato
l'incantesimo.

La distanza massima deve essere una potenza di 10, cio\`e 1, 10,
100, 1000, 10000 ecc. metri. Nei calcoli viene usata la potenza,
cio\`e:

\noindent
0 per 1 metro \\
1 per 10 metri \\
2 per 100 metri \\
3 per 1000 metri ... \\
6 per 1000 km ... 

In pratica, si conta il numero di zeri che seguono l' ``1'' nella distanza in metri.

\subsection{Modificatori Standard}

\subsubsection{Durata}

Specifica la durata dell'incantesimo, cio\`e il periodo di tempo
durante il quale l'alterazione ha effetto.

Non si conteggia nel teletrasporto, che ha sempre durata zero.

\subsubsection{Gittata} 

Indica la distanza iniziale massima del
target dal mago. Per gli incantesimi di spostamento non indica la
portata massima dello spostamento: se il mago vuole spostare di 100
metri un oggetto che si trova ad 1 metro da lui, la gittata \`e 1
metro (100 metri \`e la lunghezza dello spostamento).

\subsubsection{Campo} 

Definisce il numero di oggetti
(viventi o non viventi) soggetti all'incantesimo, o l'area entro la
quale l'incantesimo inizia il suo effetto.

\subsubsection{Tiro Resistenza} Se
il target \`e un essere vivente pu\`o resistere all'incantesimo.
Dove non specificato, il target che non intende sottoporsi
volontariamente all'incantesimo deve effettuare un TR a difficolt\`a
pari DB + 10. \`E possibile aumentare la difficolt\`a del TR con un
bonus fino a +5 (o diminuirlo fino a -5) aumentando la difficolt\`a
dell'incantesimo (o diminuendola) di 1 punto ogni +1 di bonus (o -1 di
malus).

\subsection{Note} 
Le alterazioni dello stesso tipo non sono
cumulabili, e i valori vanno sempre calcolati sulle caratteristiche di
base dei target.

\section{Attacco}

Questa categoria comprende tutti gli incantesimi che infliggono danni
diretti ai target.  Il Fallimento Catastrofico specifico porta il mago
a subire il danno dovuto al suo incantesimo e ad accorciare la gittata
di questo a 0.

\subsection{Tipi di Attacco}

\subsubsection{Danno netto}

I target dell'incantesimo perdono un numero di Punti Vita, Punti
Fisico o Punti Mente pari al risultato del tiro di un numero di dadi
stabilito e conteggiato nella DB. Il mago sceglie se agire su PV, PF,
PE o PM al momento di creare l'incantesimo.

I Punti Vita persi non sono localizzati e non sono limitati dalle
armature. Alcuni incantesimi di difesa possono invece proteggere il
bersaglio parzialmente o totalmente.

La realizzazione di un TR ad una difficolt\`a pari alla $DB+10$
consente al bersaglio di subire solo la met\`a dei danni (le
protezioni magiche devono essere conteggiate prima di dimezzare).

L'incantesimo infligge danni addizionali se l'entit\`a del danno \`e
sufficiente a produrli, esattamente come avverrebbe se il danno
fosse stato provocato da qualsiasi altro colpo.

\dt{21} 

\db{2 punti ogni D6 PV di danno}

\es{Mongan vuole attaccare un gruppo di briganti che si staglia
  davanti a lui con una sfera di elettricit\`a che parte dal suo
  corpo e colpisce tutti coloro che gli stanno davanti.
  
  La DT \`e pari a 21; la DB \`e pari a 12 (2 x 6d6 di danno) =
  12, i Modificatori sono: +3 (Gittata 10 m) + 4 (Campo d'azione 4
  metri di raggio) +3 (Azione lungo la gittata). La DA \`e pari a 21
  (DT) + 12 (DB) + 10 (Modificatori) = 43}

\es{Tom vuole che chiunque oltrepassi l'uscio della sua camera da l\`{\i} a
19 minuti subisca danno a causa di una fiammata. La DT \`e 21, la DB \`e
pari a 6 (2x 3d6 PV di danno) =6, i Modificatori sono: +1 (Campo d'azione 1
metro di raggio) +4 (Perch\'e l'area di danno perduri per 19 minuti). La
DA \`e pari a 21 (DT) + 6 (DB) +5 (Modificatori) = 32}

\subsubsection{Danno mirato}
L'incantesimo colpisce esattamente come un attacco fisico, con una base di TPC
e un valore di Danno. 

Pu\`o infliggere danni da Botta, Taglio o Punta e danni
addizionali. Le armature e gli scudi limitano il danno. 

L'attacco mirato a una parte del corpo si pu\`o portare soltanto se
l'incantesimo \`e diretto ad 1 solo target; in caso contrario il
Master determiner\`a per ognuno dei target colpiti la parte del corpo
danneggiata mediante l'uso del ``Fantoccio'' (vedi pagina
\pageref{fantoccio}).

La vittima pu\`o ``Schivare'' l'incantesimo come se fosse
un normale proiettile lanciato con un attacco da ADT. 

Se la base TPC \`e 0, la gittata deve essere nulla (a contatto). Per
infliggere il Danno, in questo caso, il mago deve realizzare un TPC
usando la sua abilit\`a CAC.

\dt{5}

\db{5 punti ogni D6 PV/PSC di danno\\ + 1 punto per ogni punto di TPC base}

\es{Zoran vuole colpire
un cinghiale, destinato al suo piatto, alla testa. Decide cos\`{\i} di scagliargli
un dardo. La DT \`e 5, la DB \`e 10 (5x 2d6 PV/PSC di danno) +15 (Base
TPC) = 25, i modificatori sono: +2 (1 target: il cinghiale) +3 (Gittata 10 metri).
La DA \`e pari a 5 (DT) + 25 (DB) + 5 (Modificatori) = 35}

\subsubsection{Stordimento}
Il target che fallisce il TR non pu\`o eseguire alcuna azione nel round in
corso e in quello successivo.

\dt{9}

\db{Difficolt\`a del TR} 

\es{Noit vuole
stordire 5 rapinatori che accerchiano lui e i suoi compagni per poter cos\`{\i} 
fuggire. La DT \`e 9, la DB \`e pari a 20 (difficolt\`a del TR scelta
da Noit), i Modificatori sono: +1 (Gittata 4 metri) +6 (Campo d'azione 1d6 target).
La DA \`e pari a 9 (DT) + 20 (DB) +7 (Modificatori) = 36}

\subsubsection{Morte} 

I PV dei target passano automaticamente a 0 e le attivit\`a vitali dei target
stessi cessano immediatamente senza che vengano prodotte lesioni (perdita PSC).

Se il target realizza un TR a difficolt\`a decisa dal mago,
l'incantesimo non ha effetto. All'aumentare della difficolt\`a del TR,
naturalmente, la DB aumenta di conseguenza.

L'incantesimo \`e reversibile con un controincantesimo specifico o, a
scelta, con un incantesimo di Resurrezione.

Applicando il modificatore Azione nel Tempo si possono infliggere
malattie mortali (o simili) producendo anche lesioni fisiche. Le
malattie possono comunque essere curate durante il loro corso.

\dt{18}

\db{Difficolt\`a del TR} 

\es{Kayo vuole uccidere una tigre stando a debita distanza.  
  
  La DT \`e pari a 18, la DB \`e pari a 25 (Difficolt\`a del
  TR), i Modificatori sono +2 (1 target, la tigre) +2 (Gittata 7
  metri). La DA \`e 18 (DT) + 25 (DB) + 4 (Modificatori) = 47}

\bigskip

\es{Shel vuol far ammalare il suo rivale in amore di una malattia
  mortale. 

  La DT \`e pari a 18, la DB \`e pari a 20
  (Difficolt\`a del TR), i Modificatori sono +2 (1 target, il
  rivale) -1 (Gittata nulla), -7 (Azione nel tempo, oltre 30 giorni).
  La DA \`e perci\`o pari a 18 (DT) + 20 (DB) -6 (Modificatori) =
  32. La malattia pu\`o essere scelta nella tabella ``Malattie'' tra
  quelle mortali con un TR minore o uguale a quello della DB}

{\raggedright \subsection{Spiegazione dei parametri della DB}}

\subsubsection{D6 PV o PV/PSC di danno} 

Numero di d6 (dadi a sei facce) che determinano l'ammontare del danno.
Il danno netto pu\`o consistere in perdita di PV, ma anche di PM, PF e
PE. Il danno mirato \`e sempre in PSC.

L'incantesimo pu\`o produrre danni addizionali nei round successivi
a seconda della natura dello stesso (ad esempio: incendio, gelo, ecc.)
come descritto a pagina \pageref{danniaddizionali}.

\subsubsection{Base di TPC} 

L'incantesimo infligge il danno mirato solo se la somma Base di TPC +
1d20 (tiro aperto) \`e maggiore del TPD del target (parare o
schivare), e solo se \`e superiore a 20, esattamente come un attacco
fisico.


\subsubsection{Tiro Resistenza} 

La riuscita di un TR annulla gli effetti dell'incantesimo nei casi
di Morte e Stordimento e dimezza i danni nel caso di Danno Netto.

\subsection{Modificatori Standard}

\subsubsection{Gittata} 

Definisce la distanza massima del mago dal/dai target sotto attacco o dal
centro dell'``esplosione''. 
\nb{Tutti i target devono trovarsi ad una distanza minore o uguale alla gittata.
L'incantesimo non avr\`a effetto sui target che si trovano ad una distanza maggiore}

\subsubsection{Campo} 

Gli incantesimi di attacco si possono applicare su uno o pi\`u target
(persone, cose, esseri ecc.)  o su un'area, agendo su tutti gli
oggetti presenti in essa. In questo caso tutti gli oggetti nell'area
subiranno lo stesso danno. L'area pu\`o essere ridotta a discrezione
del mago, al momento del lancio, fino ad 1 metro di raggio.

\iffullversion
\subsubsection{Azione lungo la gittata} 

\`E possibile fare in modo che l'attacco agisca, oltre che nel punto
di arrivo, anche nella traiettoria dal mago al punto di arrivo stesso.
In tal caso, quando l'incantesimo ha un campo d'azione ``ad area'', si
pu\`o scegliere se far seguire una traiettoria ``conica'' (dalle
mani del mago l'incantesimo parte ``espandendosi'' progressivamente
fino a raggiungere l'area colpita) o ``cilindrica'' (l'incantesimo si
``espande'' immediatamente e percorre la traiettoria colpendo una
``striscia'' di larghezza pari al doppio del raggio d'azione).

\subsubsection{Rimbalzi} Se l'incantesimo agisce lungo la gittata \`e possibile
farlo rimbalzare ``a specchio'' su oggetti inanimati per un massimo di cinque
rimbalzi. I bersagli possono essere colpiti pi\`u volte (e subiscono il danno
pi\`u volte) a seconda del rimbalzo. Per calcolare il rimbalzo pu\`o essere
necessario un tiro Calcolare a discrezione del Master.
\fi

\subsubsection{Durata} 

Questo modificatore pu\`o essere usato solo per gli incantesimi che
producono danno su un'area: ogni volta che un essere o un oggetto si
trova a transitare per quell'area (ma non se ci resta) nel corso della
Durata dell'incantesimo, subisce il danno. Per gli altri il
modificatore di Durata \`e sempre zero.

\iffullversion
\subsubsection{Azione nel tempo}
La perdita dei PV/PSC/PM/PF avviene proporzionalmente al tempo
trascorso, fino ad infliggere tutto il danno.  Nel caso di
Morte il bersaglio perder\`a gradualmente tutti i suoi PV/PSC;

I malus alle caratteristiche vengono calcolati di conseguenza.
\fi

\subsubsection{Tiro Resistenza} 

\`E possibile aumentare la difficolt\`a del TR del target con un
bonus fino a +5 (o diminuirlo fino a -5).

La difficolt\`a aumenta di 1 punto ogni +1 alla difficolt\`a del
TR e diminuisce di 1 punto ogni -1 alla difficolt\`a del TR.

\section{Controllo}

Questa categoria comprende tutti gli incantesimi che consentono di
manipolare le menti degli esseri, controllandone il comportamento o
ingannando i loro sensi con illusioni. Il Fallimento Catastrofico
specifico di questa classe porta a far accorgere il target del
tentativo di controllo e alla riuscita automatica di tutti i TV e i
TOSS del target.

\subsection{Tipi di Controllo} 

\subsubsection{Comando} 
\`E possibile
far eseguire al target azioni contro la sua volont\`a. Il target
\`e soggetto a bonus o malus al TV a seconda di quanto sarebbe, in
condizioni normali, disposto o contrario ad eseguire tali comandi.
Possono essere impartiti anche comandi che innescano reazioni
istintive, come dormire, aver fame, rimanere bloccati per la paura
ecc. Applicando la Permanenza Ciclica a questo tipo di incantesimo si
possono impartire ordini ``post-ipnotici'', cio\`e far compiere al
target una stessa azione ogni volta che si verifica un evento (ad
esempio addormentarsi ad ogni schiocco di dita). 

Cambiando il comando
si realizza un nuovo incantesimo. 

\dt{11}

\db{Difficolt\`a del TV} 

\es{Naryt vuole estorcere dei soldi ad un venditore di spezie,
impartendogli un ordine magico. La DT \`e 11, la DB \`e pari a 25
(Difficolt\`a del TV), i Modificatori sono: +2 (1 target: il
mercante) +1 (Gittata 4 metri). La DA \`e pari a 11 (DT) +25 (DB) +3
(Modificatori) = 39}

\subsubsection{Possessione}  

Il mago assume il controllo
completo del corpo del target, perdendo il controllo del proprio ed
entrando in catalessi per tutta la durata dell'incantesimo. Il mago
assume le caratteristiche fisiche del target (FOR, COS, OSS, AGI, BEL
con un malus di -1 dovuto alla differenza di conformazione fisica
rispetto a quella usuale che viene annullato dopo 48 ore), conservando
le proprie mentali e spirituali (INT, CON, CONC, CAR, PSI, PK, PE)
nonch\'e i valori VAL di tutte le abilit\`a (i bonus vanno per\`o
ricalcolati). 

La personalit\`a dell'individuo controllato scompare
fino all'esaurimento dell'incantesimo. In caso di morte del corpo del
mago, ne muore anche la personalit\`a, e la personalit\`a
originaria ritorna al possessore del corpo, il quale non ricorder\`a
niente. 

Il mago non pu\`o rientrare in possesso del proprio corpo se
si trova ad una distanza da questo maggiore della gittata. 

Quando il
corpo posseduto muore, se il corpo del mago si trova entro la gittata,
il mago torna nel suo corpo, in caso contrario muore anch'egli. 

 Se
l'incantesimo \`e reso Permanente la distruzione del corpo del mago
non comporter\`a la morte del mago, ma bens\`{\i} quella della personalit\`a
del target.  Dopodich\'e l'incantesimo non potr\`a pi\`u essere
sciolto. 

Non \`e possibile effettuare questo tipo di
incantesimo mentre ci si trova nel corpo di un target. 

Questo tipo di
incantesimi pu\`o essere applicato solo su 1 target. 

\dt{11}


\db{Difficolt\`a del TV}

\es{Jackie decide di prendere il controllo
della bellssima Hilwan, ma non chiedetevi il perch\'e. La DT \`e
11, la DB \`e 20 (Difficolt\`a del TV), i modificatori sono: +2 (1
target, la bellissima) -1 (Gittata nulla) +8 (Durata 35 minuti). La DA
\`e pari a 11 (DT) + 20 (DB) +9 (Modificatori) = 40. Se il TV
fallisce la personalit\`a di Jackie si impossessa del corpo di
Hilwan per 35 minuti mentre il corpo di Jackie si trova in catalessi.}

\es{Vicious, non pi\`u soddisfatto del suo vecchio corpo, decide di impossessarsi
permanentemente del corpo del giovane Rambus, molto pi\`u prestante di lui.
La DT \`e 11, la DB \`e 20 (Difficolt\`a del TV), i modificatori sono:
+2 (1 Target, Rambus) -1 (Gittata nulla) +15 (Durata permanente). La DA \`e
pari a 11 (DT) + 20 (DB) + 16 (Modificatori) = 47. 

Rambus fallisce il TV e la
sua personalit\`a viene sostituita da quella di Vicious. Vicious brucia il
suo vecchio corpo rendendo cos\`{\i} l'incantesimo non reversibile e causando
la dissoluzione della personalit\`a di Rambus}

\subsubsection{Illusione} 

Consiste
nel trasmettere ai target segnali sensoriali artificiali in modo da far loro
credere di vedere, sentire, toccare oggetti che in realt\`a non ci sono.

Le illusioni possono avere dimensioni e forma qualunque. Al momento del lancio
dell'incantesimo il target effettua un TOSS; se lo realizza l'illusione ha luogo
ma il target si accorge immediatamente che si tratta di un'illusione e pu\`o
farla scomparire realizzando un TV. Il TV viene ripetuto ad ogni round se fallito.
Il TOSS (e il conseguente TV) pu\`o essere ripetuto ogni qualvolta il target
ottenga nuove informazioni sull'illusione. 

L'illusione pu\`o interessare
tutti i sensi o solo alcuni di essi, sommando le DT. 

\dt{1 udito} 

\dt{2 vista}

\dt{2 gusto + olfatto} 

\dt{4 tatto}

\db{Difficolt\`a del TV e del TOSS}

\es{Maggie
vuole giocare uno scherzo a Tyra, facendole credere di incontrare il suo ex
fidanzato che le sussurra parole dolci. La DT \`e 1 (Illusione Uditiva) +2
(Illusione Visiva) = 3, la DB \`e 20 (Difficolt\`a del TOSS e del TV),
i modificatori sono: +2 (1 target, Tyra) +1 (Gittata 4 metri) +1 (Durata 7 minuti).
La DA \`e pari a 3 (DT) + 20 (DB) + 4 (Modificatori) = 27. Se Tyra realizzasse
il TOSS si avvedrebbe dell'illusione, e se realizzasse un successivo TV potrebbe
dissolverla}

\subsubsection{Miraggio} 

Consiste nel creare oggetti virtuali che vengono percepiti come se si
trattasse di oggetti reali.

I miraggi possono avere qualunque forma, e dimensioni limitate dal
campo d'azione dell'incantesimo.

Nel momento in cui un personaggio percepisce l'oggetto illusorio
pu\`o effettuare un TOSS; se lo realizza si accorge immediatamente
che si tratta di un'illusione e pu\`o, se vuole, smettere di
``percepirla'' realizzando un TV.

Il TV pu\`o essere ripetuto ad ogni round se fallito.  

Il TOSS (e il conseguente TV) pu\`o essere ripetuto ogni qualvolta
il personaggio ottenga nuove informazioni sull'illusione. 

Il miraggio pu\`o interessare tutti i sensi o solo alcuni di essi,
sommando le DT.

\dt{1 udito} 

\dt{2 vista}

\dt{2 gusto + olfatto} 

\dt{4 tatto}

\db{Difficolt\`a del TV e del TOSS}


\es{Morn vuole
mostrare agli astanti un leone sperando di intimorirli. 

La DT \`e 1 (Illusione
Uditiva) +2 (Illusione Visiva) =3, la DB \`e 20 (Difficolt\`a del TOSS
e del TV), i modificatori sono: +2 (2 m. di raggio, area in cui deve apparire
il leone ) +1 (Gittata 4 metri) +1 (Durata 7 minuti). La DA \`e pari a 3
(DT) + 20 (DB) + 4 (Modificatori) = 27. Se gli astanti realizzassero il TOSS
si avvedrebbero dell'illusione, e se realizzassero un successivo TV potrebbero
dissolverla}

\iffullversion
\subsubsection{Variazione della caratteristica VOLont\`a} 

Pu\`o essere
variata in aumento o in diminuzione la caratteristica VOL. 

Pu\`o essere effettuata
una sola variazione per volta sulla stessa caratteristica: le
variazioni non sono cumulabili.  

\dt{19}
\db{6 punti ogni Punto Caratteristica di variazione}

\es{ Calvin desidera aumentare la sua
VOL di 3 punti per il suo esame di magia. La DT \`e 19. La DB \`e pari
a: (6 x 3 punti variazione caratteristica)= 18: I modificatori sono: +2 (1
target: Calvin) -1 (gittata nulla)= 1. La DA \`e: 19 (DT) +18 (DB) +1 (modificatori)=38}

\subsubsection{Variazione della caratteristica Psiche} 

Pu\`o essere variata in aumento o in diminuzione la caratteristica
PSI. Le variazioni non sono cumulabili.

\dt{10} 
\db{1 punto ogni Punto Caratteristica di variazione}

\es{Folls sta per entrare, per scommessa, in una
casa che le leggende narrano sia infestata da spettri. Per darsi coraggio decide
di realizzare un incantesimo che incrementa la sua Psiche di 5 punti. La DT
\`e 19, la DB \`e pari a 10 (2 x 5 punti Variazione Caratteristica), i
modificatori sono +2 (1 Target, Folls), -1 (Gittata nulla), +5 (Durata: a concentrazione).
La DA \`e pari a 19 (DT) + 10 (DB) +6 (Modificatori) = 35}
\fi

{\raggedright \subsection{Spiegazione dei parametri della DB}}

\subsubsection{Difficolt\`a del TV e del TOSS} Gli incantesimi
di controllo ``inseriscono'' forzatamente comandi o informazioni
fasulle nelle menti dei target. La mente del target pu\`o ovviamente
opporsi a questa forzatura con un TV.  L'incantesimo \`e tanto pi\`u
potente quanto meglio riesce a penetrare le difese (VOL e OSS) del
target.

\subsubsection{Punti variazione caratteristica}

\`E il numero di punti utilizzabili per variare il valore di una
caratteristica del target. Per esempio: per costruire un incantesimo
che conferisce +2 alla PSI occorrono 2 punti Variazione
Caratteristica. Per costruire un incantesimo che d\`a -2 punti alla
VOL occorrono 2 punti variazione Caratteristica.

\subsection{Modificatori
Standard}

\subsubsection{Gittata.}Definisce la distanza massima del target dall'evocatore.
Il target \`e l'essere che deve essere controllato o ingannato.

\subsubsection{Campo}

Per quanto riguarda il tipo Comando, il numero di target \`e uguale al
numero di comandi moltiplicato per il numero di vittime che li
subiscono. Impartire un comando a 2 vittime ha un campo d'azione di 2
target, impartire 2 comandi a 3 vittime ha un campo d'azione pari a 6
target, e cos\`i via. I comandi sono gli stessi per tutte le vittime.

Con un incantesimo su area si pu\`o impartire un solo Comando per
volta.

Per quanto riguarda l'Illusione, il campo d'azione definisce quali
sono le vittime dell'Illusione stessa, tenendo conto che una illusione
pu\`o essere arbitrariamente complessa.

Per il Miraggio, il numero di target \`e pari al numero di oggetti
illusori creati. Se il Miraggio \`e su area, in quell'area si potranno
creare o far muovere tanti oggetti illusori quanti se ne vuole.

I TV devono essere effettuati separatamente da ogni vittima, per ciascun
comando o illusione.


\subsubsection{Durata} 

Per Comando e Possessione, la durata definisce il periodo di tempo
entro il quale gli ordini possono essere impartiti.  Se si impartisce
l'ordine di compiere una azione che si protrae nel tempo, ne viene
stabilita con questo parametro la Durata massima.  Per le illusioni e
i miraggi, stabilisce la durata delle stesse.

\subsubsection{Tiro Volont\`a/Tiro Osservazione}

 Dove non indicato
esplicitamente il TV ha difficolt\`a pari alla DB +10. 

\`E possibile aumentare
la difficolt\`a del TV o del TOSS con un bonus fino a +5 (o diminuirlo fino
a -5) aumentando la difficolt\`a dell'incantesimo (o diminuendola). Se il
target accetta di buon grado l'incantesimo, il TV non deve essere effettuato,
e deve anzi essere considerato automaticamente fallito.


\section{Difesa} 

Questa categoria comprende tutti gli incantesimi capaci di difendere
autonomamente il bersaglio da attacchi fisici o magici. Il Fallimento
Catastrofico specifico di questa classe porta il bersaglio a ricevere,
per ogni attacco subito, un Danno Addizionale pari al valore di
protezione dell'incantesimo.

\subsection{Tipi di Difesa}

\subsubsection{Armatura} 

Protegge il target da danni (PV/PSC) dovuti ad attacchi fisici o magici da
Danno Mirato, assorbendoli esattamente come un'armatura.

Le categorie di danno che questo incantesimo consente di ridurre sono:

\begin{itemize}
\itemsep -6pt
\item danni da arma/combattimento
\item danni da incantesimi di Attacco/Danno Mirato
\item qualsiasi altro danno localizzato che possa essere
limitato da una armatura non magica.  
\end{itemize}

Protegge solo esseri viventi, creature magiche comprese.

\dt{19}

\db{2 punti per ogni Punto Protezione (PP)}

\es{Jeremiah sta per affrontare
un combattimento contro un Krenay e decide di proteggersi con un'armatura
magica. La DT \`e 19, la DB \`e pari a 16 (2 x 8 PP), i Modificatori sono
+2 (1 Target, Jeremiah) -1 (Gittata nulla). La DA \`e pari a 19 (DT) +16
(DB) +1 (Modificatori) = 36}

\subsubsection{Protezione magica} 

Assorbe un numero di PV di danno dovuti a danno magico diretto
(Attacco/Danno netto), o altri danni non localizzati, di un certo
tipo, che non vengono assorbiti dalle armature (incendio, gelate,
ecc.), pari al risultato del lancio di un numero di d6 stabilito nella
DB.  L'incantesimo agisce su \textbf{ogni danno} subito entro la
durata.

La protezione magica si applica prima del TR per dimezzare i danni da
Attacco/Danno netto.  Realizzando il TR si subisce quindi la met\`a
del danno residuo.

\dt{20}

\db{2 punti per ogni D6 di protezione}

\es{Markin deve entrare nella stanza di Tom, la cui entrata \`e
  protetta da un incantesimo che infligge 3d6 PV di danno da fuoco a
  chiunque la oltrepassi.  Decide di attivare una Protezione Magica.
  La DT \`e 20, la DB \`e pari a 8, (2x 4d6 di protezione), i
  Modificatori sono +2 (1 target: Markin) -1 (Gittata nulla). La DA
  \`e pari a 20 (DT) +8 (DB) +1 (Modificatori). Markin entra nella
  stanza, riceve una fiammata che gli infligge 14 PV. La protezione
  assorbe fino a 15 PV, ottenuti dal tiro di 4d6.
  
  Markin entra incolume nella stanza. Tom si rivela essere in
  compagnia della bellissima Hilwan... Ma sar\`a lei o Jackie?}

\subsubsection{Riparazione danni}
Fa recuperare punti struttura ad oggetti danneggiati. Il numero di PS
recuperati dal target \`e pari al risultato ottenuto con il lancio
di un numero di dadi stabiliti nella DB. L'incantesimo non opera se
l'oggetto \`e totalmente o quasi totalmente distrutto (es.
polverizzato, fuso, ecc.)

\dt{15} 

\db{2 punti per ogni D6 di riparazione}

\es{Rei vuole riparare la sua spada di famiglia rotta
nell'ultimo combattimento. l'arma deve recuperare 13 PS. La DT \`e 15, la
DB \`e 8 (2 x 4d6 di riparazione), i Modificatori sono +2 (1 Target: la spada)
-1 (Gittata nulla). La DA \`e pari a 15 (DT) +8 (DB) +1 (Modificatori) =
24}

\subsubsection{Cura} 

Consente di far recuperare al target un numero di PV/PSC, PF, PM o
PSI, pari al risultato ottenuto con il lancio di un numero di dadi
stabilito nella DB.

I danni prodotti da una malattia mortale, da un veleno mortale o da un
incantesimo di Attacco/Morte con azione nel tempo possono essere
curati solo temporaneamente, rallentando ma non bloccando il loro
decorso (i PV recuperati vengono successivamente persi gradualmente).

\dt{21}

\db{2 punti per ogni D6 di riparazione}

\es{Peitho deve curare le ustioni di due cameriere coinvolte in
un incendio. La DT \`e 21, la DB \`e 8 (2x4d6), i Modificatori sono: +4
(2 target: le cameriere) -1 (Gittata nulla) -3 (Azione nel tempo: 1d6 giorni).
La DA \`e 21 (DT) +8 (DB) +0 (Modificatori) = 29}

\iffullversion
\subsubsection{Resurrezione} 

Restituisce la vita ad un essere che l'ha persa.  La resurrezione ha
successo se il numero di punti ottenuti dal \textbf{tiro del numero di d6
dell'incantesimo \`e maggiore o uguale al numero dei PV totali}
(COSx3 + Bonus FOR, escludendo qualsiasi variazione dovuta a droghe o
magia) del morto da resuscitare.

L'incantesimo elimina il danno che ha portato alla morte (ferita,
malattia, incantesimo).  Il target resuscita in stato di incoscienza e
con 1 solo PV ma in condizioni stabili.

In caso di un incantesimo di morte che agisce nel tempo o di malattia
mortale alla fine del decorso, la ``resurrezione'' pu\`o essere
anticipata, vanificando gli effetti dell'incantesimo di morte o della
malattia mortale. 

Questo procedimento vale anche come antidoto per i veleni mortali
qualora abbiano avuto effetto.

L'incantesimo non agisce su target morti di vecchiaia o di morte
``naturale'',

\dt{21}

\db{2 punti per ogni D6 di riparazione\\ + 1 punto ogni giorno trascorso dalla morte} 

\es{Morgan \`e
appena morto annegato e la sua amata Gayl decide di provare a resuscitarlo.
Morgan aveva 28 PV. La DT \`e pari a 21, la DB \`e pari a 14 (2x7d6)+ 0 (giorni dalla morte).

I Modificatori sono: +2 (1 target: Morgan), -1 (Gittata nulla). La DA \`e
pari a 21 (DT) +14 (DB) +1 (Modificatori) = 36. Il risultato dei 7d6 \`e
36. Morgan \`e ora malconcio ed incosciente (1 PV) ma vivo}

\subsubsection{Antidoto}

Annulla gli effetti dell'assunzione di una droga, di un veleno o
simili e guarisce le malattie. La DB \`e tanto pi\`u alta quanto pi\`u
difficile \`e il TR per resistere agli effetti principali o
collaterali dell'agente invalidante.

L'incantesimo non restituisce i PV persi a causa della malattia, del
veleno o della droga, ma ne arresta semplicemente il decorso. I PV
dovranno essere recuperati, a parte, con cure, incantesimi o col passare
del tempo.

Gli incantesimi di antidoto sono specifici ed attivi contro un solo
agente.  Pu\`o essere necessario un Tiro Medicina per capire di che
agente si tratta.

\dt{19}

\db{Difficolt\`a del TR contro la droga, il veleno o la malattia}

\es{Jack ha preso un Raffreddore, Patch vuole curarglielo con un incantesimo.
La DB \`e 35 (TR contro Raffreddore), la DT \`e 19, i MODificatori
sono +2 (1 target) -1 (a contatto).

L'incantesimo per curare un raffreddore ha quindi DA 55. Se
l'incantesimo riesce, Jack guarisce immediatamente senza recuperare
gli eventuali PV persi fino a quel momento.

L'incantesimo per curare il raffreddore non pu\`o essere usato contro
altre malattie.
}

\subsubsection{Rianimazione}
Fa uscire dal coma il target. Ha successo se \textbf{il numero di
  punti ottenuti dal tiro del numero di d6 dell'incantesimo \`e
  maggiore o uguale al numero dei Punti Vita totali} (COSx3 + Bonus
FOR, escludendo qualsiasi variazione dovuta a droghe o magia) del
target.

L'incantesimo elimina il danno che ha portato al coma.

\dt{15} 

\db{2 punti per ogni d6 di riparazione}

\es{Kowana \`e entrato in coma per una botta alla testa subita in una
  rissa all'osteria. Garull vuole provare a rianimarlo. Kowana ha 40
  PV. La DT \`e 15, la DB \`e 20 (2x10d6), i Modificatori sono: +2 (1
  target: Kowana), -1 (Gittata nulla). La DA \`e pari a 15 (DT) +20
  (DB) +1 (Modificatori) = 36. Il risultato dei 10d6 \`e 41. Kowana
  esce dal coma con la testa sana.}

\subsubsection{Scudo antidinamico} 

Divide per un certo fattore i danni determinati da oggetti veloci
(spade, frecce, bastoni, pietre, pugni ecc.), purch\'e questi non
derivino da un incantesimo di Attacco/Danno Netto.

Non protegge da danni di tipo chimico, da fuoco, gelo ecc.

La protezione \`e efficace contro tutti i Colpi permessi nel CAC tranne
la PREsa.

Protegge solo esseri viventi, creature magiche comprese.

\dt{21}

\db{5 punti per ogni unit\`a di Divisore}

\es{Gunnar e il suo gruppo
stanno per essere travolti da una frana. Gunnar decide di riparare il gruppo
con uno scudo antidinamico. La DT \`e 21, la DB \`e pari a 15 (5x3 unit\`a
di divisore), i Modificatori sono +2 (Campo d'azione 2 metri di raggio) -1 (Gittata
nulla). La DA \`e pari a 21 (DT) +15 (DB) +1 (Modificatori) = 37. Il danno
dovuto alla frana sar\`a diviso per 3}
\fi

\subsubsection{Deviazione} 

Ripara il target da attacchi che non derivino da un incantesimo di
Attacco/Danno Netto, tentando di deviare o parare \textbf{ogni attacco
  durante il tempo} d'azione. Il target non subisce l'attacco se la
somma del valore \textbf{Base TPD + 1d20} (tiro aperto) \`e maggiore
o uguale al valore del TPC dell'attaccante.

Non \`e possibile per il target sapere in anticipo se la deviazione
\`e stata effettuata con successo quindi il target deve dichiarare
la sua azione prima di conoscere il risultato della deviazione.

\dt{25}

\db{Base TPD}
\bigskip
\es{Mullin si trova sotto il tiro di numerosi arcieri. Per
cercare di evitare di essere trafitto si protegge con un incantesimo di Deviazione.
La DT \`e 25, la DB \`e pari a 15 (Base TPD), i modificatori sono +2 (1
target: Mullin) -1 (Gittata nulla). La DA \`e 25 (DT) +15 (DB) +1 (Modificatori)
= 41. Tutte le frecce per le quali il TPC \`e minore o uguale a 15+1d20 non
colpiranno Mullin}

\subsubsection{Parata} 

Ripara il target da attacchi fisici, che non derivino da un
incantesimo di Attacco/Danno Netto parando un certo numero di attacchi
portati entro la Durata. 

Il target non subisce l'attacco se la somma
del valore Base TPD + 1d20 (tiro aperto) \`e maggiore o uguale al
valore del TPC dell'attaccante. Non \`e possibile per il target
sapere in anticipo se la Parata magica \`e stata effettuata con successo
quindi il target deve dichiarare la sua azione prima di conoscere il
risultato della parata. 

L'incantesimo termina comunque, anche dopo un tempo inferiore alla
Durata, dopo aver effettuato con successo le parate indicate nella DB.

\dt{15} 

\db{Base TPD + 2 punti per ogni parata}

\es{Preparandosi a uno scontro, Jester decide di lanciare un
incantesimo che gli pari due attacchi con Base TPD 15. La DT \`e 15, la DB \`e 15
(Base TPD) +4 (2 x 2 parate) = 19, i Modificatori sono: +2 (1 Target:
Jester) -1 (Gittata nulla).  La DA \`e quindi 15 (DT) +19 (DB) +1
(Modificatori) = 35}

\subsubsection{Antimagia}  

Annulla durante la sua attivit\`a \textbf{ogni incantesimo} che
agisce sul target (compresi quelli diretti dal target a se stesso) se
la somma del valore Base Antimagia + 1d20 (tiro aperto) \`e maggiore
o uguale alla DA dell'incantesimo che si intende annullare. 

\dt{15}

\db{Base Antimagia}

\es{Bulhs sta per entrare nella tana di un animale magico e ritiene
  pi\`u salutare proteggersi dai suoi Poteri Speciali.
  
  La DT \`e 15, la DB \`e pari
  a 20 (Base Antimagia), i modificatori sono: +2 (1 target, Bulhs) -1
  (Gittata nulla) +2 (Durata 11 minuti).  La DA \`e 15 (DT) + 20 (DB)
  + 3 (Modificatori) = 38.}

{\raggedright \subsection{Spiegazione dei parametri della DB}}

\subsubsection{D6 di riparazione} 
\`E il numero di D6 di danno che possono essere ``curati''
dall'incantesimo.

\subsubsection{D6 di protezione} 
Determina
quanti D6 di danno magico vengono assorbiti direttamente dalla
barriera.

\subsubsection{Valore di protezione}

\`E il numero di PV/PSC di Danno assorbiti direttamente
dall'incantesimo, come se si trattasse di un'armatura.

\subsubsection{Unit\`a del Divisore}
Stabilisce di quanto vanno divisi i danni da botta, taglio e punta
nello Scudo Antidinamico.  Vale 2 per il dimezzamento, 3 per la
divisione per tre, ecc. Deve essere un numero intero maggiore di 1.

\subsubsection{Base antimagia} 

Gli incantesimi che riparano da altri incantesimi hanno un valore di Base
Antimagia. A questo valore va sommato un Tiro di 1d20 (aperto) e se
tale somma supera la DA dell'incantesimo che si intende annullare,
questo non ha effetto nel campo d'azione dell'Antimagia.

\subsubsection{Base TPD}  Indica la
Base TPD che sommata a 1d20 deve essere uguale o maggiore al TPC
dell'attacco (compresi Danno Mirato, ADT e ADL) che si intende parare.
Vedi ``Il Combattimento'' a pagina \pageref{combattimento}.

\subsection{Modificatori Standard} 

\subsubsection{Gittata} 
Definisce la distanza
massima dal mago degli oggetti o del centro dell'area da proteggere.

\subsubsection{Campo} Determina il raggio dell'area protetta, gli oggetti o
gli esseri protetti. 

\subsubsection{Durata} Specifica per quanto tempo agisce
la protezione

\iffullversion
\subsubsection{Azione nel Tempo} 

Per i tipi Cura, Riparazione Danni, Resurrezione, Rianimazione
specifica in quanto tempo i danni vengono riparati.

Non pu\`o essere usato negli altri casi.
\fi

\subsection{Note} Gli incantesimi di Difesa dello stesso tipo
non si possono sovrapporre.  Il nuovo incantesimo si sostituisce al
vecchio solo se la sua DA \`e maggiore o uguale a quella del
vecchio.

\es{Bogus ha addosso un incantesimo a DA=31 che gli d\`a PP=5. Ne
  lancia un altro, quasi identico, a DA=33 che ha PP=6. I nuovi PP
  SOSTITUISCONO i vecchi. Lo stesso mago, ora, prova a diminuire i PP
  con un incantesimo a DA 29 con PP=4. Il nuovo incantesimo non ha
  effetto, se non sciogliendo prima quello gi\`a in atto.}

In compenso si possono sovrapporre incantesimi di tipo diverso.

Gli incantesimi di difesa agiscono in quest'ordine: 

\begin{itemize}
\item Contro attacchi fisici:
  \begin{enumerate}\itemsep -3pt
  \item Deviazione
  \item Parata
  \item Scudo antidinamico
  \item Armatura
  \end{enumerate}
\item Contro attacchi magici:
  \begin{enumerate}\itemsep -3pt
  \item Antimagia
  \item Protezione magica
  \end{enumerate}
\end{itemize}

\section{Evocazione} Questa categoria include
tutti gli incantesimi che permettono di richiamare, controllare,
proteggersi dalle creature proprie di ciascuna scuola. 

Il fallimento
catastrofico specifico di questa categoria consiste nella riuscita
automatica dei TV degli esseri su cui si fa l'incantesimo. 

\subsection{Tipi di Evocazione}

\subsubsection{Richiamo} 

Richiama una creatura e permette di controllarla. Per richiamare la
creatura \`e necessario conoscerla bene (tiro su Conoscere Angeli,
Animali Magici, Esseri del Limbo, Esseri Onirici, Elementali, Demoni,
pari alla Potenza della creatura + 10).

La creatura ubbidir\`a ciecamente a chi l'ha richiamata se un
confronto di TV al momento dell'incantesimo vede il mago vincere sulla
creatura (vedi paragrafo ``Confronto di TV'' qui sotto), altrimenti la
creatura agir\`a di sua iniziativa.

L'incantesimo richiama creature di potenza minore o uguale alla DB. 

\dt{18} 
\db{Potenza della Creatura} 

\es{Stephan conosce l'incantesimo per evocare Arioch, e prova a
  richiamarne uno durante un assedio. L'incantesimo ha DT 18, DB 30
  (Potenza di Arioch), Modificatori +2 (1 target: Arioch). La DA \`e
  quindi 18 (DT) +30 (DB) +2 (Modificatori) = 50. Se il TV di Stephan
  contro quello dell'Arioch avr\`a successo, Arioch obbedir\`a a
  Stephan per 3 minuti; altrimenti, Stephan se la vedr\`a davvero
  molto brutta!}

\subsubsection{Controllo} 

Consente di controllare creature erranti (agenti di propria
iniziativa) o gi\`a evocate (sotto il controllo di altri).

La creatura ubbidir\`a ciecamente a chi lancia l'incantesimo se un
confronto di TV al momento dell'incantesimo vedr\`a il mago vincere
sulla creatura o su chi la controllava in precedenza. (vedi paragrafo
``Confronto di TV'' pi\`u avanti), altrimenti l'incantesimo non avr\`a
effetto. 

L'incantesimo consente di controllare solo creature con potenza minore
o uguale alla DB.

\dt{14}

\db{Potenza della Creatura}

\es{Gerson si trova davanti una Banshee e decide di provare a
  controllarla per poi rimandarla nel Limbo. 
  
  La DT \`e 14, la DB \`e 16 (Potenza della Banshee), i
  Modificatori sono: +2 (1 target) +1 (Gittata 4 metri). La DA \`e
  pari a 14 (DT) +16 (DB) +3 (Modificatori) = 33}

\subsubsection{Protezione}

Rende le creature incapaci di nuocere agli esseri che si trovano
all'interno del Campo di azione dell'incantesimo, cercando di impedire
alle creature di penetrarvi.

Le creature possono entrare o procedere nell'area del Campo d'azione
dell'incantesimo di protezione se realizzano un TV a difficolt\`a
pari alla DB + 10.

Il TV subisce un malus di -1 per ogni metro all'interno del cerchio.

Se il TV riesce e la creatura entra nel cerchio subisce 1d6 PV di
Danno Netto a round per ogni metro all'interno del cerchio protetto.

Le creature non possono attaccare direttamente a distanza l'evocatore:
gli incantesimi diretti verso l'interno dell'area vengono
automaticamente neutralizzati, gli attacchi fisici a distanza vanno
fuori bersaglio.

All'interno del cerchio la creatura pu\`o portare normalmente
qualunque attacco.

L'incantesimo agisce solo su creature di potenza minore o uguale alla
DB e non agisce sulle creature controllate dall'evocatore.

L'incantesimo si dissolve immediatamente se almeno uno degli esseri
protetti effettua un attacco di qualsiasi tipo contro le creature,
anche se l'attacco non va a segno.

\dt{10}

\db{Potenza della Creatura}

\es{Folin deve entrare in una stanza sorvegliata da un
Kartenius (Incubo) e decide di creare un campo di protezione. La DT \`e 10,
la DB \`e 15 (Potenza del Kartenius), i Modificatori sono +3 (Campo
d'azione 3 metri di raggio), -1 (Gittata nulla). La DA \`e pari a 10
(DT) +15 (DB) +2 (Modificatori) = 27}

{\raggedright \subsection{Spiegazione dei parametri della DB}}

\subsubsection{Potenza della Creatura} 

\`E un numero normalmente
compreso tra 1 e 30 che riassume il livello di potere della creatura
stessa; per conoscere la potenza della creatura consultate le tabelle
con le caratteristiche delle creature. Le creature con potenza
maggiore di 30 non sono evocabili.

\subsection{Modificatori standard}

\subsubsection{Gittata}  

Negli incantesimi di richiamo stabilisce a che
distanza dall'evocatore comparir\`a l'essere richiamato. Normalmente
(e al minimo) l'essere compare ad 1 metro.  Il punto in cui l'essere
comparir\`a deve essere ``a portata di vista''
dell'evocatore, quindi non \`e possibile richiamare un essere dietro
una porta chiusa, dietro un muro, ecc. 

Negli incantesimi di controllo,
definisce la distanza massima alla quale deve trovarsi l'essere da
controllare. Negli incantesimi di protezione, definisce la distanza
massima del mago dal centro dell'area in cui agisce la protezione.

\subsubsection{Campo} 

Negli incantesimi di richiamo definisce il numero di creature
richiamate.

Negli incantesimi di controllo pu\`o indicare il numero delle vittime
oppure un'area circolare all'interno della quale si trovano le vittime
su cui si vuole esercitare il controllo.

Negli incantesimi di protezione definisce il raggio dell'area
circolare protetta.

Il confronto di TV deve essere effettuato separatamente per le diverse
creature, cio\`e si deve effettuare 1 TV per ogni creatura. 

\subsubsection{Durata}  

L'essere richiamato o controllato pu\`o essere comandato a
piacimento, a voce, per tutta la durata dell'incantesimo. Prima del
termine dell'incantesimo l'essere deve essere congedato; se questo non
viene fatto, la creatura diventa errante e agisce di sua iniziativa.

\subsubsection{Confronto di TV} 

Le creature soggette ad un incantesimo di
richiamo o controllo possono ribellarsi. Al momento dell'incantesimo
sia la creatura che l'evocatore effettuano un TV. Se il TV della
creatura non supera il TV dell'evocatore, la creatura viene
controllata e ubbidisce ciecamente, in caso contrario diventa un
essere errante e decide essa stessa come comportarsi.  

Se si tenta di
controllare creature richiamate o controllate da altri maghi, il TV
deve essere confrontato con il risultato del TV realizzato
dall'evocatore originario durante il suo incantesimo. Se il TV del
mago che tenta di ``sottrarre'' il controllo fallisce, il controllo
rimane a colui che lo deteneva in precedenza.  

\`E possibile conferire al proprio tiro un bonus fino a +5 (o un malus
fino a -5) aggiungendo +1 (risp. -1) ai Modificatori per ogni punto di
bonus (risp. malus).

\subsection{Note}

Il controllo e la protezione terminano nel momento in cui l'evocatore
attacca o tenta di infliggere direttamente dei danni alle creature,
anche se l'incantesimo \`e permanente. 

Le creature controllate non ubbidiscono a comandi che le spingono a
infliggere danni a loro stesse.

\section{Informazione}

Questa categoria comprende tutti gli incantesimi capaci di fornire
all'usufruitore notizie, informazioni, dati su esseri, persone,
oggetti, incantesimi, luoghi, in forma di pensiero, immagine, parola,
ecc.

Il fallimento catastrofico porta ad ottenere notizie false o
fuorvianti.

\subsection{Tipi di Informazione} 

\iffullversion
\subsubsection{Variazione della caratteristica CONoscenza} 

Pu\`o essere variata la caratteristica CON in aumento o
in diminuzione. Pu\`o essere effettuata una sola variazione per
volta sulla stessa caratteristica.

\dt{19}

\db{6 punti ogni Punto Caratteristica di variazione}

\es{Malvin desidera aumentare la sua CON di 2 punti per il suo esame
  di ammissione all'Universit\`a. La DT \`e 19. La DB \`e: 12 (6x2
  punti variazione caratteristica): I modificatori sono: +2 (1 target:
  Malvin) -1 (gittata nulla) +5 (Durata: a concentrazione) =6. La DA
  \`e: 19 (DT) +12 (DB) +6 (Modificatori)=37} 
\fi

\subsubsection{Variazione di abilit\`a mentale} 

Consente di attribuire un bonus o un malus ad una o pi\`u abilit\`a
basate su INT, CON o CONC.

Il numero dei punti variazione abilit\`a necessari per incrementare
(o decrementare) di un punto di VAL l'abilit\`a \`e pari al Codice
di Difficolt\`a dell'abilit\`a da variare (D1, D2, D3).

Pu\`o essere effettuata una sola variazione alla volta sulla stessa
abilit\`a, cio\`e: le variazioni alla stessa abilit\`a non sono
cumulabili.

\dt{18} 

\db{1 punto ogni Punto Abilit\`a di variazione}

\es{Halween \`e un avvocato e vuole vincere assolutamente la causa.
  Decide di incrementare il TOT dell'abilit\`a Giurisprudenza di 5
  punti.  La DT \`e 18. La DB \`e: 5 (1x5 punti variazione
  abilit\`a)=5. I modificatori sono: +2 (1 target: Halween), -1
  (gittata nulla) +5 (Durata: a Concentrazione)=6.  La DA \`e pari a:
  18 (DT) +5 (DB) +6 (modificatori)=29}

\subsubsection{Interrogazione}  

Consente di estrarre direttamente informazioni, che non sono
conseguenze di ``deduzioni'', da uno o pi\`u soggetti (oggetti, esseri
o persone) a proposito delle loro esperienze \textbf{dirette}.

L'incantesimo risponde in modo veritiero in base alle esperienze
vissute dal soggetto interrogato.

Se si interrogano oggetti, l'incantesimo fornir\`a informazioni
relative al loro presente o passato, al presente o al passato degli
oggetti e persone a loro circostanti.

La DB aumenta col numero di anni trascorsi dal momento dell'esperienza
sulla quale si indaga.

Se il soggetto sotto interrogazione ha INT maggiore di 4 pu\`o
resistere all'incantesimo effettuando un TV per ogni domanda. Se il TV
fallisce l'incantesimo risponder\`a comunque in modo veritiero alla
domanda. Se il TV riesce l'incantesimo non risponder\`a alla domanda.

Ogni domanda posta \`e da considerarsi 1 target.

\dt{6} 

\db{Difficolt\`a del TV\\ + Numero di anni Trascorsi}

\es{Norbert vuole sapere se qualcuno ha aperto la porta del suo studio
  nei due anni in cui era assente. Interroga la porta del suo studio.
  L'incantesimo ha DT 6, la DB pari a 0 (La porta non ha VOLont\`a) +2
  (Anni trascorsi) =2, I Modificatori sono +2 (1 target: 1
  informazione dalla porta) -1 (Gittata nulla). La DA \`e pari a 6
  (DT) + 2 (DB) +1 (Modificatori) = 9.}

\es{Xunos sta trattando per l'acquisto di un fortral, ma non si fida
  del suo interlocutore, cos\`{\i} decide di leggergli il pensiero per
  verificare la sua sincerit\`a. L'incantesimo ha DT 6, DB pari a 20
  (Difficolt\`a del TV) +0 (Anni trascorsi). I modificatori sono: +2
  (1 target: 1 informazione dall'interlocutore). La DA \`e quindi
  pari a 6 (DT) +20 (DB) + 2 (Modificatori) = 28.}

\subsubsection{Dialogo}

Consente di parlare o comunicare telepaticamente per tutta la Durata
dell'incantesimo con uno o pi\`u target, purch\'e si trovino
entro la Gittata. Il target, per ogni domanda che gli si pone,
effettua un TV a difficolt\`a pari alla DB; se fallisce \`e tenuto
a dire la verit\`a; altrimenti la risposta sar\`a arbitraria.

Se si pone pi\`u volte la stessa domanda allo stesso target nel corso
dello stesso incantesimo, si riceve pi\`u volte la stessa risposta.

Questo tipo di incantesimo \`e applicabile solo a target con una
INT minima di 4 (o che abbiano avuto nel corso della loro vita tale valore di
INT, come, ad esempio, cadaveri). 

Nel caso di domande complesse il target potrebbe
dover effettuare un TINT per verificare se capisce la domanda. 

I target rispondono
solo a colui che pronuncia l'incantesimo. 

\dt{7} 

\db{Difficolt\`a del TV} 

\es{Gonar vuole parlare con il suo trisavolo morto senza costringerlo a dire la
verit\`a. Si reca alla sua tomba e lancia un incantesimo sulla sua mummia.
La DT \`e 7, la DB \`e 0 (Difficolt\`a del TV), I modificatori sono:
+2 (1 target) +8 (Durata 35 minuti). La DA \`e pari a 7 (DT) +0 (DB) + 10
(Modificatori) = 17.}

\iffullversion
\subsubsection{Registrazione} 

Consente di trasferire e memorizzare informazioni (vocali, scritte,
visive, ecc.) da una sorgente posta all'interno della Gittata ad uno
o pi\`u target posti anch'essi ad una distanza minore o uguale alla
Gittata. Ogni copia delle informazioni \`e 1 target. La
complessit\`a \`e un numero stabilito dal Master e indica sia la
``quantit\`a'' che la ``qualit\`a'' delle informazioni registrate.

\nb{I libri di magia o contenenti incantesimi, ecc. non possono essere
  registrati (la registrazione conterr\`a errori, o non sar\`a neppure
  possibile, ecc.)}

\dt{22} 

\db{Complessit\`a}

\es{Shen vuole copiare un breve libro di Storia (il Master ne
  stabilisce la complessit\`a, che \`e 5) che ha trovato in
  biblioteca, sul suo quaderno.  La DT \`e 22, la DB \`e 5
  (Complessit\`a), i Modificatori sono +2 (1 target) -1 (Gittata
  nulla). La DA \`e quindi 22 (DT) +5 (DB) -+1 (Modificatori)= 27}
\fi

\subsubsection{Consiglio} 

Consente di ricavare informazioni su una materia qualunque.  In
pratica l'incantesimo si comporta come un ``esperto'' di una certa
materia che abbia, in una abilit\`a scelta dal mago, un TOT dipendente
dalla DB, rispondendo ad una o pi\`u domande sulla materia stessa.

L'incantesimo, cio\`e, risponde correttamente alle domande, il
cui numero \`e indicato come numero di Target del campo
d'azione, se il TOT dell'abilit\`a + 1d20 (tiro aperto) \`e
maggiore o uguale alla difficolt\`a, stabilita dal Master, della
risposta.

\`E il Master che decide, al momento della domanda, a quale
abilit\`a associare l'incantesimo. 
Se un mago vuole conoscere il valore di una gemma, il Master far\`a,
ad esempio, un tiro su Valutare Pietre con un TOT nell'abilit\`a
deciso dal mago nella DB.

\dt{9} 

\db{TOT dell'abilit\`a}

\es{Thorn desidera sapere
quanti abitanti ha la citt\`a di Koyada. La DT \`e 9, la DB \`e 16
(Valore di base nell'abilit\`a Geografia), i Modificatori sono +2 (1 Target:
1 informazione) -1 (Gittata nulla). La DA \`e quindi 9 (DT) + 16 (DB) +1
(Modificatori) = 26}

\subsubsection{Divinazione} 

Consente al mago di ricavare informazioni su oggetti, persone, cose o
fatti non disponibili. 

\`E possibile prevedere il futuro immediato, senza per\`o ottenere
informazioni precise. La conoscenza del futuro pu\`o consentire ai
personaggi di cambiare il futuro stesso.

La Difficolt\`a dell'informazione \`e stabilita dal Master,
ad esempio: 5=banale: ``cosa c'\`e dentro questa scatola chiusa?'', 10
facile: ``c'\`e qualcuno in casa?''; 20=medio: ``ci sono non\-morti in
questa zona?'', 50=follia: ``c'\`e qualcuno che sta scavando una
trincea a 2000 km da qui?''

L'incantesimo risponde se la Difficolt\`a stabilita dal Master \`e
minore o uguale della Difficolt\`a di Reperimento Informazione della
DB dell'incantesimo.

Il numero di informazioni reperibili \`e pari al numero di Target
definiti nel Campo d'azione.

\dt{9}

\db{Difficolt\`a di Reperimento Informazione\\ + Numero di ore di previsione}

\es{Shingo vuole uscire in barca ma vede nuvole all'orizzonte. Vuole sapere
se ci sar\`a burrasca nelle prossime 6 ore. La DT \`e 9, la DB \`e
pari a 1 (Difficolt\`a di reperimento, basta guardare!) +6 (Ore di previsione)
=7; i Modificatori sono +2 (1 target: 1 informazione) -1 (Gittata nulla). La
DA \`e pari a 9 (DT) +7 (DB) +1 (Modificatori) = 17}

\subsubsection{Trasmissione/ricezione}
Consente ai target di inviare messaggi a qualunque distanza, nelle
forme pi\`u disparate, ed eventualmente ricevere la risposta. La
complessit\`a \`e un numero stabilito dal Master e indica sia la
``quantit\`a'' che la ``qualit\`a'' delle informazioni
trasmesse/ricevute.

\dt{11 (solo trasmissione)} 

\dt{15 (trasmissione+ricezione)}

\db{Complessit\`a\\ + 4 punti per ogni potenza di 10 metri di Distanza Informazione}

\es{Francisco da Terranova vuole spedire un messaggio a Matteo che si
  trova a Rylex a 1000 km da lui, e vuole ottenere anche una risposta.
  La DT \`e 15, la DB \`e 2 (Complessit\`a del messaggio stimata
  dal Master) +24 (4x6 zeri dopo l'1), i modificatori sono +2 (1
  messaggio) -1 (Gittata nulla) -3 (Azione nel tempo 1d6 giorni). La
  DA \`e pari a 15 (DT) +26 (DB) -2 (Modificatori) = 39}

{\raggedright \subsection{Spiegazione dei parametri della DB}}

\subsubsection{Difficolt\`a del TV}
Indica la difficolt\`a del TV che i target devono realizzare per sottrarsi
agli effetti dell'incantesimo. Il TV non \`e necessario se il target vuole
subire l'incantesimo. 

\subsubsection{Difficolt\`a di reperimento dell'informazione}
Per il reperimento di informazioni di qualunque tipo (divinazione) il
Master deve assegnare una Difficolt\`a.

Ad esempio, per scoprire il nome di un personaggio
storico in base a certi dati, si dovrebbe effettuare un tiro in Storia a difficolt\`a
25. Questo numero rappresenta la Difficolt\`a di Reperimento dell'informazione.

In altri casi il Master determina una Difficolt\`a arbitraria in base alla
situazione

\subsubsection{Complessit\`a}

Indica la ``quantit\`a'' e la ``qualit\`a'' dell'informazione. I
messaggi possono essere inviati in forma acustica, scritta, visiva,
tattile, ecc. I messaggi ``acustici'' sono i pi\`u semplici
(complessit\`a 1), mentre quelli compositi (visivi, acustici ecc.)
sono pi\`u complessi (complessit\`a 10 e oltre).

La complessit\`a aumenta anche all'aumentare
della ``lunghezza'' dei messaggi trasmessi. Un libro, ad esempio, ha complessit\`a
5-10; un filmato complesso, con immagini e suoni, pu\`o avere anche 15-20.
Viene determinata dal Master.

\iffullversion
\subsubsection{Punti variazione caratteristica}

\`E il numero di punti utilizzabili per modificare una caratteristica del target. Per
esempio: per costruire un incantesimo che conferisce $+2$ (o $-2$) alla CON sono
necessari 2 Punti Variazione Caratteristica. 
\fi

\subsubsection{Punti variazione abilit\`a}

Determina l'ammontare delle variazioni in aumento o in diminuzione al TOT di
una abilit\`a. Il numero dei Punti Variazione Abilit\`a necessari per
incrementare (o decrementare) di una unit\`a il TOT dell'abilit\`a \`e
pari al Codice di Difficolt\`a dell'abilit\`a da variare. Per conferire
+5 (o -5) ad una abilit\`a a D1 sono necessari 5 punti; per conferire +10
(o -10) ad una abilit\`a a D3 sono necessari 30 Punti Variazione Abilit\`a.

\subsubsection{Distanza Informazione} 

Indica la distanza massima di trasmissione e
ricezione delle informazioni in linea d'aria. Deve essere una potenza di 10,
cio\`e 1, 10, 100, 1000, 10000 ecc. metri. 

Nei calcoli viene usata la potenza, cio\`e:

\noindent
0 per 1 metro\\ 
1 per 10 metri\\ 
2 per 100 metri \\
3 per 1000 metri\\
 ...\\
6 per 1000 km ... \\
ecc.

\subsection{Modificatori Standard}

\subsubsection{Gittata} 

Definisce la distanza massima del target dal mago. Per gli incantesimi
di Divinazione e Consiglio la Gittata si valuta come se fosse nulla (``a
contatto'').

\subsubsection{Campo} 
Definisce il numero di informazioni o il numero di soggetti da cui
prelevare o a cui trasmettere le informazioni, a seconda del tipo
dell'incantesimo.

\subsubsection{Durata} Questo
modificatore pu\`o essere usato per gli incantesimi di Dialogo, quelli di
Variazione Caratteristica CON e Variazione Abilit\`a Mentale. Gli altri tipi
di incantesimo hanno durata ``immediata''.

\iffullversion
\subsubsection{Azione nel Tempo} L'incantesimo
pu\`o non agire immediatamente, ma svolgere la sua azione nel corso
del tempo; cio\`e l'informazione si ottiene o si trasferisce dopo un
tempo variabile o ``gradualmente'' (Registrazione copierebbe un libro
in 6 ore, il messaggio inviato con ``Trasmissione'' arriverebbe in 6
giorni, ecc.).  
\fi

\subsubsection{Tiro Volont\`a}

Se il target \`e dotato di VOLont\`a pu\`o resistere all'incantesimo.
\`E possibile aumentare la difficolt\`a del TV con un bonus fino a +5
(o diminuirlo fino a -5) aumentando la difficolt\`a dell'incantesimo
di 1 punto ogni +1 di bonus (o diminuendola di 1 ogni $-1$ di malus).
Se il target accetta di buon grado l'incantesimo, il TV non deve
essere effettuato.

\iffullversion
\section{Esempio}

 Ora che conoscete le regole della
Magia e sapete costruire un incantesimo, vi spieghiamo in dettaglio
ci\`o che \`e successo tra Seinen e Yoimoi all'inizio del
capitolo, nel paragrafo ``Le basi della Magia'' a pagina \pageref{basimagia}. 

\es{Yoimoi \`e
  un'allieva di Maestro Seinen Lotharielen dei Laftinais,
  illusionista di Bahuney, iscritto nelle liste ufficiali del Conclave.
  
  Il Maestro le ha appena insegnato come ricavare delle informazioni
  dalle menti delle persone che le stanno di fronte. L'incantesimo che
  la ragazza ha improvvisato \`e Lettura del Pensiero della scuola di
  Illusionismo.  
  
  Questo incantesimo fa parte della categoria \textbf{Informazione}, del
  tipo \textbf{Interrogazione}, che consente di ricavare direttamente
  informazioni da oggetti o persone. L'informazione che viene richiesta
  in questo caso \`e il ``pensiero corrente''.
  
  Yoimoi \`e un'adepta della Via del Pensiero (Illusionismo in
  Elfico), che ha una Potenzialit\`a di 16 in Informazione, e ha un
  VALore di 4 nell'abilit\`a magica.
  
  L'incantesimo fatto ha una DA pari a 6 (Difficolt\`a di Tipo DT)
  +15 (Difficolt\`a di Base DB per un TV a 15) + 2 (per 1 target)
  =23, quindi la DR \`e 23 - 16 (la potenzialit\`a)=7.
  
  Il target si trova a meno di 1 metro, quindi la Gittata non si
  conteggia e la Durata \`e minore di 4 minuti, quindi i relativi
  modificatori sono nulli.
  
  Yoimoi ha un valore di concentrazione pari a 14; si concentra per 5
  round per avere +5 al Tiro Concentrazione, che effettua con successo
  ottenendo 6 con il dado (14 +5 +6=25) quindi pu\`o tentare di
  pronunciare l'incantesimo.
  
  Tira 1d20 e ottiene 8, facendo riuscire l'incantesimo, infatti
  4 (abilit\`a magica) +8 (tiro del dado) = 12, che \`e maggiore
  della DR dell'incantesimo (7).
  
  Il target (Maestro Seinen) ha 20 in volont\`a, ma \`e
  favorevolmente predisposto, per sua scelta, per l'incantesimo,
  quindi il suo TV fallisce automaticamente.
  
  L'incantesimo ricava dalla mente di Seinen l'informazione
  desiderata...
  
  Che \`e quello che, di proposito, il furbo maestro stava
  pensando... E quindi ci\`o che ha causato il forte imbarazzo
  nell'allieva. Non specifichiamo che cos'\`e per pudore! Dopotutto
  Seinen \`e un navigatissimo elfo Luxi.
  
  Naturalmente Yoimoi si \`e stancata, perdendo 7 PM e 2 PF
  (rispettivamente, la DR e 1/3 della DR), e consumando 23 dei propri
  PE.}
\fi
%%% Local Variables: 
%%% mode: latex
%%% TeX-master: "manual"
%%% End: 
